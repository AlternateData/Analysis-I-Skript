\documentclass[12pt,a4paper,titlepage]{article} %twopage

\usepackage[ngerman]{babel}
\usepackage[utf8]{inputenc} %Umlaute
\usepackage{amsmath,amssymb,amsthm}
\usepackage{hyperref} %Hyperlinks in pdf

%Metadaten
\hypersetup{
	pdftitle = {Analysis I Skript (WS 18/19)},
	pdfauthor = {Pavel Zwerschke, Daniel Augustin}}

\theoremstyle{definition}
\newtheorem{satz}{Satz}[subsection]
\newtheorem{kor}[satz]{Korollar}
\newtheorem{lem}[satz]{Lemma}

\newtheorem{defi}[satz]{Definition}
\newtheorem*{beh}{Behauptung}

\theoremstyle{remark}
\newtheorem*{bem}{Bemerkung}
\newtheorem*{bsp}{Beispiel}

\newenvironment{bew}{\begin{proof}[Beweis]}{\end{proof}}
\newcommand{\N}{\mathbb{N}}
\newcommand{\Z}{\mathbb{Z}}
\newcommand{\Q}{\mathbb{Q}}
\newcommand{\R}{\mathbb{R}}
\newcommand{\gqq}[1]{\glqq{}#1\grqq{}}
\newcommand{\limes}[1]{\lim\limits_{#1\rightarrow\infty}}

\begin{document}
\title{Analysis I (WS 18/19)}
\date{\today}
\author{Pavel Zwerschke, Daniel Augustin}
\maketitle

%Inhaltsverzeichnis
\tableofcontents
\newpage
\setcounter{section}{1}
\section{Etwas Logik}
\subsection{Grundbegriffe}
\begin{satz}
	Folgende Aussagen sind allein aus logischen Gründen immer wahr.
	\begin{tabular}{rl}
		$\neg(\neg A) \Leftrightarrow A$ & Gesetz der doppelten Verneinung\\
		$A \Rightarrow B \Leftrightarrow \neg B \Rightarrow \neg A$ & Kontraposition\\
		$A \Rightarrow B \Leftrightarrow (\neg (A \wedge \neg B))$ & beim Widerspruchsbeweis\\
		$\neg(A \wedge B) \Leftrightarrow (\neg A \vee \neg B)$ & de Morgan\\
		$\neg(A \vee B) \Leftrightarrow (\neg A \wedge \neg B)$ & de Morgan\\
	\end{tabular}
\end{satz}
\begin{defi}
	Sei $M$ eine Menge und $A(x)$ Aussagen mit $x\in M$\\
	$\forall x \in M: A(x)$ ist wahr, falls alle $A(x)$ wahr sind.\\
	$\exists x\in M:A(x)$ ist wahr, falls mindestens eine Aussage $A(x)$ wahr ist.
\end{defi}
\begin{defi}[wichtige Mengen]
	Seien $M,N$ Mengen.
	\begin{align*}
	\emptyset &:= \text{ die Menge ohne Elemente (leere Menge)}\\
	M \cap N &:= \{x | x \in M \wedge x \in N \} \text{ (Schnitt)}\\
	M \cup N &:= \{x | x \in M \vee x \in N \} \text{ (Vereinigung)}\\
	M \setminus N &:= \{x | x \in M \wedge x \notin N \} \text{ (Differenzmenge)}\\
	\mathcal{P}(M) &:= \{A | A \subset M \} \text{ die Menge aller Teilmengen von } M \text{ (Potenzmenge)}
	\end{align*}
	Sei $I$ eine Menge und für $i\in I$ eine Menge $M_i$.
	\begin{align*}
	\bigcap\limits_{i\in I} M_i &:= \{x | \forall i \in I:x\in M_i\}.\\	
	\bigcup\limits_{i\in I} M_i &:= \{x | \exists i \in I:x\in M_i\}.
	\end{align*}
	Ist $M\cap N = \emptyset$, so heißen $M$ und $N$ divergent.\\
	$M\subset N$, falls $\forall x \in M: x \in N$ ($M$ Teilmenge von $N$).\\
	$M = N$, falls $M$ und $N$ dieselben Elemente haben.\\
	Insbesondere ist $(M=N) \Leftrightarrow M\subset N \wedge N\subset M$.\\
	$M \subsetneq N : M \subset N \wedge M \neq N$ ($M$ echte Teilmenge von $N$).
\end{defi}
\section{Die reellen Zahlen}
\subsection{Körperaxiome (engl. field)}
\begin{defi}[Körperaxiome]
	In einem Körper gelten diese Axiome:
	\begin{enumerate}
		\item Kommutativität: $\forall a,b\in \mathbb{K}: a+b=b+a, a\cdot b=b\cdot a$
		\item Assoziativität: $\forall a,b,c\in \mathbb{K}: a+(b+c) = (a+b)+c, a\cdot (b\cdot c) = (a\cdot b)\cdot c$
		\item Existenz des neutralen Elements: \\
		$\exists 0 \in \mathbb{K}: a + 0 = 0 + a = a \forall a\in \mathbb{K}\\
		\exists 1 \in \mathbb{K}: a \cdot 1 = 1 \cdot a = a \forall a\in\mathbb{K}$
		\item Existenz eines inversen Elements:\\
		$\forall a\in\mathbb{K}\exists -a\in\mathbb{K}:a+ (-a)=0\\
		\forall a\in\mathbb{K}\setminus\{0\}\exists \frac{1}{a}\in\mathbb{K}:a\cdot\frac{1}{a}=1$\\
		Es gilt: $0 \neq 1$.
		\item Distributivgesetz: $\forall a,b,c\in\mathbb{K}:a\cdot(b+c)=a\cdot b + a \cdot c$
	\end{enumerate}
\end{defi}
\begin{defi}
	Zu $a\in \mathbb{K}$ ist $-a$ das Inverse bzgl. der Addition\\
	schreibe $a-b := a + (-b)$.\\
	Zu $a\in\mathbb{K}\setminus\{0\}$ sei $a^{/1}$ das Inverse bzgl. der Multiplikation.\\
	Ist $b\neq 0$, so schreiben wir $\frac{a}{b} := a\cdot b^{-1} =b^{-1}\cdot a$.\\
	schreibe $(ab) := a\cdot b$.
\end{defi}
\begin{lem}[Rechnen in einem Körper].%FORMATIERUNG SCHLECHT
	\begin{enumerate}
		\item Umformen von Gleichungen\\
		$\forall a,b,c\in\mathbb{K}:$\\
		aus $a+b=c$ folgt $a=c-b$\\
		aus $a\cdot b=c$, $b\neq 0$ folgt $a=\frac{c}{b}$
		\item Allgemeine Rechenregeln\\
		$\begin{aligned}
		-(-a) &= a\\
		(a^{-1})^{-1}&=a\text{, falls }a\neq 0\\
		-(a+b) &= (-a) + (-b)\\
		(a\cdot b)^{-1}&=b^{-1}\cdot a^{-1}=a^{-1}\cdot b^{-1}\\
		a\cdot 0&=0\\
		a(-b)&=-(ab), (-a)(-b)=ab\\
		a(b-c) &= ab - ac\\
		ab &= 0 \Leftrightarrow a = 0 \vee b = 0 \text{ (Nullteilerfreiheit)}\\ 
		\end{aligned}$
	\end{enumerate}
\end{lem}
\subsection{Die Anordnungsaxiome}
\begin{defi}
	Sei $\mathbb{K}$ (genauer $(\mathbb{K},+,\cdot)$) ein Körper. Dann heißt $>$ eine Anordnung falls
	\begin{enumerate}
		\item Für jedes $a\in\mathbb{K}$ gilt genau eine der Aussagen $a>0,a=0,-a>0$\\
		(wenn $a\in\mathbb{K}$, mit $a>0$ positiv)
		\item Aus $a>0$ und $b>0$ folgt\\
		$a+b>0$ und $a\cdot b>0$
	\end{enumerate}
	Wir nennen $(\mathbb{K},+,\cdot,>)$ einen angeordneten Körper.
\end{defi}
\setcounter{satz}{0}%HUNDERTMARK HAT FALSCH GEZÄHLT
\begin{satz}
	Sei $(\mathbb{K}, +,\cdot, >)$ ein angeordneter Körper. Dann gilt
	\begin{enumerate}
		\item für $a,b\in\mathbb{K}$ gilt genau eine der Relationen $a>b, a=b, a<b$ (Trichotromie)
		\item Aus $a>b, b>c$ folgt $a>c$ (Transitivität)
		\item Aus $a>b$ folgt:\\
		$\begin{cases}
		a+c>b+c, \forall c\in\mathbb{K}\\
		ac>bc, \text{ falls } c>0\\
		ac<bc, \text{ falls } c<0
		\end{cases}$.
		\item Aus $a>b$ und $c>d$ folgt:\\
		$\begin{cases}
		a+c>b+d\\
		ac>bd, \text{ falls } b,d>0 %WARUM FALLS b,d>0???
		\end{cases}$
		\item Für $a\neq 0$ ist $a^2 >0$.
		\item Aus $a>0$ folgt $\frac{1}{a}>0$.
		\item Aus $a>b>0$ folgt $0<\frac{1}{a}<\frac{1}{b}$.
		\item Aus $a>b, 0<\lambda<1$ folgt $b<\lambda b + (1-\lambda)a<a$.
	\end{enumerate}
\end{satz}
\begin{defi}[Betrag]
	Sei $(\mathbb{K}, +,\cdot,>)$ ein angeordneter Körper.\\
	Betrag von $a\in\mathbb{K}$ ist gegeben durch\\
	$\left| a\right| := 
	\begin{cases}
	a, \text{ falls } a\geq 0\\
	-a, \text{ falls }a<0
	\end{cases}$\\
	auch noch $a,b\in \mathbb{K}$\\
	$\max(a,b) := 
	\begin{cases}
	a, \text{ falls } a\geq b\\
	b, \text{ falls } a < b
	\end{cases}\\
	\min(a,b) := 
	\begin{cases}
	a, \text{ falls } a\leq b\\
	b, \text{ falls } a > b
	\end{cases}$\\
\end{defi}
\begin{satz}
	$(\mathbb{K},+,\cdot,>)$ ang. Körper\\
	Dann gilt $\forall a,b\in\mathbb{K}:$
	\begin{enumerate}
		\item $\left|-a\right| = \left| a\right|$ und $a\leq\left|a\right|$
		\item $\left| a\right| \geq 0$ und $\left| a\right| = 0 \Leftrightarrow a = 0$
		\item $\left| ab\right| = \left| a\right| \left| b\right|$
		\item $\left| a+b\right| \leq \left| a\right| + \left|b\right|$ (Dreiecksungleichung)
		\item $\left|\left|a\right|-\left|b\right|\right| \leq \left| a-b\right|$ (umgekehrte Dreiecksungleichung)
	\end{enumerate}
\end{satz}
\begin{kor}[\glqq geometrisch-arithmetische Ungleichung\grqq]
	Sei $(\mathbb{K},+,\cdot,>)$ ein ang. Körper, $a,b\in\mathbb{K}\\
	\Rightarrow ab \leq \left( \dfrac{a+b}{2}\right)^2$.\\
	Wenn Gleichheit gilt, so folgt $a=b$.
\end{kor}
\subsection{Obere und untere Schranken, Supremum und Infimum}
\begin{defi}
	Sei $A\subset \mathbb{K}, A\neq \emptyset$. Dann ist $\gamma\in\mathbb{K}$ die kleinste obere Schranke (oder Supremum), falls $A\leq \gamma$ und aus $A\leq n$ folgt $\gamma \leq n$.\\
	Schreibe $\gamma = \sup A = \sup(A)$.\\
	Analog: $\beta$ it die größte untere Schranke von $A$ (Infimum), falls $\beta \leq A$ und aus $\eta \leq A$ folgt $\eta \leq \beta$\\
	Schreibe $\beta = \inf A = \inf(A)$.
\end{defi}
\begin{lem}
	$A\subset\mathbb{K}, A\neq \emptyset$.
	\begin{enumerate}
		\item $\alpha := \sup A \Leftrightarrow \alpha \geq A \wedge \forall \varepsilon > 0 \exists a \in A: \alpha - \varepsilon < a$.
		\item $\beta := \inf B \Leftrightarrow \beta \leq B \wedge \forall \varepsilon > 0 \exists b \in B: b < \beta + \varepsilon$.
	\end{enumerate}
\end{lem}
\subsection{Das Vollständigkeitsaxiom}
\begin{defi}
	Ein angeordneter Körper $(\mathbb{K},+,\cdot,>)$ erfüllt das Vollständigkeitsaxiom, falls\\
	\begin{center}
		Jede nichtleere, nach oben beschränkte Teilmenge hat ein Supremum.
	\end{center}
	Solch einen Körper nennt man ordnungsvollständig. $\mathbb{R}$, der Körper der reellen Zahlen, ist \underline{der} ordnungsvollständige Körper. (Im Wesentlichen gibt es nur einen!)
\end{defi}
\subsection{Die natürlichen Zahlen $\mathbb{N}$}
\begin{defi}
	Eine Teilmenge $M\subset \mathbb{R}$ heißt \underline{induktiv}, falls
	\begin{enumerate}
		\item $1\in M$
		\item Aus $x\in M$ folgt $x+1 \in M$
	\end{enumerate}
\end{defi}
\begin{defi}[natürliche Zahlen] .\\ %SCHLECHT FORMATIERT
	$\mathbb{N} := \{x\in \mathbb{R}: \text{ für jede induktive Teilmenge } M\in\mathbb{R} \text{ gilt } x\in M\} := \underset{M\subset \mathbb{R}\text{ ist induktiv}}{\bigcap} M$
\end{defi}
\begin{satz}[Archimedisches Prinzip für $\mathbb{R}$] .%SCHLECHT FORMATIERT
	\begin{enumerate}
		\item $\mathbb{N}$ ist (in $\mathbb{R}$) \underline{nicht} nach oben beschränkt!
		\item $\forall x\in\mathbb{R}$ mit $x>0 \exists n\in \mathbb{N}: \frac{1}{n} < x$.
	\end{enumerate}
\end{satz}
\begin{satz}[Induktionsprinzip]
	Sei $M\subset \mathbb{N}$ mit 
	\begin{enumerate}
		\item $1\in M$
		\item Ist $x\in M \Rightarrow x+1 \in M$
	\end{enumerate}
	Dann ist $M= \mathbb{N}$.
\end{satz}
\begin{kor}[Vollständige Induktion]
	Für $n\in\mathbb{N}$ seien $A(n)$ Aussagen.
	Es gelte:
	\begin{enumerate}
		\item $A(1)$ ist wahr.
		\item aus $A(n)$ ist wahr folgt $A(n+1)$ ist wahr.
	\end{enumerate}
\end{kor}
\begin{satz}[Bernoullische Ungleichung]
	$$x\in\mathbb{R}, x\geq -1, n\in\mathbb{N}_0 = \mathbb{N}\cup \{0\}$$
	gilt $(1+x)^n \geq 1+ n + x (\forall m\in\mathbb{N},x\geq -1)$\\mit \glqq $>$\grqq, falls $n>1, x\neq 0$\\
	$( \forall n\in\mathbb{N}, x \geq -1(1+x)^n \geq 1 + nx)$
\end{satz}
\begin{satz}[geometrische Summe]
	Sei $x\neq 1$, dann ist $$\sum_{k=0}^{n}x^k = \frac{1-x^{n+1}}{1-x}$$
\end{satz}
\begin{satz}[Eigenschaften von $\mathbb{N}$]
	Es gilt
	\begin{enumerate}
		\item $\forall m,n \in \mathbb{N}: n+m \in \mathbb{N}$ imd $n\cdot m \in \mathbb{N}$.
		\item $\forall n\in\mathbb{N}: n=1$ oder $(n>1$ und demnach $n-1 \in \mathbb{N})$.
		\item $\forall m,n\in\mathbb{N}:m\leq n:n-m\in\mathbb{N}_0$.
		\item $\forall n\in\mathbb{N}$ gibt es kein $m\in\mathbb{N}: n<m<n+1$.
	\end{enumerate}
\end{satz}
\section{Funktionen und Abbildungen}
\subsection{Funktion als Abbildung}
\begin{defi}
	Eine Funktion (oder Abbildung) von einer Menge $A$ in eine Menge $B$ ordnet jedem Element $a\in A$ ein \underline{eindeutiges} Element $b \in B$ zu.\\
	Wir schreiben:
	\begin{equation*}
	f: A \rightarrow B, a \mapsto f(a)\quad(=b)
	\end{equation*}	
	$A$: Definitionsbereich\\
	$B:$ Zielbereich (Target(space))\\
	z.B. $f: \mathbb{R} \rightarrow \mathbb{R}, x \mapsto x^2$\\
	Die Abbildung $f: A \rightarrow B$ ist\\
	\begin{tabular}{r|l}
		injektiv&aus $f(a) = f(a'), a, a' \in A$, folgt $a=a'$\\
		surjektiv&$\forall b\in B \exists a\in A: b=f(a)$\\
		bijektiv&sie ist injektiv und surjektiv
	\end{tabular}
\end{defi}
\setcounter{section}{9}
\section{Häufungspunkte, Limes Superior, Limes Inferior}
\begin{defi}
	Sei \((a_n)_n\) Folge. \(a\in\R\) Häufungspunkt von \((a_n)_n\), falls \(\forall\varepsilon>0 a_n\in U_\varepsilon(a) \) für unendlich viele \(n\in\N\) (d. h. \( \forall K\in\N \exists n>K:a_n \in U_\varepsilon (a) \))
\end{defi}
\begin{lem}
	\begin{enumerate}
		\item Ist \((a_n)_n\) beschränkt. So ist \(H((a_n)_n) \neq \emptyset\).
		\item Konvergiert \((a_n)_n\) gegen \(a\), so ist \(H((a_n)_n) = \{a\} \).
	\end{enumerate}
\end{lem}
\begin{satz}
	\[ b\in H((a_n)_n) \Leftrightarrow (a_n)_n \text{ hat Teilfolge } (a_{n_k})_k \text{, die gegen } b \text{ konvergiert}. \]
\end{satz}
\begin{defi}
	Geg. reelle Folge \((a_n)_n\).\\
	\begin{align*}
	\limsup\limits_{n\rightarrow\infty} a_n := \limes{n}\underset{k\geq n}{\sup} a_k \text{ (}=\inf\underset{k\geq n}{\sup}a_k\text{)}\\
	\liminf\limits_{n\rightarrow\infty} a_n := \limes{n}\underset{k\geq n}{\inf} a_k \text{ (}=\sup\underset{k\geq n}{\sup}a_k\text{)}
	\end{align*}
\end{defi}
\begin{lem}
	\begin{enumerate}
		\item Es gilt immer \( \liminf\limits_{n\rightarrow\infty}a_n \leq \limsup\limits_{n\rightarrow\infty}a_n \)
		\item \((a_n)_n\) beschränkte Folge: Dann gilt \((a_n)_n\) konvergiert \( \Leftrightarrow \liminf\limits_{n\rightarrow\infty}a_n = \limsup\limits_{n\rightarrow\infty}a_n \Leftrightarrow \liminf\limits_{n\rightarrow\infty}a_n \geq \limsup\limits_{n\rightarrow\infty}a_n \).
	\end{enumerate}
\end{lem}
\begin{satz}
	Sei \((a_n)_n\) beschränkte Folge. Dann gilt:\\
	\(a^* = \limsup\limits_{n\rightarrow\infty}a_n \Leftrightarrow \)
	\begin{enumerate}
		\item \(\forall a> a^* \) ist \( \{n\in\N |a_n>a \} \) endliche Teilmenge von \(\N\).
		\item \( \forall a<a^* \) gibt es \(\infty\)-viele \(a_n >a\). (d. h. \( \{ n\in\N|a_n>a \} \) ist unendliche Teilmenge von \(\N\)).
	\end{enumerate}
	genauso für \( \liminf \):
	\(a_* = \liminf\limits_{n\rightarrow\infty}a_n \Leftrightarrow \)
	\begin{enumerate}
		\item \(\forall a< a_* \) ist \( \{n\in\N |a_n<a \} \) endliche Teilmenge von \(\N\).
		\item \( \forall a>a_* \) gibt es \(\infty\)-viele \(a_n<a\).
	\end{enumerate}
\end{satz}
\begin{kor}
	Sei \((a_n)_n\) beschränkte Folge.
	\[ \Rightarrow H((a_n)_n) \subset [ \liminf\limits_{n\rightarrow\infty}a_n, \limsup\limits_{n\rightarrow\infty}a_n ] \text{ und } \limsup a_n, \liminf a_n \in H((a_n)_n). \]
\end{kor}
\section{Konvergente Reihen, Teil 1}
\begin{defi}
	Dieses Symbol \[ \sum_{n=1}^{\infty}a_n \] steht für die Folge \((S_n)_n\) der Partialsummen: \[S_n := \sum_{k=1}^{n}a_k .\]
	Die Reihe \( \sum_{k=1}^{\infty}a_k \) konvergiert, falls die Folge ihrer Partialsummen konvergiert. In diesem Fall setzen wir 
	\[ \sum_{k=1}^{\infty}a_k := \limes{n} \sum_{k=1}^{\infty}a_k = \limes{n} S_n \]
	Sagen auch die Reihe \( \sum_{n=1}^{\infty}a_n \) konvergiert. Die Reihe \( \sum_{n=1}^{\infty}a_n \) divergiert, falls \( (S_n)_n \) divergiert. Reihe \( \sum_{n=1}^{\infty}a_n \) heißt bestimmt divergent, falls \((S_n)_n\) bestimmt gegen \(+ \infty \) oder \(-\infty\) divergiert. Setzen dann \( \sum_{n=1}^{\infty}a_n =-\infty \), falls \(\limes{n} \sum_{k=1}^{n}a_k = -\infty \). \( \sum_{n=1}^{\infty}a_n = +\infty \), falls \( \limes{n}\sum_{k=1}^{n}a_k=+\infty \)
\end{defi}
\begin{satz}
	Sind \(a_n \geq 0, n\in\N \), dann gilt\\
	entweder ist \(S_n = \sum_{k=1}^{n}a_n\) nach oben beschränkt und dann ist \[ \sum_{k=1}^{\infty}a_k = \limes{n}S_n \in [0,\infty) \]
	oder \(S_n \rightarrow \infty, n\rightarrow\infty \) und dann ist \[ \sum_{k=1}^{\infty}a_k = \limes{n}S_n = +\infty. \]
\end{satz}
\begin{kor}
	Sind \(a_n \geq 0, n\in \N \), so ist
	\begin{align*}
	\text{entweder } &\sum_{n=1}^{\infty}a_n < \infty &\text{in diesem Fall ist die Reihe nach oben beschränkt.}\\
	\text{oder } &\sum_{n=1}^{\infty}a_n = +\infty &\text{nach oben unbeschränkt.}
	\end{align*}
\end{kor}
\begin{satz}[Cauchy-Kriterium für Reihen]
	Seien \(a_n \in\R, n\in\N\). Dann gilt: 
	\[ \sum_{n=1}^{\infty} a_n \text{ konvergiert } \Leftrightarrow \forall \varepsilon >0 \exists K \in \N : | \sum_{j=n+1}^{m}a_j|<\varepsilon \quad \forall m>n\geq K.\]
\end{satz}
\begin{kor}
	Wenn die Reihe \(\sum_{n=1}^{\infty} a_n\) konvergent, so ist \((a_n)_n\) eine Nullfolge.
\end{kor}
\begin{satz}
	Gilt \(0\leq a_n \leq b_n \quad \forall n\in\N \), so folgt:
	\begin{enumerate}
		\item Ist \( \sum_{n=1}^{\infty}b_n \) konvergent, so konvergiert auch \( \sum_{n=1}^{\infty}a_n \) und \( \sum_{n=1}^{\infty} a_n \leq \sum_{n=1}^{\infty}b_n \).
		\item Divergiert \( \sum_{n=1}^{\infty}a_n \), so divergiert auch \( \sum_{n=1}^{\infty}b_n \) (gegen \(+\infty\)).
	\end{enumerate}
\end{satz}
\begin{satz}
	Sind \( \sum_{n=1}^{\infty}a_n, \sum_{n=1}^{\infty}b_n \) konvergente (reelle Reihen), so konvergiert auch \( \sum_{n=1}^{\infty}(\lambda a_n + \mu b_n) \quad \forall \lambda, \mu \in\R \) und 
	\[ \sum_{n=1}^{\infty} (\lambda a_n + \mu b_n) = \lambda \sum_{n=1}^{\infty}a_n + \mu \sum_{n=1}^{\infty}b_n. \]
\end{satz}
\begin{satz}[Cauchys Verdichtungskriterium]
	Sei \((a_n)_n\) mon. fallende Nullfolge, so gilt:
	\[ \sum_{n=0}^{\infty} \text{ konvergiert } \Leftrightarrow \text{ die \glqq kondensierte\grqq{} Reihe} \sum_{n=0}^{\infty}2^n a_{2^n} = a_1 + 2a_2 + 4a_4 + 8a_8 + \dots \text{ konvergiert.} \]
\end{satz}
\end{document}