\documentclass[12pt,a4paper,titlepage]{article} %twopage

\usepackage[ngerman]{babel}
\usepackage[utf8]{inputenc} %Umlaute
\usepackage{amsmath,amssymb,amsthm}
\usepackage{hyperref} %Hyperlinks in pdf

%Metadaten
\hypersetup{
	pdftitle = {Analysis I Skript (WS 18/19)},
	pdfauthor = {Pavel Zwerschke, Daniel Augustin}}

\theoremstyle{definition}
\newtheorem{satz}{Satz}[subsection]
\newtheorem{kor}[satz]{Korollar}
\newtheorem{lem}[satz]{Lemma}

\newtheorem{defi}[satz]{Definition}
\newtheorem*{beh}{Behauptung}

\theoremstyle{remark}
\newtheorem*{bem}{Bemerkung}
\newtheorem*{bsp}{Beispiel}

\newenvironment{bew}{\begin{proof}[Beweis]}{\end{proof}}
\newcommand{\N}{\mathbb{N}}
\newcommand{\Z}{\mathbb{Z}}
\newcommand{\Q}{\mathbb{Q}}
\newcommand{\R}{\mathbb{R}}
\newcommand{\gqq}[1]{\glqq{}#1\grqq{}}
\newcommand{\limes}[1]{\lim\limits_{#1\rightarrow\infty}}

\begin{document}
\title{Analysis I (WS 18/19)}
\date{\today}
\author{Pavel Zwerschke, Daniel Augustin}
\maketitle

%Inhaltsverzeichnis
\tableofcontents
\newpage
\section{Was ist Analysis?}
\subsection{Grundbegriffe}
\begin{satz}
	Folgende Aussagen sind allein aus logischen Gründen immer wahr.
	\begin{tabular}{rl}
		$\neg(\neg A) \Leftrightarrow A$ & Gesetz der doppelten Verneinung\\
		$A \Rightarrow B \Leftrightarrow \neg B \Rightarrow \neg A$ & Kontraposition\\
		$A \Rightarrow B \Leftrightarrow (\neg (A \wedge \neg B))$ & beim Widerspruchsbeweis\\
		$\neg(A \wedge B) \Leftrightarrow (\neg A \vee \neg B)$ & de Morgan\\
		$\neg(A \vee B) \Leftrightarrow (\neg A \wedge \neg B)$ & de Morgan\\
	\end{tabular}
\end{satz}
\end{document}