\documentclass[12pt,a4paper,titlepage]{article} %twopage

%Befehle

%Grafik einbinden
%\centering
%\includegraphics[width=0.7\textwidth]{Profilbild.png}


%\usepackage{german} %deutsches Format
\usepackage[ngerman]{babel}
\usepackage[utf8]{inputenc} %Umlaute
\usepackage{graphicx} %Grafiken einbinden
\usepackage{amsmath}
\usepackage{amssymb}
\usepackage{amsthm}
\usepackage{nicefrac}
\usepackage{hyperref} %Hyperlinks in pdf

%Metadaten
\hypersetup{
	pdftitle = {Analysis I Skript (WS 18/19)},
	pdfauthor = {Pavel Zwerschke}}


\newtheorem{satz}{Satz}[section]
\newtheorem{lem}{Lemma}

\theoremstyle{definition}
\newtheorem*{defi}{Definition}
\newtheorem*{beh}{Behauptung}

\theoremstyle{remark}
\newtheorem*{bem}{Bemerkung}
\newtheorem*{bsp}{Beispiel}

\newenvironment{bew}{\begin{proof}[Beweis]}{\end{proof}}


\begin{document}
\title{Analysis I (WS 18/19)}
\date{\today}
\author{Pavel Zwerschke}
\maketitle

%Inhaltsverzeichnis
\tableofcontents
\newpage

\setcounter{section}{-1}
\section{Organisatorisches}
\textbf{Dozent}\\
Prof. Dr. Dirk Hundertmark (20.30, 2.028)\\
\href{mailto:dirk.hundertmark@kit.edu}{dirk.hundertmark@kit.edu}\\
\textbf{Übungsleiter}\\
Dr. Markus Lange (20.30, 2.030)\\
\href{mailto:markus.lange@kit.edu}{markus.lange@kit.edu}\\
\textbf{Übungszettel}\\
Ausgabe:\\
donnerstags unter \href{http://www.math.kit.edu/iana1/lehre/ana12018w/}{\texttt{www.math.kit.edu/iana1/lehre/ana12018w/}}\\
Abgabe:\\
bis mittwochs um 19:00 in den Abgabekästen des Foyers des Mathematikgebäudes (20.30)\\
getackert, mit Namen, Matrikelnummer, Tutoriennummer und Deckblatt (optional) in das Fach mit der richtigen Kennzeichnung legen\\
Zettel dürfen zu zweit abgegeben werden\\
\textbf{Übungsschein}\\
Jede K-Aufgabe wird mit 4 Punkten bewertet. Einen Übungsschein erhält wer 50\% der Punkte aller K-Aufgaben erzielt.\\
\textbf{Klausur}\\
Die Anmeldung findet über das Online-Portal statt. Die Klausur findet in KW 8 2019 statt. Der Übungsschein ist Voraussetzung für die Teilnahme an der Klausur.

\newpage

\section{Was ist Analysis?}
\textbf{Zentrale Begriffe:}\\
Grenzwerte von Folgen und Reihen, Funktionen, stetig, differenzierbar, integrieren, Differential- und Integralrechnung, Differentialgleichungen (Newton, Maxwell, Schrödinger), unendlich dimensionale Räume
\begin{bsp}
	$S = \frac{1}{2} + \frac{1}{4} + \dots + \frac{1}{2^n} + \dots\\
	2S = 1 + \frac{1}{2} + \dots + \nicefrac{1}{2} + \dots\\
	2S = 1 + S$\\
	$S$ entspricht der Wahrscheinlichkeit, dass irgendwann mal Kopf in einem Münzwurf kommt.\\
	Vorsicht!\\
	$S = 1 + 2 + 4 + \dots\\
	2S = 2 + 4 + 8 + \dots = -1 + 1 + 2 + 4 + \dots = -1 + S\\
	S = -1$\\
	Natürlich Quatsch!\\
  	Formales Rechnen kann gefährlich sein!

	\begin{itemize}
		\item Was sind mathematische Aussagen?
		\item Wie macht man Beweise, wie findet man sie? (learning by doing)
		\item logische Zusammenhänge
	\end{itemize}
\end{bsp}
\section{Etwas Logik}
Eine (mathematische) Aussage ist ein Ausdruck, der wahr oder falsch ist.\\
z. B.
\begin{enumerate}
  \item $A :$ \glqq $1+1=2$.\grqq (auch \glqq $1+1=3$\grqq, \glqq $1+1=0$\grqq)
  \item $B :$ \glqq Es gibt unendlich viele Primzahlen.\grqq
  \item $C :$ \glqq Es gibt unendlich viele Primzahlen $p$ für die $p + 2$ auch eine Primzahl ist.\grqq
  \item $D :$ \glqq Die Gleichung $m \ddot{x} = F$ hat geg. $\dot{x}(0) = v_0, x(0) = x_0$ immer genau eine Lösung.\grqq
  \item $E :$ \glqq Jede gerade natürliche Zahl größer als 2 ist die Summe zweier Primzahlen.\grqq
  \item $F :$ \glqq Morgen ist das Wetter schön.\grqq
  \item $G :$ \glqq Ein einzelnes Atom im Vakuum mit der Kernladungszahl $Z$ kann höchstens $Z+1$ Elektronen binden.\grqq
  \item $H(k,m,n) :$ \glqq Es gilt: $k^2 + m^2 = n^2$.\grqq (z. B. $H(3,4,5)$ ist wahr.)    
\end{enumerate}
Gegeben für natürliche Zahlen $n$, Aussagen $A(n)$, dann gilt:\\
Für jede nat. Zahl $n$ ist $A(n)$ wahr, genau dann, wenn 
\begin{enumerate}
  \item $A(1)$ ist wahr.
  \item Unter der Annahme, dass $A(n)$ wahr ist, folgt, dass $A(n+1)$ wahr ist.
\end{enumerate}
\bsp 
\begin{center}
$A(n): 1+2+3+\dots+n=\frac{n(n+1)}{2}$.
\end{center}
\begin{bew} 
	Vollständige Induktion\\
	Induktionsanfang:\\
	$1 = \frac{1(1+1)}{2}$ \checkmark\\
	Induktionsschluss:\\
	Wir nehmen an, dass $A(n)$ wahr ist (für $n\in\mathbb{N}$)\\
	D. h. Induktionsannahme:\\
	$1+2+3+\dots+n=\frac{n(n+1)}{2}$\\
	Dann folgt:\\
	$\underbrace{1+2+\dots+n}_{=\frac{n(n+1)}{2}}+(n+1)=\frac{n(n+1)}{2}+(n+1) \\
	=\frac{n(n+1) + 2(n+1)}{2} = \frac{(n+1)(n+2)}{2} = \frac{(n+1)((n+1)+1)}{2}$
\end{bew}
\begin{bem}
	Gaußscher Trick:\\
	1)\\
	$S=1+2+3+\dots+n=n+(n-1)+(n-2)+\dots+2+1\\
	2S = \underbrace{(n+1)+(n+1)+\dots+(n+1)}_{n\text{-mal}} \Leftrightarrow S = 
	\frac{n(n+1)}{2}$.\\
	2)\\
	$S_n = 0 + 1 + 2 + \dots + n\\=$ Anzahl der Punkte in\\
	%BILD EINFUEGEN
	$\approx$ Fläche eines rechtwinkligen Dreiecks $=\frac{1}{2}*n*n$.\\
	Also: Ansatz (\glqq geschicktes Raten\grqq , \glqq scientific guess\grqq , englisch: ansatz):\\
	$S_n = \underbrace{a_2 n^2 + a_1 n + a_0}_{\text{Polynom 2. Grades in n}}\\
	a_2 = \frac{1}{2}$\\
	Wie bekommt man $a_0, a_1, (a_2)$?
	$n=0: S_0 = 0 = a_2*0^2+a_1*0+a_0 \Rightarrow a_0 = 0$.\\
	$n=1: S_1 = 1 = a_2*1^2+a_1*1^2 = a_2 + a_1 = \frac{1}{2} + a_1$.\\
	also: $a_1 = \frac{1}{2}$\\
	$\Rightarrow$ Raten: $S_n = \frac{1}{2}n^2 + \frac{1}{2} n = \frac{n(n+1)}{2}$.
\end{bem}

\subsection{Grundbegriffe}
Aussagen: Notation\\
\begin{tabular}{r|l}
	$:$ & \glqq so, dass gilt\grqq\\
	$\exists$ & \glqq es gibt mindestens ein\grqq , \glqq es existiert\grqq\\
	$\forall$ & \glqq für alle\grqq\\
	$\Rightarrow$ & \glqq impliziert\grqq($A \Rightarrow B$ \glqq aus $A$ folgt $B$\grqq)\\
	$\Leftrightarrow$ & \glqq genau dann, wenn\grqq\\
	$\neg A$ & nicht $A$\\
	$A \wedge B$ & $A$ und $B$\\
	$A \vee B$ & $A$ oder $B$\\
	$A := B$ & $A$ ist per Definition gleich $B$
\end{tabular}

\begin{satz}
	Folgende Aussagen sind allein aus logischen Gründen immer wahr.
	\begin{tabular}{rl}
		$\neg(\neg A) \Leftrightarrow A$ & Gesetz der doppelten Verneinung\\
		$A \Rightarrow B \Leftrightarrow \neg B \Rightarrow \neg A$ & Kontraposition\\
		$A \Rightarrow B \Leftrightarrow (\neg (A \wedge \neg B))$ & beim Widerspruchsbeweis\\
		$\neg(A \wedge B) \Leftrightarrow (\neg A \vee \neg B)$ & de Morgan\\
		$\neg(A \vee B) \Leftrightarrow (\neg A \wedge \neg B)$ & de Morgan\\
	\end{tabular}
\end{satz}
\begin{bem}
	$A \Rightarrow B \Leftrightarrow B $ ist mindestens so wahr wie $A \Leftrightarrow A$ ist mindestens so falsch wie $B \Leftrightarrow \neg B \Rightarrow \neg A$.\\$(A\Leftrightarrow B) \Leftrightarrow (A \Rightarrow B \wedge B \Rightarrow A)$.
\end{bem}
\begin{bsp}
	$n \in \mathbb{N}$ ist gerade, falls $k \in \mathbb{N}$ existiert mit $n = 2k$.\\
\end{bsp}
\end{document}
