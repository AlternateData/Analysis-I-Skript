\documentclass[12pt,a4paper,titlepage]{article} %twopage

%\usepackage{german} %deutsches Format
\usepackage[ngerman]{babel}
\usepackage[utf8]{inputenc} %Umlaute
\usepackage{graphicx} %Grafiken einbinden
\usepackage{amsmath,amssymb,amsthm}
\usepackage{nicefrac}
\usepackage{marvosym}[Lightning]
\usepackage{array}
\usepackage{dsfont} %statt \mathbb{1}
\usepackage{tikz}
\usetikzlibrary{matrix}
\usepackage{hyperref} %Hyperlinks in pdf

%Metadaten
\hypersetup{
	pdftitle = {Analysis I Skript (WS 18/19)},
	pdfauthor = {Pavel Zwerschke, Daniel Augustin}}

\theoremstyle{definition}
\newtheorem{satz}{Satz}[subsection]
\newtheorem{kor}[satz]{Korollar}
\newtheorem{lem}[satz]{Lemma}

\newtheorem{defi}[satz]{Definition}
\newtheorem*{beh}{Behauptung}

\theoremstyle{remark}
\newtheorem*{bem}{Bemerkung}
\newtheorem*{bsp}{Beispiel}

\newenvironment{bew}{\begin{proof}[Beweis]}{\end{proof}}
\newcommand{\N}{\mathbb{N}}
\newcommand{\Z}{\mathbb{Z}}
\newcommand{\Q}{\mathbb{Q}}
\newcommand{\R}{\mathbb{R}}
\newcommand{\C}{\mathbb{C}}
\newcommand{\gqq}[1]{\glqq{}#1\grqq{}}
\newcommand{\limes}[1]{\lim\limits_{#1\rightarrow\infty}}
\newcommand{\limessup}[1]{\limsup \limits_{#1\rightarrow\infty}}
\newcommand{\limesinf}[1]{\liminf \limits_{#1\rightarrow\infty}}

\begin{document}
\title{Analysis I (WS 18/19)}
\date{\today}
\author{Pavel Zwerschke, Daniel Augustin}
\maketitle

%Inhaltsverzeichnis
\tableofcontents
\newpage

\setcounter{section}{-1}
\section{Organisatorisches}
\textbf{Dozent}\\
Prof. Dr. Dirk Hundertmark (20.30, 2.028)\\
\href{mailto:dirk.hundertmark@kit.edu}{dirk.hundertmark@kit.edu}\\
\textbf{Übungsleiter}\\
Dr. Markus Lange (20.30, 2.030)\\
\href{mailto:markus.lange@kit.edu}{markus.lange@kit.edu}\\
\textbf{Übungszettel}\\
Ausgabe:\\
donnerstags unter \href{http://www.math.kit.edu/iana1/lehre/ana12018w/}{\texttt{www.math.kit.edu/iana1/lehre/ana12018w/}}\\
Abgabe:\\
bis mittwochs um 19:00 in den Abgabekästen des Foyers des Mathematikgebäudes (20.30)\\
getackert, mit Namen, Matrikelnummer, Tutoriennummer und Deckblatt (optional) in das Fach mit der richtigen Kennzeichnung legen\\
Zettel dürfen zu zweit abgegeben werden\\
\textbf{Übungsschein}\\
Jede K-Aufgabe wird mit 4 Punkten bewertet. Einen Übungsschein erhält wer 50\% der Punkte aller K-Aufgaben erzielt.\\
\textbf{Klausur}\\
Die Anmeldung findet über das Online-Portal statt. Die Klausur findet am 26.03.2019 von 08:00 bis 10:00 statt. Der Übungsschein ist Voraussetzung für die Teilnahme an der Klausur.

\newpage
%17.10.2018
\documentclass[../ana1.tex]{subfiles}
\begin{document}

\begin{prosa}
\section{Was ist Analysis?}
\begin{description}
	\item[Mathematik]
		Streng logisches Herleiten neuer Aussagen (aus möglichst wenigen Grundannahmen, sogenannten Axiomen).
	\item[Analysis]
		Aus dem altgriechischen \glqq Auflösen\grqq. Analysis hat ihre Grundlage in der \glqq Infinitesimalrechnung\grqq\: von Leibnitz und Newton.
	\item[Zentrale Begriffe]
		Grenzwerte von Folgen und Reihen, Funktionen, Stetigkeit, Differenzierbarkeit, Integrierbarkeit, Differential- und Integralrechnung, Differentialgleichungen (Newton, Maxwell, Schrödinger), unendlich dimensionale Räume.
\end{description}
\begin{bspe}\leavevmode
	\begin{enumerate}[(1)]
		\item Summe über den Kehrwert von Zweierpotenzen.
			\begin{alignat*}{3}
							   &&             S &= \frac{1}{2} + \frac{1}{4} + \cdots + \frac{1}{2^{n}} + \cdots \\
				\Longrightarrow&\quad& 		 2S &= 1 + \frac{1}{2} + \cdots + \frac{1}{2^{n-1}} + \cdots \\
				\Longrightarrow&\quad& 	     2S &= 1 + S \\
				\Longrightarrow&&             S &= 1
			\end{alignat*}
			\(S\) entspricht der Wahrscheinlichkeit, dass bei wiederholtem Werfen \\
			einer Münze irgendwann Kopf vorkommt.\\
		\item Vorsicht!
			\begin{alignat*}{3}
							   &&       S &= 1 + 2 + 4 + \cdots\\
				\Longrightarrow&\quad& 2S &= 2 + 4 + 8 + \cdots = S - 1\\
				\Longrightarrow&\quad&  S &= -1
			\end{alignat*}
			Natürlich Quatsch! Formales Rechnen kann gefährlich sein!
	\end{enumerate}
\end{bspe}
\begin{description}
	\item[Fragestellungen in dieser Vorlesung]\leavevmode
		\begin{itemize}[-]
			\item Was sind mathematische Aussagen?
			\item Wie macht man Beweise, wie findet man sie? (learning by doing)
			\item logische Zusammenhänge
			\item Was sind Zahlen?
		\end{itemize} 
\end{description}
\end{prosa}
\end{document}
\documentclass[../ana1.tex]{subfiles}
\begin{document}
\setcounter{section}{1}

\section{Etwas Logik}
\begin{defi*}
	Eine (mathematische) Aussage ist ein Ausdruck, der entweder wahr oder falsch ist.
\end{defi*}

\begin{bspe}\leavevmode
	\begin{enumerate}[(1)]
		\item \(A\): \gqq{\(1+1=2\).}  (auch \gqq{\(1+1=3\)}, \gqq{\(1+1=0\)})
		\item \(B\): \gqq{Es gibt unendlich viele Primzahlen.}
		\item \(C\): \gqq{Es gibt unendlich viele Primzahlen \(p\), für die \(p + 2\) auch eine Primzahl ist.} (Primzahlzwillingsvermutung)
		\item \(D\): \gqq{Die Gleichung \(m \ddot{x} = F\) hat, gegeben \(\dot{x}(0) = v_0, x(0) = x_0\), immer genau eine Lösung.} (Lösung der Newtonschen Gleichung)
		\item \(E\): \gqq{Jede gerade natürliche Zahl größer als 2 ist die Summe zweier Primzahlen.} (Goldbach Vermutung)
		\item \(F\): \gqq{Morgen ist das Wetter schön.}
		\item \(G\): \gqq{Ein einzelnes Atom im Vakuum mit der Kernladungszahl \(Z\) kann höchstens \(Z+1\) Elektronen binden.} (Ionisierungsvermutung, es ist noch nicht einmal bekannt, ob es eine Zahl \(Z\) gibt, sodass höchstens \(Z+1\) Elektronen gebunden werden.)
		\item \(H(k,m,n)\): \gqq{Es gilt: \(k^2 + m^2 = n^2\).} (z.B. \(H(3,4,5)\) ist wahr.)
	\end{enumerate}
\end{bspe}

\begin{bem}
	Sei  \(A(n)\) eine Aussage für jede natürliche Zahl \(n\in \N\). Dann gilt, dass
	\(A(n)\) für jedes \(n\) wahr ist, genau dann, wenn gilt
	\begin{enumerate}[(i)]
		\item \(A(1)\) ist wahr.
		\item \(A(n)\) wahr \(\implies A(n+1)\) ist wahr.
	\end{enumerate}
\end{bem}

\begin{bsp}
	\(A(n): 1+2+3+\cdots+n = \frac{n(n+1)}{2}\).
	\begin{bew}(Durch vollständige Induktion)
		\induktion{
			\(1 = \frac{1(1+1)}{2}\).
		}{
			\(A(n)\) gelte für ein \(n\in\mathbb{N}\).
		}{
			\(\begin{aligned}[t]
				\underbrace{1+2+\cdots+n}_{=\frac{n(n+1)}{2}\IV} + (n+1) &= \frac{n(n+1)}{2}+(n+1)
				=\frac{n(n+1) + 2(n+1)}{2}\\
				&= \frac{(n+1)(n+2)}{2} = \frac{(n+1)((n+1)+1)}{2}.
			\end{aligned}\)
			\newline\qedhere
		}
	\end{bew}
\end{bsp}
%18.10.2018
\begin{bem}[Gaußsche Summenformel]\leavevmode
	\begin{enumerate}[(a)]
		\item \(\begin{aligned}[t]
				&	         		 & S &= 1 + 2 + 3 + \cdots + n = n + (n-1)+ (n-2) + \cdots + 2 + 1\\
				&\implies 		     &2S &= \underbrace{(n+1)+(n+1)+\dots+(n+1)}_{n\text{-mal}}\\
				&\Longleftrightarrow & S &= \frac{n(n+1)}{2}.
			  \end{aligned}\)
		\item \( S_{n} \coloneqq 0 + 1 + 2 + \cdots + n = \) entspricht der Anzahl der Punkte in einem 
			  rechtwinkligen Dreieck mit dem Flächeninhalt \(\frac{1}{2}\cdot n \cdot n\).\newline
			  Ansatz (\glqq geschicktes Raten\grqq , \glqq scientific guess\grqq , englisch: ansatz):
			  \[S_{n} = \underbrace{a_{2}n^{2} + a_{1}n + a_{0}\text{,}}_{\text{Polynom }2\text{. Grades in }n}\quad a_{2} = \frac{1}{2}\]
			  Wie bekommt man die Werte von \(a_0, a_1,\) und \(a_2\)?
			  \[\begin{aligned}[t]
				n=0\colon& S_{0} = 0 = a_{2}0^{2} + a_{1}0 + a_{0}               &\implies& a_{0} = 0.\\ 
				n=1\colon& S_{1} = 1 = a_{2}1^{2} + a_{1}1 = \frac{1}{2} + a_{1} &\implies& a_{1} = \frac{1}{2}.
			  \end{aligned}\]
			  \(\implies S_{2} = \frac{1}{2}n^{2} + \frac{1}{2}n = \frac{n(n+1)}{2}\).
	\end{enumerate}
\end{bem}

\subsection{Grundbegriffe}
\begin{notation}\leavevmode
	\begin{center}
		\begin{tabular}{r|l}
			\(\colon\)               & \glqq so, dass gilt\grqq                                                    \\
			\(\exists\)         & \glqq es gibt mindestens ein\grqq , \glqq es existiert\grqq                 \\
			\(\forall\)         & \glqq für alle\grqq                                                         \\
			\(\Rightarrow\)     & \glqq impliziert\grqq(\(A \Rightarrow B\) \glqq aus \(A\) folgt \(B\)\grqq) \\
			\(\Leftrightarrow\) & \glqq genau dann, wenn\grqq                                                 \\
			\(\neg A\)          & nicht \(A\)                                                                 \\
			\(A \wedge B\)      & \(A\) und \(B\)                                                             \\
			\(A \vee B\)        & \(A\) oder \(B\)                                                            \\
			\(A \coloneqq B\)          & \(A\) ist per Definition gleich \(B\)
		\end{tabular}
	\end{center}
\end{notation}

\begin{satz}
	Folgende Aussagen sind allein aus logischen Gründen immer wahr.
	\begin{center}
		\begin{tabular}{rl}
			\(\neg(\neg A) \Leftrightarrow A\)                            & Gesetz der doppelten Verneinung \\
			\(A \Rightarrow B \Leftrightarrow \neg B \Rightarrow \neg A\) & Kontraposition                  \\
			\(A \Rightarrow B \Leftrightarrow (\neg (A \wedge \neg B))\)  & beim Widerspruchsbeweis         \\
			\(\neg(A \wedge B) \Leftrightarrow (\neg A \vee \neg B)\)     & de Morgan                       \\
			\(\neg(A \vee B) \Leftrightarrow (\neg A \wedge \neg B)\)     & de Morgan                       \\
		\end{tabular}
	\end{center}
\end{satz}

\begin{bem}\leavevmode
	\begin{enumerate}[(a)]
		\item \(\begin{aligned}[t]
					(A \Rightarrow B) &\Leftrightarrow B \text{ ist mindestens so wahr wie } A\\
									&\Leftrightarrow A \text{ ist mindestens so falsch wie } B\\
									&\Leftrightarrow \neg B \Rightarrow \neg A.
			    \end{aligned}\)
		\item \(\begin{aligned}[t]
			    	(A \Leftrightarrow B) &\Leftrightarrow ((A \Rightarrow B) \wedge (B \Rightarrow A)).
			    \end{aligned}\)
	\end{enumerate}
\end{bem}
\begin{defi*}
	Sein \(n \in \N\), dann definiere:
	\begin{enumerate}[(a)]
		\item \(n\) ist gerade \(\longeq \exists k \in \N \colon n = 2k\).
		\item \(n\) ist ungerade \(\longeq \exists k \in \N_{0} \colon n = 2k + 1\).
	\end{enumerate}
\end{defi*}
\begin{bsp}
	Es gilt $n$ ist gerade \(\iff n^2\) ist gerade.
	\begin{bew}
		\equirl{
			\(n\) gerade \(\implies n=2k\) für ein \(k \in \N\).\\
			\(\implies n^{2} = (2k)^{2} = 4k^{2} = 2(2k^{2})\) ist gerade.
		}{
			(Durch Kontraposition) \(n\) ungerade \(\implies n=2k + 1\) für ein \(k \in \N_{0}\).\\
			\(\implies n^{2} = (2k+1)^{2} = 4k^{2} + 4k + 1 = \underbrace{2(2k^{2} + 2k)}_{\text{gerade}} + 1\) ist ungerade.
		}
	\end{bew}
\end{bsp}

\begin{defi*}[Informelle Mengendefinition nach Cantor]\leavevmode\\
	Eine Menge ist eine Sammlung von Objekten (Elemente) zu einem neuen Objekt.
\end{defi*}

\begin{notation}\leavevmode
	\begin{enumerate}[(a)]
		\item \(a\) ist ein Element von \(M \longeq a \in M\).\\
			  \(a\) ist kein Element von \(M \longeq a \notin M\).
		\item Beschreibung durch Auflisten: \(M = \set{x_{1} \ko x_{2} \ko x_{3} \ko \ldots \ko x_{17}}\).\\
			  Beschreibung durch Eigenschaften: \(M = \set{a \; \vert \; a \text{ hat Eigenschaft } E}\).
	\end{enumerate}
\end{notation}

\begin{bspe}\leavevmode
	\begin{enumerate}[(1)]
		\item \(\N \coloneqq \set{1 \ko 2 \ko 3 \ko \ldots}\)
		\item \(-\N \coloneqq \set{-n \; \vert \; n \in \N}\)
		\item \(\Z \coloneqq \set{x \; \vert \; x \in \N \; \vee \; x \in -\N \; \vee \; x = 0 }\)
	\end{enumerate}
\end{bspe}

\begin{defi}
	Sei $M$ eine Menge und $A(x)$ Aussagen mit $x\in M$.
	\begin{enumerate}[(a)]
		\item \(\forall x \in M \colon A(x)\) ist wahr, falls alle \(A(x)\) wahr sind.
		\item \(\exists x \in M \colon A(x)\) ist wahr, falls mindestens eine Aussage \(A(x)\) wahr ist.
	\end{enumerate}
	Achtung: Reihenfolge der Quantoren ist wichtig!
\end{defi}

\begin{bsp}
	Töpfe \(\coloneqq\) Menge der Töpfe, Deckel \(\coloneqq\) Menge der Deckel.\\
	\(\forall T \in \text{Töpfe} \quad \exists D \in \text{Deckel} \colon D \text{ passt auf } T\)\\
	\(\iff\) Für jeden Topf gibt es mindestens einen Deckel, der passt.
	\(\exists D \in \text{Deckel} \quad \forall T \in \text{Töpfe} \colon  D \text{ passt auf } T\)\\
	\(\iff\) Es existiert mindestens ein Deckel, der auf alle Töpfe passt.
\end{bsp}

\begin{bem}[Negation von quantifizierten Aussagen]
	\begin{flalign*}
		\neg (\forall x \in M \colon A(x)) &\Longleftrightarrow \exists x \in M \colon \neg A(x).&&\\
		\neg (\exists x \in M \colon A(x)) &\Longleftrightarrow \forall x \in M \colon \neg A(X).&&
	\end{flalign*}
\end{bem}

\begin{defi}[Mengenoperationen]
	Seien \(M,N,I\) Mengen und für \(i \in I\) sei \(M_{i}\) eine Menge.
	\begin{enumerate}[(a)]
		\item \(\emptyset\) ist die Menge ohne Elemente (leere Menge).
		\item \(M \subseteq N \longeq \forall x \in M \colon x \in N\). M ist Teilmenge von N.
		\item \(M \subsetneq N \longeq M \subseteq N \vee M \neq N\). M ist echte Teilmenge von N. 
		\item \(M \cap N \coloneqq \set{x \; \vert \; x \in M \wedge x \in N}\) heißt Schnitt von \(M\) und \(N\).
		\item \(M \cup N \coloneqq \set{x \; \vert \; x \in M \vee x \in N}\) heißt Vereinigung von \(M\) und \(N\).
		\item \(M \setminus N \coloneqq \set{x \; \vert \; x \in M \wedge x \notin N}\) heißt Differenz von \(M\) und \(N\).
		\item \(\PO(M) \coloneqq \set{A \; \vert \; A \subset M}\) heißt Potenzmenge von \(M\).
		\item \(\bigcap\limits_{i \in I} M_{i} \coloneqq \set{x \; \vert \; \forall i\in I \colon x \in M_{i}}\).
		\item \(\bigcup\limits_{i \in I} M_{i} \coloneqq \set{x \; \vert \; \exists i\in I \colon x \in M_{i}}\).
	\end{enumerate}
	Ist \(M \cap N = \emptyset\) so heißen \(M\) und \(N\) disjunkt.
\end{defi}

\begin{bsp}\leavevmode
	\begin{enumerate}[(1)]
		\item Es gilt immer \(\emptyset \subseteq M\), für jede Menge \(M\).
		\item \(M = \set{1\ko2} \implies \PO(M) = \set{\emptyset\ko\set{1}\ko\set{2}\ko\set{1\ko2}}\)
	\end{enumerate}
\end{bsp}

\begin{bem}[Eigenschaften von \(\subseteq\)] Seien \(M\ko N\ko A\ko B\ko C\) Mengen. Dann gilt:
	\begin{enumerate}[(a)]
		\item \(\emptyset \subseteq M\).
		\item \(M \subseteq M\).
		\item \(M = N \iff M \subseteq N \vee N \subseteq M\).
		\item \(A\subset B \wedge B \subset C \iff A \subset C\).
		\item \(\begin{rcases*}
				(A\cup B) \cup C = A \cup (B \cup C)\;\\
				(A\cap B) \cap C = A \cap (B \cap C)\;
			  \end{rcases*}\) Assoziativität.
		\item \(\begin{rcases*}
			    A\cup B = B \cup A\;\\
		        A\cap B = B \cap A\;
			  \end{rcases*}\) Kommutativität.
		\item \(\begin{rcases*}
			    A \cap (B\cup C) = (A\cap B) \cup (A\cap C)\;\\
			    A \cup (B\cap C) = (A\cup B) \cap (A\cup C)\;
			  \end{rcases*}\) Distributivgesetze.
	\end{enumerate}
\end{bem}

\end{document}
\documentclass[../ana1.tex]{subfiles}
\begin{document}
\setcounter{section}{2}

\section{Die reellen Zahlen}
\subsection{Körperaxiome (engl. field)}
$\mathbb{K:}$ Menge mit zwei Operationen \glqq $+$\grqq und \glqq $\cdot$\grqq.\\
$\forall a,b \in \mathbb{K}$ ist $a+b\in \mathbb{K} \wedge a\cdot b \in \mathbb{K}$ erklärt sollen kompatibel sein.
\begin{defi}[Körperaxiome]
	In einem Körper gelten diese Axiome:
	\begin{enumerate}
		\item Kommutativität: $\forall a,b\in \mathbb{K}: a+b=b+a, a\cdot b=b\cdot a$
		\item Assoziativität: $\forall a,b,c\in \mathbb{K}: a+(b+c) = (a+b)+c, a\cdot (b\cdot c) = (a\cdot b)\cdot c$
		\item Existenz des neutralen Elements: \\
		      $\exists 0 \in \mathbb{K}: a + 0 = 0 + a = a \forall a\in \mathbb{K}\\
			      \exists 1 \in \mathbb{K}: a \cdot 1 = 1 \cdot a = a \forall a\in\mathbb{K}$
		\item Existenz eines inversen Elements:\\
		      $\forall a\in\mathbb{K}\exists -a\in\mathbb{K}:a+ (-a)=0\\
			      \forall a\in\mathbb{K}\setminus\{0\}\exists \frac{1}{a}\in\mathbb{K}:a\cdot\frac{1}{a}=1$\\
		      Es gilt: $0 \neq 1$.
		\item Distributivgesetz: $\forall a,b,c\in\mathbb{K}:a\cdot(b+c)=a\cdot b + a \cdot c$
	\end{enumerate}
\end{defi}
%23.10.2018
\begin{bsp}
	$\mathbb{Q} = \frac{m}{n}, n \in \mathbb{N}, m\in\mathbb{Z}$ ist ein Körper.\\
	$\mathbb{F}_2: $
	\begin{tabular}{c|cc}
		$+$ & $0$ & $1$ \\
		\hline
		$0$ & $0$ & $1$ \\
		$1$ & $1$ & $0$
	\end{tabular}
	\begin{tabular}{c|cc}
		$\cdot$ & $0$ & $1$ \\
		\hline
		$0$     & $0$ & $0$ \\
		$1$     & $0$ & $0$
	\end{tabular}
	ist ein Körper.
\end{bsp}
\begin{bem}
	\begin{enumerate}.%FORMATIERUNG SCHLECHT
		\item Somit ist ein Körper $\mathbb{K}$ mit \glqq $+$\grqq eine kommutative Gruppe und $\mathbb{K} \setminus \{0\}$ mit \glqq $\cdot$\grqq auch eine kommutative Gruppe.
		\item Die neutralen Elemente sind eindeutig bestimmt.\\
		      z.B.: angenommen, $0_1$ und $0_2$ sind neutrale Elemente mit \glqq $+$\grqq .\\
		      $\Rightarrow 0_1 \overset{(3)}{=} 0_1 + 0_2 \overset{(1)}{=} 0_2 + 0_1 \overset{(2)}{=} 0_2$\\
		      analog für Multiplikation
	\end{enumerate}
\end{bem}
\begin{defi}
	Zu $a\in \mathbb{K}$ ist $-a$ das Inverse bzgl. der Addition\\
	schreibe $a-b := a + (-b)$.\\
	Zu $a\in\mathbb{K}\setminus\{0\}$ sei $a^{/1}$ das Inverse bzgl. der Multiplikation.\\
	Ist $b\neq 0$, so schreiben wir $\frac{a}{b} := a\cdot b^{-1} =b^{-1}\cdot a$.\\
	schreibe $(ab) := a\cdot b$.
\end{defi}
\begin{lem}[Rechnen in einem Körper].%FORMATIERUNG SCHLECHT
	\begin{enumerate}
		\item Umformen von Gleichungen\\
		      $\forall a,b,c\in\mathbb{K}:$\\
		      aus $a+b=c$ folgt $a=c-b$\\
		      aus $a\cdot b=c$, $b\neq 0$ folgt $a=\frac{c}{b}$
		\item Allgemeine Rechenregeln\\
		      $\begin{aligned}
				      -(-a)           & = a                                                               \\
				      (a^{-1})^{-1}   & =a\text{, falls }a\neq 0                                          \\
				      -(a+b)          & = (-a) + (-b)                                                     \\
				      (a\cdot b)^{-1} & =b^{-1}\cdot a^{-1}=a^{-1}\cdot b^{-1}                            \\
				      a\cdot 0        & =0                                                                \\
				      a(-b)           & =-(ab), (-a)(-b)=ab                                               \\
				      a(b-c)          & = ab - ac                                                         \\
				      ab              & = 0 \Leftrightarrow a = 0 \vee b = 0 \text{ (Nullteilerfreiheit)} \\
			      \end{aligned}$
	\end{enumerate}
\end{lem}
\begin{bew}
	$0 = a + (-a) = (-a) + a\\
		\Rightarrow -(-a) = a\\
		(a+b) + ((-a)+(-b)) = (a+(-a))+(b+(-b))=0+0=0\\
		\Rightarrow -(a+b)=(-a)+(-b)$\\
	benutzen wir auch Eindeutigkeit des inversen Elements\\
	analog zeigt man $(a^{-1})^{-1} = a$ und $(ab)^{-1}= b^{-1}a^{-1}=a^{-1}b^{-1}$\\
	z.B.: $(ab)\cdot (b^{-1}a^{-1})=a(b\cdot b^{-1}) a^{-1} = (a\cdot 1)a^{-1} = a\cdot b^{-1}=1$\\
	Ferner $a\cdot 0 = a\cdot(0+0)=a\cdot 0 + a\cdot 0 = a\cdot 0 + 0\\
		\Rightarrow a\cdot 0 = a\cdot 0 - a\cdot 0 = 0\\
		\Rightarrow a\cdot b + a\cdot (-b) = a\cdot (b+(-b)) = a\cdot 0=0\\
		\overset{\text{Eind. d. Inv.}}{\Rightarrow} -ab=a(-b)$\\
	Somit auch $(-a)(-b) = -((-a)b) = -(b(-a)) = (-ba) = -(-ab) = ab$\\
	und $a(b-c) = a(b+(-c))=ab+a(-c)=ab+(-ac)=ab-ac$.\\
	ist $ab = 0$ und $a\neq 0 \Rightarrow 0=(ab)\frac{1}{a}=\frac{1}{a}\cdot (ab)=(\frac{1}{a}\cdot a)b = 1b = b$\\also ist $b=0$.
\end{bew}
\begin{satz}[Bruchrechnen]
	$a,b,c,d\in\mathbb{K}, c\neq 0, d\neq 0$.\\
	Dann gilt
	\begin{enumerate}
		\item $\frac{a}{c}+\frac{b}{d}=\frac{ad+bc}{cd}$
		\item $\frac{a}{c}\cdot\frac{b}{d}=\frac{ab}{cd}$
		\item $\frac{\nicefrac{a}{c}}{\nicefrac{b}{d}} = \frac{ad}{bc}$, falls auch $b\neq 0$ ist.
	\end{enumerate}
\end{satz}
\begin{bew}
	Übung
\end{bew}
\begin{bsp}
	rationale Zahlen sind ein Körper\\
	schreiben $(\mathbb{K},+,\cdot)$ für einen Körper
\end{bsp}
\subsection{Die Anordnungsaxiome}
\begin{defi} %EIGENTLICH DEF 3.2.0
	Sei $\mathbb{K}$ (genauer $(\mathbb{K},+,\cdot)$) ein Körper. Dann heißt $>$ eine Anordnung falls
	\begin{enumerate}
		\item Für jedes $a\in\mathbb{K}$ gilt genau eine der Aussagen $a>0,a=0,-a>0$\\
		      (wenn $a\in\mathbb{K}$, mit $a>0$ positiv)
		\item Aus $a>0$ und $b>0$ folgt\\
		      $a+b>0$ und $a\cdot b>0$
	\end{enumerate}
	Wir nennen $(\mathbb{K},+,\cdot,>)$ einen angeordneten Körper.
\end{defi}
\begin{bem}
	Statt $-a>0$ schreiben wir $a<0$\\
	Statt $a-b>0$ schreiben wir $a>b$\\
	Bild:
	\begin{center}
		\subgraphic{0.4}{img02.pdf}
	\end{center}
	Statt $a-b<0$ schreiben wir $a<b$.\\
	$a \geq b$, falls $a>b \vee a=b$\\
	$a \leq b$, falls $a<b \vee a=b$.
\end{bem}
\begin{satz}
	Sei $(\mathbb{K}, +,\cdot, >)$ ein angeordneter Körper. Dann gilt
	\begin{enumerate}
		\item für $a,b\in\mathbb{K}$ gilt genau eine der Relationen $a>b, a=b, a<b$ (Trichotromie)
		\item Aus $a>b, b>c$ folgt $a>c$ (Transitivität)
		\item Aus $a>b$ folgt:\\
		      $\begin{cases}
				      a+c>b+c, \forall c\in\mathbb{K} \\
				      ac>bc, \text{ falls } c>0       \\
				      ac<bc, \text{ falls } c<0
			      \end{cases}$.
		\item Aus $a>b$ und $c>d$ folgt:\\
		      $\begin{cases}
				      a+c>b+d \\
				      ac>bd, \text{ falls } b,d>0 %WARUM FALLS b,d>0???
			      \end{cases}$
		\item Für $a\neq 0$ ist $a^2 >0$.
		\item Aus $a>0$ folgt $\frac{1}{a}>0$.
		\item Aus $a>b>0$ folgt $0<\frac{1}{a}<\frac{1}{b}$.
		\item Aus $a>b, 0<\lambda<1$ folgt $b<\lambda b + (1-\lambda)a<a$.
	\end{enumerate}
\end{satz}
\begin{bem}
	Auf $\mathbb{F}_2$ kann es keine Anordnung geben!
\end{bem}
\begin{bew}
	\begin{enumerate}%schlecht formatiert
		\item Direkt aus (A.1) und Def. von $a>b$.
		\item $a-c = \underbrace{(a-b)}_{>0}+\underbrace{(b-c)}_{>0} \overset{\text{(A.2)}}{>} 0$.
		\item $(a+c)-(b+c)=a-b>0\\
			      ac-bc=\underbrace{(a-b)}_{>0}\cdot c \overset{\text{(A.2)}}{>}0$, falls $c>0$\\
		      Ist $c<0$, so ist $-c>0\\
			      \Rightarrow bc-ac=\underbrace{(a-b)}_{>0}\cdot\underbrace{(-c)}_{>0} \overset{\text{(A.2)}}{>} 0$\\
		      $ac-bd=ac-bc+bc-bd=\underbrace{(a-b)}_{>0} \cdot \underbrace{c}_{>0} + \underbrace{b}_{>0} \cdot \underbrace{(c-d)}_{>0} \overset{\text{(A.2)}}{>}0$.
		\item $(a+c)-(b+d) = (a-b)+(c-d)>0$ nach (A.2)\\
		      $ac-bd = ac-bc + bc-bd = (a-b)c + b(c-d)$\\
		      Ist $b=0 \Rightarrow a> b = 0 \Rightarrow ac > 0 = bd$\\
		      Ist $b<0 \Rightarrow (-b)d > 0 \Rightarrow -bd > 0 \Rightarrow bd < 0 \Rightarrow ac<-bd \Rightarrow \underbrace{ac}_{>0} + \underbrace{(-bd)}_{>0} \overset{\text{(A.2)}}{>} 0$.
		\item  Fallunterscheidung:\\
		      ist $a>0\Rightarrow a^2 = a\cdot a > 0$ (A.2)\\
		      ist $a<0\Rightarrow a^2 = (-a)\cdot(-a) > 0$ (A.2)
		\item sei $a>0:$\\
		      $\overset{\text{5.}}{\Rightarrow} \left(\frac{1}{a}\right) > 0 \Rightarrow \frac{1}{a} = \underbrace{\left(\frac{1}{a}\right)^2}_{>0} \cdot \underbrace{a}_{>0} > 0$.
		\item aus $a>b>0\\
			      \Rightarrow \frac{1}{b} - \frac{1}{a} = \frac{1}{b}(a-b)\frac{1}{a}>0$.
		\item $a>b, 0>\lambda>1 \Rightarrow \lambda > 0 \wedge 1-\lambda > 0\\
			      b=\lambda b + \underbrace{(1-\lambda)b}_{<(1-\lambda)a}\\
			      <\lambda b + (1-\lambda) a < \lambda a + (1-\lambda)a = a\\
			      \Rightarrow b<\lambda b + (1-\lambda)a =a$.\\
		      Insbesondere $\lambda = \nicefrac{1}{2}\Rightarrow b< \nicefrac{1}{2} b + \nicefrac{1}{2}a = \frac{a+b}{2} < a$.
	\end{enumerate}
\end{bew}
\begin{defi}[Betrag]
	Sei $(\mathbb{K}, +,\cdot,>)$ ein angeordneter Körper.\\
	Betrag von $a\in\mathbb{K}$ ist gegeben durch\\
	$\left| a\right| :=
		\begin{cases}
			a, \text{ falls } a\geq 0 \\
			-a, \text{ falls }a<0
		\end{cases}$\\
	auch noch $a,b\in \mathbb{K}$\\
	$\max(a,b) :=
		\begin{cases}
			a, \text{ falls } a\geq b \\
			b, \text{ falls } a < b
		\end{cases}\\
		\min(a,b) :=
		\begin{cases}
			a, \text{ falls } a\leq b \\
			b, \text{ falls } a > b
		\end{cases}$\\
\end{defi}
\begin{bem} . %SCHLECHT FORMATIERT
	\begin{enumerate}
		\item $a,b\in \mathbb{K}\\
			      \left| a-b\right| = $ Abstand von $a$ zu $b$.\\
		      $\left| a\right| = \left| a-0\right| = $ Abstand von $a$ zu $0$.
		\item $\left| a\right| = \max(a, -a)$.
	\end{enumerate}
\end{bem}
\begin{satz}
	$(\mathbb{K},+,\cdot,>)$ ang. Körper\\
	Dann gilt $\forall a,b\in\mathbb{K}:$
	\begin{enumerate}
		\item $\left|-a\right| = \left| a\right|$ und $a\leq\left|a\right|$
		\item $\left| a\right| \geq 0$ und $\left| a\right| = 0 \Leftrightarrow a = 0$
		\item $\left| ab\right| = \left| a\right| \left| b\right|$
		\item $\left| a+b\right| \leq \left| a\right| + \left|b\right|$ (Dreiecksungleichung)
		\item $\left|\left|a\right|-\left|b\right|\right| \leq \left| a-b\right|$ (umgekehrte Dreiecksungleichung)
	\end{enumerate}
\end{satz}
\begin{bew} . %SCHLECHT FORMATIERT
	\begin{enumerate}
		\item $\left| -a \right| =
			      \begin{cases}
				      -a, -a \geq 0 \\
				      -(-a), -a \leq 0
			      \end{cases}
			      = \begin{cases}
				      -a, a \leq 0 \\
				      a, a \geq 0
			      \end{cases}
			      = \left| a\right|\\
			      \left|a\right| - a=\begin{cases}
				      a-a,a\geq 0 \\
				      -a-a, a<0
			      \end{cases}
			      = \begin{cases}
				      0, a \geq 0 \\
				      -(a+a), a<0
			      \end{cases} \geq 0$.\\
		      alternativ: $a \leq \max(a, -a) = \left|a\right|$.
		\item
		\item Hier ändern sich die linke und rechte Seite \underline{nicht}, wenn man $a$ bzw. $b$ durch $-a$ bzw. $-b$ ersetzt.\\
		      Also, o.B.d.A. können wir annehmen, dass $a,b \geq 0$.\\
		      $\Rightarrow \left| ab\right| = ab = \left|a\right|\left|b\right|$.
		\item $\overset{\text{Satz 1 (5)}}{\Rightarrow} \left| a+b\right|^2 = (a+b)^2=a^2+2ab+b^2=\left|a\right|^2 + 2\underbrace{ab}{\leq \left| ab\right|} + \left|b\right|^2\\
			      \overset{\text{(2)}}{\leq} \left|a\right|^2 + 2\left|ab\right| + \left|b\right|^2\\
			      \overset{\text{(3)}}{\left|a\right|^2+2\left|a\right|\left|b\right| + \left|b\right|^2}$.\\
		      Also $(a+b)^2 \leq (\left|a\right| + \left| b\right|)^2\\
			      \overset{\text{H.A.}}{\Rightarrow}\left|a+b\right| \leq \left|a\right|+\left|b\right|$.\\
		      H.A. aus $\left|c\right|^2 \leq \left|d\right|^2$ folgt $|c| \leq |d|$ (Kontraposition).
		\item $|a| = \left| a-b+b\right| = \left| (a-b) + b\right| \overset{\text{(4)}}{\leq} \left|a-b\right| + |b|\\
			      |a|-|b|\leq \left|a-b\right| \forall a,b\in\mathbb{K}$.\\
		      Jetzt: Symmetrieargument. (Vertausch von $a$ und $b$)\\
		      $\Rightarrow |b| - |a| \leq \left|b-a\right| = \left|(-b-a)\right| = \left|a-b\right|$\\
		      also $|b|-|a| \leq \left|a-b\right|\\
			      |a|-|b| \leq \left|a-b\right|\\
			      \left| |a|-|b|\right| = \max(\left|a\right|-\left|b\right|, -(|a|-|b|))=\max(|a|-|b|, |b|-|a|) \leq |a-b|$.
	\end{enumerate}
\end{bew}
\begin{bsp}
	Sei $a,b\in\mathbb{K}$ ein angeordneter Körper. Aus $|b-a|\leq \nicefrac{b}{2}, 2 = 1+1$ folgt $a \geq \nicefrac{b}{2}$
	Bild: %BILD EINFUEGEN\\
	\begin{bew}
		$b-a \leq |b-a| \leq \nicefrac{b}{2} \Rightarrow a \geq b - \nicefrac{b}{2} = \nicefrac{b}{2}$.
	\end{bew}
\end{bsp}
\begin{kor}[\glqq geometrisch-arithmetische Ungleichung\grqq]
	Sei $(\mathbb{K},+,\cdot,>)$ ein ang. Körper, $a,b\in\mathbb{K}\\
		\Rightarrow ab \leq \left( \dfrac{a+b}{2}\right)^2$.\\
	Wenn Gleichheit gilt, so folgt $a=b$.
\end{kor}
\begin{bew}
	In Übung
\end{bew}
\textbf{Fakt:}
\begin{itemize}
	\item In jedem angeordneten Körper gilt $0<1$!
	\item Es gibt keine Anordnung, die $\mathbb{F}_2$ zu einem angeordneten Körper macht. (H.A.)
\end{itemize}
\subsection{Obere und untere Schranken, Supremum und Infimum}
Notation: $a$ ist nicht negativ, falls $a\geq 0$.
\\natürlich $a=b\Leftrightarrow a\leq b \wedge a \geq b$.\\
Im Folgenden ist $\mathbb{K}$ immer ein angeordneter Körper. $A,B \subset \mathbb{K}, A,B\neq \emptyset$ und $\gamma \in \mathbb{K}$, so bedeutet $A\leq \gamma: \forall a\in A: a \leq \gamma$ ($\gamma$ it obere Schranke für $A$).\\
$B\geq \beta: \forall b\in B: b\geq \beta$ ($\beta$ ist untere Schranke für $B$).\\
Analog sind $a<\gamma, A>\gamma, A<B,$ usw. definiert.\\
Hat $A$ eine obere Schranke, so heißt $A$ nach oben beschränkt. Hat $B$ eine untere Schranke, so ist $B$ nach unten beschränkt. $A$ ist beschränkt, falls es nach oben und unten beschränkt ist.\\
Ist $A\leq \alpha$ und $\alpha\in A$, so heißt $\alpha$ größtes (maximales) Element von $A$, schreibe $\alpha = \max A$ (Maximum).\\
Ist $B\geq \beta$ und $\beta\in B$, so heißt $B$ kleinstes (minimales) Element von $B$, schreibe $\beta = \min B$ (Minimum).\\
Man zeige, dass $\max$ und $\min$ eindeutig sind, sofern sie existieren.\\
$[0,1) := \{x\in\mathbb{K}|0\leq x\leq 1\}$ hat kein Maximum bzw. kein maximales Element.
				\begin{defi}
					Sei $A\subset \mathbb{K}, A\neq \emptyset$. Dann ist $\gamma\in\mathbb{K}$ die kleinste obere Schranke (oder Supremum), falls $A\leq \gamma$ und aus $A\leq n$ folgt $\gamma \leq n$.\\
					Schreibe $\gamma = \sup A = \sup(A)$.\\
					Analog: $\beta$ it die größte untere Schranke von $A$ (Infimum), falls $\beta \leq A$ und aus $\eta \leq A$ folgt $\eta \leq \beta$\\
					Schreibe $\beta = \inf A = \inf(A)$.
				\end{defi}
				\begin{bsp}
					$P := \{x\in\mathbb{K}|x>0\}\\
						\Rightarrow$
					\begin{enumerate}
						\item $P$ ist nicht nach oben beschränkt.
						\item $P$ hat kein Minimum, aber $\inf P = 0$.
					\end{enumerate}
				\end{bsp}
				\begin{bew} . %Schlecht formatiert
					\begin{enumerate}
						\item Ang. $\gamma$ ist obere Schranke für $P$. D.h. $\forall x\in P$ folgt $0<x\leq\gamma\Rightarrow\gamma >0\Rightarrow \gamma \in P \Rightarrow 0 < \gamma = \gamma + 0 < \gamma + 1 \in P\Rightarrow\gamma + 1\in P$ und $\gamma +1>\gamma \gamma$ ist nicht obere Schranke für $P$ (Widerspruch!) \Lightning
						\item $2:= 1+1 > 1>0$\\
						      Ang. $\min P := \eta$ existiert. $\Rightarrow \eta \in P, \eta > 0, \tilde{x} := \frac{\eta}{2} = \frac{0 + \eta}{2} < \eta$.\\
						      Es gilt $0 = \inf P$.\\
						      Sicherlich $0<P$, also ist $0$ eine untere Schranke für $P$.\\
						      $0$ ist die größte untere Schranke, denn nach obigem Argument ist jede Zahl $>0$ keine untere Schranke für $P$!
					\end{enumerate}
				\end{bew}
				\begin{lem}
					$A\subset\mathbb{K}, A\neq \emptyset$.
					\begin{enumerate}
						\item $\alpha := \sup A \Leftrightarrow \alpha \geq A \wedge \forall \varepsilon > 0 \exists a \in A: \alpha - \varepsilon < a$.
						\item $\beta := \inf B \Leftrightarrow \beta \leq B \wedge \forall \varepsilon > 0 \exists b \in B: b < \beta + \varepsilon$.
					\end{enumerate}
				\end{lem}
				\begin{bew} .%SCHLECHT FORMATIERT
					\begin{enumerate}
						\item \glqq $\Rightarrow$\grqq: Sei $\alpha = \sup A$. Also $\alpha$ ist die kleinste obere Schranke für $A$. D.h. $\alpha \geq A$ und $\forall \varepsilon > 0$ ist $\varepsilon>0<\alpha$, also ist $\alpha-\varepsilon$ keine obere Schranke für $A$. D.h. $\exists a\in A:\alpha - \varepsilon <a$.\\
						      \glqq $\Leftarrow$\grqq: Sei $\alpha \geq A \wedge \forall \varepsilon > 0 \exists a \in A: \alpha - \varepsilon < a$. Also ist $\alpha$ eine obere Schranke für $A$. Sei $\tilde{\alpha}<\alpha$.\\
						      Setze $\varepsilon:= \alpha -\tilde{\alpha} > 0 \Rightarrow \exists a \in A: \tilde{\alpha} = \alpha - \varepsilon < a \Rightarrow \tilde{\alpha}$ ist keine obere Schranke für $a$. $\Rightarrow\alpha$ ist die kleinste obere Schranke.
						\item $A:= -B = \{-b|b \in B\}$. Beachte: $\sup A = \sup(-B) = -\inf B$.
					\end{enumerate}
				\end{bew}
				%30.10.2018
				\subsection{Das Vollständigkeitsaxiom}
				\begin{defi}
					Ein angeordneter Körper $(\mathbb{K},+,\cdot,>)$ erfüllt das Vollständigkeitsaxiom, falls\\
					\begin{center}
						Jede nichtleere, nach oben beschränkte Teilmenge hat ein Supremum.
					\end{center}
					Solch einen Körper nennt man ordnungsvollständig. $\mathbb{R}$, der Körper der reellen Zahlen, ist \underline{der} ordnungsvollständige Körper. (Im Wesentlichen gibt es nur einen!)
				\end{defi}
			$\mathbb{Q}; A:= \{r\in\mathbb{Q}|r^2 < 2\}$\\
				Notation: $a,b\in\mathbb{R} \quad a<b$\\
			$[a,b] := \{x\in\mathbb{R}|a\leq x\leq b\}$ abgeschlossenes Intervall\\
			$(a,b) := \{x\in\mathbb{R} | a<x<b\}$ offenes Intervall\\
			$[a,b) := \{x\in\mathbb{R}|a\leq x<b\}$ nach rechts halboffenes Intervall\\
					$(a,b] := \{x\in\mathbb{R}|a<x\leq b\}$ nach links halboffenes Intervall\\
				Intervalllänge: $b-a$\\
				unbeschränkte Intervalle:\\
			$(-\infty, a] := \{x\in\mathbb{R}|x\leq a\}$\\
$[a,\infty) := \{x\in\mathbb{R}|x\geq a\}$\\
$(-\infty, a) := \{x\in\mathbb{R}|x<a\}$\\
$(a, \infty) := \{x\in\mathbb{R}|x>a\}$.
\subsection{Die natürlichen Zahlen $\mathbb{N}$}
(als Teilmenge von $\mathbb{R}$)\\
$n$ natürliche Zahl, $n=\underbrace{1+1+\ldots + 1}_{n\text{-mal}}$ (zirkulär \Lightning)
\begin{defi}
	Eine Teilmenge $M\subset \mathbb{R}$ heißt \underline{induktiv}, falls
	\begin{enumerate}
		\item $1\in M$
		\item Aus $x\in M$ folgt $x+1 \in M$
	\end{enumerate}
\end{defi}
\begin{bsp}
	$[1,\infty)$ ist induktiv.\\
	$\mathbb{R}$ ist induktiv.\\
	$(1,\infty)$ ist nicht induktiv.\\
	$\{1\} \cup [1+1,\infty)$ ist induktiv.
\end{bsp}
\textbf{Beobachtung:} Ein beliebiger Schnitt induktiver Mengen ist wieder induktiv.\\
$J$: Indexmenge $A_0$ induktiv $\forall j\in J$\\
$\Rightarrow \forall i\in J: 1\in A_j \Rightarrow 1\in \underset{j\in J}{\bigcap} A_j$\\
Ist $x\in \underset{j\in J}{\bigcap} A_j\Rightarrow \forall j \in J: x\in A_j \Rightarrow x+1 \in A_j \Rightarrow x+1 \in \underset{j\in J}{\bigcap} A_j$.
\begin{defi}[natürliche Zahlen] .\\ %SCHLECHT FORMATIERT
	$\mathbb{N} := \{x\in \mathbb{R}: \text{ für jede induktive Teilmenge } M\in\mathbb{R} \text{ gilt } x\in M\} := \underset{M\subset \mathbb{R}\text{ ist induktiv}}{\bigcap} M$
\end{defi}
\begin{bem}
	$\mathbb{N}$ ist induktiv und $\mathbb{N}$ ist die kleinste induktive Teilmenge von $\mathbb{R}$.
\end{bem}
\begin{satz}[Archimedisches Prinzip für $\mathbb{R}$] .%SCHLECHT FORMATIERT
	\begin{enumerate}
		\item $\mathbb{N}$ ist (in $\mathbb{R}$) \underline{nicht} nach oben beschränkt!
		\item $\forall x\in\mathbb{R}$ mit $x>0 \exists n\in \mathbb{N}: \frac{1}{n} < x$.
	\end{enumerate}
\end{satz}
\begin{bew}
	\begin{enumerate}
		\item Angenommen, $\mathbb{N}\subset{R}$ ist nach oben beschränkt.\\
		      $\mathbb{N} \neq \emptyset$ (da $1\in\mathbb{N}$)\\
		      Vollständigkeitsaxiom $\Rightarrow \alpha := \sup \mathbb{N} \in \mathbb{R}$.\\
		      Setze $\varepsilon = 1$ in Lemma 3.3.2\\ %NOCH VERKNÜPFUNG EINFÜGEN
		      $\alpha -1$ ist nicht obere Schranke für $\mathbb{N}$.\\
		      $\exists n\in \mathbb{N}: n>\alpha-1\\
			      \Rightarrow n+1 > \alpha \in \mathbb{N}$ \Lightning  zu $\alpha$ ist obere Schranke von $\mathbb{N}$.
		\item Sei $x>0 \overset{\text{Satz 3.2.1 (6)}}{\Rightarrow} \frac{1}{x} > 0\Rightarrow \exists n\in \mathbb{N}: n > \frac{1}{x}\underset{\text{Satz 3.2.1 (7)}}{\Rightarrow} x = \dfrac{1}{\nicefrac{1}{x}} > \frac{1}{n}$. %VERLINKUNGEN ZU SÄTZEN
	\end{enumerate}
\end{bew}
\begin{satz}[Induktionsprinzip]
	Sei $M\subset \mathbb{N}$ mit
	\begin{enumerate}
		\item $1\in M$
		\item Ist $x\in M \Rightarrow x+1 \in M$
	\end{enumerate}
	Dann ist $M= \mathbb{N}$.
\end{satz}
\begin{bew}
	$\Rightarrow M$ ist induktiv. $\mathbb{N}$ kleinste induktive Teilmenge von $\mathbb{R}$\\
	$\Rightarrow \mathbb{N} \subset M$\\
	$M\subset \mathbb{N} \wedge \mathbb{N} \subset M \Leftrightarrow M = \mathbb{N}$.
\end{bew}
\begin{kor}[Vollständige Induktion]
	Für $n\in\mathbb{N}$ seien $A(n)$ Aussagen.
	Es gelte:
	\begin{enumerate}
		\item $A(1)$ ist wahr.
		\item aus $A(n)$ ist wahr folgt $A(n+1)$ ist wahr.
	\end{enumerate}
\end{kor}
\begin{bew}
	Definiere $M := \{n\in \mathbb{N}| A(n) \text{ ist wahr}\} \subset \mathbb{N}$.
	\begin{enumerate}
		\item $\Rightarrow 1\in M$, da $A(1)$ wahr ist
		\item $\Rightarrow$ sei $n\in M$, d.h. $A(n)$ ist wahr $\Rightarrow A(n+1)$ ist wahr, d.h. $n+1\in M$.
	\end{enumerate}
	$\overset{\text{Ind.prinzip Satz 4}}{\Rightarrow} M = \mathbb{N}$, also sind alle $A(n)$ wahr! %LINK ZU SATZ 4!
\end{bew}
Notation: Induktive Definition von Summen und Produkten.\\
$a_1+a_2+\ldots + a_n$ vage \dots\\
\textbf{Summe:} \[\sum_{k=1}^{1} a_k := a_1, (n = 1), \sum_{k=1}^{n+1} a_k := \left( \sum_{k=1}^{n} a_k\right) + a_{n+1}, n\in\mathbb{N}\]
Allgemein: untere Grenze $k=m$, obere Grenze $k=n$, Laufindex kann verschoben werden.\\
z.B.: $k= j+1$
\[\sum_{k=m}^{n}a_k = \sum_{j=m-1}^{n-1}a_{j+1} = \ldots = \sum_{l = 0}^{n-m}a_{l+m}\]
Ist $m>n$, definieren $\sum_{k=m}^{n-m}a_k := 0$ (leere Summe)\\
\textbf{Produkt:} $$\prod_{k=1}^{1}a_k := a_1, \prod_{k=1}^{n+1}a_k := \left( \prod_{k=1}^{n}a_k\right) \cdot a_{n+1}, n\in\mathbb{N}$$
Ähnlich $\prod_{k=m}^{n}a_n$, setzen für $m>n \prod_{k=m}^^n a_k := 1$ (leeres Produkt)\\
z.B. $$a\in\mathbb{R}, a^n = \prod_{k=1}^n a \text{, d.h. } a^1=a, a^{n+1} = a^n \cdot a, n\in\mathbb{N} \text{ (induktive Definition)}$$
Rechenregeln gelten z.B. $$\sum_{k=1}^{n} (a_k+b_k) = \sum_{k=1}^{n} a_k + \sum_{k=1}^{n} b_k\\
	a_j, b_j \in\mathbb{R}, j=1,\ldots, n$$
$$c \in\mathbb{R}, \sum_{n}^{k=1} (c\cdot a_k) = c\cdot \sum_{k=1}^{n} a_k$$
\begin{satz}[Bernoullische Ungleichung]
	$$x\in\mathbb{R}, x\geq -1, n\in\mathbb{N}_0 = \mathbb{N}\cup \{0\}$$
	gilt $(1+x)^n \geq 1+ n + x (\forall m\in\mathbb{N},x\geq -1)$\\mit \glqq $>$\grqq, falls $n>1, x\neq 0$\\
	$( \forall n\in\mathbb{N}, x \geq -1(1+x)^n \geq 1 + nx)$
\end{satz}
\begin{bew}
	Vollständige Induktion:\\
	Induktionsanfang:
	$$n=0: (1+x)^0=1=1+0x \checkmark$$
	$$n=1: (1+x)^1=1+x=1+1x \checkmark$$
	Induktionsschritt:
	Induktionsvoraussetzung: es gelte für ein festes, aber beliebiges $n\in\mathbb{N}$:
	\begin{equation*}
		\begin{aligned}
			(1+x)^n     & \geq 1+nx                                                                                              \\%LINEBREAK KLAPT NICHT
			(1+x)^{n+1} & = \underbrace{(1+x)^{n}}_{\geq 1+nx} \cdot \underbrace{(1+x)}_{> 0} \geq (1+nx)(1+x) = 1+(n+1)x + nx^2 \\
			            & =\begin{cases}
				\geq 1+(n+1)x, x>-1 \\
				> 1+(n+1)x, x>-1, x\neq 0
			\end{cases}
		\end{aligned}
	\end{equation*}
\end{bew}
\begin{satz}[geometrische Summe]
	Sei $x\neq 1$, dann ist $$\sum_{k=0}^{n}x^k = \frac{1-x^{n+1}}{1-x}$$
\end{satz}
\begin{bew}
	Vollständige Induktion:\\
	IA:
	\begin{equation*}
		\begin{aligned}
			n=0: & \sum_{k=0}^{0} x^k = x^0 = 1 = \frac{1-0}{1+0}\checkmark                       \\
			n=1: & \sum_{n=0}^{1} x^k = 1+x = \frac{1-x}{1-x} (1+x) = \frac{1-x^2}{1-x}\checkmark
		\end{aligned}
	\end{equation*}
	IS:\\
	IV: Es gelte für ein festes, aber beliebiges $n\in\mathbb{N}$: $$\sum_{k=0}^{n} x^k = \frac{1-x^{n+1}}{1-x}$$
	\begin{equation}
		\begin{aligned}
			\Rightarrow \sum_{k=0}^{n} x^k + x^{n+1} \overset{\text{IV}}{=} \frac{1-x^{n+1}}{1-x} + x^{n+1} \\
			= \frac{1-x^{n+1} + (1-x)x^{n+1}}{1-x} = \frac{1-x^{n+2}}{1-x}.
		\end{aligned}
	\end{equation}
\end{bew}
\begin{bew} ohne vollständige Induktion:
	\[
		\begin{aligned}
			S_n                  & := \sum_{k=0}^{n} x^k                                                                 \\
			x\cdot S_n           & = \sum_{k=0}^{n} x^k = \sum_{k=0}^{n} x^{k+1} = \sum_{j=1}^{n+1} x^j,                 \\
			\Rightarrow (1-x)S_n & = S_n - x S_n = \sum_{k=0}^{n} x^k - \sum_{k=1}^{n+1} x^k = x^0 - x^{n+1} = 1-x^{n+1} \\
			\Rightarrow S_n      & = \frac{1-x^{n+1}}{1-x}
		\end{aligned}\]
\end{bew}
\begin{satz}[Eigenschaften von $\mathbb{N}$]
	Es gilt
	\begin{enumerate}
		\item $\forall m,n \in \mathbb{N}: n+m \in \mathbb{N}$ imd $n\cdot m \in \mathbb{N}$.
		\item $\forall n\in\mathbb{N}: n=1$ oder $(n>1$ und demnach $n-1 \in \mathbb{N})$.
		\item $\forall m,n\in\mathbb{N}:m\leq n:n-m\in\mathbb{N}_0$.
		\item $\forall n\in\mathbb{N}$ gibt es kein $m\in\mathbb{N}: n<m<n+1$.
	\end{enumerate}
\end{satz}
\begin{bew}.%Falsch formatiert
	\begin{enumerate}
		\item Gegeben $m\in\mathbb{N}: A :=\{n\in \mathbb{N}| n+m\in \mathbb{N}\}\subset \mathbb{N}$\\
		      \begin{enumerate}
			      \item $1\in A$, denn $m\in\mathbb{N}: 1+m = m+1\in\mathbb{N}$.
			      \item Angenommen, $n\in A \Rightarrow (n+1) + m = \underbrace{n+m}_{\in\mathbb{N}} + 1 \in\mathbb{N}\\
				            \Rightarrow n+1\in A$
		      \end{enumerate} somit ist $A$ induktiv, also $\mathbb{N}\subset A \Rightarrow A = \mathbb{N}$.
		\item Definiere $B:= \{n\in \mathbb{N}|n=1 \vee (n-1\in\mathbb{N}\wedge n-1\geq 1)\} \subset \mathbb{N}$\\
		      Dann ist $B$ induktiv, denn
		      \begin{enumerate}
			      \item $1\in B, 2=1+1\in B$
			      \item Sei $1\neq n\in B$, so folgt $1\leq n-1$ und somit $n=(\underbrace{n-1}_{\in\mathbb{N}}) +1\in\mathbb{N}$\\
			            $\Rightarrow n+1\in\mathbb{N}$ und $(n+1)-1=n\geq 1+1>1$.
			            Somit ist $n+1\in B$.
		      \end{enumerate}
		\item $C:= \{n\in \mathbb{N} | \forall m \in \mathbb{N}$ mit $m\leq n$ ist $n-m\in \mathbb{N}_0\}\Rightarrow$
		      \begin{enumerate}
			      \item $1\in C$, denn ist $m\in \mathbb{N}$ und $m=1$.\\
			            folgt nach b): $m=1$\\
			            $\Rightarrow n - m = 1 - 1 = 0\in \mathbb{N}_0$.
			      \item ang. $n\in C$ und $m\in\mathbb{N}$ mit $m\leq n+1$.\\
			            Fallunterscheidung:
			            \begin{itemize}
				            \item $n=1 \Rightarrow n+1-m=(n+1)-1=n\in\mathbb{N}.\checkmark\\
					                  \Rightarrow n+1\in C.$
				            \item $n > 1$ (und $m \leq n+1$)\\
				                  $\overset{\text{b)}}{\Rightarrow} m-1\in \mathbb{N}$ und $m-1 \leq (n+1)-1 = n$\\
				                  Da $n\in C, m-1\in\mathbb{N}, m-1\leq n \Rightarrow \underbrace{n-(m-1)}_{=(n+1)-m} \in \mathbb{N}_0\\
					                  \Rightarrow n+1\in C$.
			            \end{itemize}
		      \end{enumerate}
		\item H.A.
	\end{enumerate}
\end{bew}

\end{document}
\newpage
%02.11.2018
\documentclass[../ana1.tex]{subfiles}
\begin{document}
\setcounter{section}{3}

\section{Abbildungen und Funktionen}

\subsection{Funktionen als Abbildungen}

\begin{defi}
	Eine Funktion (oder Abbildung) von einer Menge \(A \) in eine Menge \(B \) ordnet jedem Element \(a \in A \)
	ein eindeutiges Element \(b \in B \) zu. Schreibe:
	\[f \colon A \longrightarrow B \ko \quad a \mapsto f(a) \]
	Die Abbildung \(f \colon A \longrightarrow B \) heißt
	\begin{enumerate}[(a)]
		\item injektiv \(\longeq \) Aus \(f(a) = f(a^\prime) \) mit \(a \ko a^\prime \in A \) folgt \(a = a^\prime \).
		\item surjektiv \(\longeq \forall b \in B \, \exists a \in A \colon \; f(a) = b \).
		\item bijektiv \(\longeq \) f ist injektiv und surjektiv. 
	\end{enumerate}
\end{defi}

\begin{bsp}
	\(f \colon \R \longrightarrow \R_{\geq 0} \coloneqq \set{x \in \R \; \vert \; x \geq 0} \ko \; x \mapsto x^{2} \) ist surjektiv und nicht injektiv.
\end{bsp}

\begin{bem}
	Sei \(f \colon A \longrightarrow B \) eine Funktion.
	\begin{enumerate}[(a)]
		\item \(f \injektiv \iff a \ko a^\prime \in A \ko \; a \neq a^\prime \Rightarrow f(a) \neq f(a^\prime) \)
		\item \(f \bijektiv \iff \forall \, b \in B \, \existse \,a \in A \colon \; f(a) = b \)
	\end{enumerate}
\end{bem}

\begin{defi*}
	Es sei \(f \colon A \longrightarrow B \) eine Funktion. \\
	Ist \(f \) bijektiv so definiert
	\[\inverse{f} \colon B \longrightarrow A \ko \quad b \mapsto a \text{ mit } f(a) = b \]
	die Umkehrfunktion von \(f \).
	Ist \(f \) nicht bijektiv so definiert
	\[\inverse{f} \colon \PO(B) = \PO(A) \ko \quad B \supseteq M \mapsto \set{a \in A \; \vert \; f(a) \in M} \]
	die Urbildfunktion von \(f \) eine verallgemeinerte Umkehrfunktion.
	Ist nun \(g \colon B \longrightarrow C \) eine weiter Funktion so heißt
	\[g \circ f \colon A \longrightarrow C \ko \quad (g \circ f)(a) \coloneqq g \circ f(a) \coloneqq g(f(a)) \]
	die Verkettung von \(f \) und \(g \). Diese verkettete Funktion lässt sich durch folgendes Diagramm darstellen:
	\[A \overset{f}{\longrightarrow} B \overset{g}{\longrightarrow} C. \]
\end{defi*}

\begin{defi*}[Identitätsabbildung]
	Sei \(A \) eine Menge, dann ist die Identität auf \(A \) definiert durch
	\[\id_{A} \colon A \longrightarrow A \ko \quad a \mapsto a. \]
\end{defi*}


\begin{bem}
	Sei \(f \colon A \longrightarrow B \) eine Funktion. Dann gilt
	\[f \bijektiv \implies \inverse{f} \circ f = \id_{A} \text{ und } f \circ \inverse{f} = \id_{B}. \]
\end{bem}


\subsection{Abbildungen als Graph}

\begin{defi}
	Seien \(A \ko B \) Mengen und \(a \in A \ko \, b \in B \). \\
	Dann heißt \((a \ko b) \) Tupel.
	In der Mengenlehre sind diese definiert durch 
	\[(a \ko b) \coloneqq \set{\set{a}\ko \set{a \ko b}}. \]
	Die Reihenfolge der Elemente spielt also eine wichtige Rolle und im Allgemeinen gilt \((a \ko b) \neq (b \ko a) \).
	Die Menge
	\[A \times B \coloneqq \set{(a \ko b) \; \vert \; a \in A \ko \, b \in B} \]
	heißt kartesisches Produkt von \(A \) und \(B \).
\end{defi}

\begin{bsp}\(\R \times \R \) \\
	\begin{figure}[h!]
		\centering
		\subgraphic{0.3}{img03.pdf}
		\caption{Kartesisches Koordinatensystem der Ebene}
	\end{figure}
\end{bsp}

\begin{defi*}
	Seien \(A \ko B \) Mengen. Die Abbildungen
	\[\Pi_1 \coloneqq \Pi_A \colon A \times B \rightarrow A \ko \quad (a \ko b) \mapsto a \]
	\[\Pi_2 \coloneqq \Pi_B \colon A \times B \rightarrow B \ko \quad (a \ko b) \mapsto b \]
	heißen Projektion auf die erste beziehungsweise zweite Koordinate.
	Es gilt also \(\Pi_A(a \ko b) = a \) und \(\Pi_B(a \ko b) = b\).
\end{defi*}

\begin{defi*}
	Es seien \(A_1 \ko \ldots \ko A_n \) Mengen für \(n \in \N \). Dann sei \(A_1 \times A_2 \) definiert wie oben.
	Ferner definiere induktiv
	\[A_1 \times \cdots \times A_j \coloneqq (A_1 \times \cdots \times A_{j-1}) \times A_j \ko \quad j \in \set{2 \ko \ldots \ko n}.\]
\end{defi*}

\begin{bem}
	Seien \(A \ko B \ko C\) Mengen. Dann gilt
	\[(A \times B) \times C = A \times (B \times C) \text{ mit } ((a \ko b) \ko c) = (a \ko (b \ko c)).\]
	Es existiert also eine Bijektion \(\Phi \colon (A \times B) \times C \longrightarrow A \times (B \times C)\).
\end{bem}

\begin{defi}[Graph einer Abbildung]\leavevmode \\
	Sei \(f \colon A \longrightarrow B \) Funktion. Die Menge
	\[\Gamma \coloneqq \Gamma_f \coloneqq \set{(a \ko b) \in A \times B \colon \; b = f(a)} \varsubsetneq A \times B \]
	heißt Graph von \(f\).
	\iftoggle{short}{}{
		\begin{figure}[h!]
			\centering
			\subfloat[Graph]{\subgraphic{0.3}{img05.pdf}}
			\qquad
			\subfloat[Kein Graph]{\subgraphic{0.3}{img06.pdf}}
		\end{figure}
	}
\end{defi}

\begin{notation}[Einschränkung von Funktionen]\leavevmode \\
	Sei \(f \colon A \longrightarrow B \) eine Funktion und \(X \subseteq A\) eine Teilmenge von \(A \).
	\[\restr{f}{X} \colon X \longrightarrow B \ko \quad x \mapsto f(x)\]
	heißt Einschränkung von \(f \) auf \(X \).
\end{notation}

\begin{satz}\label{satz:graph}
	\(\Gamma \subset A \times B \) ist genau dann Graph einer Abbildung \(f \colon A \longrightarrow B \),
	wenn die Projektion \(\restr{\Pi_A}{\Gamma} \colon \Gamma \longrightarrow A \) bijektiv ist. \\
\end{satz}
\begin{bew}
	\equirl{
		Sei \(\Gamma = \Gamma_f \) wobei \(f \colon A \rightarrow B \)eine Funktion ist.\\
		\(\overunderset{(a \ko b) \in \Gamma_f }{ \Leftrightarrow b = f(a)}{\implies} \forall \, a \in A \)
		existiert genau ein \(b\in B \) mit \(f(a) = b \). \\
		\(\overunderset{\phantom{(a \ko b) \in \Gamma_f}}{\phantom{\Leftrightarrow b = f(a)}}{\implies} \restr{\Pi_A}{\Gamma} \) ist bijektiv.
	}{
		Sei \(\restr{\Pi_A}{\Gamma} \rightarrow A \) bijektiv. Dann gilt für \( (a_j \ko b_j) \in \Gamma \ko j \in \set{1 \ko 2} \) \\
		mit \( \Pi_A(a_1 \ko b_1) = \Pi_A(a_2 \ko b_2) \) auch \((a_1 \ko b_1) = (a_2 \ko b_2) \).\\
		\(\implies a_1 = a_2 \ko \, b_1 = b_2 \implies \forall \, a \in A \, \existse \, b \in B \colon \; (a \ko b) \in \Gamma \),\\
		da \(b = \Pi_B(a \ko b) = \Pi_B(\inverse{\left(\restr{\Pi_A}{\Gamma}\right)}(a)) \).\\
		Definiere \(f \coloneqq \Pi_B \circ \inverse{\left(\restr{\Pi_A}{\Gamma}\right)} \colon A \longrightarrow B\). Nachrechnen zeigt \(\Gamma = \Gamma_f \).\qedhere
		\begin{figure}[H]
			\centering
			\subgraphic{0.25}{img07.pdf}
			\caption{Situation im Beweis von \autoref{satz:graph}}
		\end{figure}
	}
\end{bew}

\begin{bem}
	In \autoref{satz:graph} gilt \(f = \Pi_B \circ \inverse{\left(\restr{\Pi_A}{\Gamma}\right)} \). %ENTSPRICHT 4.2
\end{bem}

\begin{bsp}
	Ist \(f \colon A \longrightarrow B \) bijektiv, also \(f(a) = b \Leftrightarrow \inverse{f}(b) = a\), so gilt \\
	\[\inverse{\Gamma_f} = \set{(b \ko \inverse{f}(b)) \; \vert \; b \in B} = \set{(f(a), a) \; \vert \; a \in A} = S(\Gamma_f) \ko \]
	wobei \(S \colon A \times B \longleftrightarrow B \times A \ko \, (a \ko b) \mapsto (b \ko a)\).
	\begin{figure}[h!]
		\centering
		\subfloat{\subgraphic{0.3}{img08.pdf}}
		\qquad
		\subfloat{\subgraphic{0.3}{img09.pdf}}
		\caption{\(\protect\Gamma_{f^{^{\protect\shortminus 1}}} \) entspricht Spiegeln von \(\protect\Gamma_{f} \) an der Winkelhalbierenden.}
	\end{figure}
\end{bsp}


\subsection{Schubfachprinzip und endliche Mengen}

\begin{notation}
	Sei \(n \in \N \). \([n] \coloneqq \set{1 \ko \ldots \ko n} \) ist induktiv gegeben durch
	\[[1] \coloneqq \set{1} \]
	\[[n + 1] \coloneqq [n] \cup \set{n + 1} \ko \quad n \in \N.\]
	Es bezeichne weiterhin \(\#[n] \coloneqq n \) die Mächtigkeit oder Größe von \([n] \).
\end{notation}

\begin{satz*}[Schubfachprinzip]\label{satz:schubfach}\leavevmode \\
	Ist \(f \colon [m] \longrightarrow [n] \) eine Funktion mit \(m > n \), so existiert ein \(k \in [n]\) mit 
	\(\inverse{f}(\set{k}) \subseteq \set{m_1 \ko m_2} \) wobei \(m_1 \neq m_2 \) ist.
\end{satz*}
\begin{bew}
	Angenommen für alle \(k \in [n] \) gilt \(\inverse{f}(\set{k}) \subseteq \set{j_k} \) für ein \(k_j \in [m] \).
	Ferner gilt \(\#[n] = n \). Da durch \(f \) jedem Element in \([m] \) genau ein Element in \([n] \) zugeordnet wird
	sind die \(j_k \), falls existent, alle unterschiedlich.
	Somit folgt, dass \(\inverse{f}([n]) \subseteq \set{j_1 \ko \ldots \ko j_n} \) für \(j_k \) paarweise verschieden. \Lightning{ zu \(m > n\).}
\end{bew}

\begin{kor}\label{satz:schubfach:kor}\leavevmode \\
	Sind \(n \ko m \in\N \) und ist \(f \colon [m] \longrightarrow[n] \) injektiv \(\implies m \leq n \).
\end{kor}
\begin{bew}
	Angenommen \(m > n\), dann folgt aber, dass ein \(k \in [n] \) existiert mit \(\#\inverse{f}(\set{k}) \geq 2 \).
	Es gibt also \(m_1 \ko m_2 \in [m] \ko \; m_1 \neq m_2 \) \\
	mit \(f(m_1) = k = f(m_2) \).\Lightning{ zu \(f \) injektiv.}
\end{bew}
%%%%%%%%%%%%%%%%%%%%%%%%%%%%%%%%%%%%%%%%%%%%%%%%%%%%%%%%%%%%%%%%%%%%%%%%%%%%%%%%%%%%%%%%%%%%%%%%%%%%%%%%%%%%%%%%%%%%%%%%%%%%%%%%%%%%%%%%%
%%%Irgendwie ist dieser Beweis sehr umständlich und Lang. Ich habe mir die Freiheit genommen ihn durch eine kurze Version zu ersetzen.%%%
%%%%%%%%%%%%%%%%%%%%%%%%%%%%%%%%%%%%%%%%%%%%%%%%%%%%%%%%%%%%%%%%%%%%%%%%%%%%%%%%%%%%%%%%%%%%%%%%%%%%%%%%%%%%%%%%%%%%%%%%%%%%%%%%%%%%%%%%%
%%%\begin{bew}
%%%	Fassen obige Aussage als \(A(n) \) auf, die für alle \(m\in\N \) zu zeigen ist. \\
%%%	Induktionsanfang:
%%%	\[ n=1: f:[m] \rightarrow \{1\} \injektiv \Rightarrow m=1, \text{ da sonst } f(1) = 1 = f(2) \text{\Lightning{} zu Injektivität}. \]
%%%	Induktionsschritt: \\
%%%	Induktionsvoraussetzung: \(A(n) \) ist wahr für \(n\in\N \). \\
%%%	Zu zeigen: \(A(n+1) \) ist wahr. \\
%%%	Angenommen, \(f:[m]\rightarrow[n+1] = [n] \cup \{n+1\} \) sei injektiv. \\
%%%	Zu zeigen: \(m\leq n+1 \) \\
%%%	Fallunterscheidung:
%%%	\begin{enumerate}
%%%		\item Ang.\  \(m=1 \Rightarrow m=1\leq n+1\checkmark{} \)
%%%		\item Ang.\  \(m>1, m\in\N \overset{\text{Satz 3.5.8}}{\Rightarrow} m-1\in\N \) \\
%%%		      \((*) \) Beh.: \(\exists \) inj.\  \(\tilde{f}: \{1,\ldots,m-1\}\rightarrow \{1,\ldots,n\} \). \\
%%%		      \[ \overset{(*) + \text{IV}}{\Rightarrow} m-1\leq n, \text{ d.h. } m\leq n+1 \Rightarrow A(n+1) \text{ ist wahr}. \]
%%%	\end{enumerate}
%%%	Beweis von \((*) \): \\
%%%	Angenommen, \(\exists f: [m]\rightarrow[n+1] \) inj. \\
%%%	Dann \(\exists \tilde{f}: [m+1]\rightarrow [m+1]\rightarrow[n] \) inj. \\
%%%	Fallunterscheidung:
%%%	\begin{itemize}
%%%		\item Ang. \(f(k)\in [n] \forall 1\leq k \leq m-1 \). Dann setze \(\tilde{f}(k) \coloneqq f\vert_{[m-1]} \\
%%%			  \tilde{f}(k) \coloneqq f(k), 1\leq k\leq m-1 \) \\
%%%		      (Nachrechnen \(\tilde{f} \) ist injektiv.)
%%%		\item \(\exists j\in\N, 1\leq j\leq m-1 \) mit \(f(j) = n+1 \). \\
%%%		      Dann definiere \(\tilde{f}: [m-1]\rightarrow [n] \)
%%%		      \[\tilde{f}(k) \coloneqq
%%%			      	\begin{cases}
%%%				    	f(k), 1 \leq k \leq m-1, k\neq j \\
%%%				      	f(m), k=j
%%%			      	\end{cases} \]
%%%		      Man prüfe nach \(\tilde{f}: [m-1]\rightarrow [n] \) injektiv!
%%%	\end{itemize}
%%%\end{bew}
%%%%%%%%%%%%%%%%%%%%%%%%%%%%%%%%%%%%%%%%%%%%%%%%%%%%%%%%%%%%%%%%%%%%%%%%%%%%%%%%%%%%%%%%%%%%%%%%%%%%%%%%%%%%%%%%%%%%%%%%%%%%%%%%%%%%%%%%%

\begin{kor}\label{satz:schubfach:kor_2}\leavevmode \\
	Sind \(n \ko m \in\N \) und ist \(f \colon [m] \longrightarrow[n] \) bijektiv, so gilt \(m = n \).
\end{kor}
\begin{bew}
	Nach Voraussetzung sind sowohl \(f \) als auch \(\inverse{f} \) injektiv. \\
	\(\implies m \leq n \wedge n \leq m \implies m = n \).
\end{bew}

\begin{defi}
	Seien \(M \ko A \ko B \) Mengen.
	\begin{enumerate}[(a)]
		\item \(M \) heißt endlich \(\longeq M = \emptyset \) oder \(\exists \, n \in \N \, \exists \, f \colon [n] \longrightarrow M \bijektiv \). \\
			  In diesem Fall heißt \(\#M \coloneqq n \) die Mächtigkeit von M. \\
			  Für die leere Menge setzen wir \(\#\emptyset \coloneqq 0 \). \\
		\item \(M \) heißt unendlich \(\longeq M \) ist nicht endlich.
		\item \(A \ko B \) heißen gleichmächtig \(\longeq \exists \, f \colon A \longrightarrow B \, \bijektiv \). \\
			  Schreibe auch \(A \sim B\).
		\item \(M \) heißt abzählbar \(\longeq \exists \, f \colon \N \longrightarrow M \, \bijektiv \).
		\item \(M \) heißt abzählbar unendlich \(\longeq M \) ist unendlich und abzählbar.
	\end{enumerate}
\end{defi}

\begin{bem}
	Ist \(M \) endlich, so ist \(\#M \) wohldefiniert.
\end{bem}
\begin{bew}
	Seien \(f \colon [n] \rightarrow M \ko g \colon [m] \rightarrow M \) bijektiv.
	\[\begin{tikzcd}
		& M &  \\
	   {[n]} \arrow[ru, "f"] &  & {[m]} \arrow[lu, "g"'] \arrow[ll, "h"']
	\end{tikzcd}\]
	\( h \coloneqq \inverse{f} \circ g \colon [m] \longrightarrow [n] \) ist auch bijektiv. \(\overset{\autoref{satz:schubfach:kor_2}}{\implies} m=n \).
\end{bew}

\begin{bem}[Satz von Cantor und Berenstein]\label{satz:cantor_Berenstein}\leavevmode \\
	Seien \(A \ko B \) Mengen und \(f \colon A \longrightarrow B \) sowie \(g \colon B \longrightarrow A \) injektiv.
	Dann existiert eine Bijektion \(h \colon A \longrightarrow B \).
\end{bem}
\begin{bew}
	Siehe Kolmogorov-Fomin: Introductory Real Analysis. \\
	Könnten definieren \(A\leq B \), falls es eine inj. Funktion \(f: A\rightarrow B \) gibt. \\
	\(A\leq B \wedge B\leq A \Leftrightarrow A\sim B \).
\end{bew}

\begin{bem}
	\(A\leq B \) heißt Kardinalität von \(A \) ist kleiner gleich der Kardinalität von \(B \).
	\begin{enumerate}
		\item Ist \(B\subset A \) und \(A \) endlich, so ist \(B \) endlich und \( \#B \leq \# A \).
		\item \(A,B \) endlich und disjunkt, \(A\cap B = \emptyset \Rightarrow \#(A\cup B) = \#A + \#B \).
	\end{enumerate}
\end{bem}

\begin{satz}
	Zwei Aussagen:
	\begin{enumerate}
		\item Jede Teilmenge einer abzählbaren Menge ist abzählbar.
		\item Sind für \(j\in\N \quad A_j \) abzählbare Mengen. Dann ist \(\bigcup_{j\in\N} \) abzählbar.
	\end{enumerate}
\end{satz}
\begin{bew}
	Beweis unterteilen:
	\begin{enumerate}
		\item Sei \(A \) abzählbar. Ist \(A \) endlich, so ist auch jedes \(B\subset A \) endlich, und somit abzählbar. \\
		      Sei \(A \) abzählbar unendlich. Dann existiert eine Bijektion \(f: \N \rightarrow A \) und setzen wir \(a_n \coloneqq f(n) \), so ist
		      \[ A = \bigcup_{n\in\N} \{a_n\} = \{a_1, a_2,\ldots \}. \]
			  Ist \(B\subset A \), so existieren \(n_j \in\N, 1\leq n_1<n_2<\ldots \) mit 
			  \[ B = \{a_{n_1}, a_{n_2}, \ldots \}. \]
		      Gibt es nur endlich viele \(n_j \), so ist \(B \) endlich, andernfalls ist \(h: \N \rightarrow B, g \mapsto h(j) \coloneqq a_{n_j} \)
		      eine Bijektion.
		\item o.\ B.\ d.\ A.\ sind alle \(A_j \) paarweise verschieden, \(A_l \cap A_m \neq \emptyset \) für \(l\neq m \).
		      Wenn nicht, betrachte
		      \[ B_1 \coloneqq A_1, \quad B_2 \coloneqq A_2 \setminus A_1, \]
		      \[ B_3 \coloneqq A_3 \setminus \{A_1 \cup A_2\}, \ldots, \quad B_{n+1} \coloneqq A_{n+1} \setminus \{A_1\cup\ldots\cup A_n\} \]
		      Dann sind \(B_n \) paarweise verschieden und \[\bigcup_{l\in\N} B_l = \bigcup_{l\in\N} A_l. \]
		      Schreiben \(A_l \) als Liste \(A_l = \{a_{1l}, a_{2l}, \ldots \} \) \\
		      Bild: \\
		      \begin{tikzpicture}
			      \matrix(m)[matrix of math nodes,column sep=1cm,row sep=1cm]{
				      s_{11}  & s_{12}  & s_{13}  & s_{14}  & \cdots \\
				      s_{21}  & s_{22} & s_{23} & s_{24} & \cdots \\
				      s_{31} & s_{32} & s_{33} & s_{34} & \cdots \\
				      s_{41} & s_{42} & s_{43} & \cdots                                  \\
			      };
			      \draw[->]
			      (m-1-1)edge(m-1-2)
			      (m-1-2)edge(m-2-1)
			      (m-2-1)edge(m-3-1)
			      (m-3-1)edge(m-2-2)
			      (m-2-2)edge(m-1-3)
			      (m-1-3)edge(m-1-4)
			      (m-1-4)edge(m-2-3)
			      (m-2-3)edge(m-3-2)
			      (m-3-2)edge(m-4-1);
		      \end{tikzpicture}
		      \\Jetzt können wir das obige rechteckige Schema diagonal abzählen! Dies liefert uns eine Bijektion von \(\N \) nach \(\bigcup_l\in\N A_l \).
	\end{enumerate}
\end{bew}

\begin{bem}
	Als Übung: Man gebe explizit eine Bijektion \(f : \N \rightarrow \N \times \N \) an!
\end{bem}

\textbf{Permutationen:}
\begin{defi}
	Eine bijektive Abbildung \(\sigma : \{1,\ldots,n\} \rightarrow \{1,\ldots,n\} \) heißt Permutation.
	\[n! = \prod_{k=1}^{n}k \]
\end{defi}

\begin{satz}
	\[n\in\N, S_n = \{\sigma: \{1,\ldots,n\}\rightarrow \{1,\ldots,n\} | \sigma \text{ ist bijektiv.}\} \Rightarrow \# S_n = n! \]
\end{satz}
\begin{bew} per Induktion \\
	\(n = 1 \) ist klar.
	Beobachtung: Permutation \(\sigma \in S_n \) identifizieren mit \(n \)-Tupel \( (\sigma(1), \sigma(2), \ldots, \sigma(n)) \) \\
	Induktionsannahme: \( \# S_n = n! \) für ein \(n\in\N \) \\
	Die Menge \(S_{n+1} \) ist die disjunkte Vereinigung der Teilmengen
	\[S_{n+1, k} \coloneqq \{\tau \in S_{n+1} | \tau_k = n+1 \} \quad k = 1,\ldots, n+1. \]
	z.B.: \[S_{4,2} = \{ (1,4,2,3), (2,4,3,1), (3,4,1,2), (1,4,3,2), (2,4,1,3), (3,4,1,2) \} \]
	Beobachtung: Jedem \(\tau = (\sigma_1,\ldots,\sigma_n)\in S_n \) können wir die Permutation \( (\sigma_1, \ldots, \sigma_{k-1}, \underbrace{n+1}_{k\text{-te Stelle}}, \sigma_k, \ldots, \sigma_n) \in S_{n,k} \) zuordnen und diese Abbildung ist bijektiv (nachprüfen).
	\[\Rightarrow \# S_{n+1,k} = \# S_n \]
	\[ S_{n+1} = \# (\bigcup_{k=1}^{n+1} S_{n+1,k} ) = \sum_{k=1}^{n+1} \underbrace{\# S_{n+1,k}}_{=\#S_n = n!} = \sum_{k=1}^{n+1} n! = (n+1)n! = (n+1)! \]
\end{bew}

\begin{defi}[Binomialkoeffizient]
	Für \(\alpha \in\R \) und \( k\in\N \) setzen wir
	\[ \binom{\alpha}{k} \coloneqq \frac{\alpha(\alpha-1)\cdot\cdots\cdot(\alpha-k+1)}{k!}, \text{ sowie } \binom{\alpha}{0} \coloneqq 1. \]
\end{defi}

\begin{lem}[Rekursionsformel für Binomialkoeffizienten]
	Für \( \alpha\in\R, k\in\N \) gilt \[ \binom{\alpha + 1}{k} = \binom{\alpha}{k} + \binom{\alpha}{k-1}. \]
\end{lem}
\begin{bew}
	Für \(k=1 \) ist dies einfach zu sehen. Für \(k\geq 2 \) gilt
	\begin{align*}
		 & \binom{\alpha}{k} + \binom{\alpha}{k-1}                                                                                                                                 \\
		 & = \frac{\alpha (\alpha-1)\cdot\cdots\cdot(\alpha-k+1)}{1\cdot2\cdot\cdots\cdot k} + \frac{\alpha (\alpha-1)\cdot\cdots\cdot(\alpha-k+2)}{1\cdot2\cdot\cdots\cdot (k-1)} \\
		 & = \frac{\alpha (\alpha-1)\cdot\cdots\cdot(\alpha-k+2)(\alpha-k+1+k)}{1\cdot2\cdot\cdots\cdot k}                                                                         \\
		 & = \frac{(\alpha + 1)\alpha (\alpha-1)\cdot\cdots\cdot((\alpha-1)-k+1)}{1\cdot2\cdot\cdots\cdot k} = \binom{\alpha + 1}{k}
	\end{align*}
\end{bew}

\begin{bem}
	\begin{enumerate}
		Pascal Dreieck: \\
		\item Ist \(\alpha = n\in\N_0 \), so können wir \(\binom{n}{k} \) ausrechnen mit dem Dreiecksschema von Blaise Pascal (1623\textendash1662). \\
		      \begin{tabular}{>{\(n=}l<{ \)\hspace{12pt}}*{13}{c}}
			      0 &   &   &   &   &    &    & 1  &    &    &   &   &   &   \\
			      1 &   &   &   &   &    & 1  &    & 1  &    &   &   &   &   \\
			      2 &   &   &   &   & 1  &    & 2  &    & 1  &   &   &   &   \\
			      3 &   &   &   & 1 &    & 3  &    & 3  &    & 1 &   &   &   \\
			      4 &   &   & 1 &   & 4  &    & 6  &    & 4  &   & 1 &   &   \\
			      5 &   & 1 &   & 5 &    & 10 &    & 10 &    & 5 &   & 1 &   \\
			      6 & 1 &   & 6 &   & 15 &    & 20 &    & 15 &   & 6 &   & 1
		      \end{tabular}
		\item Ist \( \alpha=n\in\N_0 \), so folgt durch Erweitern mit \((n-k)! \) für \(n \in\N_0, k\in \{0,1,\ldots,n\} \) \[ \binom{n}{k} = \frac{n!}{k!(n-k)!} = \binom{n}{n-k}. \]
	\end{enumerate}
\end{bem}

\begin{satz}
	Zahl der Kombinationen: \\
	Sei \( n\in\N_0, k\in \{1,\ldots,n\} \). Dann ist die Anzahl der \(k \)-elementigen Teilmengen von \( \{1,\ldots,n\} \) gleich \( \binom{n}{k} \).
\end{satz}
\begin{bew}
	Die Behauptung gilt für \(k=0 \) und beliebiges \(n\in\N \), da die leere Menge die einzige Teilmenge von \( \{1,\ldots,n\} \) mit 0 Elementen ist und nach Definition ist \( \binom{n}{0} = 1 \). \\
	Insbesondere gilt die Behauptung dann für \(n=0 \). \\
	Induktiv über \(n \), wobei Behauptung für alle \(k\in \{0,1,\ldots,n\} \) zu zeigen ist. \\
	Induktionsschluss: Bestimme die Anzahl der \(k \)-elementigen Teilmengen von \( \{ 1,\ldots,n+1 \} \) (wobei wir \(k\geq 1 \) annehmen können). \\
	Sei \(A\subset \{1,\ldots,n+1\} \) mit \( \#A = k\geq 1 \). \\
	Diese fallen in 2 Klassen:
	Klasse 1: \(n+1 \notin A \). \\
	Klasse 2: \(n+1 \in A \). \\
	Die Mengen der Klasse 1 bestehen genau aus den \(k \)-elementigen Teilmengen von \( \{1,\ldots,n\} \). \\
	Die Mengen der 2. Klasse erhält man aus den \((k-1) \)-elementigen Teilmengen von \( \{1,\ldots,n\} \) durch Vereinigung mit \( \{n+1\} \). \\
	Also ist nach Induktionsannahme
	\begin{align*}
		 &  \# \{k\text{-elementige Teilmengen von } \{1,\ldots,n+1\} \}                                     \\
		 &= \# \{k\text{-elementige Teilmengen von } \{1,\ldots,n\} \}                                      \\
		 &+ \# \{(k-1)\text{-elementige Teilmengen von } \{1,\ldots,n\} \} \\
		 &  \overset{\text{IV}}{=} \binom{n}{k} + \binom{n}{k-1} \overset{\text{Lem. 8}}{=} \binom{n+1}{k}.
	\end{align*}
\end{bew}

\begin{satz}[Binomische Formel]
	\[ a,b\in\R,n\in\N:{(a+b)}^n = \sum_{k=0}^{n} \binom{n}{k} a^k b^{n-k}. \]
\end{satz}

\begin{bem}
	\begin{align}
		{(a+b)}^1 & = a+b                     \\
		{(a+b)}^2 & = a^2 + 2ab + b^2         \\
		{(a+b)}^3 & = a^3 +3a^2b + 3ab^2 +b^3
	\end{align}
\end{bem}
\begin{bew} Induktion
	\[ n=1: {(a+b)}^1 = a+b = \sum_{k=0}^{1}\binom{1}{k} a^k b^{1-k} \checkmark{} \]
	Induktionsvoraussetzung: für ein beliebiges, aber festes \(n \in\N \) gilt:
	\[ {(a+b)}^n = \sum_{k=0}^{n} \binom{n}{k} a^k b^{n-k}. \]
	Induktionsschritt:
	\begin{align*}
		 &{(a+b)}^{n+1} \\
		=& (a+b) {(a+b)}^n = (a+b) \sum_{k=0}^{n} \binom{n}{k} a^k b^{n-k} \\
		=& \sum_{k=1}^{n+1} \binom{n}{k-1} a^k b^{n-(k-1)} + \sum_{k=0}^{n} \binom{n}{k} a^k b^{n+1-k} \\
		=& \binom{n}{n} a^{n+1} + \sum_{k=1}^{n} \binom{n}{k-1} a^k b^{n+1-k} + \sum_{k=1}^{n} \binom{n}{k} a^k b^{n+1-k} + \binom{n}{0} b^{n+1} \\
		=& \binom{n+1}{n+1} a^{n+1} + \sum_{k=1}^{n} \underbrace{\left(\binom{n}{k-1} + \binom{n}{k}\right)}_{=\binom{n+1}{k}} a^k b^{n+1-k} + \binom{n+1}{0} b^{n+1} \\
		=& \sum_{k=0}^{n+1} \binom{n+1}{k} a^k b^{n+1-k} \\
		=& \sum_{k=0}^{n+1} \binom{n+1}{k} a^k b^{n+1-k}.
	\end{align*}
\end{bew}

\textbf{Notation:} Geg. Menge \(A \), sei \[ \{0,1\}^A \coloneqq \{ \text{Funktion }f:A\rightarrow \{0,1\} \} \] \(= \) Menge aller \(\{0,1\} \)-wertigen Funktionen mit Definitionsbereich \(A \). \\
Allg.: \(A,B \) Mengen, \(B^A \coloneqq \{ \text{Funktion } f: a\rightarrow B\} \).

\begin{satz}
	Sei \(A\neq\emptyset \) eine endliche Menge. Dann ist 
	\[ \#( {\{0,1\}}^A) = 2^{\#A}. \]
\end{satz}
\begin{bew}
	Sei \(n \coloneqq \# A \in\N \). \\
	\(\Rightarrow \) Bijektion \(h:\{1,\ldots,n\} \rightarrow A \). \\
	\( \Rightarrow \) können annehmen \(A = \{1,2,\ldots,n\} \) \\
	z.z. \( \#( {\{0,1\}}^{[n]} ) = 2^n \). \\
	Induktion: \(n=1 \exists \) Fkt. \(f_1,f_2 : \{1\} \rightarrow \{0,1\} \\
	f_1(1) = 0 \quad f_2(1) = 1 \) \\
	Formel stimmt für \(n=1 \).
	Ang. Formel stimmt für \(n\geq 1 \). Fkt. \(f: \{1,\ldots, n+1\} \rightarrow \{0,1\} \) \\
	2 Klassen:
	\begin{enumerate}
		\item \(S_0 = \{f: \{1,\ldots,n\} \rightarrow \{0,1\}: f(n+1) = 0 \} \)
		\item \(S_1 = \{f: \{1,\ldots,n\} \rightarrow \{0,1\}: f(n+1) = 1 \} \)
	\end{enumerate}
	\[S_0 \cap S_1 = \emptyset, {\{0,1\}}^{[n+1]} = S_0 \cup S_1 \]
	\[\underbrace{\# S_0}_{=\#S_1} = \# ( {\{0,1\}}^{[n]} )  \overset{\text{IA}}{=} 2^n \]
	\[ \Rightarrow \# ( {\{0,1\}}^{n+1} ) = \#S_0+\#S_1 = 2^n + 2^n = 2^{n+1}. \]
\end{bew}

\begin{kor}
	Sei \(A \) endliche Menge. \\
	\(\mathcal{P}(A) = \) Potenzmenge = \( \{B | B \subseteq A\} \) \\
	\(\Rightarrow \#\mathcal{P}(A = 2^{\#A}) \).
\end{kor}
\begin{bew}
	Sei \(A\neq \emptyset \Rightarrow\mathcal{P}(\emptyset) = \{\emptyset \}, \quad 2^0 = 1 \checkmark{} \) \\
	Sei \( \#A \in\N \). Nach Satz 11 reicht eine Bijektion \( \varphi : \mathcal{P} \rightarrow \underbrace{{\{0,1\}}^A}_{= \{f: A \rightarrow \{0,1 \} \}} \). Dies wird in Lemma 13 für beliebige Mengen \(A \) gemacht.
\end{bew}

\begin{lem}
	Sei \(A\neq\emptyset{} \). Dann sind \( \mathcal{P}(A) \) und \( {\{0,1\}}^A \) gleichmächtig.
\end{lem}
\begin{bew}
	Brauchen \(\varphi : \mathcal{P}(A) \rightarrow {\{0,1\}}^A \). \\
	Sei \(B \subseteq A \), Indikatorfunktion \[ \mathds{1}_B(x) \coloneqq \begin{cases}
			1, & x\in B            \\
			0, & x\in A\setminus B
		\end{cases}
		, \mathds{1}_B : A \rightarrow \{0,1\}. \]
	Beachte: \(B = \{x\in A | \mathds{1}_B(x) = 1 \} \). \\
	Definiere \( \varphi : \mathcal{P}(A) \rightarrow {\{0,1\}}^A, B \mapsto \mathds{1}_B \). \\
	Beh.: \(\varphi \) ist bijektiv.
	\begin{enumerate}
		\item \( \varphi \) ist surjektiv. \\
		      Sei \(f: A\rightarrow \{0,1\} \) (nachrechnen)
			  \[ B_f \coloneqq f^{-1} (\{1\}) = \{a \in A \; \vert \; f(a) = 1 \} \Rightarrow \varphi(B_f) = \mathds{1}_{B_f} = f. \]
		\item \( \varphi \) ist injektiv. \\
		      Seien \( B_1,B_2\subseteq A, B_1 \neq B_2 \).
		      \begin{align*}
			      B_1\setminus B_2 \neq \emptyset \vee B_2\setminus B_1 \neq \emptyset                                    \\
			      \text{o.B.d.A. } B_1 \setminus B_2 \neq \emptyset \Rightarrow \exists x \in B_2 \setminus B_1 \subset A \\
			      \mathds{1}_{B_1}(x) = 0 \neq 1 = \mathds{1}_{B_2}(x)                                                    \\
			      \Rightarrow \varphi(B_1) = \mathds{1}_{B_1} \neq \mathds{1}_{B_2} = \varphi(B_2).
		      \end{align*}
	\end{enumerate}
\end{bew}

\begin{lem}
	Sei \(A \) Menge. Dann gibt es keine surj. Fkt. \(f: A\rightarrow \mathcal{P}(A) \).
\end{lem}

\begin{bem}
	Ist \(A \) endlich \( \Rightarrow \# \mathcal{P} (A) = 2^{\# A} > \#A \). \\
	\( \varphi: \mathcal{P}(A) \rightarrow {\{0,1\}}^A, \quad A\supset B \mapsto \mathds{1}_B \).
\end{bem}
\begin{bew}
	Sei \(f: A \rightarrow \mathcal{P}(A) \)
	\[ f(A) \subset A \quad \forall a \in A. \]
	Definiere \(R \coloneqq \{a\in A, a \neq f(a)\} \subset A \). \\
	Angenommen \(f: A \rightarrow \mathcal{P}(A) \) ist surjektiv.
	\[ \Rightarrow \forall b\in A \exists b : B=f(b) \Rightarrow \exists a\in A : R = f(a). \]
	\(\Rightarrow \) 2 Möglichkeiten:
	\begin{enumerate}
		\item \(a\in R \) \[ a\in f(a = R = \{x\in A| x\notin f(x)\}) \text{\Lightning{}} \]
		\item \(a \notin R = f(a) \Rightarrow a \notin f(a) \Rightarrow a\in R \) \Lightning{} \\
		      \( f \) kann nicht surjektiv sein!
	\end{enumerate}
\end{bew}

\end{document}
\documentclass[../ana1.tex]{subfiles}
\onlyinsubfile{\sectionNumbering} %Use numbering relative to sections and not subsection

\begin{document}
\setcounter{section}{4}

%09.11.2018
\section{Starke Induktion und das Wohlordnungsprinzip}

\begin{satz}[Starke Induktion]\label{satz:starke_ind}
	Seien \(A(n) \) Aussagen für alle \(n \in \N \). Gilt
	\begin{enumerate}[(i)]
		\item \(A(1) \) ist wahr.
		\item \(\forall \, n \in \N \colon \; A(1) \ko \ldots \ko A(n) \) wahr \(\implies A(n + 1) \) wahr.
	\end{enumerate}
	So ist ist \(A(n) \) wahr für alle \(n \in \N \).
\end{satz}
\begin{bew}
	Definiere die Aussage \(B(n) \coloneqq \set{ \text{alle } A(k) \text{ mit } k\leq n \text{ sind wahr} } \). Dann gilt
	\begin{enumerate}[(i)]
		\item \(B(1) \) ist wahr
		\item Ist \(B(n) \) wahr für ein \(n \in\N \), so ist \(B(n+1) \) wahr
	\end{enumerate}
	\(\implies B(n) \) ist wahr für alle \(n \in \N \).
\end{bew}

\begin{bem} \leavevmode \\
	\(\displaystyle\left[\forall \, n \in \N \, \forall \, k < n \colon \; A(k) \implies A(n)\right] \iff \, \forall \, n \in \N \colon \; A(n). \)
\end{bem}

\begin{satz}[Wohlordnungsprinzip für \(\N \)]\label{satz:wohlordprinz} \leavevmode \\
	Jede nichtleere Teilmenge der natürlichen Zahlen \(\N \) besitzt ein kleinstes Element.
\end{satz}
\begin{bew}
	Sei \(A(n)\coloneqq \set{\forall \, \emptyset \neq B \subseteq \N \text{ mit } n \in B \text{ hat ein kleinstes Element}} \).
	Zu zeigen ist also: \(A(n) \) ist wahr für alle \(n \in \N \). Es gilt
	\begin{enumerate}[(i)]
		\item \(A(1) \) ist wahr, denn ist \(B \subseteq \N \) mit \(1 \in B \), so folgt \(\forall \, k \in B \colon \; k \geq 1 \).
			  Also ist \(1 \) kleinstes Element in \(B \).
		\item Angenommen für \(n \in \N \) sind \(A(1) \ko \ldots \ko A(n) \) wahr. Sei \(B \subseteq \N \) mit \(n + 1 \in B \).
			  \begin{faelle}
				\item{Fall 1:} \(\set{1 \ko \ldots \ko n} \cap B = \emptyset \implies n + 1 \) ist kleinstes Element in \(B \).
				\item{Fall 2:} \(\set{1 \ko \ldots \ko n} \cap B \neq \emptyset \implies \, \exists \, k \in \set{1 \ko \ldots \ko n} \) kleinstes Element von \(B \) nach \(\IV \).
			  \end{faelle}
		      \(\implies B \) hat ein kleinstes Element.
	\end{enumerate}
	In beiden Fällen hat \(B \) ein kleinstes Element, also ist \(A(n+1) \) wahr.\\
	\(\overset{\text{\autoref{satz:starke_ind}}}{\implies} \forall \, n \in \N \colon \; A(n) \) wahr.
\end{bew}

\iftoggle{short}{}{\newpage}%Formatierung ausführliches Skript

\begin{notation}\leavevmode
	\begin{enumerate}[(a)]
		\item \(\Z \coloneqq \minus \N \cup \N_0 = \set{0 \ko \pm 1 \ko \pm 2  \ko \ldots} \) ist die Menge der \underline{ganze Zahlen}.
		\item \(\Q \coloneqq \set{\frac{m}{n} \; \vert \; n \in \N \ko m \in \Z } \) ist die Menge der \underline{rationale Zahlen}.
	\end{enumerate}
\end{notation}

\begin{kor}
	Jede nichtleere, nach unten beschränkte Teilmenge in \(\Z \) besitzt ein kleinstes Element.
\end{kor}
\begin{bew}
	Sei \(\emptyset \neq A \subsetneq \Z \ko A \geq \beta \) für ein \( \beta \in\Z \) \\
	Setze \(B \coloneqq A + \abs{\beta} + 1 = \set{\alpha + \abs{\beta} + 1 \; \vert \; \alpha \in A} \subseteq \N \ko B \neq \emptyset. \\
	\overset{\text{\autoref{satz:wohlordprinz}}}{\implies} \exists \, n_0 \coloneqq \min B\\
	\overset{\phantom{\text{\autoref{satz:wohlordprinz}}}}{\implies} z_0 \coloneqq n_0 - \abs{\beta} - 1 \in \Z \) ist kleinstes Element von \(A \).
\end{bew}

\begin{lem}\label{satz:gauss_klammer}
	Sei \( a \in\R \) mit \( a > 0 \). Dann existiert \(q \in \N_0 \) mit \(q \leq a < q + 1 \).
\end{lem}
\begin{bew}
	Ist \( 0 < a < 1 \), so nehme \( q = 0 \). Sei also \(a \geq 1 \) und setze \(B \coloneqq \set{ n \in \N \; \vert \; a < n } \). \\
	Da \( \N \) nicht nach oben beschränkt (\autoref{C3-satz:arch_prinz})ist, gilt \(B \neq \emptyset \). \\
	\(\overset{\text{\autoref{satz:wohlordprinz}}}{\implies} m \coloneqq \min B \) existiert. Da \(m \in B \), ist \(m > a \geq 1 \). \\
	Somit gilt nach \autoref{C3-satz:prop_N}, dass \(q\coloneqq m-1\in\N \).\\
	Da \( m \) die kleinste natürliche Zahl mit \(m < a \) ist, folgt \(m - 1 \leq a < m \).
\end{bew}

\begin{bem}
	Sei \(a \in \R \). Dann existiert \(q \in \Z \) mit \(q \leq a < q + 1 \).
\end{bem}
\begin{bew}
	Analog zu \autoref{satz:gauss_klammer}.
\end{bew}

\begin{satz}[\(\Q \) ist dicht in \(\R \)]
	Seien \(a \ko b \in \R \ko a < b \). Dann existiert \( r \in \Q \) mit \(a < r < b \).
\end{satz}
\begin{bew}
	\obda  sei \(b \geq 0 \), ansonsten betrachte \( a^\prime = \minus{a} \ko b^\prime = \minus{b} \). \\
	Weiter können wir \(a \geq 0 \) annehmen, sonst nehme \(r = 0 \). \\
	Sei also \(0 \leq a < b \overset{\text{\autoref{C3-satz:arch_prinz}}}{\implies} \exists \, n \in \N \colon \; n(b-a) > 1 \). \\
	Setze \( B \coloneqq \set{ l \in \N \vert l > na } \subseteq \N \).
	\( \overset{\text{\autoref{satz:wohlordprinz}}}{\implies} m = \min B \) existiert.\\
	Ferner gilt \(m-a \leq na < m \) und somit
	\[ na < m = \underbrace{m-1}_{< na} + \underbrace{1}_{< n(b-a)} = nb \]
	\(\implies na < m  < nb \iff a < \frac{m}{n} < b\).
\end{bew}

\iftoggle{short}{}{\newpage}%Formatierung ausführliches Skript

\begin{bem}
	\(\sqrt{2} \in \R \setminus \Q \).
\end{bem}
\begin{bew}[Beweis durch Widerspruch]
	Angenommen \( r^2 = 2 \) mit \( r = \frac{p}{q} \ko q \in \N \ko p \in \Z \). \\
	Wir definieren 
	\[A \coloneqq \set{n \in \N \; \vert \; \exists \, m \in \Z \colon \; \frac{m^2}{n^2} = 2} \neq \emptyset \]
	\(\overset{\text{\autoref{satz:wohlordprinz}}}{\implies} n_* = \min A \in \N \) existiert. Also existiert \( m \in \Z_+ \) mit
	\(m^2 = 2n_*^2\). Somit ist \(m > n_* \).
	Ferner gilt
	\[ m= \sqrt{2}n_* \overset{\sqrt{2} > 1}{\iff} 0 < \underbrace{m - n_*}_{\in \N} = \underbrace{(\sqrt{2} - 1)}_{\in (0 \ko 1)} n_* < n_* \]
	Nun gilt: 
	\[ \sqrt{2} = \frac{m}{n_*} = \frac{m(m - n_*)}{n_*(m-n_*)} \overset{m^2 = 2n_*^2}{=} \frac{2n_*^2 - mn_*}{n_*(m - n_*)} = \frac{2n_* - m}{m - n_*} \]
	Somit \(2n_* - m \in \Z \ko m - n_* < n_* \). \Lightning{ zu \(n_* = \min A \)} \\
	Somit kann kein \(m \in \Z \) existieren, sodass \(\frac{m^2}{n^2} = 2 \) für beliebiges \(n \in \N \).
\end{bew}

\begin{satz}\label{satz:sqrt_irr}
	Sei \(k \in \N \), dann ist entweder \(\sqrt{k} \in \N \) oder \(\sqrt{k} \in \R \setminus \Q \).
\end{satz}
\begin{bew}
	Sei \(k \in \N \) und \(\sqrt{k} \notin \N \). \\
	Angenommen \(\sqrt{k} \in \Q \), also \(\sqrt{k} = \frac{m}{n} \ko m \in \Z \ko n \in \N \). \\
	Definiere \(A \coloneqq \set{n \in \N \; \vert \; \exists \, m \in \Z \colon \; \frac{m^2}{n^2} = k} \)
	\(\overset{\text{\autoref{satz:wohlordprinz}}}{\implies}\exists \, n_* = \min A \in \N \). \\
	Sei \(\frac{m}{n_*} = \sqrt{k} \), dann gilt
	Ferner existier \(q \in \N \) mit \(q \leq \sqrt{k} < q+1 \) nach \autoref{satz:gauss_klammer} \\
	Also gilt
	\[ 0 < \underbrace{m-qn_*}_{\in \N} = (\underbrace{\sqrt{k}-q}_{\in (0 \ko 1)})n_* < n_* \]
	Somit 
	\[ \sqrt{k} = \frac{m}{n_*} = \frac{m(m - qn_*)}{n_*(m - qn_*)} = \frac{kn_*^2 - mqn_*}{n_*(m - qn_*)}=\frac{kn_* - mq}{m - qn_*} \]
	\Lightning{} zu \(n_* = \min A \ko m-qn_* < n_* \)
\end{bew}

\iftoggle{short}{}{\newpage}%Formatierung ausführliches Skript

\begin{bem}(Erweiterung von \autoref{satz:sqrt_irr} auf \(n \)-te Wurzeln) \leavevmode \\
	Seien \(n \in \N \ko k \in \Z \) und \(x \in \R \) mit \(x^n = k \). Dann ist entweder \(x \in \N \) oder \(x \in \R \setminus \Q \).
\end{bem}
\begin{bew}
	Sei \(\frac{a}{b} \eqqcolon x \) mit \(x^n = k \ko a \in \Z \ko b \in \N \) und \(a \ko b\) teilerfremd \\
	\dphp für \(q \in \N \) mit \(a = qp_a \ko b = qp_b \) für \(p_a \ko p_b \in \Z \) folgt \(q = 1 \).
	Dann gilt
	\[{\frac{a}{b}}^n = \frac{a^n}{b^n} = k \iff \frac{a^n}{b} = \underbrace{kb^{n - 1}}_{\in \Z}. \]
	\(\implies \frac{a^n}{b} \in \Z \). Also muss \(b \) entweder \(1 \) oder \(\minus 1 \) sein.
	\(\implies \frac{a}{b} \in \N \).
\end{bew}

\end{document}
\documentclass[../ana1.tex]{subfiles}
\onlyinsubfile{\sectionNumbering} %Use numbering relative to sections and not subsection

\begin{document}
\setcounter{section}{5}

\section{Existenz von Wurzeln in \(\R \)}

\begin{notation}
	Sei \( n \in\N \) und \(a > 0 \). Eine Zahl \(\alpha \in \R \) 
	heißt dann \textit{\(n\)-te Wurzel von \(a \)}, 
	falls \(\alpha^n = a \).\\
	In diesem Fall schreibe auch 
	\( \alpha \eqqcolon a^{\frac{1}{n}}\) 
	oder \(\alpha \eqqcolon \sqrt[n]{a} \).
\end{notation}

\begin{satz}\label{satz:ex_wurzel}
	Sei \(\alpha \in\R, a>0\) und \(n \in \N \). Dann existiert die \(n\)-te Wurzel von \(a\) als eindeutige reelle Zahl,
	\dphp{} \(\existse \, \alpha \in \R \) mit \(\alpha > 0 \) und \(\alpha^n = a \).
\end{satz}
\begin{bew}
	Angenommen, die Behauptung gilt für \(a \geq 1 \). \\
	Sei \(0 < b < 1\) und setze \(a \coloneqq \frac{1}{b} > 1 \implies \existse \, \alpha > 0 \colon \alpha^n = a = \frac{1}{b} \). \\
	Setze ferner \(\beta \coloneqq \frac{1}{\alpha}\). \\
	Dann gilt also
	\[ \beta^n = {\left(\frac{1}{\alpha}\right)}^n = \frac{1}{\alpha^n} = \frac{1}{a} = b.\]
	Sei also \(a \geq 1 \). Ist \(a = 1 \), so ist \(\alpha = 1 \) die einzige Lösung von \(\alpha^n = 1 \). Außerdem können wir annehmen, dass \(n > 1 \).
	Es seien nun also \(a > 1 \ko n > 1 \). Setze
	\[ A \coloneqq \set{ x \in \R \; \vert \; 0 < x \ko x^n< a }. \]
	Dann ist \(1 \in A \) und somit \(A \neq \emptyset \). Außerdem ist \(A \) nach oben beschränkt, denn ist \(y \geq a\), so folgt
	\[ y^n \geq a^n = \underbrace{a \cdot a \cdots a}_{n\text{-mal}} > \underbrace{1 \cdot 1 \cdots \cdot 1}_{n-1\text{-mal}} \cdot a = a \]
	Somit ist \(A \leq a \). \(\overset{\text{\autoref{ax:V}}}{\implies} \alpha \coloneqq \sup A \in \R \) existiert. Da \(1 \in A \) folgt \(\alpha \geq 1 > 0 \). \\
	Für \(\alpha \) gilt entweder \(\alpha^n <a \ko \alpha^n > a \) oder \(\alpha^n = a \). Ist \(\alpha^n = a \), so sind wir fertig. Bleibt also noch zu zeigen,
	dass die anderen beiden Fälle nicht auftreten können.
	\begin{faelle}
		\item[Fall \(\alpha^n < a\):] Sei \(0 < \delta \leq 1 \). Dann gilt
			\!\begin{align*}
				{(\alpha + \delta)}^n & = \sum_{k=0}^{n} \binom{n}{k} \alpha^k\delta^{n-k} \\
								  	  & = \alpha^n + \sum_{k=0}^{n-1} \binom{n}{k} \alpha^k\delta^{n-k} \\
									  & = \alpha^n + \sum_{k=1}^{n-1} \binom{n}{k-1} \underbrace{\alpha^{k-1}}_{\leq a^{k-1}}\underbrace{\delta^{n+1-k}}_{\leq \delta \cdot \delta^{n-k} \leq \delta} \\
									  & \leq \alpha^n \delta \sum_{k=1}^{n-1} \binom{n}{k-1} \alpha^{k-1} \\
									  & \leq \alpha^n \delta \sum_{k=0}^{n} \binom{n}{k} \alpha^{k} \\
									  & = \alpha^n + \delta {(a+1)}^n. \tag{\(*\)}
			\end{align*}
			Es gilt \(\alpha^n < a\) nach Annahme 
			und \(\delta \coloneqq \frac{1}{2} 
			\min{\left( 1 \ko \frac{a-\alpha^n}{{(a+1)}^n} \right)} \) \\
			Dann gilt \(0<\delta\leq 1\) und \((*) \). Also
			\begin{align*}
				&{(\alpha + \delta)}^n \\
				\leq \; &\alpha^n + \delta {(a+1)}^n \\
				\leq \; &\alpha^n + \frac{1}{2} (a-\alpha^n) \\
				= \; &\frac{1}{2} (\alpha^n+a) \\
				< \; &\frac{1}{2} (a+a) \\
				= \; &a
			\end{align*}
			Somit ist \(\alpha < \alpha + \delta \). \Lightning{} zu \(\alpha \) ist \(\sup A\). \\
		\item[Fall \(\alpha^n > a\):] Sei \(0 < \delta \leq 1 \). Dann gilt
			\!\begin{align*}
				{(\alpha - \delta)}^n & = \sum_{k=0}^{n} \binom{n}{k} \alpha^{n-k} {(-\delta)}^k = \sum_{k=0}^{n} \binom{n}{k} \alpha^{n-k} {(-1)}^k\delta^k \\
								      & = \alpha^n + \sum_{k=0}^{n-1} \binom{n}{k+1} \alpha^{n-1-k} {(-1)}^{k+1}\delta^{k+1} \\
								   	  & = \alpha^n - \sum_{k=0}^{n-1} \binom{n}{k+1} \alpha^{n-1-k} {(-1)}^{k} \delta^k \\
									  & \geq \alpha^n - \delta \sum_{k=0}^{n-1} \binom{n}{k+1} a^{n-k+1} \\
									  & = \alpha^n - \delta \sum_{k=1}^{n} \binom{n}{k} a^k \\
					 				  & \geq \alpha^n - \delta {(a+1)}^n \tag{\(**\)}
			\end{align*}
			Setze nun \( \delta \coloneqq \frac{1}{2} \min \left(1 \ko \frac{\alpha^n-a}{{(a+1)}^n} \right) \). Dann gilt \(0 \leq \delta \leq \frac{1}{2} < 1 \). Somit gilt
			\[{(\alpha -\delta)}^n \geq \alpha^n - \frac{1}{2} (\alpha^n + a) = \frac{1}{2} (\alpha^n + a) > \frac{1}{2} (a + a) \geq a. \]
			Also ist \( \alpha - \delta \) eine obere Schranke für \(A \). Da \(\alpha - \delta < \alpha \), ist das ein Widerspruch zu \(\alpha = \sup A\). Somit bleibt nur \(\alpha^n = a\).
	\end{faelle}
\end{bew}

\begin{bem}
	Für die rationalen Zahlen ist die Aussage aus \autoref{satz:ex_wurzel} falsch. \zB{} \(\sqrt{2} \notin \Q \).
\end{bem}

\end{document}
\documentclass[../ana1.tex]{subfiles}
\onlyinsubfile{\sectionNumbering} %Use numbering relative to sections and not subsection

\begin{document}
\setcounter{section}{6}

\section{Folgen und Konvergenz}

\begin{defi}
	Sei \(X\neq\emptyset \) eine Menge. Eine Folge (mit Werten in \(X\) oder auch \(X\)-wertige Folge) ist eine Funktion 
	\[f: \N \rightarrow X, n \mapsto f(n)\in X\]
	Wir setzen \(a_n := f(n), n\in\N \).\\
	\(a_n\) heißt \(n\)-tes Folgenglied. Wir schreiben auch \({(a_n)}_{n\in\N}\) oder kurz \({(a_n)}_n\).\\
	Ist \(X = \R \), so heißt die Folge reellwertig oder reelle Folge (Folge reeller Zahlen). \({(a_n)}_{n\in\N} \subset \R \).
\end{defi}

\begin{defi}[Konvergenz (reeller Folgen)]
	Eine reelle Folge \({(a_n)}_{n\in\N}\) konvergiert (mit \(n\rightarrow\infty \)) gegen ein \(a\in\R \), falls 
	\[\forall \varepsilon > 0 \exists k \in \N: \forall n\geq k \text{ folgt } |a_n - a| < \varepsilon.\]
	Die Zahl \(a\) heißt Grenzwert der Folge, wir schreiben \(\limes{n}a_n = a\) oder \(a_n\rightarrow a\) (für \(n\rightarrow\infty \)).\\
	Eine (reelle) Folge heißt konvergent, falls ein \(a\in\R \) der Grenzwert der Folge ist, andernfalls heißt die Folge divergent.
\end{defi}

\begin{bem}
	\[\limes{n}a_n = a \Leftrightarrow \forall \varepsilon > 0 \exists k\in\N: \forall n\geq k \text{ folgt } |a_n -a|< \varepsilon.\]
\end{bem}

\begin{bsp}
	Beweis mit \(\varepsilon \)-Methode:
	\begin{enumerate}
		\item \(a_n := \frac{1}{n}\) konvergiert gegen \(0\). Denn zu geg.\  \(\varepsilon > 0\) wähle \(k\in\N \) mit \(k>\frac{1}{\varepsilon}\). Dann gilt für \(n\geq k\)
		\[|a_n-a| = |\frac{1}{n} - 0| = \frac{1}{n} \leq \frac{1}{k} < \varepsilon.\]
		\item Konstante Folge. Sei \(a\in\R \) und sei \(a_n = a\) für \(n\in\N \).\\
		Dann folgt \(\limes{n} a_n = a\), denn für \(\varepsilon > 0\)
		\[|a_n - a| = |a-a|=0<\varepsilon \text{, wähle } k=1\]
		\item Sei \(a_n := {(-1)}^n\), also \(a_1 = -1, a_2 = 1, a_3 = -1, \ldots \) \\
		Dann ist \({(a_n)}_{n\in\N}\) nicht konvergent.
		\begin{bew}
			Angenommen: \({(a_n)}_n\) konvergiert und \(a\in\R \) ist Grenzwert. Zu \(\varepsilon = 1\) existiert dann \(k\in\N \) so, dass \(|a_n - a| < \varepsilon = 1 \quad \forall n\geq k\) \\
			Also gilt für \(n\geq k\):
			\[2 = |a_n - a_{n+1}| = |a_n - a + a - a_{n+1}| \leq |a_n - a| + |a-a_{n+1}| < 1 + 1 = 2 \text{ \Lightning}\]
		\end{bew}
		\item Die Folge \((a_n)\) konvergiert gegen \(a\). Dann konvergiert auch \({(|a_n|)}_n\) gegeen \(|a|\). (Hinweis: Umgekehrte Dreiecksungleichung)
		\item Geometrische Folge:\\
		Sei \(q\in\R, |q|<1\). Dann gilt
		\[\limes{n} q^n = 0.\]
		\begin{bew}
			Annahme: \(q\neq 0\), dann gilt \(\frac{1}{|q|}>1\) und es existiert \(x>0\), sodass \(\frac{1}{|q|} = 1 + x\).\\
			Aus Bernoullischer Ungleichung folgt 
			\[ {(1+x)}^n \geq 1+nx \]
			und somit 
			\[ |q^n-0| = |q^n| = |q|^n = \frac{1}{{(1+x)}^n} \leq \frac{1}{1+nx}. \]
			Also zu \(\varepsilon > 0\) wähle \(k\in\N \forall n\geq k\) gilt \(nx > \frac{1}{\varepsilon}\).
			\[|q^n-0|\leq \frac{1}{1+nx} \leq \frac{1}{nx} < \varepsilon \text{ für } n\geq k.\]
		\end{bew}
		\item Sei \(a\in\R \) mit \(a>0\). Dann konvergiert die \(a_n = a^{\nicefrac{1}{n}}\) gegen \(1\).
		\begin{bew}
			Fall 1: Die Beh.\ stimmt für \( a=1 \).\\
			Fall 2: \(a>1\). Dann ist \(a_n = a^{\nicefrac{1}{n}}>1\) und somit \(q_n := a_n-1 = a^{\nicefrac{1}{n}} -1 >0\).\\
			\[a_n = a^{\frac{1}{n}} = 1+q_n \Rightarrow a = {(1+q_n)}^n \underset{\text{Bern.\ Ungl.}}{\geq} 1+ nq_n\]
			\[\Rightarrow 0 \leq q_n \leq \frac{a-1}{n} \forall n\in\N \]
			Zu \(\varepsilon > 0\) wähle \(K\in\N \) mit \(K>\frac{a-1}{\varepsilon}\).\\
			Dann \(n\geq K\)
			\[|a_n-1| = |a^{\nicefrac{1}{n}}-1|= a^{\nicefrac{1}{n}} -1 = q_n \leq \frac{a-1}{n} < \varepsilon.\]
			Fall 3: \(0<a<1\). Dann ist \(b := \frac{1}{a}>1\).
			\[\overset{\text{Fall 2}}{\Rightarrow} \limes{n} b^{\frac{1}{n}} = 1\]
			\begin{align*}
				|a^{\nicefrac{1}{n}}-1|&=a^{\nicefrac{1}{n}}\left|1-\frac{a}{a^{\nicefrac{1}{n}}}\right|\\
				&= a^{\nicefrac{1}{n}}\left| 1 - {\left(\frac{1}{a}\right)}^{\nicefrac{1}{n}} \right|\\
				&= a^{\nicefrac{1}{n}} \left|1-b^{\nicefrac{1}{n}}\right|\\
				&\leq \left|1-b^{\nicefrac{1}{n}} \right|\underset{n\rightarrow\infty}{\longrightarrow} 0
			\end{align*}
			Somit gilt
			\[\limes{n} a^{\nicefrac{1}{n}} = 1\]
		\end{bew}
	\item Es gilt \(\limes{n}n^{\nicefrac{1}{n}}=1\).
	\begin{bew}
		1. Versuch:\\
		Setze \(q_n := n^{\nicefrac{1}{n}} -1>0\) für \(n>1\) \\
		\[\Rightarrow n={(1+q_n)}^n \geq 1+nq_n\]
		\[\Rightarrow |n^{\nicefrac{1}{n}}-1| = q_n \leq \frac{n-1}{n} = 1-\frac{1}{n}\]
		funktioniert nicht\dots \\
		Frage: Kann Bernoullische Ungleichung verbessert werden?\\
		\begin{align*}
			{(1+q)}^n &= \sum_{k=0}^{n} \binom{n}{k} q^k1^{n-k}\\
			&=1+\binom{n}{1}q + \binom{n}{2}q^2 + \sum_{k=3}^{n}\binom{n}{k} q^k1^{n-k}\\
			&\geq 1+nq + \frac{n(n-1)}{2} q^2\\
			&\geq 1+\frac{n(n-1)}{2} q^2 \quad\text{falls }	q\geq 0.\\
			(*)
		\end{align*}
		Setzen \(q_n := n^{\nicefrac{1}{n}} -1 >0\) für \(n\geq 2\).
		\[\Rightarrow n = {(1+q_n)}^n \overset{(*)}{\geq}1+\frac{n(n-1)}{2} q_n^2\]
		\[\Rightarrow q_n^2 \leq \frac{2(n-1)}{n(n-1)} = \frac{2}{n}\]
		\[\Rightarrow q_n \leq \sqrt{\frac{2}{n}} \forall n\geq 2\]
		Zu \(\varepsilon >0\) wähle \(K\in\N \) mit \(\sqrt{\frac{2}{K}}<\varepsilon \).
		\[\Rightarrow |n^{\nicefrac{1}{n}} -1| = q_n \leq \sqrt{\frac{2}{n}} \overset{n\geq K}{<} \varepsilon.\]
		Somit gilt \(\forall\varepsilon>0\exists k\in\N \), sodass für \(n\geq K\) gilt
		\[|n^{\nicefrac{1}{n}} - 1| <\varepsilon.\]
		Also per Definition
		\[\limes{n} n^{\nicefrac{1}{n}} = 1.\]
	\end{bew}
	\end{enumerate}
\end{bsp}

\begin{satz}
	Falls die reelle Folge \({(a_n)}_{n\in\N}\) konvergiert, so ist ihr Grenzwert eindeutig bestimmt.
\end{satz}
\begin{bew}
	Annahme: \({(a_n)}_{n\in\N}\) konvergiert gegen \(a\) und \(b\in\R \). Und \(a\neq b\) o.\ B.\ d.\ A.\ gilt \(a<b\).
	Wissen: 
	\begin{align*}
		\forall \varepsilon>0\exists K_1\in\N: \forall n\geq K_1 \quad |a_n -a| <\varepsilon \\
		\forall \varepsilon>0\exists K_2\in\N: \forall n\geq K_2 \quad |a_n -b| <\varepsilon
	\end{align*}
	Setze \(\varepsilon:=\frac{b-a}{2} >0\).\\
	Dann folgt für \(n\geq \max \{K_1,K_2\} \)
	\[b-a = b-a_n + a_n -a \leq \underbrace{|b-a|}_{<\varepsilon} + \underbrace{|a_n-a|}_{<\varepsilon} < 2\varepsilon = b-a \text{\Lightning}\]
	Somit muss \(a=b\) gelten!
\end{bew}

Bild:\\
\begin{center}
\begin{tikzpicture}[scale=2]
	\draw (0,0) -- (3,0);
	\draw (1,0) node[below=2mm] {\(a\)} (2,0) node[below=2mm] {\(b\)};
	\draw (1,-1/32) -- (1,1/32);
	\draw (2,-1/32) -- (2,1/32);
	\draw[<->] (1.5,1/8) -- (2,1/8);
	\draw (0.5,0) node {\((\)} (1.5,0) node {\()(\)} (2.5,0) node {\()\)};
	\draw (1.75, 1/8) node[above] {\(\varepsilon= \frac{b+a}{2} \)};
	\draw (1.5,-0.5) node {\(a_n\)};
	\draw[->] (1.5,-0.4) -- (1.75,-0.1);
	\draw[->] (1.5,-0.4) -- (1.25,-0.1);
\end{tikzpicture}
\end{center}

\begin{defi}[\(\varepsilon \)-Umgebung]
	Die \(\varepsilon \)-Umgebung um \(a\in\R \) ist die Menge
	\[U_\varepsilon(a):= \{x\in\R: |x-a| <\varepsilon \} = (a-\varepsilon, a+\varepsilon).\]
\end{defi}

Beobachtung: Sei \({(a_n)}_{n\in\N}\) konvergent gegen \(a\in\R \).
\[\Leftrightarrow \forall\varepsilon >0 \exists K\in\N: a_n \in U_\varepsilon(a) \forall n\geq K.\]

\begin{defi}[Beschränktheit von Folgen]
	Eine Folge \({(a_n)}_{n\in\N} \subset \R \) heißt beschränkt, wenn für \(C\geq 0\) gilt \(|a_n|\leq C \quad \forall n\in\N \) \\
	nach oben beschränkt, wenn es ein \(C\in\R \) gibt mit \(a_n\leq C \quad \forall n\in\N \) \\
	nach unten beschränkt, wenn es ein \(C\in\R \) gibt mit \(a_n\geq C \quad \forall n\in\N \).
\end{defi}

\begin{bem}
	beschränkt \(\Leftrightarrow \) nach oben und nach unten beschränkt
\end{bem}

\begin{satz}
	Jede konvergente Folge ist beschränkt.
\end{satz}
\begin{bew}
	Sei \(\limes{n} a_n = a\). Zu \(\varepsilon = 1\) wähle \(K\in\N \).
	\(|a_n-a|<1 \quad \forall n\geq K\).
	\begin{align*}
		n\geq K &\Rightarrow |a_n| = |a_n -a+a| \leq |a_n -a| + |a| < 1 + |a|\\
		n\leq K-1 &\Rightarrow |a_n| \leq \max \{|a_1|,\ldots,|a_{K-1}|\}.
	\end{align*}
	Setze \(C:= \max \{|a_1|,\ldots,|a_{K-1}|, 1 + |a|\} \), so folgt
	\[|a_n| \leq C \quad\forall n\in\N.\]
\end{bew}

\begin{lem}
	Die Folge \({(b_n)}_n \subset \R \) konvergiert gegen \(b\neq 0\). Dann existiert \(K\in\N \), sodass 
	\[|b_n| \geq \frac{|b|}{2}.\]
\end{lem}
\begin{bew}
	Bild:\\
	\begin{center}
	\begin{tikzpicture}
		\draw (-1,0) -- (5,0);
		\foreach \x in {0, 2, 4} 
		{
			\draw (\x,-1/8) -- (\x,1/8);
		}
		\draw (0,0) node[below=2mm] {\(0\)};
		\draw (2,0) node[below=2mm] {\(\frac{b}{2}\)}; 
		\draw (2,-1.5) node {\(b_n\)};
		\draw[->] (2,-1.2) -- (3,-0.1);
		\draw (4,0) node[below=2mm] {\(b\)};
	\end{tikzpicture}
	\end{center}
	Setze \(\varepsilon := \frac{|b|}{2}>0\). Dann existiert \(K\in\N \) mit 
	\[|b_n-b| < \varepsilon = \frac{|b|}{2} \quad\forall n\geq K, n\geq K\]
	\begin{align*}
		&\Rightarrow |b| = |b-b_n + b_n| \leq |b-b_n| + |b_n| \overset{n\geq K}{<} \frac{|b|}{2} + |b_n|\\
		&\Rightarrow |b_n| > |b| - \frac{|b|}{2} = \frac{|b|}{2} \quad \forall n\geq K.
	\end{align*}
\end{bew}

\begin{satz}[Rechenregel für Grenzwerte]
	Es gelte \(a_n\rightarrow a, b_n \rightarrow b\) für \(n\rightarrow\infty \).
	\begin{enumerate}
		\item \(\forall \lambda, \mu \in\R \) ist \( {(\lambda a_n + \mu b_n)}_{n\in\N} \) konvergent mit Grenzwert 
		\[\limes{n} (\lambda a_n + \mu b_n) = \lambda a + \mu b.\]
		\item Die Folge \( {(a_n b_n)}_{n\in\N} \) konvergiert mit Grenzwert 
		\[\limes{n} {(a_n b_n)} = ab.\]
		\item Falls \(b\neq 0\), so gibt es ein \(K_0 \in\N \) mit \(b_n \neq 0 \, \forall \, n\geq K\) und die Folge \( {\left(\frac{a_n}{b_n}\right)}_{n\geq K_0}\) ist konvergent mit Grenzwert 
		\[\limes{n} \frac{a_n}{b_n} = \frac{a}{b}.\]
	\end{enumerate}
\end{satz}
\begin{bew}
	\begin{enumerate}
		\item 1. Fall \(\lambda = \mu = 1\).\\
		Zu \(\varepsilon>0 \exists K_1, K_2 \in\N \), sodass
		\begin{align*}
			|a_n-a|<\frac{\varepsilon}{2} \quad \forall n\geq K_1\\
			|b_n-b|<\frac{\varepsilon}{2} \quad \forall n\geq K_2
		\end{align*}
		Setze \(K := \max \{K_1, K_2\} \). Dann folgt 
		\[|a_n + b_n - (a+b)| = |(a_n-a)+(b_n-b)| \leq |a_n-a|+|b_n-b| < \frac{\varepsilon}{2} + \frac{\varepsilon}{2} = \varepsilon \quad \forall n\geq K.\]
		Also ist \(\limes{n}a_n+b_n=a+b\).
		Fall 2: allg.\  \(\lambda,\mu\in\R \) \\
		Aus 2.\ folgt 
		\[\limes{n} \lambda a_n = \lambda \limes{n} a_n\]
		\[\limes{n} \mu b_n = \mu \limes{n} b_n \qquad(*)\]
		\(\overset{\text{Fall 1}}{\Rightarrow} \lambda a_n + \mu b_n\) ist konvergent und 
		\[\limes{n} (\lambda a_n + \mu b_n) = \limes{n} (\lambda a_n) + \limes{n} (\mu b_n) \overset{(*)}{=} \lambda \limes{n} a_n + \mu \limes{n} b_n = \lambda a + \mu b.\]
		\item Sei \(n\in\N \). Dann folgt \(a_n b_n-ab = a_n b_n -a_n b + a_n b - ab=a_n(b_n-b)+(a_n-a)b\)
		\[\Rightarrow |a_n b_n -ab|\leq |a_n| |b_n-b| + |a_n-a||b|.\]
		Nach Satz 6 existiert \(C\geq 0\) mit \(|a_n|\leq C \forall n\in\N \). Setze \(D:= \max \{C, |b|\} \).
		\[|a_n b_n - ab| \leq D(|a_n-a|+|b_n-b|) \,\forall \, n\in\N.\]
		Zu \(\varepsilon > 0\) wähle \(K_1,K_2\in\N \) mit 
		\[|a_n-a|<\frac{\varepsilon}{2(0+1)} \quad \forall n\in K_1\]
		\[|b_n-b|<\frac{\varepsilon}{2(0+1)} \quad \forall n\in K_2\]
		Dann folgt \(\forall n\geq K := \max \{K_1, K_2\} \)
		\[|a_n b_n - ab|<\frac{\varepsilon}{2} + \frac{\varepsilon}{2} = \varepsilon.\]
		Also \(\limes{n} a_n b_n = ab\).
		\item o.\ B.\ d.\ A.\  \(a_n = 1\). (aus 2.\ folgt dann der allgemeine Fall mit \( \frac{a_n}{b_n} = a_n \cdot\frac{1}{b_n} \))\\
		Da \(b_n \rightarrow b \neq 0\), folgt mit Lemma 7, dass ein \(K_0\in\N \) existiert mit \(|b_n| >\frac{|b|}{2} \) für \(n\geq K_0\) 
		\[\frac{1}{b_n} \text{ ist wohldefiniert } \forall n\geq K_0.\]
		Es gilt: \( \frac{1}{b} - \frac{1}{b_n} = \frac{b_n - b}{b b_n} \) und somit 
		\[ \left| \frac{1}{b} - \frac{1}{b_n} \right| = \frac{|b_n-b|}{|b| \cdot |b_n|} \leq \frac{2 |b_n - b|}{|b|^2}. \]
		Zu \(\varepsilon >0 \) wähle \(K_1 \in \N \) mit \(|b_n - b| < \frac{|b|^2\varepsilon}{2} \quad \forall n\geq K_1 \).\\
		Dann folgt 
		\[ \left| \frac{1}{b} - \frac{1}{b_n} \right| \leq \frac{2 \cdot |b_n - b|}{|b|^2} < \varepsilon \quad \forall n\geq\max \{ K_0,K_1\}. \]
		Somit folgt \( \frac{1}{b_n} \rightarrow \frac{1}{b} \) (für \(n\rightarrow \infty \)). Somit folgt die allg. Aussage aus Teil 2 von Satz 7.1.8.
	\end{enumerate}
\end{bew}

%20.11.2018
reelle Folgen \(f = {(f_n)}_n, g = {(g_n)}_n  \) \\
\[ {(f+g)}_n := f_n + g_n, \quad n\in\N \]
\(x {(\lambda f)}_n := \lambda f_n \Rightarrow {(\lambda f + \mu g)}_n = \lambda f_n + \mu g_n \) ist eine Linearkombination.\\
\(\Rightarrow{}\) Raum der reellen Folgen ist ein reeller Vektorraum.
\begin{align*}
	&\{ \text{Raum der (reellen) Folgen} \} \\
	&\supsetneq \{ \text{Raum der beschränkten (reellen) Folgen} \} \\
	&\supsetneq \{ \text{Raum der (reellen) konvergenten Folgen} \} \\
	&\supsetneq \{ \text{Raum der (reellen) Nullfolgen} \}.
\end{align*}
\({(f_n)}_{n\in\N}\) ist eine Nullfolge, falls \( \limes{n} f_n = 0 \).

\begin{bsp}[1]
	\( p,q \) Polynome vom Grad \(m,n\in\N \).\\
	D. h. \[ p(x) = a_m x^m + a_{m-1} x^{m-1} + \cdots + a_1 x + a_0 \quad \forall x\in\R \]
	\[q(x) = b_n x^n + b_{n-1} x^{n-1} + \cdots + b_1 x + b_0 \quad b_n \neq 0 \neq a_m \]
	\begin{align*}
		k\in\N. \frac{p(k)}{q(k)} = \frac{a_m x^m + a_{m-1} x^{m-1} + \cdots + a_1 x + a_0}{b_n x^n + b_{n-1} x^{n-1} + \cdots + b_1 x + b_0}\\
		= k^{m-n} \frac{a_m + a_{m-1} k^{-1} + \cdots + a_1 k^{1-m} + a_0 k^{-m} }{b_n + b_{n-1} k^{-1} + \cdots + b_1 k^{1-n} + b_0 k^{-n}}\\
		\overset{\text{Satz 8}}{\longrightarrow}
		\begin{cases}
			0, \text{ falls } n>m.\\
			\frac{a_n}{b_n}, \text{ falls } n = m.
		\end{cases}
	\end{align*}
\end{bsp}

\begin{bsp}[Geometrische Reihe]
	\begin{align*}
		-1<q<1.\\
		a_n &:= 1 + q + q^2 + \cdots + q^n\\
		&= \sum_{l=0}^{n} q^l \overset{\text{Satz 3.5.7}}{=} \frac{1-q^{n+1}}{1-q}\\
		\Rightarrow \limes{n} a_n &= \frac{1- \limes{n} q^{n+1}}{1-q} = \frac{1}{1-q}.
	\end{align*}
	Da \(q^n \rightarrow 0, n\rightarrow\infty \), Bsp. 6 oben.
	Schreiben hierfür \[ \sum_{l=0}^{n} q^l = \frac{1}{1-q}, \quad -1<q<1 \]
\end{bsp}

\begin{bsp}
	Ist \( {(b_n)}_n \) beschränkt, \( {(a_n)}_n \) Nullfolge. \( \Rightarrow {(b_n a_n)}_n \) Nullfolge. (Hausaufgabe)
\end{bsp}

\textbf{Notation:} Wir sagen die Aussagen \( A(n), n\in\N \) gelten für fast alle \( n\in\N \), falls \( K_0\in\N \) existiert, sodass \(A(n)\) wahr ist für alle \( n\geq K_0 \) (d.\ h.\ für alle genügend großen \( n \), d.\ h.\  \( A(n) \) wahr für alle bis auf endlich viele \(n\in\N \)).

\begin{bsp}
	\[ a_n\rightarrow a,n\rightarrow\infty \Leftrightarrow \forall \varepsilon > 0 \text{ ist } a_n\in U_\varepsilon(a) = (a-\varepsilon, a+\varepsilon \text{ für fast alle } n.)  \]
\end{bsp}

\begin{satz}
	Seien \( {(a_n)}_n, {(b_n)}_n\) konvergente reelle Folgen, \( a_n\rightarrow a, b_n \rightarrow b, n\rightarrow\infty \). Dann gilt
	\begin{enumerate}
			\item Aus \(a_n\leq b_n\) für fast alle \(n\) folgt \(a\leq b\).
			\item Sind \(c,d\in\R, c\leq a_n\leq d \) für fast alle \(n \Rightarrow c\leq a\leq d\)
			\item (Sandwichlemma) Ist \(a_n \leq c_n \leq b_n \) für fast alle \(n\) (\( {(c_n)}_n \) weitere reelle Folge) und \( a=b \Rightarrow {(c_n)}_n \) konvergiert und \( \lim\limits_{b\rightarrow\infty}c_n = a \) (\( =b \)).
	\end{enumerate}
\end{satz}
\begin{bew}\leavevmode
	\begin{enumerate}
		\item Bild: Ang.\  \( a>b \) \\
		\( \varepsilon = \frac{b-a}{2} >0 \)
		\begin{center}
		\begin{tikzpicture}[scale=2]
			\draw (0,0) -- (4,0);
			\draw (1,-1/8) -- (1,1/8);
			\draw (1,0) node[below=2mm] {\(b\)};
			\draw (3,-1/8) -- (3,1/8);
			\draw (3,0) node[below=2mm] {\(a\)};
			\draw (2,0) node {\()(\)} (1.5,0) node[below] {\(b_n\)} (2.5,0) node[below] {\(a_n\)};
			\draw[<->] (2,1/4) -- (3,1/4);
			\draw (2.5,1/4) node[above] {\(\varepsilon\)};
		\end{tikzpicture}
		\( \Rightarrow b_n > a_n \) \Lightning{}
		\end{center}
		\underline{Formal:} \(\exists K_0 \in\N:a_n\leq b_n \quad \forall n\geq K_0. \)
		\begin{align*}
		&\forall \varepsilon > 0 \exists K_1 \in\N,K_2\in\N: &a_n \in U_\varepsilon(a) &\forall n\geq K_1, &a - \varepsilon < a_n < a+\varepsilon \\
		&\Rightarrow K := \max(K_0, K_1, K_2) &b_n \in U\varepsilon (b) &\forall n\geq K_2,  &b-\varepsilon < b_n < b+\varepsilon.
		\end{align*}
		Ang. \(a>b: \varepsilon := \frac{a-b}{2} > 0 \Rightarrow \) \\
		\(K \) wie oben \(:\Rightarrow a< a_n + \varepsilon \leq b_n + \varepsilon < b + e\varepsilon = b + 2 \frac{a-b}{2} = a \)
		\( \Rightarrow a<a\) \Lightning{} \( \Rightarrow a\leq b \checkmark{}\).
		Andere Möglichkeit:\\
		\[ a_n \leq b_n, \forall \varepsilon > 0: a-\varepsilon < a_n<a + \varepsilon, b-\varepsilon < b_n < b + \varepsilon \quad \forall n\geq K. \]
		\[ a < a_n + \varepsilon \leq b_n + \varepsilon < b + 2\varepsilon \Rightarrow \underbrace{a-b < 2\varepsilon \quad \forall \varepsilon > 0}_{\Rightarrow a-b\leq 0 \Leftrightarrow a\leq b.}. \]
		\item Nehme \(b_n = c, b_n \rightarrow c\).\\
		Da \(b_n = c \leq a_n \overset{1.}{\Rightarrow} c= \lim b_n \leq \lim a_n = a \).\\
		Nehme auch \(b_n = d, a_n \leq d = b_n \overset{1.}{\Rightarrow} a \leq \lim b_n = d. \checkmark \)
		\item Haben \(\forall \varepsilon > 0\).
		\begin{align*}
			&\exists K_0 \in\N: &a_n \leq c_n \leq b_n \quad \forall n\geq K_0\\
			&\exists K_1, K_2 \in\N: &a-\varepsilon < a_n < a + \varepsilon \quad \forall n\geq K_1\\
			&&\underbrace{b-\varepsilon}_{=a-\varepsilon} < b_n < \underbrace{b + \varepsilon}_{=a+\varepsilon} \, \forall \, n \geq K_2. \text{ (da) }b=a
		\end{align*}
		\[\forall  n\geq K: a-\varepsilon < a_n \leq c_n \leq b_n < a+\varepsilon \]
		\[ \Rightarrow a-\varepsilon < c_n < a_n + \varepsilon \Leftrightarrow c_n\in U_\varepsilon(a) \quad \forall n\geq K \] 
		\( \Leftrightarrow \) konvergiert \( {(c_n)}_n \) gegen \( a \)!
	\end{enumerate}
\end{bew}

Achtung! \( a_n < b_n \forall n, a_n \rightarrow a, b_n \rightarrow b \nRightarrow  a<b\).\\
Bsp. \(a_n = 0, b_n = \frac{1}{n}\).

\begin{defi}[Uneigentliche Konvergenz]
	Die Folge \( {(a_n)}_n \) konvergiert uneigentlich (divergiert bestimmt) gegen \(+\infty \), falls 
	\[ \forall R>0 \exists K\in\N \text{ mit } a_n >R \quad \forall n\geq K. \]
	Schreiben \( \limes{n} a_n = \infty \) oder \( a_n \rightarrow +\infty, n\rightarrow \infty \)
	Analog für \( \limes{n} a_n = -\infty \), falls 
	\[ \forall R<0 \exists K\in\N: a_n < R \forall n\geq K. \]
\end{defi}

\begin{bsp}
	Ist \(a>1 \Rightarrow \limes{n}q^n = +\infty, 0< \frac{1}{q} < 1. \)
\end{bsp}
\begin{bew}
	\( \frac{1}{q^n} = {\left( \frac{1}{q} \right)}^n \rightarrow 0, n\rightarrow \infty \) \\
	d.\ h.\ zu \(R>0 \exists K\in\N: \frac{1}{q^n} < \frac{1}{R} \quad \forall n\geq K. \) \\
	\( \Leftrightarrow q^n > R \quad \forall n\geq K. \) Also \(\lim q^n = +\infty \) nach Def.\\
	Insgesamt: 
	\begin{align*}
		&q>1 &\Rightarrow \limes{n}q^n = +\infty.\\
		&q=1 &\Rightarrow \limes{n}q^n = 1.\\
		&-1<q<1 & \Rightarrow \limes{n} q^n = 0.\\
		&q\leq 1 & \Rightarrow {(q^n)}_n \text{ ist nicht konvergent.}
	\end{align*}
	Ist \( q<1 \Rightarrow {(q_n)}_n \) nicht beschränkt ist.
\end{bew}

\begin{satz}[Kehrwerte]
	\begin{enumerate}
		\item Aus \( |a_n| \rightarrow \infty, n\rightarrow\infty \) folgt \(\frac{1}{a_n} \rightarrow0,n\rightarrow\infty \).
		\item Aus \( a_n\rightarrow 0, a_n > 0 \) (bzw. \(a_n<0\)) \( \forall n \) folgt \( \frac{1}{a_n} \rightarrow\infty, n\rightarrow\infty \) (\( \frac{1}{a_n} \rightarrow -\infty, n\rightarrow\infty \)).
	\end{enumerate}
\end{satz}
\begin{bew}\phantom{\qedhere}
	Übung.
\end{bew}

\end{document}
\documentclass[../ana1.tex]{subfiles}
\onlyinsubfile{\sectionNumbering} %Use numbering relative to sections and not subsection

\begin{document}
\setcounter{section}{7}

\section{Monotone Konvergenz}
\begin{defi}
	Eine Folge \({(a_n)}_n\) heißt \\
	monoton wachsend, falls \(a_n\leq a_{n+1} \quad \forall n\in\N \).\\
	monoton fallend, falls \(a_{n+1} \leq a_n \quad\forall n\in\N \).\\
	Ähnlich: monoton wachsend (fallend) für fast alle \(n\in\N \), falls \( K\in\N \) existiert mit \( a_n\leq a_{n+1} \quad \forall n\geq K \) (bzw. \(a_{n+1}\leq a_n \quad \forall n\geq K \)).\\
	Ist \\
	\( a_n<a_{n+1} \quad \forall n\in\N \), so heißt \(a_n\) streng monoton wachsend.\\
	\( a_{n+1}<a_n \quad \forall n\in\N \), so heißt \(a_n\) streng monoton fallend.
\end{defi}
\begin{satz}[Monotone Konvergenz]
	Jede monoton wachsende, nach oben beschränkte Folge ist konvergent. Jede monoton fallende, nach unten beschränkte Folge ist konvergent.
\end{satz}
\begin{bew}
	\({(a_n)}_{n\in\N}, a_{n+1} \geq a_n \quad \forall n\in\N \) (oder \( \forall n\geq K \in\N \))\\
	und \( \exists C\in\R: a_n \leq C \quad \forall n\in\N \) (oder \( \forall n\geq K \in\N \))\\
	\[ B:= \{a_n, n\in\N \} \subset \R, b\neq \emptyset \text{ und } B\leq C. \]
	\(\overundersett{Vollst.}{axiom}{\Rightarrow} L := \sup B\) die kleinste obere Schranke für \(B\).\\
	\(\Rightarrow a_n \leq L \quad \forall n\in\N \).\\
	Und: \(L\) kleinste ob. Schranke \( \Rightarrow \forall \varepsilon>0: L-\varepsilon \) keine obere Schranke für \(B\).\\
	\[ \Rightarrow \exists K \in\N: L-\varepsilon < a_K \leq a_{K+1} \leq a_{K+2} \leq \cdots \leq a_n \quad \forall n\geq K. \]
	\[ \forall n\geq K: L-\varepsilon < a_n \leq L < L + \varepsilon \Leftrightarrow a_n \in (L-\varepsilon, L + \varepsilon) \]
	\( \Leftrightarrow {(a_n)}_n \) konvergiert gegen \(L\).\\
	Ist \(a_{n+1} \leq a_n, a_n\geq C \quad \forall n\in\N \), so betrachte \(b_n := -a_n \leq -C \) und \(b_{n+1} \geq b_n\). Dann ersten Fall anwenden!
\end{bew}
\begin{bsp}[1]
	\(x_0 > 0, \quad x_{n+1} := \frac{1}{2} \left( x_n + \frac{a}{x_n} \right) \) konvergent gegen \(\sqrt{a}, a>0 \).\\
	Ang.: \( \limes{n} x_n = l \) existiert, dann auch \( \limes{n} x_{n+1} = l, l>0 \)
	\[ \overundersett{Grenzwert-}{sätze}{\Longrightarrow} l = \limes{n} x_{n+1} = \limes{n} \left( \frac{1}{2} \left(x_n + \frac{a}{x_n} \right) \right) = \frac{1}{2} \left( l + \frac{a}{l} \right) \Rightarrow l^2 = a, l = \sqrt{a}. \]
\end{bsp}
\begin{bsp}[2]
	\(f_n := {\left( 1 + \frac{1}{n} \right)}^n \) Grenzwert \( \limes{n} f_n = \limes{n} {\left( 1 + \frac{1}{n} \right)}^n \) existiert \( =: e \).\\
	Beh. 1: \( f_n\) ist nach oben beschränkt.\\
	Beh. 2: \( f_n\) ist monoton wachsend.
\end{bsp}
%22.11.2018
\begin{bew}
	Beh. 1:
	\begin{align*}
		f_n = {\left(1+\frac{1}{n}\right)}^n  &= \sum_{k=0}^{n} \binom{n}{k} {\left( \frac{1}{n} \right)}^k\\
		&= \sum_{k=0}^{n} \underbrace{\frac{n!}{k!(n-k)!}}_{=\frac{n(n-1)\ldots (n-k+1)}{k!}} \\
		&= \sum_{k=0}^{n} \frac{1}{k!} \frac{n}{n} \underbrace{\frac{n-1}{n}}_{<1} \underbrace{\frac{n-k+1}{n}}_{<1}\\
		&\leq \sum_{k=0}^{n} \frac{1}{k!} =: e_n, \quad f_n \leq e_n \forall n\in\N \\
		e_{n+1} = e_n + \frac{1}{(n+1)!} > 1_n.
	\end{align*}
	Beachte: \( k! = k(k-1)(k-2)\ldots3\cdot2\cdot1 \qquad k\geq 2 \) \\
	\( \geq 3\cdot3\ldots3\cdot2\cdot1 = 3^{k-2}\cdot2 (*) \) \\
	Also ist \(n\geq3\).
	\begin{align*}
		e_n = \sum_{k=0}^{n} \frac{1}{k!} = 1+1 + \sum_{k=2}^{n} \frac{1}{k!} \overset{(*)}{\leq} 2 + \sum_{k=2}^{n} \frac{1}{2 \cdot 3^{k-2}}\\
		= 2 + \frac{1}{2} \underbrace{\sum_{l=0}^{n-2} 
		{\left(\frac{1}{3}\right)}^l}_{
			=\ensuremath{ \frac{1 
			- {\left( \nicefrac{1}{3} \right)}^{n-1} } }
			{1 - \nicefrac{1}{3}}}\\
		\leq 2 + \frac{1}{2} \cdot \frac{3}{2} 
		= 2 + \frac{3}{4} = 2,75.\\
		\Rightarrow e_n \leq 2,75 \forall n\geq 2.
	\end{align*}
	\(\Rightarrow {(e_n)}_n \) ist nach oben beschränkt.\\
	\( \overset{\text{Mon.\ Konv.}}{\Rightarrow} \limes{n} e_n \) existiert \(\leq 2,75\).\\
	Auch \( f_n\leq e_n\leq 2,75 \quad \forall n\geq 2 \).\\
	\( \Rightarrow {(f_n)}_n \) ist auch oben beschränkt.
	Beh. 2:\\
	\begin{align*}
		\frac{f_n}{f_{n-1}} &= \frac{{(1+\frac{1}{n})}^n}{{(1+\frac{1}{n-1})}^{n-1}} \qquad n\geq 2\\
		&= \frac{{(\frac{n+1}{n})}^n}{{(\frac{n}{n-1})}^{n-1}} = \frac{n}{n-1} \frac{{(\frac{n+1}{n})}^n}{{(\frac{n}{n-1})}^{n}} = \frac{n}{n-1} {\left( \frac{\overbrace{(n+1)(n-1)}^{n^2-1}}{n^2} \right)}^n\\
		&= \frac{n}{n-1} {\left( \frac{n^2-1}{n^2} \right)}^n 
		= \frac{n}{n-1} \underbrace{{\left( 1-\frac{1}{n^2} \right)}^n}_{
			\geq 1-n \frac{1}{n^2} \text{ (Bern. Ungl.)}}\\
		&\geq \frac{n}{n-1} \left(1-n \frac{1}{n^2} \right) 
		= \frac{n}{n-1} \left( 1-\frac{1}{n} \right) 
		= 1 \Rightarrow f_n \geq f_{n-1} \forall n\geq 2.
	\end{align*}
	\( \Rightarrow \limes{n} f_n \) existiert!
\end{bew}
\begin{defi}[Eulersche Zahl]
	\( e:= \limes{n} {(1+\frac{1}{n})}^n \) (\(\leq 2,75\))
\end{defi}
\begin{bem}\leavevmode
	\begin{enumerate}
		\item Es gilt auch \( \limes{n} {(1+\frac{x}{n})}^n \) existiert. \( \forall x\in\R \) (H.A.)
		\item Alternative Darstellung für \(e\):\\
		Hatten gesehen: \(f_n \leq e_n = \sum_{k=0}^{n} \frac{1}{k!} \quad \forall n \) \\
		\(e_n \leq 2,75, e_{n+1} > e_n \) \\
		\( \Rightarrow \) es existiert \( \limes{n} e_n = \limes{n} \sum_{k=0}^{n} \frac{1}{k!} =: \sum_{k=0}^{\infty} \frac{1}{k!} \) und somit auch \(e = \limes{n} f_n\leq \limes{n} e_n = \sum_{k=0}^{\infty} \frac{1}{k!} \).
	\end{enumerate}
	Beobachtung: 
	\[ f_n = {\left(1+\frac{1}{n}\right)}^n = \sum_{k=0}^{n} \frac{1}{k!} \underbrace{\frac{n}{n}\frac{(n-1)}{n} \ldots\frac{n-k+1}{n}}_{\geq 0} , \quad {\left(1+\frac{1}{n}\right)}^n = \sum_{k=0}^{n} \binom{n}{k} \frac{1}{n^k}. \]
	Nehme \(m\in\N \) fest.
	\[ n\geq m \geq \sum_{k=0}^{m} \frac{1}{k!} 1 \underbrace{\frac{n-1}{n}}_{\overset{n\rightarrow\infty}{\rightarrow} 1} \ldots \underbrace{\frac{n-k+1}{n}}_{\overset{n\rightarrow\infty}{\rightarrow} 1} \]
	Grenzwertsätze \(\Rightarrow \sum_{k=0}^{m} \frac{1}{k!} \) für \(n\rightarrow\infty \).
	\( \Rightarrow \) Für jedes \(m\in\N \) ist
	\[e_m \leq \limes{n} f_n = e \]
	Auch, \(e_m = \sum_{k=0}^{m} \frac{1}{k!} \) hat den Grenzwert \(m\rightarrow\infty \)!
	\[ \overset{\text{Satz 7.1.9.}}{\lim\limits{m\rightarrow\infty}} e_m \leq e. \]
	\[\Rightarrow e = \lim\limits_{m\rightarrow\infty} \sum_{k=0}^{m} \frac{1}{k!} =: \sum_{k=0}^{\infty} \frac{1}{k!} \]
\end{bem}
\begin{satz} 
	\(e\) ist irrational!
\end{satz}
\begin{bew}
	\(e_n = \sum_{k=0}^{n} \frac{1}{k!} \) approximiert \(e\) extrem gut\\
	\[ 0< e-e_n = \sum_{k=0}^{\infty} \frac{1}{k!} - \sum_{k=0}^{n} \frac{1}{k!} = \sum_{k=n+1}^{\infty} \frac{1}{k!}. \]
	\[e-e_n = \limes{n}\left( \sum_{k=0}^{m} \frac{1}{k!} - \sum_{k=0}^{n} \frac{1}{k!} \right) = \limes{m} \left(\sum_{k=n+1}^{n} \frac{1}{k!} \right) \]
	\begin{align*}
	 	(*)\quad k \geq n+1\\
		k! &= k(k-1)\dots 3\cdot 2\cdot 1\\
		&=\underbrace{k(k-1)\dots(n+2)}_{\geq 2^{k-n}}\underbrace{(n+1)n\dots 2\cdot 1}_{=(n+1)!} \geq (n+1)!\\
		&\geq 2^{k-n}(n+1)!
	\end{align*}
	\begin{align*}
		m>n: &\sum_{k=n+1}^{m} \frac{1}{k!}\\
		&\overset{(*)}{\leq} \sum_{k=n+1}^{m} \frac{1}{(n+1)!} {\left(\frac{1}{2}\right)}^{k-(n+1)}\\
		&\leq \frac{1}{(n+1)!} \sum_{k=n+1}^{m} {\left(\frac{1}{2}\right)}^{k-(n+1)}\\
		&= \frac{1}{(n+1)!} \sum_{l=0}^{m-(n+1)} {(\frac{1}{2})}^l = \frac{1}{(n+1)!} \frac{1-{(\frac{1}{2})}^{m-(n-1)}}{1-\frac{1}{2}}\\
		&\leq \frac{1}{(n+1)!} \frac{1}{1-\frac{1}{2}} = \frac{2}{(n+1)!}\\
		&\Rightarrow 0 < e - e_n \leq \frac{2}{(n+1)!} (*) \qquad\forall n\geq 2.
	\end{align*}
	Wäre \(e\) rational, \(\Rightarrow p\in\N,q\in\N:e=\frac{p}{q} \)
	\[ \Rightarrow n! e = n! \frac{p}{q} \in\N \forall n\geq q \]
	Auch: \( n! e_n = n! \sum_{k=0}^{n} \frac{1}{k!} \in\N \) \\
	\( \Rightarrow n!(e-e_n) \in\N_0 \qquad \forall n \), die groß genug sind
	\[ \overset{(*)}{\Rightarrow} 0<n!(e-e_n) \leq \frac{n!2}{(n+1)!} = \frac{2}{n+1} < 1 \quad \forall n\geq 3 \]
	\Lightning{} also ist \(e\) irrational!
\end{bew}
\textbf{Anwendungen:}
\begin{satz}[Invervallschachtelungsprinzip]
	Seien \( a_n\leq b_n, I_n := [a_n,b_n] \) abgeschlossene Intervalle und \[I_{n+1} \subset I_n \quad \forall n\in\N \]
	sowie \( |I_n| := b_n - a_n \overset{n\rightarrow\infty}{\rightarrow} 0 \).\\
	Dann besteht \(\bigcap_{n\in\N} I_n\) aus genau einem Punkt!\\
	Bild: 
	\begin{center}
		\begin{tikzpicture}
			\draw (0,0) -- (8,0);
			\draw (1,0) node {\([\)} node[below=2mm] {\(a_1\)};
			\draw (1.5,0) node {\([\)} node[below=2mm] {\(a_2\)};
			\draw (2.5,0) node {\([\)} node[below=2mm] {\(a_3\)};
			\draw (3,0) node {\(]\)} node[below=2mm] {\(b_1\)} node[above=2mm] {\(I_1\)};
			\draw (5,0) node {\(]\)} node[below=2mm] {\(b_2\)} node[above=2mm] {\(I_2\)};
			\draw (6,0) node {\(]\)} node[below=2mm] {\(b_3\)} node[above=2mm] {\(I_3\)};
		\end{tikzpicture}
	\end{center}
\end{satz}
\begin{bew}\leavevmode
	\begin{enumerate}
		\item \( \bigcap_{n\in\N} I_n \) besteht aus höchstens einem Punkt in \(\R \).\\
		Ang. \( \exists a, a^2 \in \bigcap_{n\in\N} I_n \quad a\neq a^2 \) (o.B.d.A. \(\tilde{a}>a\)).\\
		\[I_{n+1} \subset I_n \quad\forall n \Rightarrow I_n \subset I_{n-1} \subset \ldots\subset I_m \quad \forall n>m.\]
		\[ \Rightarrow \bigcap_{n\in\N} I_n = \{ x\in\R| x\in I_n \forall n\in\N \} \subset \{ x\in\R|x\in I_k \quad \forall 1\leq k \leq m \} = \bigcap_{k=1}^m I_k = I_m \]
		\[ \Rightarrow \{ a,\tilde{a} \} \subset I_m \quad\forall m\in\N. \]
		\[a,\tilde{a} \in I_m = [a_m,b_m]. \]
		\begin{center}
			\begin{tikzpicture}
				\draw (0,0) -- (5,0);
				\draw (1,0) node {\([\)} node[below=2mm] {\(a_m\)};
				\draw (4,0) node {\(]\)} node[below=2mm] {\(b_m\)};
				\draw (2,-0.1) -- (2,0.1) (3,-0.1) -- (3,0.1);
				\draw (2,0) node[below] {\(a\)} (3,0) node[below] {\(\tilde{a}\)};
			\end{tikzpicture}
		\end{center}
		\[\Rightarrow 0<\tilde{a}-a \leq b_m - a_m = |I_m| \rightarrow 0 \quad m\rightarrow\infty \]
		\Lightning{} für \(m\) groß \(\bigcap_{n\in\N} I_n\) hat höchstens ein Element!
		\item \( \bigcap_{n\in\N} I_n \neq \emptyset \qquad I_n = [a_n,b_n] \) \\
		\[I_{n+1} \subset I_n \Leftrightarrow a_{n+1} \geq a_n \wedge b_{n+1} \leq b_n \quad \forall n. \]
		Auch: \( a_n \leq b_n \leq b_{n-1} \leq \cdots\leq b_1 \)
		\( \Rightarrow \) Folge \({(a_n)}_n\) ist nach oben beschränkte monoton wachsende Folge.\\
		\( \overset{\text{Mon.Konv.}}{\Rightarrow} a:= \limes{n} a_n \) existiert und \(a\geq a_n \quad \forall n \).\\
		Sei \(n\geq m\):
		\[a-n\leq b_n\leq\cdots\leq b_m \Rightarrow a = \limes{n} a_n \leq b_m \]
		\[\Rightarrow a_m\leq a_{n-1}\leq a_n\leq a\leq b_m. \]
		\( \Rightarrow a\in I_m\) für jedes \(m\in\N \) \\
		\[  \Rightarrow \{a\}\subset \bigcap_{m\in\N} I_m \]
		\[ \text{d.\ h.\ } \bigcap_{m\in\N} I_m \neq \emptyset \]
	\end{enumerate}
\end{bew}
\begin{satz}[\(k\)-adische Darstellung reeller Zahlen]
	\(k\in\N, k\geq Z\) und \(x\in\R \). Dann gibt es \(z_0\in \Z \) und \(l_j \in \{0,1,\ldots,k-1\} \) derart, dass \(x = z_0 + \limes{n} \sum_{j=1}^{n} l_j k^{-j} = z_0 + \sum_{j=1}^{\infty} l_j k^{-j} \).\\
	\(Z_0 := \lfloor x \rfloor := \min(p\in\Z, p>x) -1 = \max(q\in\Z, q\leq x). \) \\
	\( 0\leq x-\lfloor x\rfloor <1. \) \\
	\(\Rightarrow \) \obda{} sei \( 0\leq x<1\).
	\begin{center}
		\begin{tikzpicture}
			\draw (0,0) -- (8,0);
			\foreach \x in {0,...,8}
			\draw (\x, 0.1) -- (\x, -0.1);
			\draw (3,0) node[above = 1mm, font=\fontsize{6}{0}] {\(l_1 = 3\)};
			\draw (4,0) node[above = 1mm, font=\fontsize{6}{0}] {\(l_1 + 1 = 4\)};
			\draw (0,0) node[below = 2mm] {\(0\)};
			\draw (8,0) node[below = 2mm] {\(1\)};
			\foreach \x in {1,...,9}
			\draw (3.\x, 0.05) -- (3.\x, -0.05);
			\draw (3.5,0) node[below = 2mm] {\(x\)};
		\end{tikzpicture}
	\end{center}
	iteriere diesen Prozess.
	\(\rightarrow \) kriegen 
	\(l_j \in \{0,1,2,\ldots,k-1 \} \) 
	und 
	\[ \sum_{j=1}^{n} l_j k^{-j} 
	\leq x < \sum_{j=1}^{n-1} l_j k^{-j + \frac{l_n + 1}{k^n}} \tag{\(*\)}\]
	\[ \overset{(*)}{\Rightarrow} x = \limes{n} \sum_{j=1}^{n} l_j k^{-j}. \]
\end{satz}
%27.11.2018
\begin{bsp}
	\( p := \lfloor x \rfloor := \max (z\in\Z:z\leq x) \Rightarrow p\leq x < p+1 \Rightarrow \tilde{x} := x - p \in [0,1) \). Also reicht es, \(x\in[0,1)\) zu betrachten!\\
	Bild: \(k=8\) \\
	\begin{center}
		\begin{tikzpicture}
		\draw (0,0) -- (8,0);
		\foreach \x in {1,...,7}
		\draw (\x, 0.1) -- (\x, -0.1);
		\draw (3,0) node[above = 1mm, font=\fontsize{6}{0}] {\(l_1 = 3\)} node[below = 1mm] {\(\frac{3}{8}\)};
		\draw (4,0) node[above = 1mm, font=\fontsize{6}{0}] {\(l_1 + 1 = 4\)} node[below = 1mm] {\(\frac{4}{8}\)};
		\draw (0,0) node {\([\)} node[below = 2mm] {\(0\)};
		\draw (8,0) node {\()\)} node[below = 2mm] {\(1\)};
		\end{tikzpicture}
	\end{center}
	\(l_1 = \lfloor kx \rfloor \)
	\[ \frac{l_1}{k} \leq x < \frac{l_1 + 1}{k} \Rightarrow 0 \leq x - \frac{l_1}{k} < \frac{1}{k} \]
	\[ \Rightarrow 0 \leq k\left( x - \frac{l_1}{k} \right) < 1 \Leftrightarrow 0 \leq k^2 [x-\frac{l_1}{k}] < k. \]
	\[ l_2 := \lfloor k^2\left( x - \frac{l_1}{k} \right) \rfloor \overset{\text{wie vorher}}{\Rightarrow} \frac{l_2}{k} \leq k(x-\frac{l_1}{k}) < \frac{1}{k} \]
	\[ \Rightarrow l_1/k + l_2/k^2 \leq x < l_1/k + \frac{l_2 + 1}{k^2} \]
	induktiv weitermachen. Geg. \( l_1, l_2, \ldots, l_n \in \{ 0,1,2,\ldots,k-1 \} \). 
	\[ \text{mit } l_1/k + l_2/k^2 + \cdots + l_n/k^n \leq x < l_1 /k + l_2/k^2 + l_{n-1}/k^{n-1} + \frac{l_n + 1}{k^n} \]
	\[ \Rightarrow 0\leq x - \sum_{j=1}^{n} \frac{l_j}{k^j} < \frac{1}{k^n} \]
	\[ \text{oder } k^{n+1} [x-\sum_{j=1}^{n} \frac{l_j}{k^j} ] < k\in\N \]
	\[ \text{definiere } l_{n+1} := \lfloor k^{n+1} (x-\sum_{j=1}^{n} \frac{l_j}{k^j} ) \rfloor  \]
	\[ \text{kriegen } a_n := \sum_{j-1}^{n} \frac{l_j}{k^j} \leq x < \sum_{j=1}^{n} l_j/k^j + 1/k^n =: b_n \]
	\[ \Rightarrow a_n \leq x < b_n \]
	\[ \text{und } b_n - a_n = 1/k^n \rightarrow 0, n\rightarrow\infty. \]
	Entweder: \(I_n := [a_n,b_n] \) sind geschachtelt \(I_{n+1} \subset I_n \)
	Lange \(I_n - |I_n| = b_n - a_n \rightarrow 0. \)
	Intervallschachtelungsprinzip \(\Rightarrow \{x\} = \bigcap_{n\in\N} I_n \)
	\( \Rightarrow x= \limes{n} a_n = \limes{n} \sum_{j=1}^{n} l_j/k^j \).\\
	Alternative:\\
	\begin{align*}
		a_{n+1} &\geq a_n\\
		a_n &\leq b_n = \sum_{j=1}^{n} \frac{l_j}{k^j} + \frac{1}{k^n}\\
		&\leq \sum_{j=1}^{n} \frac{k-1}{k^j} + \frac{1}{k^n}\\
		&= (k-1)\underbrace{ \sum_{j=1}^{n} {( \frac{1}{k})}^j + \frac{1}{k^n} }_{= \frac{1}{k} \sum_{j=0}^{n-1} {(\frac{1}{k})}^j = \frac{1}{k} \frac{1-{(\frac{1}{k})}^n}{1-\frac{1}{k}} }\\
		&=\frac{k-1}{k}-\frac{1-{(\frac{1}{k})}^n}{1-\frac{1}{k}} + \frac{1}{k^n}\\
		&= 1- {\left(\frac{1}{k}\right)}^n + \frac{1}{k^n} = 1.\\
		\overundersett{Mon.}{Konv.}{\Rightarrow} a &= \limes{n} a_n \text{ existiert.}\\
	\end{align*}
	Beachte: \(0 \leq x-a_n < \frac{1}{k^n} \rightarrow\overset{n \rightarrow \infty}{\rightarrow} 0\) \\	 
	\begin{align*}
		a_n \leq x < a_n + \frac{1}{k^n} \Rightarrow\overset{\text{Sandwich}}{\Rightarrow}
		a &= \limes{n} a_n \leq x \leq \limes{n} a_n + \limes{n} \frac{1}{k^n}\\
		&= a + 0 = a\\
		&\Rightarrow a \leq x \leq a \Rightarrow a = x
	\end{align*}
\end{bsp}
\begin{bsp}
	\(k\)-adische Darstellung\\
	\(k = 10\) \\
	Behauptung: \(0\text{,}\overline{9} = 1\)
	\begin{align*}
		0\text{,}\overline{9} &= \limes{n} \sum_{j=1}^{n} \frac{9}{10^j}\\
		&=\limes{n} \frac{9}{10}\cdot \sum_{j=0}^{n-1}{\left(\frac{1}{10}\right)}^j\\
		&=\frac{9}{10}\cdot\frac{10}{9} = 1
	\end{align*}
\end{bsp}
\begin{kor}
	Die reellen Zahlen sind überabzählbar!
\end{kor}
\begin{bew}
	Es reicht zu zeigen \( [0,1) \) ist überabzählbar. Es reicht, eine Teilmenge von \(A \subset [0,1) \) anzugeben, die nicht abzählbar ist.\\
	nehmen: \(k=3\)
	\[ A:= \{ x\in[0,1) : \exists l_j \in \{0,1 \}: x = \sum_{j=1}^{\infty} \frac{l_j}{3^j} = \limes{n} \sum_{j=1}^{n}\frac{l_j}{3^j} \}. \]
	\(A\) hat die gleiche Mächtigkeit wie die Menge der \( \{0,1\} \)-wertigen Folgen. Hat dieselbe Mächtigkeit wie \(\mathcal{P}(\N)\). Diese ist überabzählbar.
\end{bew}

\end{document}
\documentclass[../ana1.tex]{subfiles}
\onlyinsubfile{\sectionNumbering} %Use numbering relative to sections and not subsection

\begin{document}
\setcounter{section}{8}

\section{Cauchyfolgen}
\begin{defi}[Cauchyfolge]
	Eine Folge \({(a_n)}_n\) heißt Cauchyfolge (kurz Cauchy), falls
	\[\forall \varepsilon > 0 \,\exists K\in\N: |a_n-a_m| < \varepsilon \, \forall n,m\geq K \]
\end{defi}
\begin{bem}
	Reicht \(m,n\in\N \) mit \( m>n\geq K \) zu betrachten, da Def.\ symmetrisch in \(m,n\) ist und falls \(m=n \Rightarrow a_m-a_n=0\)
\end{bem}
\begin{lem}
	Jede konvergente Folge ist eine Cauchyfolge.
\end{lem}
\begin{bew}
	\( {(a_n)}_n \quad a_n \rightarrow a \) für \(n\rightarrow\infty \) \\
	d.\ h.\  \( \forall \varepsilon > 0 \exists K \in \N: \forall n\geq K \) ist \( |a_n-a| < \varepsilon/2 \).\\
	Ist \(m>n\geq K: |a_m -a_n| = |a_m-a+a-a_n| \leq \underbrace{|a_m-a|}_{<\varepsilon/2} + \underbrace{ |a-a_n| }_{<\varepsilon/2}. \) Also ist \({(a_n)}_n\) Cauchyfolge.
\end{bew}
\begin{lem}
	Jede Cauchyfolge ist beschränkt.
\end{lem}
\begin{bew}
	Sei \(a_n\) Cauchyfolge. 
	\[ \forall \varepsilon > 0 \, \exists K\in\N:|a_m-a_n| < \varepsilon \quad \forall m,n\geq K. \]
	Wähle \(\varepsilon = 1, k_0: |a_m-a_n| < 1 \quad \forall n,m\geq K_0 \) \\
	Sei \(n \geq k_0 \Rightarrow |a_{K_0} - a_n| < 1. \)
	\[ \Rightarrow |a_n| = |a_n - a_{K_0} + a_{K_0}| \leq |a_n - a_{K_0}| + |a_{K_0}| < 1 + |a_{K_0}| \] für alle \(n\geq K_0\) \\
	Also setze: \( C := \max ( |a_1|,|a_2|,\ldots,|a_{K_0}|,1+|a_{K_0}| ) < \infty \) 
	\[ \Rightarrow \forall n\in\N \text{ ist } |a_n| \leq C. \]
\end{bew}
\begin{bsp}
	\begin{align*}
		a_n = {(-1)}^n \text{ (oder } (-1)^n + 1/n \text{)} \\
		n = 2k, k\in\N \Rightarrow a_{2k} = {(-1)}^{2k} = 1 \\
		a_{2k+1} = {(-1)}^{2k+1} = -1.
	\end{align*}
	Die neue Folge \( {(a_{2k})}_{k\in\N} \) ist konstant, also konvergiert sie.
\end{bsp}
\begin{bsp}
	\(B= \{b_1, b_2,\ldots,b_R \} \subset \R \quad R\in\N \).\\
	\( {(a_n)}_n \) Folge mit Werten in \(B\).
	\(a_n \in B \, \forall n\in\N \) \\
	\(B\) endliche Menge!\\
	\( \Rightarrow \) Es gibt mindestens ein \(r_0 \in \{1,2,\ldots,R\} \), sodass \( a_n=b_{r_0}  \) für unendlich viele \(n\).\\
	\[ \Leftrightarrow \forall K\in\N \,\exists n>K : a_n = b_{r_0} \]
	Jetzt induktiv: \\
	\begin{align*}
		n_1 \in\N: a_{n_1} = b_{r_0}.\\
		n_2 := \min (n>n_1:a_n=b_{r_0}) >n_1 \quad a_{n_2} = b_{r_0}\\
		\text{induktiv}\\
		\text{geg. } n_1 < n_2<\cdots<n_k\\
		a_{n_l} = b_{r_0} \quad l = \{1, \ldots,k \} \\
		n_{k+1} = \min(n>n_k:a_n = b_{r_0}) > n_k \text{ und } a_{n_{k+1}} = b_{r_0}.
	\end{align*}
	Erhalte \(n_k\in\N \quad \forall n\in\N \) mit \( n_{k+1} > n_k \forall k\in\N \) und \( a_{n_k} = b_{r_0} \) \\
	\( \Rightarrow \) Folge \( {(a_{n_k})}_{k\in\N} \) die konstant ist.\\
	Außerdem \( {(a_{n_k})}_k \) ist Teil der Folge \( {(a_n)}_n \)!
\end{bsp}
\begin{defi}[Teilfolge]
	Eine Funktion \( \sigma:\N \rightarrow\N \) heißt Ausdünnung, falls \( \sigma (n+1) > \sigma(n) \quad \forall n\in\N \) (d.\ h.\  \(\sigma \) ist streng monoton wachsend).
\end{defi}
Erinnerung: Folge ist eine Funktion \(f: \N\rightarrow X \) \\
\( \sigma : \N\rightarrow\N \Rightarrow f \circ \sigma : \N\rightarrow X, n\mapsto f(\sigma(n))  \) ist auch eine Folge.\\
\(a_n = f(n),\qquad a_{\sigma(n)} = f(\sigma(n)) = (f\circ\sigma)(n) \) \\
Geg. Folge \( {(a_n)}_{n\in\N} \) und eine Ausdünnung \( \sigma : \N\rightarrow \N. \) Setzen wir \( n_k := \sigma (k), k\in\N \) und \( {(a_{\sigma(k)})}_k = {(a_{n_k})}_{k\in\N} \) Teilfolge von \( {(a_n)}_{n\in\N} \).\\
Beobachtung: abgeschlossene und beschränkte Intervalle sind aus Folgensicht fast so gut wie endliche Mengen!
\begin{lem}
	Sei \(I = [b,c], \quad b,c\in\R, b\leq c.\ {(a_n)}_n \subset I \). \Dphp{} \(a_n \in I \quad\forall n\in\N \), dann gibt es eine Teilfolge, von \({(a_n)}_n\), die mit Grenzwert in \(I\) konvergiert.
\end{lem}
\begin{bew}
	\(I_0 = [a,b]\) \\
	Bild:\\
	\begin{tikzpicture}[scale = 8]
		\draw (0,0) -- (1,0);
		\draw (0,0) node {\([\)} node[below = 2mm] {\(b\)};
		\draw (1,0) node {\(]\)} node[below = 2mm] {\(c\)};
		\draw (1/2,0) node {\(]\)} node[below = 2mm] {\(z_1:=\frac{b+c}{2}\)};
		\draw (1/4,0) node {\([\)};
		\draw (3/4,0) node[above] {endl.};
		\draw (3/8,0) node[above] {endl.};
		\draw (1/8,0) node[above] {\(\infty\)};
	\end{tikzpicture}
	\\
	%29.11.2018
	Fallunterscheidung:\\
	1.) Es sind \(\infty \)-viele \( a_n\in I_1 := [b,z_1] \). Dann setze \(b_1 := b, c_1 := z_1, I_1 = I_{1,-} = [b_1,c]\).\\
	2.) Nun endlich viele \(a_n\in I_{1,-} \Rightarrow b_1 := z_1, c_1 := c, I_1 := I_1 := I_{1,+} := [z_1,c] = [b_1,c_1] \Rightarrow \exists \infty \)-viele \(a_n \in I_1\).\\
	\(|I_1| = \) Länge von \(I_1 = c_1-b_1 = \frac{c-b}{2} \). \(n_1 = \min (n\in\N:a_n \in I_1)  \Rightarrow n_1 \geq 1. a_{n_1} \in I_1 \).
	Dann \(z_2 := \frac{b_1 + c_1}{2}\) \\
	\(\Rightarrow \) Sind \(\infty \)-viele \(a_n\in I_{2,-} := [b_1,z_2] \), so setze \(I_2 := I_{2,-}\). Somit sind \(\infty \)-viele \(a_n\in I_{2,+} = [z_1,c_1] \). Setze dann \(I_2 := I_{2,+}\).\\
	\[ n_2 := \min(n>n_1:a_n\in I_2) \Rightarrow a_{n_2} \in I_2 \text{ und } n_2>n_1\geq 1. |I_2| = \frac{|I_1|}{2} = \frac{c-b}{2}. \]
	Iteriere dies: Ang.\ haben \(I_k = [b_k, c_k] \subset I_{k-1} \subset \ldots \subset I_1 \subset I_0 = [b,c]. |I_k| = \frac{|I_0|}{2^k}. \) \\
	\(z_{k+1} := \frac{b_k + c_k}{2} \) Mittelpunkt von \(I_k\).\\
	Sind \(\infty \)-viele \(a_n\) in \(I_{k+1} := [b_k, z_{k+1}] \), so setze \(I_{k+1} := I_{k+1,-} \qquad b_{k+1} = b_k, c_{k+1} = z_k+1 \).\\
	Somit \(I_{k+1} := I_{k+1,+} = [z_{k+1, c_k}]/ b_{k+1} = z_{k+1}, c_{k+1} = c_k. \) \\
	Nach Konstruktion sind \(\infty \)-viele \(a_n\in I_{k+1} \). (d.\ h.\  \(\forall K\in\N \exists n>k:a_n\in I_{k+1} \))\\
	\( n_{k+1} := \min (n>n_k: a_n \in I_{k+1}) > n_k\\
	\Rightarrow \) Folge von Indizes \(n_1 <n_2<\cdots<n_k<n_{k+1}<\ldots \\
	n_k\in\N \) mit \( a_{n_k} \in I_k, |I_k| = \frac{|I_0|}{2^k}\\
	I_k = [b_k, c_k] \)
	\[b_k \leq a_{n_k} \leq c_k \quad \forall k\in\N (*) \]
	\( {(b_k)}_k \) monoton wachsende Folge \(b_k\leq c \forall k \).\\
	\( {(c_k)}_k \) monoton fallende Folge \( c_k \geq b \forall k \).\\
	\( \overset{\text{Mon. Konv.}}{\Rightarrow} \\b= \limes{k} b_k \) exist.\\
	\(c = \limes{k} c_k \) exist.\\
	und \(0 \leq c_k - b_k \leq \frac{c-b}{2^k} \rightarrow0 (k\rightarrow\infty) \).\\
	\(\Rightarrow b=c \overset{(*)}{\Rightarrow} {(a_{n_k})}_k \) konvergiert gegen \(b\).\\
	d.\ h.\  \( {(a_{n_k})}_k \) ist konv.\ Teifolge von \( {(a_n)}_n \).
	%Der unendliche Intervall wird immer halbiert in einen endlichen und unendlichen Teil (rekursiv).
\end{bew}
\begin{kor}
	Jede beschränkte (reelle) Folge hat eine konvergente Teilfolge (Satz von Bolzano-Weierstraß).
\end{kor}
\begin{bew}
	Sei \( 0\leq C < \infty. |a_n| \leq C \quad\forall n\in\N. \) \\
	\[ \Rightarrow -C \leq a_n \leq C, a_n \in [-C, C] \quad\forall n\in\N. \]
	\( \Rightarrow \) Beh.\ folgt aus Lemma 5!
\end{bew}
\begin{kor}
	Jede Cauchy-Folge hat eine konv. Teilfolge.
\end{kor}
\begin{bew}
	Nach Lemma 3 ist \( {(a_n)}_n \) beschränkt. Wende Korrolar 6 an.
\end{bew}
Hauptbeobachtung:
\begin{lem}
	Sei \({(a_n)}_n\) eine Cauchyfolge. Dann gilt
	\[ {(a_n)}_n \text{ konvergiert } \Leftrightarrow {(a_n)}_n \text{ hat eine konvergente Teilfolge.} \]
\end{lem}
\begin{bem}
	Ist \( {(a_n)}_n \) konvergente Folge, so konvergiert jede Teilfolge gegen den gleichen Grenzwert von \({(a_n)}_n\).
\end{bem}
\begin{bem}
	\( {(a_n)}_n \) Cauchyfolge \\
	\begin{align*}
		&\Leftrightarrow \forall \varepsilon>0 \,\exists K\in\N:|a_m-a_n|<\varepsilon \quad \forall m>n\geq K.\\
		&\Leftrightarrow \forall \varepsilon>0 \,\exists K\in\N:|a_{n+p}-a_n|<\varepsilon \quad \forall n\geq K, p\in\N \\
		&\Leftrightarrow \forall \varepsilon>0 \,\exists K\in\N:\sup \underbrace{\set{|a_{n+p}-a_n|}}_{\substack{\text{muss gleichm.}\\\text{in }p\in\N\text{ klein sein}\\ \text{für } n \text{ groß} }}<\varepsilon \quad \forall n\geq K, p\in\N.
	\end{align*}
\end{bem}
\begin{bew}
	\glqq{}\( \Rightarrow \)\grqq{}: Nach Bem 1 konvergiert jede Teilfolge von \( {(a_n)}_n \) gegen denselben Grenzwert.\\
	\glqq{}\(\Leftarrow \)\grqq{}: Ang. Teilfolge \( {(a_{n_k})}_k \) konvergiert gegen \(L\in\R \). \(L = \limes{k} a_{n_k} \) existiert.\\
	\( n_k\in\N, n_1<n_2<\cdots<n_k<n_{k+1}<\ldots \) \\
	Auch: \({(a_n)}_n\) ist Cauchy, d.\ h.\ 
	\[ \forall \varepsilon>0\exists K\in\N:|a_m-a_n|<\varepsilon \quad\forall m>n\geq K (*). \]
	\[n_1 \geq 1 \Rightarrow n_2 > n_1 \geq 1 \Rightarrow n_2 \geq 2 \ldots n_k \geq k \forall k \]
	D.\ h.\ ist \(k>n \Rightarrow m=n_k \geq k>n \overset{(*)}{\Rightarrow} |a_{n_k}-a_n|<\varepsilon \quad \forall k>n\geq K \).\\
	Sei \( \varepsilon>0, K\) gegeben\\
	\[ \Rightarrow |a_{n_k} - a_n| <\varepsilon \quad \forall k>n\geq K \Rightarrow \limes{k} |a_{n_k}-a_n| = \limes{k} |a_{\sigma(k)}-a_n| = |L-a_n| \leq \varepsilon \]
	\(\sigma(k) = n_k, \limes{k} a_{\sigma(k)} = L \)
	\[ \forall \varepsilon>0 \exists K\in\N: |L-a_n|\leq \varepsilon \forall n\geq K \]
	\(\Rightarrow {(a_n)}_n\) konvergiert gegen \(L\)!
\end{bew}
\begin{satz}
	Eine (reelle) Folge \({(a_n)}_n\) konvergiert \( \Leftrightarrow {(a_n)}_n \) ist Cauchyfolge.
\end{satz}
\begin{bew}
	\glqq{}\(\Rightarrow \)\grqq{} ist Lemma 2.\\
	\glqq{}\( \Leftarrow \)\grqq{} \({(a_n)}_n\) Cauchy \( \overset{\text{Kor. 7}}{\Rightarrow} \exists \) konv. Teilfolge von \({(a_n)}_n \overset{\text{Lem. 8}}{\Rightarrow} {(a_n)}_n \) konvergiert. 
\end{bew}
z. B. \({(a_n)}_n\) Cauchyfolge. \(\Rightarrow \exists \) Teilfolge mit \(|a_{n_{k+1}}-a_{n_k}| <2^{-k}\).

\end{document}
\documentclass[../ana1.tex]{subfiles}
\onlyinsubfile{\sectionNumbering} %Use numbering relative to sections and not subsection

\begin{document}
\setcounter{section}{9}

\section{Häufungspunkte, Limes Superior, Limes Inferior}
\( {(a_n)}_n \) konvergente Folge \(a= \limes{n} a_n \)
\[\Leftrightarrow\forall\varepsilon>0 \text{ ist } a_n\in U_\varepsilon(a) = (a-\varepsilon, a+\varepsilon) \text{ für fast alle } n\in\N. \]
\begin{defi}
	Sei \({(a_n)}_n\) Folge. \(a\in\R \) Häufungspunkt von \({(a_n)}_n\), falls \(\forall\varepsilon>0 a_n\in U_\varepsilon(a) \) für unendlich viele \(n\in\N \) (d.\ h.\  \( \forall K\in\N \exists n>K:a_n \in U_\varepsilon (a) \))
\end{defi}
\begin{bem}
	Eine konvergente Folge hat nur einen Häufungspunkt.\\
	Bild:\\
	\begin{tikzpicture}[scale = 4]
		\draw (0,0) -- (1,0);
		\draw (1/4, 1/16) -- (1/4, -1/16) node[below] {\(a\)};
		\draw (3/4, 1/16) -- (3/4, -1/16) node[below] {\(\tilde{a}\)};
		\draw (1/2,0) node {\( )( \)};
		\draw (0,0) node {\((\)};
		\draw (1,0) node {\()\)};
	\end{tikzpicture}
\end{bem}
\( H({(a_n)}_n) = H((a_n)) :=  \) Menge der Häufungspunkte von \({(a_n)}_n\).
\begin{lem}
	\begin{enumerate}
		\item Ist \({(a_n)}_n\) beschränkt. So ist \(H({(a_n)}_n) \neq \emptyset \).
		\item Konvergiert \({(a_n)}_n\) gegen \(a\), so ist \(H({(a_n)}_n) = \{a\} \).
	\end{enumerate}
\end{lem}
\begin{bew}
	\begin{enumerate}
		\item Nach Kor. 9.6 hat \({(a_n)}_n\) konvergente Teilfolge \({(a_{n_k})}_k, L= \limes{k} a_{n_k} \). Dann ist \( \forall \varepsilon >0  \)
		\begin{align*}
			a_{n_k} \in U_\varepsilon(L) \text{ für fast alle }k\in\N \\
			\Rightarrow a_{n_k} \in U_\varepsilon(L) \text{ für unendlich viele } n\in\N \\
			\Rightarrow a_n\in U_\varepsilon(L)\\
			\Rightarrow L \text{ ist Häufungspunkt}.
		\end{align*}
		\item Sind \( \tilde{a}>a\) Häufungspunkt, \(a=\limes{n}a_n \quad \varepsilon = \frac{\tilde{a}-a}{2} \)
		Bild:\\
		\begin{tikzpicture}[scale = 4]
			\draw (0,0) -- (1,0);
			\draw (1/4, 1/16) -- (1/4, -1/16) node[below] {\(a\)};
			\draw (3/4, 1/16) -- (3/4, -1/16) node[below] {\(\tilde{a}\)};
			\draw (1/2,0) node {\( )( \)};
			\draw (0,0) node {\((\)};
			\draw (1,0) node {\()\)};
		\end{tikzpicture}
		\\
		\( \Rightarrow \) fast alle \(a_n \in U_\varepsilon(a), U_\varepsilon(a) \cap U_\varepsilon(\tilde{a}) = \emptyset \) \\
		\( \Rightarrow \) nur endlich viele \(a_n\) kann in \(U_\varepsilon(\tilde{a})\) sein.
	\end{enumerate}
\end{bew}
\begin{satz}
	\[ b\in H({(a_n)}_n) \Leftrightarrow {(a_n)}_n \text{ hat Teilfolge } {(a_{n_k})}_k \text{, die gegen } b \text{ konvergiert}. \]
\end{satz}
\begin{bew}
	\glqq{}\(\Rightarrow \)\grqq{}: Ist \(\lim a_{n_k} = b \Leftrightarrow \forall \varepsilon >0 \) ist \( a_{n_k} \in U_\varepsilon(b) \) für fast alle \(k\).\\
	\( \Rightarrow a_n \in U_\varepsilon(b) \) für unendlich viele \(n\).\\
	\glqq{}\(\Leftarrow \)\grqq{}: Sei \(b \in H((a_n)). \Rightarrow \forall L\in \N \) gibt es \(\infty \)-viele \( a_n \in (b-1/L, b+1/L) (*) \).\\
	\begin{align*}
		L = 1 &\overset{*}{\Rightarrow} \exists n_1 \in \N: a_{n1} \in (b-1, b+1)\\
		L = 2 &\overset{*}{\Rightarrow} \exists n_2 > n_1 \in \N: a_{n2} \in \left(b-\frac{1}{2}, b+\frac{1}{2}\right)\\
		&\vdots \text{ induktiv}\\
		L = k &\overset{*}{\Rightarrow} \exists n_k > n_{k-1} \in \N: a_{nk} \in \left(b-\frac{1}{k}, b+\frac{1}{k}\right)\\
		&\Rightarrow \text{Teilfolge } {(a_{nk})}_k \text{ mit } b - \frac{1}{k} < a_{nk} < b + \frac{1}{k} \forall k \in \N \\
		&\overset{\text{Sandwich}}{\Rightarrow} \limes{n} a_{nk} = b
	\end{align*}
\end{bew}
Offensichtliche Frage: Was ist \( \sup H({(a_n)}_n) \) (größter Häufungspunkt) und \(  \inf H({(a_n)}_n) \) (kleinster Häufungspunkt)?\\
Situation: \({(a_n)}_n\) reelle Folge\\
Geg. \(n\in\N \quad d_n := \underset{k\geq n}{\sup}a_k \in \R \Leftrightarrow {(a_n)}_n\) ist nach oben beschränkt. (somit setze \(d_n = \sup a_k = -\infty \).\\
\( d_{n+1} = \underset{k\geq n+1}{\sup}a_k \leq \underset{k\geq n}{\sup}a_k = d_n \Longrightarrow {(d_n)}_n \) ist monoton fallend.\\
\( \Rightarrow \) ist \({(d_n)}_n\) nach unten beschränkt (\( \Leftarrow {(a_n)}_n \) ist nach unten beschränkt), dann existiert \( \limes{n} d_n \) (Satz von mon. Konv.)\\
\[\text{Setzen } \limsup\limits(n\rightarrow\infty) a_n := \limes{n} d_n = \limes{n} \underset{k\geq n}{\sup} a_k. \]
\[ A\subset B\subset \R \Rightarrow \sup A \leq \sup B \]
\[ d_n - \underset{k\geq n}{\sup} a_k = \max(a_n, \underbrace{\underset{k\geq n+1}{\sup} a_k}_{d_{n+1}}) \]
Genauso: \(b_n = \underset{k\geq n}{\inf} a_k \)
\[ b_{n+1} = \underset{k\geq n+1}{\inf} a_k \geq \underset{k\geq n}{\inf} a_k = b_n \]
\[ \Rightarrow \liminf\limits_{n\rightarrow\infty} a_n := \limes{n} \underset{k\geq n}{\inf} a_k = \underset{n\in\N}{\sup} \underset{k\geq n}{\inf} a_k \]
\[ \limsup a_n = \underset{n\in\N}{\inf} \underset{k\geq n}{\sup} a_k \]
Haben immer \(b_n \leq d_n\)
\[ \Rightarrow \liminf\limits_{n\rightarrow\infty} a_n \leq \limsup\limits_{n\rightarrow\infty} a_n. \]
%04.12.2018
\begin{align*}
	{(a_n)}_n &b_n = \underset{k\geq n}{\sup} a_k = \sup \{ a_k | k\geq n \} \\
	&d_n = \underset{k\geq n}{\inf} a_k = \inf \{ a_k | k\geq n \} \\
	\Rightarrow b_{n+1} \leq b_n, d_{n+1} \geq d_n\\
	\overset{\text{Mon. Konv.}}{\Rightarrow} &\limes{n} b_n \text{ existiert in }\R \cup \{-\infty, \infty \} \\
	&\limes{n} d_n \text{ existiert in }\R \cup \{-\infty, \infty \}
\end{align*}
(\({(a_n)}_n\) nach oben beschränkt \(\Leftrightarrow \limes{n} b_n < \infty \) \\
\({(a_n)}_n\) nach unten beschränkt \(\Leftrightarrow \limes{n} d_n > -\infty \))
\begin{defi}
	Geg.\ reelle Folge \({(a_n)}_n\).
	\begin{align*}
		\limsup\limits_{n\rightarrow\infty} a_n := \limes{n}\underset{k\geq n}{\sup} a_k \text{ (}=\inf\underset{k\geq{}n}{\sup}a_k \text{)}\\
		\liminf\limits_{n\rightarrow\infty} a_n := \limes{n}\underset{k\geq n}{\inf} a_k \text{ (}=\sup\underset{k\geq{}n}{\inf}a_k \text{)}
	\end{align*}
\end{defi}
\begin{lem}
	\begin{enumerate}
		\item Es gilt immer \( \liminf\limits_{n\rightarrow\infty}a_n \leq \limsup\limits_{n\rightarrow\infty}a_n \)
		\item \({(a_n)}_n\) beschränkte Folge: Dann gilt \({(a_n)}_n\) konvergiert \( \Leftrightarrow \liminf\limits_{n\rightarrow\infty}a_n = \limsup\limits_{n\rightarrow\infty}a_n \Leftrightarrow \liminf\limits_{n\rightarrow\infty}a_n \geq \limsup\limits_{n\rightarrow\infty}a_n \).
	\end{enumerate}
\end{lem}
\begin{bew}
	\begin{enumerate}
		\item Sicherlich \( \underset{k\geq n}{\inf}a_k \leq \underset{k\geq n}{\sup} a_k \Rightarrow \) Beh. \checkmark{}
		\item \glqq{}\(\Rightarrow \)\grqq{}: Sei \(a:= \limes{n}a_n\in\R \), d.\ h.\  \( \forall \varepsilon > 0 \exists M\in\N: a-\varepsilon < a_k < a+ \varepsilon \) für alle \(k \geq M \).\\
		\( \Rightarrow \) Sei \(n\geq M\) 
		\[ \underset{k\geq n}{\sup} a_k \leq a+\varepsilon \]
		\[ \underset{k\geq n}{\inf} a_k \geq a-\varepsilon \]
		\[ \Rightarrow a-\varepsilon \leq \underbrace{\limes{n}\underset{k\geq n}{\inf}a_k}_{=\liminf a_n} \leq \underbrace{\limes{n}\underset{k\geq n}{\sup}a_k}_{=\limsup a_n} \leq a+\varepsilon \text{ für jedes } \varepsilon>0. \]
		\[ a\leq \liminf a_n \leq \limsup a_n \leq a_n \checkmark. \]
		\glqq{}\(\Leftarrow \)\grqq{}: Ang.\  \( \liminf a_n = \limsup a_n =: a \) (d.\ h.\  \(b_n =  \underset{k\geq n}{\sup} a_k \rightarrow a \) für \( n\rightarrow\infty \).)\\
		d.\ h.\  \( \forall \varepsilon>0: \exists M_1,M_2 \in\N: \)
		\begin{align*}
			a-\varepsilon < \underset{k\geq n}{\sup} a_k < a + \varepsilon \quad \forall n\geq M_1\\
			a-\varepsilon < \underset{k\geq n}{\inf} a_k < a + \varepsilon \quad \forall n\geq M_2
		\end{align*}
		Setze \(M := \max(M_1, M_2) \) \\
		\begin{align*}
			\Rightarrow a-\varepsilon < \underset{k\geq n}{\inf} a_k \leq a_n \leq \underset{k\geq n}{\sup} a_k < a + \varepsilon \quad \forall n\geq M\\
			\Rightarrow \text{ fast alle } a_n\in(a-\varepsilon,a+\varepsilon)\text{, d.\ h.\ }a_n\rightarrow a \text{ für } n\rightarrow \infty.
		\end{align*}
	\end{enumerate}
\end{bew}
\begin{satz}
	Sei \({(a_n)}_n\) beschränkte Folge. Dann gilt:\\
	\(a^* = \limsup\limits_{n\rightarrow\infty}a_n \Leftrightarrow \)
	\begin{enumerate}
		\item \(\forall a> a^* \) ist \( \{n\in\N |a_n>a \} \) endliche Teilmenge von \(\N \).
		\item \( \forall a<a^* \) gibt es \(\infty \)-viele \(a_n >a\). (d.\ h.\  \( \{ n\in\N|a_n>a \} \) ist unendliche Teilmenge von \(\N \)).
	\end{enumerate}
	genauso für \( \liminf \):
	\(a_* = \liminf\limits_{n\rightarrow\infty}a_n \Leftrightarrow \)
	\begin{enumerate}
		\item \(\forall a< a_* \) ist \( \{n\in\N |a_n<a \} \) endliche Teilmenge von \(\N \).
		\item \( \forall a>a_* \) gibt es \(\infty \)-viele \(a_n<a\).
	\end{enumerate}
\end{satz}
\begin{bew}
	Nur für \( \limsup a_n \), denn \( \limsup\limits_{n\rightarrow\infty}(-a_n) = -\liminf\limits_{n\rightarrow\infty} a_n \) \\
	\glqq{}\(\Rightarrow \)\grqq{}: \(b_n = \underset{n\geq k}{\sup}a_k \rightarrow a^* \), d.\ h.\  \(\forall \varepsilon>0\exists M\in\N: a^*-\varepsilon < b_n < a^* + \varepsilon \quad \forall n\geq M. \) \\
	\[ \Rightarrow a_n \leq b_n < a + \varepsilon \quad n\geq M. \]
	Sei \(  a>a^*, \varepsilon := a-a^* >0 \Rightarrow a_n < a^* + \varepsilon = a \quad \forall n\geq M. \Rightarrow \) höchstens endlich viele \( a_n >a^* \Rightarrow 1. \) \\
	Sei \( a< a^*, \varepsilon := a^*-a >0 \Rightarrow a=a^*-\varepsilon < \underbrace{b_n}_{=\underset{k\geq n}{\sup} a_k} \quad \forall n\geq M. \) \\
	\[ \Rightarrow a < \underset{k\geq n}{\sup} a_k \quad \forall n\geq M. \]
	\(a=a^* - \varepsilon \).
	Ang.: Fast alle \(a_n \leq a \) (=Negierung von \( \infty \)-viele \(a_n>a\)).\\
	\[ \Rightarrow b_K = \underset{k\geq K}{\sup} a_k \leq a. \]
	\[ \Rightarrow L = \max(M,K) \Rightarrow b_n \leq b_K \leq a \quad \forall n\geq L \text{ und } a<b_n=\underset{k\geq n}{\sup}a_k \forall n\geq L \text{\Lightning} \]
	\( \Rightarrow \) es gibt \(\infty \)-viele \(a_n>a\)!\\
	\glqq{}\(\Leftarrow \)\grqq{}: Sei \(a>a^* \overset{\text{1.}}{\Rightarrow} a_n \leq a \) für fast alle \(n\in\N \). \\
	\( \Rightarrow b_n = \underset{k\geq n}{\sup} a_n \leq a \) für fast alle \(n\in\N \). \\
	\[ \Rightarrow \limsup\limits_{n\rightarrow\infty} a_n = \lim b_n \leq a \quad \forall a> a^* \]
	\[ \limsup\limits_{n\rightarrow\infty} a_n \leq a^*. \]
	Sei \(a<a^* \overset{\text{2.}}{\Rightarrow} \exists \infty \)-viele \(a_n > a \Rightarrow \underset{k\geq n}{\sup} a_k > a \) für jedes \(n\in\N \) 
	\[ \Rightarrow \limes{n}\underset{k\geq n}{\sup}a_k \geq a \text{ für jedes }a<a^*) \]
	\[ \Rightarrow \limsup\limits_{n\rightarrow\infty} a_n\geq a^*. \]
\end{bew}
\begin{kor}
	Sei \({(a_n)}_n\) beschränkte Folge.
	\[ \Rightarrow H({(a_n)}_n) \subset [ \liminf\limits_{n\rightarrow\infty}a_n, \limsup\limits_{n\rightarrow\infty}a_n ] \text{ und } \limsup a_n, \liminf a_n \in H({(a_n)}_n). \]
\end{kor}
\begin{bew}
	Übung
\end{bew}
\(a_n \rightarrow a \Leftrightarrow \limsup\limits_{n\rightarrow\infty}|a_n - a| = 0 \)

\end{document}

\documentclass[../ana1.tex]{subfiles}
\onlyinsubfile{\sectionNumbering} %Use numbering relative to sections and not subsection

\begin{document}
\setcounter{section}{10}

\section{Konvergente Reihen, Teil 1}
hatten: endl. Summen: \[ a_1+a_2+\cdots+a_n = \sum_{k=1}^{n}a_k \]
Ziel: unendl. Summen: \[ a_1+a_2+a_3+\cdots = \sum_{k=1}^{\infty}a_k = \sum_{n=1}^{\infty}a_n \]
Was soll das sein?
\begin{defi}
	Dieses Symbol \[ \sum_{n=1}^{\infty}a_n \] steht für die Folge \({(S_n)}_n\) der Partialsummen: \[S_n := \sum_{k=1}^{n}a_k .\]
	Die Reihe \( \sum_{k=1}^{\infty}a_k \) konvergiert, falls die Folge ihrer Partialsummen konvergiert. In diesem Fall setzen wir 
	\[ \sum_{k=1}^{\infty}a_k := \limes{n} \sum_{k=1}^{\infty}a_k = \limes{n} S_n \]
	Sagen auch die Reihe \( \sum_{n=1}^{\infty}a_n \) konvergiert. Die Reihe \( \sum_{n=1}^{\infty}a_n \) divergiert, falls \( {(S_n)}_n \) divergiert. Reihe \( \sum_{n=1}^{\infty}a_n \) heißt bestimmt divergent, falls \({(S_n)}_n\) bestimmt gegen \(+ \infty \) oder \(-\infty \) divergiert. Setzen dann \( \sum_{n=1}^{\infty}a_n =-\infty \), falls \(\limes{n} \sum_{k=1}^{n}a_k = -\infty \). \( \sum_{n=1}^{\infty}a_n = +\infty \), falls \( \limes{n}\sum_{k=1}^{n}a_k=+\infty \)
\end{defi}
\begin{bem}
	Sei \({(b_n)}_n\) eine Folge. \(a_1 = b_1, a_n = b_n-b_{n-1}, n\geq 2 \Rightarrow \sum_{k=1}^{n}a_k = b_n \).\\
	Auch Reihen der Form \( \sum_{n=0}^{\infty}a_n \) oder \( \sum_{n=42}^{\infty}a_n\), (\(v\in\Z \)) \(\sum_{n=v}^{\infty}a_n\).
\end{bem}
Beobachtung: Sind \(a_n \geq 0 \Rightarrow S_n = \sum_{k=1}^{\infty}a_k \) monoton wachsend in \(n\).\\
\( \overset{\text{Mon. Konv.}}{\Rightarrow}\) \\
entweder \({(S_n)}_n\) ist nach oben beschränkt \( \Rightarrow \limes{n} S_n \in [0,\infty) \) \\
oder \({(S_n)}_n\) ist nach oben beschränkt \( \Rightarrow \limes{n} S_n  = +\infty \) \\
\( \Rightarrow \)
\begin{satz}
	Sind \(a_n \geq 0, n\in\N \), dann gilt\\
	entweder ist \(S_n = \sum_{k=1}^{n}a_n\) nach oben beschränkt und dann ist \[ \sum_{k=1}^{\infty}a_k = \limes{n}S_n \in [0,\infty) \]
	oder \(S_n \rightarrow \infty, n\rightarrow\infty \) und dann ist \[ \sum_{k=1}^{\infty}a_k = \limes{n}S_n = +\infty. \]
\end{satz}
\begin{bew}
	Oben.
\end{bew}
\begin{kor}
	Sind \(a_n \geq 0, n\in \N \), so ist
	\begin{align*}
		\text{entweder } &\sum_{n=1}^{\infty}a_n < \infty &\text{in diesem Fall ist die Reihe nach oben beschränkt.}\\
		\text{oder } &\sum_{n=1}^{\infty}a_n = +\infty &\text{nach oben unbeschränkt.}
	\end{align*}
\end{kor}
\begin{satz}[Cauchy-Kriterium für Reihen]
	Seien \(a_n \in\R, n\in\N \). Dann gilt: 
	\[ \sum_{n=1}^{\infty} a_n \text{ konvergiert } \Leftrightarrow \forall \varepsilon >0 \exists K \in \N : | \sum_{j=n+1}^{m}a_j|<\varepsilon \quad \forall m>n\geq K.\]
\end{satz}
\begin{bew}
	Reihe konvergiert per Def.\ genau dann, wenn Folge der Partialsummen \( {(S_n)}_n, S_n = \sum_{k=1}^{n}a_k \) konvergiert.
	\[ \overset{\text{Cauchy-Krit.\ Folgen}}{\Longleftrightarrow} \forall \varepsilon > 0 \exists K\in\N: |\underbrace{S_m-S_n}_{=\sum_{k=1}^{m}a_k-\sum_{k=1}^{n}a_k=\sum_{k=n+1}^{m}a_k}|<\varepsilon \quad \forall m>n\geq K. \]
\end{bew}
%06.12.2018
\begin{bew}
	Partialsummen \({(S_n)}_n\)
	\[ S_n := \sum_{j=1}^{n} a_j \]
	Cauchy-Kriterium  für Folgen:
	\[ \Rightarrow {(S_n)}_n \text{ konvergiert } \Leftrightarrow \forall \varepsilon > 0 \exists K\in\N:|S_m-S_n|<\varepsilon \quad \forall m>n\geq K. \]
	\[ S_m - S_n = \sum_{j=1}^{m}a_j - \sum_{j=1}^{n} a_j = \sum_{j=n+1}^{m} a_j. \]
	D. h. \({(S_n)}_n\) konvergiert \[ \Leftrightarrow \forall \varepsilon > 0 \exists K\in \: \left|\sum_{j=n+1}^{m}a_j\right|<\varepsilon \quad \forall m>n\geq K. \]
\end{bew}
\begin{bsp}[Geometrische Reihe]
	Sei \(|q|<1\).
	\[ \Rightarrow \sum_{n=0}^{\infty} q^n = \frac{1}{1-q} \]
\end{bsp}
\begin{bsp}
	\[S_n = \sum_{j=0}^{n} q^n = \frac{1-q^{n+1}}{1-q} \rightarrow \frac{1}{1-q} \text{ (da } \limes{n}q^{n+1}=0 \text{)} \]
\end{bsp}
\begin{bsp}
	\[\sum_{n=2}^{\infty}\frac{1}{n(n-1)}=1 \text{ (teleskopierende Summe)} \]
\end{bsp}
\begin{bew}
	\begin{align*}
		\frac{1}{n(n-1)} =\frac{1}{n-1} - \frac{1}{n} \quad\forall n\geq 2.\\
		\Rightarrow S_n &= \sum_{j=2}^{n} \frac{1}{j(j-1)} = \sum_{j=2}^{n} \left( \frac{1}{j-1} - \frac{1}{j} \right)\\
		&= \sum_{j=2}^{n} \frac{1}{j-1} - \sum_{j=2}^{n} \frac{1}{j} = \sum_{j=1}^{n-1}\frac{1}{j} - \sum_{j=2}^{n}\frac{1}{j}\\
		&= 1 - \frac{1}{n} \rightarrow 1, n\rightarrow \infty
	\end{align*}
\end{bew}
\begin{bsp}
	\[ \sum_{n=1}^{\infty} \frac{1}{n^2} \]
\end{bsp}
\begin{bew}
	\begin{align*}
		S_n = \sum_{j=1}^{n}\frac{1}{j^2} = 1 + \sum_{j=2}^{n} \frac{1}{j^2}\\
		\frac{1}{j^2} \leq \frac{1}{j(j-1)} \quad \forall j\geq 2.\\
		\Rightarrow S_n = 1 + \sum_{j=2}^{n} \frac{1}{j^2}.
	\end{align*}
	Also folgt aus \( \leq 1 + \sum_{j=2}^{n} \frac{1}{j(j-1)} = 1 + 1 - \frac{1}{n} \).\\
	Monotone Konvergenz, da \({(S_n)}_n\) konvergiert, also konvergiert \(\sum_{j=1}^{\infty} \frac{1}{j^2} = 2 - \frac{1}{n} \leq 2 \quad\forall n \).
\end{bew}
\begin{kor}
	Wenn die Reihe \(\sum_{n=1}^{\infty} a_n\) konvergent, so ist \({(a_n)}_n\) eine Nullfolge.
\end{kor}
\begin{bew}
	Satz 4 \( \Rightarrow \forall \varepsilon >0 \exists K\in\N:|\sum_{j=n+1}^{m}a_j| < \varepsilon \quad \forall m>n\geq K \).\\
	Setze \[ m=n+1 : \sum_{j=n+1}^{n+1} a_j = a_{n+1} \Rightarrow |a_{n+1}|<\varepsilon \quad \forall n\geq K. \]
	\[ \Rightarrow \limes{n}a_{n+1}=0=\limes{n}a_n. \]
\end{bew}
\begin{bem}
	Warnung: die Umkehrung gilt nicht!
\end{bem}
\begin{bsp}[harmonische Reihe]
	%BEGIN ALIGN MACHEN TODO
	\[ \sum_{n=1}^{\infty} \frac{1}{n} \text{ divergiert, obwohl } \frac{1}{n} \rightarrow 0. \]
	\[ S_n = \sum_{j=1}^{n}\frac{1}{j} \]
	\[ S_m -S_n \text{ wähle }m=2n \]
	\[ S_{2n} - S_n = \sum_{j=1}^{2n}\frac{1}{j} - \sum_{j=1}^{n}\frac{1}{j} = \sum_{j=n+1}^{2n}\frac{1}{j} \]
	\[ = \frac{1}{n+1} + \frac{1}{n+2} + \cdots + \frac{1}{2n} \]
	\[ \leq \frac{1}{2n} + \frac{1}{2n} + \cdots + \frac{1}{2n} \]
	\[ = \frac{1}{2} \]
	\[ \Rightarrow |S_{2n}-S_n| = S_{2n}-S_n \rightarrow \frac{1}{2} \quad\forall n>1 \]
	also kann \({(S_n)}_n\) nicht konvergieren!
	\[ \Rightarrow \sum_{n=1}^{\infty} \frac{1}{n} = +\infty. \]
\end{bsp}
\begin{satz}
	Gilt \(0\leq a_n \leq b_n \quad \forall n\in\N \), so folgt:
	\begin{enumerate}
		\item Ist \( \sum_{n=1}^{\infty}b_n \) konvergent, so konvergiert auch \( \sum_{n=1}^{\infty}a_n \) und \( \sum_{n=1}^{\infty} a_n \leq \sum_{n=1}^{\infty}b_n \).
		\item Divergiert \( \sum_{n=1}^{\infty}a_n \), so divergiert auch \( \sum_{n=1}^{\infty}b_n \) (gegen \(+\infty \)).
	\end{enumerate}
\end{satz}
\begin{bew}
	\begin{align*}
		S_n = \sum_{j=1}^{n}a_j\\
		t_n = \sum_{j=1}^{n}b_j\\
		\Rightarrow S_n \leq t_n \quad \forall n\in\N \\
		t_{n+1}\geq t_n, S_{n+1}\geq S_n
	\end{align*}
	\begin{enumerate}
		\item Ist \( \sum_{n=1}^{\infty}b_n \) konvergent, so konvergiert \({(t_n)}_n\).\\
		\[ t:= \limes{n}t_n\in [0,\infty) \Rightarrow t_n \leq t. \]
		\[ \Rightarrow S_n \leq t_n \leq t \Rightarrow {(S_n)}_n \text{ nach oben beschränkt.} \]
		\[ \overset{\text{Mon. Konv.}}{\Rightarrow} s = \limes{n} S_n \text{ existiert und } \underbrace{S}_{= \sum_{n=1}^{\infty}a_n} \leq t = \sum_{n=1}^{\infty} b_n. \]
		\item Aus 1.\ folgt ist \( \sum_{n=1}^{\infty}a_n =\infty \) divergent, so muss auch \( \sum_{n=1}^{\infty}b_n = \infty \).
	\end{enumerate}
\end{bew}
\begin{satz}
	Sind \( \sum_{n=1}^{\infty}a_n, \sum_{n=1}^{\infty}b_n \) konvergente (reelle Reihen), so konvergiert auch \( \sum_{n=1}^{\infty}(\lambda a_n + \mu b_n) \quad \forall \lambda, \mu \in\R \) und 
	\[ \sum_{n=1}^{\infty} (\lambda a_n + \mu b_n) = \lambda \sum_{n=1}^{\infty}a_n + \mu \sum_{n=1}^{\infty}b_n. \]
\end{satz}
\begin{bew}
	\[ S_n = \sum_{j=1}^{n}a_j, t_n = \sum_{j=1}^{n}b_j, S = \limes{n}S_n, t = \limes{n}t_n \]
	\[ \sum_{j=1}^{n}(\lambda a_n + \mu b_n) = \lambda\sum_{j=1}^{n}a_j + \mu \sum_{j=1}^{n}b_j = \lambda S_n + \mu t_n \rightarrow \lambda S + \mu t, n\rightarrow\infty. \]
\end{bew}
%11.12.2018
\begin{satz}[Cauchyscher Verdichtungssatz]
	Sei \({(a_n)}_n\) monoton fallende Nullfolge. Dann gilt
	\[\sum_{n=1}^{\infty} a_n \text{ konvergiert }\]
	\[\Leftrightarrow \text{ die \glqq{}verdichtete\grqq{} Reihe } \sum_{n=0}^{\infty}2^n a_{2^n} = a_1 + 2a_2 + 4a_4 + 8 a_8 + \dots \text{ konvergiert}.\]
\end{satz}
\begin{bew}
	\( a_{n+1} \leq a_n \quad\forall n, a_n\rightarrow0 \quad n \rightarrow\infty \Rightarrow a_n \geq 0 \quad\forall n\in\N \).
	\[ S_n := \sum_{j=1}^{n}a_j, t_n := \sum_{j=0}^{K}2^j a_{2^j} \text{ sind mon.\ wachsende Folgen.} \]
	\glqq{}\(\Leftarrow{}\)\grqq{}: Beobachtung: Jedes \(n\in\N \) ist in genau einem \glqq{}dyadischen\grqq{} Intervall. \( I_l := \{ 2^l, 2^l + 1, \dots, 2^{l+1} - 1 \} \)
	\[ I_0 = \{1\}, I_1 = \{2,3\}, I_3 = \{4,5,6,7\}, \#I_l = 2^{l+1} - 2^l = 2^l \]
	Ang., \(n<2^k\). 
	\[ S_n = \sum_{j=1}^{n}a_j \leq \sum_{j=1}^{2^k - 1}a_j = \sum_{l=0}^{k}\sum_{j\in I_l} \underbrace{a_j}_{\leq a_{2^l}} \]
	Bemerkung: \(I_l \cap I_m = \emptyset \; l \neq m, \quad \bigcup_{l=0}^{k} I_l = \{n \in \N | n \leq 2^k -1\} \)
	\[\leq \sum_{l=0}^{k} \#I_l \cdot a_{2^l} = \sum_{l=0}^{k} 2^l \cdot a_{2^l} = t_k\]
	\[ \Rightarrow S_n \leq t_k, \text{ falls } n\leq 2^k - 1 \]
	Annahme: \( t = \lim t_k \) existiert. \( \Rightarrow t_k \leq t \quad\forall k \).
	\[ \Rightarrow S_n \leq \limes{k}t_k = t. \]
	\[S_n \leq t \quad\forall n\in\N.\]
	\( \overset{\text{Mon. Konv.}}{\Rightarrow} \limes{n} S_n \) existiert. \checkmark{}\\
	\glqq{}\(\Rightarrow \)\grqq{}: Beachte: Jedes \(n \in\N, n\geq 2 \) ist in genau einem Block.
	\[ \tilde{I}_l = \{ 2^{l-1} + 1, 2^{l-1} + 2, \dots, 2^l \}, l\in\N. \]
	Sei \(n \geq 2^k \Rightarrow \)
	\begin{align*}
		S_n = \sum_{j=1}^{n}a_j \geq \sum_{j=1}^{2^k}a_j = a_1 + \sum_{j=2}^{2^k} a_j\\
		= a_1 + \sum_{l=1}^{k}\sum_{j\in\tilde{I}_l} a_j \geq a_1 + \sum_{l=1}^{k} \underbrace{\# \tilde{I}_l}_{=2^{l-1}} a_{2^l}
	\end{align*}
	Für \(n \geq 2^k \) ist
	\begin{align*}
		S_n &\geq a_1 + \sum_{l=1}^{k} 2^{l-1}a_{2^l}\\
		&=a_1 + \frac{1}{2} \underbrace{ \sum_{l=1}^{k}2^l a_{2^l} }_{=t_k - a_1}\\
		&= \frac{1}{2}a_1 + \frac{1}{2}t_k
	\end{align*}
	\[ \Rightarrow t_k \leq 2S_n \quad\forall n\geq 2^k. \]
	Wir nehmen an, dass \( \limes{n}S_n = S \) existiert.
	\[ S_n \leq S_{n+1} \leq \cdots \leq S \]
	Halte \( k\in\N \) fest.
	\begin{align*}
		&\Rightarrow t_k \leq 2S_n \text{ für fast alle } n.\\
		&\Rightarrow t_k \leq \limes{n} 2S_n = 2S\\
		&\Rightarrow t_k \leq 2S \quad\forall k\in\N.
	\end{align*}
	\[ \overset{\text{Mon. Konv.}}{\Rightarrow}\limes{k}t_k \text{ existiert.} \]
\end{bew}
\begin{bsp}
	\[ \sum_{n=1}^{\infty} \frac{1}{n^p} \text{ konv. } \Leftrightarrow p>1. \]
\end{bsp}
\begin{bew}
	Verdichtete Reihe ist 
	\[ \sum_{l=0}^{\infty}2^l \frac{1}{{(2^l)}^p} = \sum_{l=0}^{\infty} {(2^{1-p})}^l. \]
	geometrische Reihe, sie konvergiert genau dann, wenn \( 2^{1-p} < 1 \Leftrightarrow p>1. \)
\end{bew}
\begin{bsp}
	\[ \sum_{n=1}^{\infty} \frac{1}{n{(1+\ln_2 n )}^p}.\]
\end{bsp}
\begin{satz}[Leibniz]
	Ist \( {(a_n)}_n \) eine monoton fallende Nullfolge, so konvergiert die alternierende Reihe
	\[ a_1 - a_2 + a_3 - a_4 + \cdots = \sum_{n=1}^{\infty} {(-1)}^{n+1} a_n \]
\end{satz}
\begin{bsp}
	\[ 1 - 1/2 + 1/3 - 1/4 + \cdots = \sum_{n=1}^{\infty}{(-1)}^{n+1}\frac{1}{n} \text{ konv. } (=\log 2). \]
	\[ 1 - 1/3 + 1/5 - 1/7 + \cdots = \sum_{n=0}^{\infty}{(-1)}^n \frac{1}{2n+1} \text{ konv. } (=\frac{\pi}{4}) \]
	Beachte: \( \sum_{n=1}^{\infty}\frac{1}{n} = +\infty \).
\end{bsp}
\begin{bew}
	Aus \( a_{n+1} \leq a_n, a_n \rightarrow 0 \Rightarrow a_n \geq 0 \quad\forall n\in\N. \)
	\[ S_k := \sum_{j=1}^{k} {(-1)}^{j+1} a_j \quad k\in\N. \]
	\[S_{2n} = a_1 - a_2 + a_3 - a_4 + \cdots + a_{2n-1} - a_{2n} = \underbrace{(a_1 - a_2)}_{\geq 0} + \underbrace{(a_3 - a_4)}_{\geq 0} + \cdots + \underbrace{(a_{2n-1} - a_{2n})}_{\geq 0} \]
	\begin{align*}
		S_{2n+1} = \sum_{j=1}^{2n+1}{(-1)}^{j+1}a_j &= a_1 - a_2 + a_3 - \cdots - a_{2n} + a_{2n+1}\\
		&= a_1 - \underbrace{a_2 - a_3}_{\geq 0} - \cdots - \underbrace{(a_{2n} - a_{2n+1})}_{\geq 0}\\
		\Rightarrow S_{2(n+1)} = S_{2n+2} = S_{2n} + (a_{2n+1} - a_{2n+2}) \geq S_{2n}.\\
		S_{2(n+1)+1} = s_{2n+3} = S_{2n+1} - (a_{2n+2} - a_{2n+3}) \leq S_{2n+1}
	\end{align*}
	\[ \text{und } 0 \leq S_{2n} = \sum_{j=1}^{2n} {(-1)}^{j+1} a_j = S_{2n+1} - {(-1)}^{2n+2} \cdot a_{2n+1} = S_{2n+1} - a_{2n+1} \leq S_{2n+1}\]
	\[\Rightarrow S_{2n} \leq S_{2n+2}, \quad S_{2n+1} \geq S_{2n+3} \text{ und } 0 \leq S_{2n} \leq S_{2n+1} - a_{2n+1} \leq S_{2n+1} \leq a_n\]
	\( \overset{\text{Mon. Konv.}}{\Rightarrow} S_1 = \limes{n}S_{2n} \) und \( S_2 = \limes{n}S_{2n+1} \) existieren.\\	
	und: 
	\[ \underbrace{S_{2n} - S_{2n+1}}_{S_1 - S_2} = a_{2n+1} \rightarrow \quad n\rightarrow \infty \]
	\[ \Rightarrow S_1 - S_2 = \limes{n}(S_{2n} - S_{2n+1}) = \limes{n}a_{2n+1} = 0 \]
	\[ \Rightarrow S_1 = S_2 \Rightarrow {(S_n)}_n \text{ konvergiert auch gegen } S_1 (=S_2). \]
\end{bew}
\begin{defi}
	Eine Reihe \( \sum_{n=1}^{\infty} a_n \) heißt absolut konvergent, falls \( \sum_{n=1}^{\infty}|a_n| \) konvergiert, d.\ h.\  \(\sum_{n=1}^{\infty}|a_n|<\infty \).
\end{defi}
\begin{satz}
	Eine absolut konvergente Reihe \( \sum_{n=1}^{\infty}a_n, a_n \in\R \) ist konvergent, und
	\[ \left| \sum_{n=1}^{\infty}a_n \right| \leq \sum_{n=1}^{\infty}|a_n|.\] (Dreiecksungleichung für Reihen)
\end{satz}
\begin{bew}
	Annahme: \( \sum_{n=1}^{\infty} |a_n| \) konvergiert.
	\[ \overset{\text{Cauchy Krit.}}{\Leftrightarrow} \forall \varepsilon>0 \exists K\in\N: \sum_{j=n+1}^{m} |a_j| < \varepsilon\quad\forall m>n\geq K. \]
	Beachte: 
	\[ \left| \sum_{j=n+1}^{m}a_j \right| \leq \sum_{j=n+1}^{m}|a_j| < \varepsilon \quad \forall m>n\geq K \]
	\[\Rightarrow \forall \varepsilon > 0 \exists k \in \N: \left| \sum_{j = n+1}^{m} a_j \right| < \varepsilon \quad \forall m>n\geq K\]
	\[\overset{\text{Cauchy}}{\underset{\text{Kriterium}}{\Rightarrow}} \sum_{n=1}^{\infty} a_n \text{ konvergiert}\]
	Auch: \(m\) fest.
	\[ \underbrace{\left| \sum_{n=1}^{m}a_n \right|}_{=S_m}  = |S_m| \leq \sum_{n=1}^{m}|a_n| \leq \sum_{n=1}^{\infty}|a_n| \]
	\[ \Rightarrow |S_m| \leq \sum_{n=1}^{\infty}|a_n| \]
	Wissen \( S_m\rightarrow \sum_{n=1}^{\infty}a_n, m\rightarrow\infty \)
	\[ \Rightarrow \left|\sum_{n=1}^{\infty}a_n \right| = |S| = |\limes{m}S_m| = \limes{m}|S_m| \leq \sum_{n=1}^{\infty}|a_n|. \]
\end{bew}
\begin{bem}
	Warnung: Umkehrung von Satz 11 ist falsch! (Bsp.\ alternierende harmonische Reihe)
\end{bem}
\begin{defi}
	Wir nennen eine Reihe \( \sum_{n=0}^{\infty}c_n \) eine Majorante, von \( \sum_{n=0}^{\infty}a_n, a_n\in\R \), falls \( |a_n| \leq c_n \) für fast alle \(n\).
\end{defi}
\begin{kor}
	Hat die Reihe \( \sum_{n=0}^{\infty}a_n \) eine konvergente Majorante, so konvergiert \( \sum_{n=0}^{\infty}a_n \) absolut und ist somit auch konvergent.
\end{kor}
\begin{bew}
	Folgt direkt aus Satz 6, Def. 12 und Satz 11.
\end{bew}
\begin{satz}[Quotientenkriterium]
	Sei \( \sum_{n=0}^{\infty}a_n \) Reihe, \(a_n\neq0\), und es gebe ein \(q\) mit \(0<q<1\), sodass
	\[(*) \frac{|a_{n+1}|}{|a_n|} \leq q \text{ für fast alle }n. \]
	Dann ist \( \sum_{n=0}^{\infty}a_n \) absolut konvergent.
\end{satz}
\begin{bew}
	\begin{enumerate}
		\item \( (*) \Rightarrow \limsup\limits_{n\rightarrow\infty} \frac{|a_{n+1}|}{|a_n|} \leq q. \)
		\item \( \tilde{q} = \limsup\limits_{n\rightarrow\infty} \frac{|a_{n+1}|}{|a_n|} \Rightarrow \forall\varepsilon>0: \frac{|a_{n+1}|}{|a_n|} \leq \tilde{q} + \varepsilon \) für fast alle \(n\).
	\end{enumerate}
	\[ (*)\Rightarrow\exists n_0 \in\N: \frac{|a_{n+1}|}{|a_n|} \leq q \quad\forall n\geq n_0. \]
	\[ p\in\N_0, n=n_0 + p \]
	\begin{align*}
		\Rightarrow |a_{n+1}| \leq q |a_n| \leq q^2|a_{n-1}| \leq \cdots\leq q^{p+1}|a_{n_0}|\\
		\Rightarrow |a_n| \leq q^p |a_{n_0}| = q^n \underbrace{q^{-n_0}|a_{n_0}|}_{=M} \leq M q^n =: c_n
	\end{align*}
	d.\ h.\  \( \sum_{n=0}^{\infty}a_n \) hat \( \sum_{n=0}^{\infty}M q^n \) (geom. Reihe, sie konvergiert, da \( 0<q<1 \)) als Majorante.
\end{bew}
\begin{bem}
	\( a_n = M q^n \)
	\[ \Rightarrow \left| \frac{a_{n+1}}{a_n} \right| = \frac{M q^{n+1}}{M q^n} = q \]
	\[ a_n = \frac{1}{n^p} \quad \frac{a_{n+1}}{a_n} = \frac{1}{{(n+1)}^p} \frac{n^p}{1} = {\left(\frac{n}{n+1}\right)}^p = {\left(1 - \frac{1}{n+1}\right)}^p \rightarrow 1, n\rightarrow\infty \]
	\[ \Rightarrow \limsup\limits{n\rightarrow\infty} \frac{a_{n+1}}{a_n} = \limes{n} \frac{a_{n+1}}{a_n} = 1. \]
\end{bem}
\begin{satz}[Wurzelkriterium]
	\( \sum_{n}a_n \) Reihe mit \[ (**) \limsup\limits_{n\rightarrow\infty} |a_n|^{\frac{1}{n}} = \limsup\limits_{n\rightarrow\infty} \sqrt[n]{|a_n|} < 1 \Rightarrow \sum_n a_n \text{ konv.\ abs.}.\]
	Ist \( \limsup\limits_{n\rightarrow\infty} |a_n|^{\frac{1}{n}} > 1 \), so ist die Reihe divergent. (ohne Bew.) 
\end{satz}
\begin{bem}
	Bei \( \limsup\limits_{n\rightarrow\infty} |a_n|^{\frac{1}{n}} = 1 \) ist keine Aussage möglich.
\end{bem}
\begin{bem}
	\( \limsup\limits_{n\rightarrow\infty} |a_n|^{\frac{1}{n}} \leq \limsup\limits_{n\rightarrow\infty} \frac{|a_{n+1}|}{|a_n|}, a_n \neq 0, \text{ für fast alle } n \). (H. A.)
\end{bem}
\begin{bew}
	\( (**) \Rightarrow \exists 0<q<1. K\in\N \) mit \[ |a_n|^{\frac{1}{n}} \leq q \quad\forall n\geq K. \]
	\[\left(\tilde{q} = \limessup{n} |a_n|^{\frac{1}{n}} = \limes{n} \underbrace{\sup |a_k|^{\frac{1}{n}}}_{\rightarrow \tilde{q} < 1 (n \rightarrow \infty)}< 1\right)\] \\
	\(\forall \varepsilon > 0 \exists k \in \N: \underset{k \geq n}{\sup} |a_k|^{\frac{1}{k}} < \tilde{q} + \varepsilon \quad \forall n \geq K \) \\
	d.h.\\
	\[\underset{k \geq n}{\sup} |a_k|^{\frac{1}{k}} < \tilde{q} + \varepsilon = q \; (< 1)\]
	\(\tilde{q} < 1\) wähle \(s > 0: q = \tilde{q} + s < 1\) \\
	\[\Rightarrow |a_n| \leq q^n \quad \forall n \geq K \]
	Damit haben wir konv. Majorante \( \sum_n q^n \).
\end{bew}
\begin{satz}[\gqq{Mutter aller Konvergenzkriterien}]
	Sei \( \sum_n a_n \) Reihe mit \(a_n \neq 0 \) für fast alle \(n\). Dann gilt:
	\[ \sum_n a_n \text{ konv.\ abs.\ } \Leftrightarrow \exists c_n > 0, \sum_n c_n < \infty \text{ und } \frac{|a_{n+1}|}{|a_n|} \leq \frac{c_{n+1}}{c_n} \text{ für fast alle }n. \]
\end{satz}
\begin{bew}Übung.
	Scharfes Hinschauen auf Beweis des Quotientenkriteriums.
\end{bew}
\begin{kor}
	\( \sum_n a_n \) und \( p>1: \frac{|a_{n+1}|}{|a_n|} \leq 1 - \frac{p}{n+1} \) für fast alle \(n \Rightarrow \sum_n a_n \) konv.\ abs.\ (Übung).
\end{kor}
\begin{bem}
	Ist \( \liminf\limits_{n\rightarrow\infty} |a_n|^{\frac{1}{n}} > 1 \Rightarrow a_n \) divergent.\\
	Ist \( \liminf\limits_{n\rightarrow\infty} \frac{|a_{n+1}|}{|a_n|} > 1 \Rightarrow \sum_n a_n \) divergent.\\
	Ist \( \liminf\limits_{n\rightarrow\infty} |a_n|^{\frac{1}{n}} = \limes{n}(\underset{k\geq n}{\inf} |a_n|^{\frac{1}{k}}) > 1 \Rightarrow \exists q>1: k \in\N \).
	\[ |a_n|^{1/n} \geq \underset{k\geq n}{\inf} |a_k|^{1/k} \geq q > 1 \quad\forall n\geq K. \]
	\[ \Rightarrow |a_n| > q^n \rightarrow \infty, n\rightarrow\infty \Rightarrow a_n \text{ keine Nullfolge } \Rightarrow \sum_n a_n \text{ divergent}. \]
\end{bem}
\begin{bsp}[Exponentialreihe]
	\[ x\in\R,\qquad \sum_{n=0}^{\infty} \underbrace{\frac{x^n}{n!}}_{=a_n} =: \exp x. \]
	\begin{align*}
		|\frac{a_{n+1}}{a_n}| = |\frac{x^{n+1}}{(n+1)!} \frac{n!}{x^n}| = \frac{|x|}{n+1} \rightarrow 0, n\rightarrow\infty \\
		\Rightarrow \frac{|a_{n+1}}{a_n} \leq 1/2 \text{ für fast alle } n.\\
		\Rightarrow \sum_{n=0}^{\infty} \frac{x^n}{n!} = 1 + x + 1/2 x^2 + 1/6 x^3 + \dots \text{ konv.\ absolut.}
	\end{align*}
\end{bsp}
\begin{bem}
	\begin{align*}
		S_n(x):= \sum_{j=1}^{n} \frac{x^j}{j!} = 1 + x + 1/2x^2 + \cdots + \frac{x^n}{n!}\\
		S'_n(x) = 1 + x + 1/2x^2 + \cdots + \frac{x^{n-1}}{(n-1)!} = S_{n-1}(x)\\
		\Rightarrow S'_n(x) = S_{n-1}(x) = S_n(x) - \frac{x^n}{n!}
	\end{align*}
	Falls gilt: \( S'(x) = ( \limes{n}S_n(x) )' = \limes{n}(S'_n(x)) = \limes{n}S_{n-1}(x) = S(x) \)
	\( \Rightarrow \exp'x = \exp x? \) (im Allg.\ falsch, aber für Potenzreihen wahr)
\end{bem}
\end{document}
\documentclass[../ana1.tex]{subfiles}
\begin{document}

\section{Der Euklidische Raum \( \R^d \), komplexe Zahlen \( \C \) (und \(\C^d \))}
Grundlagen: \( \R^d := \R \times \dots \times \R = \{ (x_1, \dots, x_d) | x_j \in\R, j\in\{1,\dots, d \} \)\\
Addition:
\[x, y \in \R^d: x + y := (x_1, \dots , x_d) + (y_1, \dots, y_d) = (x_1 + y_1, \dots, x_d + y_d) \]
Skalare Multiplikation:
\[\lambda \in \R: \lambda x := \lambda(x_1, \dots , x_d) = (\lambda x_1, \dots, \lambda x_d) \]
\( \R^d \) ist ein (reeller) \(d\)-dim. Vektorraum.\\
z.B. \(\R^2\): \(x = (x_1, x_2) \in \R^2\)\\
\textbf{Bild}\\%BILD
Länge von \(x = |x| = (x_1^2 + x_2^2)^{1/2}\)
\begin{defi}[Euklidische Länge und Skalarprodukt]
	\[ x,y \in\R^d, x\cdot y = <x,y> := \sum_{j=1}^{d}x_j y_j \]
	\[ |x| = ||x||_2 := (\sum_{j=1}^{d} x_j^2 )^{1/2} = <x,x>^{1/2} = \sqrt{<x,x>}. \]
\end{defi}
\begin{bem}
	Euklidisches Skalarprodukt erfüllt die Axiome eines allg. Skalarprodukts (auf reellen VR.).\\
	(S1) \( <x,y> = <y,x> \quad\forall x,y\in R^d \) (Symmetrie)\\
	(S2) \( <x, \lambda y + \mu z> = \lambda <x,y> + \mu<x,z> \quad \forall x,y,z\in\R^d, \lambda, \mu \in\R \) (Bilinearität)\\
	(S3) \( <x,x> \geq 0 \) und \( <x,x> = 0 \Rightarrow x=0 = (0,\dots,0) \in\R^d \)
\end{bem}
\begin{bem}
	(S1) und (S2) \( \Rightarrow <\lambda x + \mu y, z> = \lambda <x,z> + \mu <y,z> \).
\end{bem}
\begin{satz}[Cauchy-Schwarz-Ungleichung, CSU]
	\[ \forall x,y\in\R^d: |<x,y>|\leq |x||y|\]
	und \glqq\(=\)\grqq{} gilt, \( \Leftrightarrow x,y \) sind linear abhängig.
\end{satz}
\begin{bew}
	Haben immer \( x,y\in\R^d, t\in\R. \)
	\begin{align*}
		0\leq |x+ty|^2 = <x+ty, x+ty>\\
		\overset{\text{(S2)}}{=} <x+ty,x> + t<x+ty,y>\\
		\overset{\text{(S1)(S2)}}{=} <x,x> + t<y,x> + t<x,y> + t^2<y,y>\\
		= |x^2| + 2t<x,y> + t^2|y|^2 =: g(t).\\
		g(t) = at^2 + 2bt + c
	\end{align*}
	Ang. \(|y| > 0, a = |y|^2 > 0 \)
	\begin{align*}
		g(t) = at^2 + 2bt + c\\
		= a(t^2 + 2b/a t + c/a + (b/a)^2 - (b/a)^2) a((t+b/a)^2 + c/a - b^2/a^2)\\
		= a(t+b/a)^2 + c - b^2/a \geq 0 \quad\forall t\in\R.
	\end{align*}
	\begin{align*}
		\Rightarrow g(t) \geq 0 \quad\forall t \Leftrightarrow b^2 \leq ac (*) \Rightarrow <x,y>^2 \leq |x|^2|y|^2\\
		\text{und }g(t) > 0 \quad\forall t \Leftrightarrow b^2 < ac (**)\\
		\text{Haben }  a=|y|^2, c=|x|^2\\
		\Rightarrow |<x,y>|\leq |x||y|.
	\end{align*}
	Fall 1: \(|y| = 0 \Leftrightarrow y = 0_v = (0, 0)\)
	\[<x, 0_v> = <x, 0 \cdot 0_v> = 0 \cdot <x, 0_v> = 0\]
	Fall 2: \(|<x, y>| < |x||y| \Leftrightarrow b^2 < ac \Leftrightarrow g(t) > 0 \quad \forall t \in \R\)\\
	\[\Leftrightarrow x + ty \neq 0 \quad \forall t \in \R \Leftrightarrow x, y \text{ linear unabhängig} \]
\end{bew}
\begin{defi}
	Eine Funktion \( ||\cdot || : \R^d \rightarrow \R \) heißt Norm (auf \( \R^d \)), falls \( \forall x,y\in\R^d, \lambda\in\R: \)
	\begin{enumerate}
		\item \( ||x||\geq 0, ||x|| = 0 \Rightarrow x = 0 \)
		\item \( ||\lambda x|| = |\lambda|\cdot||x|| \)
		\item \( ||x+y|| \leq ||x|| + ||y|| \) (Dreiecksungl.)
	\end{enumerate}
\end{defi}
\begin{satz}
	\[ |x| := \sqrt{<x,x>} = \left(\sum_{j=1}^{d} x_j^2\right)^{1/2} \text{ ist eine Norm}. \]
\end{satz}
\begin{bew}
	Eigenschaften 1. und 2. sind einfach nachzurechnen.\\
	Zu 3.:
	\begin{align*}
		|x+y|^2 = <x+y,x+y>=<x+y,x>+<x+y,y>\\
		= <x,x> + <y,x> + <x,y> + <y,y>\\
		=|x|^2 + 2<x,y> + |y|^2\\
		\leq |x|^2 + 2|<x,y>| + |y|^2 \overset{12.0.2}{\leq} |x^2| + 2|x||y| + |y|^2 = (|x|+|y|)^2
	\end{align*}
	\[ \Rightarrow |x+y| \leq |x| + |y| \]
\end{bew}
\begin{bem}
	Auch:
	\begin{align*}
		&|x - y| \geq 0 \text{ und } |x - y| = 0 \Leftrightarrow x = y\\
		&|x - y| = |y - x|\\
		&|x - y| = |x-z + z-y| \leq |x - z| + |y - z| \quad \forall z \in \R^d\\
		&||x| + |y|| \leq |x - y|
	\end{align*}
	 %Bild
\end{bem}
Komplexe Zahlen:\\
\( z=x+iy \quad\forall x,y\in\R \) und \( i^2 = -1 \).\\
\( z_1 = x_1 + i y_1, z_2 = x_2 + i y_2 \)
\[ z_1 + z_2 := (x_1 + x_2) + i(y_1 + y_2). \]
\[ z_1 \cdot z_2 := (x_1 + i y_1)(x_2 + i y_2) = x_1x_2 - y_1y_2 + i(x_1y_2 + x_2y_1) \]
Rigoros: \( \R^2, z\in\R^2, z=(x,y) \).\\
Addition als addieren von Vektoren: \[ z_1 + z_2 = (x_1 + x_2, y_1 + y_2) \]
neue Multiplikation: \[ z_1 \cdot z_2 = (x_1 x_2 - y_1y_2, x_1y_2 + x_2y_1) \]
\( z = x+i 0 = (x,0) \)\\
\( (x_1,0) + (x_2,0) = (x_1+x_2,0) \)\\
\( (x_1,0)\cdot(x_2,0) = (x_1x_2,0) \)\\
\( z=(x,y) = (x,0) + (0,y) = x(1,0) + y(0,1) = x e_1 + y e_2 \)
%BILD
\( \R^2 \) mit obiger Addition und \glqq komplexen \grqq{} Multiplikation erfüllt alle Körperaxiome.\\
\( e_2 e_2 = (0,1)(0,1) = (0-1,0 + 0) = (-1,0) = -(1,0) = -e_1 \)\\
\(e_1 = 1\cdot e_1 = (1,0) \)\\
\( e_2^2 = -e_1 = -1 (1,0) + 0(0,1) \)
\[ z = (x,y) = x e_1 + y e_2 = x \cdot 1 + y \cdot i = x + i y. \]

\end{document}

\end{document}