\documentclass[12pt,a4paper,titlepage,draft]{article}
% draft rauslöschen, für releases (Sonst wird an manchen Stellen schwarze Balken gerendert und
% Bilder laden nicht)

% Vorlage für chapter[x].tex:
%
% \documentclass[../ana1.tex]{subfiles}
% \begin{document}
% \setcounter{section}{[x - 1]}
% \section{[chaptername]}
% ...
% \end{document}

\IfFileExists{skript.sty}{\usepackage{skript}}{\usepackage{../skript}}

%Metadaten
\hypersetup{
	pdftitle = {Analysis I Vorlesungsmitschrieb (WS 18/19)},
	pdfauthor = {Pavel Zwerschke}
}

\begin{document}

%%%%%%%%%%%%%%%
%<*tag-skript>%
%%%%%%%%%%%%%%%

\renewcommand{\onlyinsubfile}[1]{}
\renewcommand{\notinsubfile}[1]{#1}

\title{Analysis I (WS 18/19)\\ {\normalsize Vorlesungsmitschrieb}}
\date{\today}
\maketitle

\section*{Disclaimer} 
Bei diesem Skript handelt es sich um einen inoffiziellen Mitschrieb 
\gqq{Analysis~I} von Prof.\ Dr.\ Dirk Hundertmark im Wintersemester 
2018/2019 an dem Karlsruher Institut für Technologie (KIT).	Die 
Mitschriebe der Vorlesung wurden mit ausdrücklicher
Genehmigung der Dozenten von Pavel Zwerschke erstellt und veröffentlicht.
Weder Prof.\ Hundertmark noch Dr.\ Lange sind für den Inhalt 
verantwortlich.
\newpage

%Listenstil
%Itemize   [-]
%Enumerate [(a)] Für Sätze und Bemerkungen
%		   [(1)] Für Beispiele
%		   [(i)] Für Voraussetzungen in Bemerkungen/Sätzen die nummeriert sind. Zusammengehörige
%                Aussagen die aus mehreren teilaussagen bestehen

\iftoggle{short}{}{
	%Inhaltsverzeichnis
	\tableofcontents
	\newpage

	\subfile{chapters/chapter0.tex}
	\newpage
	\subfile{chapters/chapter1.tex}
}
\foreach \c in {2,...,17}{
	\subfile{chapters/chapter\c.tex}
	\iftoggle{short}{}{\newpage}
}

%%%%%%%%%%%%%%%
%</tag-skript>%
%%%%%%%%%%%%%%%

\end{document}