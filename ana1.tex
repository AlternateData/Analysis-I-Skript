\documentclass[12pt,a4paper,titlepage,draft]{article} %twopage
  
%\usepackage{german} %deutsches Format
\usepackage[ngerman]{babel}
\usepackage[utf8]{inputenc} %Umlaute
\usepackage{graphicx} %Grafiken einbinden
\usepackage{amsmath,amssymb,amsthm}
\usepackage{nicefrac}
\usepackage{marvosym}[Lightning]
\usepackage{array}
\usepackage{dsfont} %statt \mathbb{1}
\usepackage{tikz} 
\usetikzlibrary{matrix}
\usepackage{hyperref} %Hyperlinks in pdf
\usepackage{subfiles}
\usepackage[shortlabels]{enumitem} 

%Metadaten
\hypersetup{
	pdftitle = {Analysis I Skript (WS 18/19)},
	pdfauthor = {Pavel Zwerschke, Daniel Augustin}}

\theoremstyle{definition}
\newtheorem{satz}{Satz}[subsection]
\newtheorem{kor}[satz]{Korollar}
\newtheorem{lem}[satz]{Lemma}

\newtheorem{defi}[satz]{Definition}
\newtheorem*{beh}{Behauptung}
\newtheorem{bem}{Bemerkung}[satz]

\theoremstyle{remark}
\newtheorem{bsp}[bem]{Beispiel}
\newtheorem{bspe}[bem]{Beispiele}

\newenvironment{bew}{\begin{proof}[Beweis]}{\end{proof}}
\newcommand{\N}{\mathbb{N}}
\newcommand{\Z}{\mathbb{Z}}
\newcommand{\Q}{\mathbb{Q}}
\newcommand{\R}{\mathbb{R}}
\newcommand{\C}{\mathbb{C}}
\renewcommand{\Re}{\operatorname{Re}}
\renewcommand{\Im}{\operatorname{Im}}
\newcommand{\gqq}[1]{\glqq{}#1\grqq{}}
\newcommand{\limes}[1]{\lim\limits_{#1\rightarrow\infty}}
\newcommand{\limessup}[1]{\limsup\limits_{#1\rightarrow\infty}}
\newcommand{\limesinf}[1]{\liminf\limits_{#1\rightarrow\infty}}
\newcommand{\onlyinsubfile}[1]{#1}
\newcommand{\notinsubfile}[1]{}
\newcommand{\subgraphic}[2]{
	\onlyinsubfile{\includegraphics[width=#1\textwidth]{images/#2}}
		\notinsubfile{\includegraphics[width=#1\textwidth]{chapters/images/#2}}
}

\begin{document}
\renewcommand{\onlyinsubfile}[1]{}
\renewcommand{\notinsubfile}[1]{#1}

\title{Analysis I (WS 18/19)}
\date{\today}
\author{Pavel Zwerschke, Daniel Augustin}
\maketitle

%Inhaltsverzeichnis
\tableofcontents
\newpage

\setcounter{section}{-1}
\section{Organisatorisches}
\begin{description}[style=nextline]
	\item[Dozent]
		Prof.\ Dr.\ Dirk Hundertmark (Gebäude 20.30, Raum 2.028)\\
		\href{mailto:dirk.hundertmark@kit.edu}{dirk.hundertmark@kit.edu}
	\item[Übungsleiter]
		Dr.\ Markus Lange (Gebäude 20.30, Raum 2.030)\\
		\href{mailto:markus.lange@kit.edu}{markus.lange@kit.edu}
	\item[Übungsaufgaben]
		\begin{description}
			\item[]
			\begin{enumerate}
			\item[]
			\item[Ausgabe:] Die Übungsblätter werden jeden Donnerstag veröffentlicht unter\\
						    \href{http://www.math.kit.edu/iana1/lehre/ana12018w/}{\texttt{www.math.kit.edu/iana1/lehre/ana12018w/}}.
			\item[Abgabe:] Die bearbeiteten Übungsblätter können bis mittwochs um 19:00 Uhr in den Abgabekästen im Foyer\\
			    		   des Mathebaus (Gebäude 20.30) abgegeben werden.\\
						   Die Abgaben werden getackert sowie mit Namen, Matrikelnummer und Tutoriennummer beschriftet in das Fach mit
						   der richtigen Kennzeichnung geworfen.\\
						   Übungsaufgaben dürfen zu zweit abgegeben werden.\\
			\end{enumerate}
		\end{description}
	\item[Übungsschein]
		Jede K-Aufgabe wird mit 4 Punkten bewertet. Einen Übungsschein erhält wer 50\% der Punkte aller K-Aufgaben erzielt.
	\item[Klausur]
		Die Anmeldung findet über das Online-Portal statt. Die Klausur findet am 26.03.2019 von 08:00 bis 10:00 statt.
		Der Übungsschein ist Voraussetzung für die Teilnahme an der Klausur.
	\item[Proseminar] 
		\href{http://www.math.kit.edu/lehre/seite/prosemanmeld/}{\texttt{www.math.kit.edu/lehre/seite/prosemanmeld/}} 
\end{description}
\newpage
\foreach \c in {1,...,12}{\subfile{chapters/chapter\c.tex}}
\end{document}
