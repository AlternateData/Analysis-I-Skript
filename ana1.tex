\documentclass[12pt,a4paper,titlepage]{article} %twopage

%Befehle

%Grafik einbinden
%\centering
%\includegraphics[width=0.7\textwidth]{Profilbild.png}


%\usepackage{german} %deutsches Format
\usepackage[ngerman]{babel}
\usepackage[utf8]{inputenc} %Umlaute
\usepackage{graphicx} %Grafiken einbinden
\usepackage{amsmath}
\usepackage{amssymb}
\usepackage{amsthm}
\usepackage{nicefrac}
\usepackage{hyperref} %Hyperlinks in pdf

%Metadaten
\hypersetup{
	pdftitle = {Analysis I Skript (WS 18/19)},
	pdfauthor = {Pavel Zwerschke}}

\newcommand{\bsp}{\textbf{Beispiel:}\\}
\newcommand{\definition}{\textbf{Definition:\\}}
\newcommand{\satz}{\textbf{Satz:\\}}
\newcommand{\behauptung}{\textbf{Behauptung:\\}}
\newcommand{\beweis}{\textbf{Beweis:\\}}
\newcommand{\bemerkung}{\textbf{Bemerkung:\\}}


\begin{document}
\title{Analysis I (WS 18/19)}
\date{\today}
\author{Pavel Zwerschke}
\maketitle

%Inhaltsverzeichnis
\tableofcontents
\newpage

\setcounter{section}{-1}
\section{Organisatorisches}
\textbf{Dozent}\\
Prof. Dr. Dirk Hundertmark (20.30, 2.028)\\
\href{mailto:dirk.hundertmark@kit.edu}{dirk.hundertmark@kit.edu}

\textbf{Übungsleiter}\\
Dr. Markus Lange (20.30, 2.030)\\
\href{mailto:markus.lange@kit.edu}{markus.lange@kit.edu}

\textbf{Übungszettel}\\
Ausgabe:\\
donnerstags unter \href{http://www.math.kit.edu/iana1/lehre/ana12018w/}{\texttt{www.math.kit.edu/iana1/lehre/ana12018w/}}\\
Abgabe:\\
bis mittwochs um 19:00 in den Abgabekästen des Foyers des Mathematikgebäudes (20.30)\\
getackert, mit Namen, Matrikelnummer, Tutoriennummer und Deckblatt (optional) in das Fach mit der richtigen Kennzeichnung legen\\
Zettel dürfen zu zweit abgegeben werden

\textbf{Übungsschein}
Jede K-Aufgabe wird mit 4 Punkten bewertet. Einen Übungsschein erhält wer 50\% der Punkte aller K-Aufgaben erzielt.

\textbf{Klausur}\\
Die Anmeldung findet über das Online-Portal statt. Die Klausur findet in KW 8 2019 statt. Der Übungsschein ist Voraussetzung für die Teilnahme an der Klausur.

\newpage

\section{Was ist Analysis?}
\textbf{Zentrale Begriffe:}\\
Grenzwerte von Folgen und Reihen, Funktionen, stetig, differenzierbar, integrieren, Differential- und Integralrechnung, Differentialgleichungen (Newton, Maxwell, Schrödinger), unendlich dimensionale Räume

\bsp
$S = \frac{1}{2} + \frac{1}{4} + \dots + \frac{1}{2^n} + \dots\\
2S = 1 + \frac{1}{2} + \dots + \nicefrac{1}{2} + \dots\\
2S = 1 + S$\\
$S$ entspricht der Wahrscheinlichkeit, dass irgendwann mal Kopf in einem Münzwurf kommt.\\
Vorsicht!\\
$S = 1 + 2 + 4 + \dots\\
2S = 2 + 4 + 8 + \dots = -1 + 1 + 2 + 4 + \dots = -1 + S\\
S = -1$\\
Natürlich Quatsch!\\
Formales Rechnen kann gefährlich sein!

\begin{itemize}
  \item Was sind mathematische Aussagen?
  \item Wie macht man Beweise, wie findet man sie? (learning by doing)
  \item logische Zusammenhänge
\end{itemize}

\section{Etwas Logik}
Eine (mathematische) Aussage ist ein Ausdruck, der wahr oder falsch ist.\\
z. B.
\begin{enumerate}
  \item $A :$ \glqq $1+1=2$.\grqq (auch \glqq $1+1=3$\grqq, \glqq $1+1=0$\grqq)
  \item $B :$ \glqq Es gibt unendlich viele Primzahlen.\grqq
  \item $C :$ \glqq Es gibt unendlich viele Primzahlen $p$ für die $p + 2$ auch eine Primzahl ist.\grqq
  \item $D :$ \glqq Die Gleichung $m \ddot{x} = F$ hat geg. $\dot{x}(0) = v_0, x(0) = x_0$ immer genau eine Lösung.\grqq
  \item $E :$ \glqq Jede gerade natürliche Zahl größer als 2 ist die Summe zweier Primzahlen.\grqq
  \item $F :$ \glqq Morgen ist das Wetter schön.\grqq
  \item $G :$ \glqq Ein einzelnes Atom im Vakuum mit der Kernladungszahl $Z$ kann höchstens $Z+1$ Elektronen binden.\grqq
  \item $H(k,m,n) :$ \glqq Es gilt: $k^2 + m^2 = n^2$.\grqq (z. B. $H(3,4,5)$ ist wahr.)    
\end{enumerate}
Gegeben für natürliche Zahlen $n$, Aussagen $A(n)$, dann gilt:\\
Für jede nat. Zahl $n$ ist $A(n)$ wahr, genau dann, wenn 
\begin{enumerate}
  \item $A(1)$ ist wahr.
  \item Unter der Annahme, dass $A(n)$ wahr ist, folgt, dass $A(n+1)$ wahr ist.
\end{enumerate}
\bsp 
$A(n): 1+2+3+\dots+n=\frac{n(n+1)}{2}$.
\begin{proof} Vollständige Induktion\\
  Induktionsanfang:\\
  $1 = \frac{1(1+1)}{2}$ \checkmark\\
  Induktionsschluss:\\
  Wir nehmen an, dass $A(n)$ wahr ist (für $n\in\mathbb{N}$)\\
  D. h. Induktionsannahme:\\
  $1+2+3+\dots+n=\frac{n(n+1)}{2}$\\
  Dann folgt:\\
  $\underbrace{1+2+\dots+n}_{=\frac{n(n+1)}{2}}+(n+1)=\frac{n(n+1)}{2}+(n+1) \\
  =\frac{n(n+1) + 2(n+1)}{2} = \frac{(n+1)(n+2)}{2} = \frac{(n+1)((n+1)+1)}{2}$
\end{proof}
\bemerkung
Gaußscher Trick:\\
$S=1+2+3+\dots+n=n+(n-1)+(n-2)+\dots+2+1\\
2S = \underbrace{(n+1)+(n+1)+\dots+(n+1)}_{n\text{-mal}} \Leftrightarrow S = \frac{n(n+1)}{2}$.



\end{document}
