\documentclass[12pt,a4paper,titlepage,draft]{article} %twopage

%\usepackage{german} %deutsches Format
\usepackage[ngerman]{babel}
\usepackage[utf8]{inputenc} %Umlaute
\usepackage{graphicx} %Grafiken einbinden
\usepackage{amsmath,amssymb,amsthm}
\usepackage{nicefrac}
\usepackage{marvosym}[Lightning]
\usepackage{array}
\usepackage{dsfont} %statt \mathbb{1}
\usepackage{tikz}
\usetikzlibrary{matrix}
\usepackage{hyperref} %Hyperlinks in pdf
\usepackage{subfiles}

%Metadaten
\hypersetup{
	pdftitle = {Analysis I Skript (WS 18/19)},
	pdfauthor = {Pavel Zwerschke, Daniel Augustin}}

\theoremstyle{definition}
\newtheorem{satz}{Satz}[subsection]
\newtheorem{kor}[satz]{Korollar}
\newtheorem{lem}[satz]{Lemma}

\newtheorem{defi}[satz]{Definition}
\newtheorem*{beh}{Behauptung}

\theoremstyle{remark}
\newtheorem*{bem}{Bemerkung}
\newtheorem*{bsp}{Beispiel}

\newenvironment{bew}{\begin{proof}[Beweis]}{\end{proof}}
\newcommand{\N}{\mathbb{N}}
\newcommand{\Z}{\mathbb{Z}}
\newcommand{\Q}{\mathbb{Q}}
\newcommand{\R}{\mathbb{R}}
\newcommand{\C}{\mathbb{C}}
\newcommand{\gqq}[1]{\glqq{}#1\grqq{}}
\newcommand{\limes}[1]{\lim\limits_{#1\rightarrow\infty}}
\newcommand{\limessup}[1]{\limsup\limits_{#1\rightarrow\infty}}
\newcommand{\limesinf}[1]{\liminf\limits_{#1\rightarrow\infty}}
\newcommand{\onlyinsubfile}[1]{#1}
\newcommand{\notinsubfile}[1]{}
\newcommand{\subgraphic}[2]{
	\onlyinsubfile{\includegraphics[width=#1\textwidth]{images/#2}}
		\notinsubfile{\includegraphics[width=#1\textwidth]{chapters/images/#2}}
}

\begin{document}
\renewcommand{\onlyinsubfile}[1]{}
\renewcommand{\notinsubfile}[1]{#1}

\title{Analysis I (WS 18/19)}
\date{\today}
\author{Pavel Zwerschke, Daniel Augustin}
\maketitle

%Inhaltsverzeichnis
\tableofcontents
\newpage

\setcounter{section}{-1}
\section{Organisatorisches}
\textbf{Dozent}\\
Prof.\ Dr.\ Dirk Hundertmark (20.30, 2.028)\\
\href{mailto:dirk.hundertmark@kit.edu}{dirk.hundertmark@kit.edu}\\
\textbf{Übungsleiter}\\
Dr.\ Markus Lange (20.30, 2.030)\\
\href{mailto:markus.lange@kit.edu}{markus.lange@kit.edu}\\
\textbf{Übungszettel}\\
Ausgabe:\\
donnerstags unter \href{http://www.math.kit.edu/iana1/lehre/ana12018w/}{\texttt{www.math.kit.edu/iana1/lehre/ana12018w/}}\\
Abgabe:\\
bis mittwochs um 19:00 in den Abgabekästen des Foyers des Mathematikgebäudes (20.30)\\
getackert, mit Namen, Matrikelnummer, Tutoriennummer und Deckblatt (optional) in das Fach mit der richtigen Kennzeichnung legen\\
Zettel dürfen zu zweit abgegeben werden\\
\textbf{Übungsschein}\\
Jede K-Aufgabe wird mit 4 Punkten bewertet. Einen Übungsschein erhält wer 50\% der Punkte aller K-Aufgaben erzielt.\\
\textbf{Klausur}\\
Die Anmeldung findet über das Online-Portal statt. Die Klausur findet am 26.03.2019 von 08:00 bis 10:00 statt. Der Übungsschein ist Voraussetzung für die Teilnahme an der Klausur.

\newpage
\foreach \c in {1,...,12}{\subfile{chapters/chapter\c.tex} }
\end{document}