\documentclass[../ana1u.tex]{subfiles}
\begin{document}
\setcounter{section}{4}

\section{Folgenkonvergenz (23.11.18)}
\begin{bsp}
    Quadratwurzel\\
    \(a > 0, a \in \R\), wir wollen\(\sqrt{a}\) bestimmen.\\
    es gilt \(x^2 = a\), dann \(x = \frac{a}{x}\)\vspace{5mm}\\
    Babylonier verwendeten Iterationsverfahren zur Berechnung der Quadratwurzel, welches in jedem Schritt als Näherungswerte das arithmetische Mittel\vspace{5mm}\\
    \(x := \frac{1}{2}(x+\frac{a}{x})\)
    \begin{beh}
        Seien \(a > 0\) und \(x_0 > 0\) reelle Zahlen\\
        Die Folge \((x_n)_n\) sei gegeben durch \(x_{n+1} := \frac{1}{2}(x_n + \frac{a}{x_n})\) dann konvergiert die Folge gegen \(\sqrt{a}\), d.h. gegen die eindeutig bestimmte positive Lösung der Gleichung \(x^2 = a\)
    \end{beh}		
    Vorgehen im Beweis:
    \begin{enumerate}
        \item \(x_n > 0, \forall n \geq 0\)
        \item \(x_n^2 \geq a, \forall n \in \N\)
        \item \(x_{n+1} \leq x_n, \forall n \in \N\)
        \item Existenz des Grenzwertes
        \item Eindeutigkeit des Grenzwertes
    \end{enumerate}
    \begin{bew}
        Umsetzen des Plans:\\
        \begin{enumerate}				
            \item Schritt:\\
                Beweis durch vollständige Induktion:\\
                (IA):\\
                \(n=0: x_0 > 0\) per Voraussetzung\\
                \(n=1: x_1 = \frac{1}{2}(\underbrace{x_0}_{>0} + \underbrace{\frac{a}{x_0}}_{>0}) > 0\) ist wohldefiniert, da \(x_0 > 0\)\\
                (IS): \(n \rightarrow n+1\): für beliebiges, aber festes n sei gezeigt, dass \(x_n > 0\)\\
                Dann gilt \(x_{n+1} = \frac{1}{2}(\underbrace{x_n}_{>0} + \underbrace{\frac{a}{x_n}}_{>0}) > 0\) wohldefiniert, weil \(x_n > 0\)\\
                Somit gilt nach dem Induktionsprinzip \(x_n > 0\) \(\forall n \in \N\)
            \item Schritt:\\
                \(x_n^2 \geq a \Leftrightarrow x_n^2 - a \geq 0 \)
                \begin{align*}
                    x_n^2 - a &= \frac{1}{4}\left(x_{n-1} + \frac{a}{x_{n-1}}\right)^2 - a\\
                    &= \frac{1}{4}\left(x_{n-1}^2 + 2a + \frac{a^2}{x_{n-1}^2}\right) - a\\
                    &= \frac{1}{4}\left(x_{n-1}^2 + 2a + \frac{a^2}{x_{n-1}^2} - 4a\right)\\
                    &= \frac{1}{4}\left(x_{n-1} - \frac{a}{x_{n-1}}\right)^2 \geq 0
                \end{align*}
            \item Schritt:\\
                \begin{align*}
                    x_{n+1} \leq x_n &\Leftrightarrow x_n - x_{n+1} \geq 1\\
                    x_n - x_{n+1} &= x_n - \frac{1}{2}(x_n + \frac{a}{x_n})\\
                    &=\frac{1}{2}x_n - \frac{a}{2x_n}\\
                    &= \underbrace{\frac{1}{2x_n}}_{>0} \cdot \underbrace{(x_n^2 - a)}_{>0} \geq 0
                \end{align*}
            \item Schritt:\\
                Wegen Schritt 3 gilt die Folge \((x_n)_n\) ist monoton fallend und durch 0 nach unten beschränkt. Somit gilt nach Satz 8.2, dass \((x_n)_n\) konvergent ist.\\
                Aus Satz 7.1.9:
                \(\limes{n} x_n = x \geq 0\)
                \begin{align*}
                    x_{n+1} = \frac{1}{2}(x_n + \frac{a}{x_n}) \overset{\cdot 2x_n} {\Leftrightarrow} 2x_{n+1}x_n = x_n^2 + a \;(\ast)\\						
                \end{align*}
                Betrachte Grenzwert von (\(\ast\))
                \begin{align*}
                    \limes{n} (2x_{n+1}x_n) &= \limes{n} (x_n^2 + a)\\
                    \overset{\text{7.1.8}}{\Leftrightarrow} 2x^2 &= x^2 + a\\
                    \Leftrightarrow x^2 = a						
                \end{align*}
                Somit konvergiert \((x_n)_n\) gegen eine Quadratwurzel von a.
            \item Schritt:\\
                Annahme: \(y\) ist einer weitere Lösung von der Gleichung \(y^2 = a\) mit \(x \neq y\).\\
                Dann gilt:
                \begin{align*}
                    0 = x^2 - y^2 = (x-y)\underbrace{(x+y)}_{>0} \text{ \Lightning} x \neq y\\
                    \Rightarrow x = y
                \end{align*}
                Also ist x die einzige Lösung der Gleichung \(x^2 = a\).
        \end{enumerate}
    \end{bew}
    \subsection{Geschwindigkeit der Konvergenz}
    Definieren relativen Fehler \(f_n\) im n-ten Schritt\\
    \(x_n = \sqrt{a} \cdot (1+f_a)\)\\
    Da \((x_n)_n\) monoton fallend ist, gilt \(f_n \geq 0 \forall n \geq 1\)\\
    Einsetzen in Rekursionsformel:\\
    \begin{align*}
        x_{x+1} &= \frac{1}{2}\left(x_n + \frac{a}{x_n}\right)\\
        \Leftrightarrow \sqrt{a}(1+f_{n+1}) &= \frac{1}{2}\left(\sqrt{a}(f_n+1) + \frac{a}{\sqrt{a}(1+f_n)} \right)\\
        \Leftrightarrow 1 + f_n &= \frac{1}{2}\left(1+f_n+\frac{1}{f_n + 1}\right)\\
        \Rightarrow f_{n+1} &= \frac{1}{2} \cdot \frac{f_n^2}{1 + f_n}\\
        \text{Nebenüberlegung:}\\
        &f_n < f_n^2: f_n > 1: \frac{f_n^2}{1+f_n} < \frac{f_n^2}{f_n} = f_n\\
        &f_n^2 < f_n: f_n < 1: \frac{f_n^2}{1+f_n} < f_n^2 \\
        \Rightarrow f_{n+1} &= \frac{1}{2} \cdot \frac{f_n^2}{1+f_n} \leq \frac{1}{2} \cdot min(f_n, f_n^2)
    \end{align*}
    Der relative Fehler von \(f_n\) geht sehr schnell gegen Null.
    \begin{bsp}
        \(0\leq f_n < 1 \Rightarrow f_1 < \frac{1}{4} < f_2 < \frac{1}{40} < f_3 < \frac{1}{1600} < \ldots < f_6 < 10^{-30}\)
    \end{bsp}
    Gilt sowas auch für k-te Wurzeln?
    \begin{satz}
        Sei \(k \geq 2 \) eine natürliche Zahl und \(a > 0\) eine positive reelle Zahl, dann konvergiert für jeden Angangswert \(x_0 > 0\) die durch\\
        \(x_{x+1} := \frac{1}{k} \left((k-1)x_n + \frac{a}{x_n^{k-1}}\right)\)\\
        rekursiv definierte Folge \((x_n)_n\) gegen die eindeutig bestimmte positive Lösung der Gleichung \(x^k = a\)
    \end{satz}
    \begin{bew}
        Wie Satz 1 zeigt man, dass es nicht mehr als eine Lösung geben kann.\\Wissen also \(0 \leq z_1 < z_2 \Rightarrow z_1^k < z_2^k\)\\
        Nun zeigen wir, dass der Grenzwert \(x := \limes{n} x_n\) im Fall der Existenz die Gleichung \(x^k = a\) erfüllt.\\
        Wir multiplizieren die Rekursionsgleichung mit \(k \cdot x_n^{k-1}\)\\
        \(\Rightarrow k \cdot x_{n+1} \cdot x_n^{k-1} = (k-1)x_n^k + a\)\\
        Durch Grenzübergang gilt:\\
        \(k \cdot x^k = (k-1)x^k + a \Rightarrow x^k = a \)\\
        Es bleibt zu zeigen, dass die Folge \((x_n)_n\) konvergiert.
        \begin{align*}
            x_{n+1} &= \frac{1}{k}\left((k-1)x_n + \frac{a}{x_n^{k-1}}\right)\\
            &= \frac{1}{k} \cdot x_n \left((k-1) + \frac{a}{x_n^{k}}\right)\\
            &= x_n \left(1 + \frac{1}{k}\left(\frac{a}{x_n^k} - 1\right)\right) \text{ (\(\ast \ast\))}\\
            x_{n+1}^k &= x_n^k\left(1+\frac{1}{k}\left(\frac{a}{x_n^k} - 1 \right)\right)^k\\
            &\overset{\text{Bernoulli}}{\Rightarrow} x_n^k\left(1+\frac{1}{k}\left(\frac{a}{x_n^k} - 1 \right)\right)^k\\
            &\geq x_n^k\left(1+\left(\frac{a}{x_n^k} - 1 \right)\right)^k\\
            &= x_n^k + a - x_n^k = a\\
            \text{Also: } x_{n+1}^k &\geq a \Rightarrow \frac{a}{x_{n+1}^k} \leq 1 \forall n \geq 1\\
        \end{align*}
        Also \((x_n)_n\) monoton fallend und durch 0 nach unten beschänkt. Also existiert \(x = \limes{n} x_n\) nach Satz 8.2 und x erfüllt nach 2) die Gleichung \(x^k = a\)
    \end{bew}
\end{bsp}
\begin{bsp}
    Es seien \((f_n)_n\) die Folgen der Fibonacci Zahlen, weiter sei \(a_n := \frac{f_{n+1}}{f_n}\) für \(n \in \N\) und \(a := \frac{1}{2}(1 + \sqrt{5}), b := \frac{1}{2}(1 - \sqrt{5})\)\\
    \begin{enumerate}
        \item \(\frac{1}{\sqrt{5}}\left(a^n-b^n\right)\)
        \item \(\limes{n} a_n = a\)
    \end{enumerate}
    \begin{bew}
        \begin{enumerate}
            \item Aussage:\\
                Vollständige Induktion:\\
                (IA):
                \begin{align*}
                    &n = 0: \; f_0 = \frac{1}{\sqrt{5}} \left(a^0-b^0 \right) = 0\\		
                    &n = 1: \; f_1 = \frac{1}{\sqrt{5}} \left(a^1-b^1 \right) = \frac{1}{\sqrt{5}} \left(\frac{1}{2}(1 + \sqrt{5})-\frac{1}{2}(1 - \sqrt{5}) \right) = \frac{\sqrt{5}}{\sqrt{5}} = 1\\		
                \end{align*}
                (IS): \(n \rightarrow n+1\)\\
                Angenommen es gilt für ein \(n \in \N: f_k = \frac{1}{\sqrt{5}}\left(a^k-b^k\right) \forall k \geq n\)\\
                Dann folgt:\\
                \(f_{n+1} = f_n + f_n\)\\
                \(= \frac{1}{\sqrt{5}}\left[a^n - b^n +a^{n-1} - b^{n-1}\right]\)\\
                \(= \frac{1}{\sqrt{5}}\left[a^{n-1}(a+1) - b^{n-1}(b+1)\right]\)\\
                \(= \frac{1}{\sqrt{5}}\left[\left(\frac{1}{2}\left(1 + \sqrt{5}\right)\right)^{n-1} \cdot\left(\frac{1}{2}\left(1 + \sqrt{5}\right)+1\right) - \left(\frac{1}{2}\left(1 - \sqrt{5}\right)\right)^{n-1}\cdot\left(\frac{1}{2}\left(1 - \sqrt{5}\right)+1\right)\right]\)\\
                \(= \frac{1}{\sqrt{5}}\left[\left(\frac{1}{2}\left(1 + \sqrt{5}\right)\right)^{n-1}\cdot\left(\frac{1}{4}\left(6 + 2\sqrt{5}\right)\right) - \left(\frac{1}{2}\left(1 - \sqrt{5}\right)\right)^{n-1}\cdot\left(\frac{1}{4}\left(6 - 2\sqrt{5}\right)\right)\right]\)\\
                \(= \frac{1}{\sqrt{5}}\left[\left(\frac{1}{2}\left(1 + \sqrt{5}\right)\right)^{n-1}\cdot\left(\frac{1}{4}\left(1 + 2\sqrt{5} + 5\right)\right) - \left(\frac{1}{2}\left(1 - \sqrt{5}\right)\right)^{n-1}\cdot\left(\frac{1}{4}\left(1 - 2\sqrt{5} + 5\right)\right)\right]\)\\
                \(= \frac{1}{\sqrt{5}}\left[\left(\frac{1}{2}\left(1 + \sqrt{5}\right)\right)^{n-1}\cdot\left(\frac{1}{4}\left(1+\sqrt{5}\right)^2\right) - \left(\frac{1}{2}\left(1 - \sqrt{5}\right)\right)^{n-1}\cdot\left(\frac{1}{4}\left(1 - \sqrt{5}\right)^2\right)\right]\)\\
                \(= \frac{1}{\sqrt{5}}\left[\left(\frac{1}{2}\left(1 + \sqrt{5}\right)\right)^{n+1} - \left(\frac{1}{2}\left(1 - \sqrt{5}\right)\right)^{n+1}\right]\)\\
                \(= \frac{1}{\sqrt{5}}\left[a^{n+1} - b^{n+1}\right]\)\\
                Aus dem Induktionsprinzip folgt somit \(f_n = \frac{1}{\sqrt{5}}\left(a^n-b^n\right) \forall n \in \N \)
            \item Aussage:\\
            \begin{align*}
                a_n = \frac{f_{n+1}}{f_n} \overset{\text{1)}}{=} \frac{a^{n+1} - b^{n+1}}{a^n - b^n} = \frac{a - b\left(\frac{b^n}{a^n}\right)}{1 - \left(\frac{b^n}{a^n}\right)} = \frac{a - b\left(\frac{b}{a}\right)^n}{1 - \left(\frac{b}{a}\right)^n}
            \end{align*}
            Wissen \(a > |b| \Rightarrow \frac{|b|}{a} < 1 \Rightarrow \left(\frac{b}{a}\right)^n \overset{n \rightarrow \infty}{\rightarrow} 0\)\\
            Nach Satz 7.1.8 gilt:
            \begin{align*}
                \limes{n} a_n = \limes{n} \frac{a - b\left(\frac{b}{a}\right)^n}{1 - \left(\frac{b}{a}\right)^n} = \frac{a}{1} = a
            \end{align*}
        \end{enumerate}
    \end{bew}
\end{bsp}

\end{document}