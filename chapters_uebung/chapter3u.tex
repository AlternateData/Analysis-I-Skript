\documentclass[../ana1u.tex]{subfiles}
\begin{document}
\setcounter{section}{2}

\section{Mehr zu Wurzeln (09.11.18)}
\begin{bsp} [Exkurs]
    \begin{beh}
        \(\sqrt{2} \in \R \setminus \Q\)
    \end{beh}
    \begin{bew}
        Beweis durch Widerspruch:
        \[\exists \, r \in \Q \text{ mit } r^2 = 2\]
        Sei \(r^2 = 2\) mit \(r = \frac{m}{n}\), \(n \in \N\, m \in \Z\)\\
        Wir definieren:
        \[A := \{n \in \N: \; \exists \, m \in \Z: \frac{m^2}{n^2} = 2\} \neq \emptyset\]
        \[\overset{\text{Satz 5.1.2}}{\Rightarrow} n_* = \min A \in \N \text{ also existiert } m \in \Z_+ \text{ mit}\]
        \[m^2 = 2n_*^2 \Rightarrow m > n_* \]
        Außerdem gilt:
        \[m = \sqrt{2}n_* \overset{\sqrt{2} > 1}{\Leftrightarrow} 0 < \underset{\in \N}{m - n_*} = \underset{> 0 \text{ und } < 1}{(\sqrt{2} - 1)}n_* < n_*\]
        \[\sqrt{2} = \frac{m}{n_*} = \frac{m(m - n_*)}{n_*(m - n_*)} = \frac{m^2 - mn_*}{mn_* - n_*^2} = \frac{2n_*^2 - mn_*}{n_*(m-n_*)} = \frac{2n_*-m}{m-n_*} \text{ \Lightning}\]
        weil:\\
        \(2n_* - m \in \Z\)\\
        \(m - n_* < n_*\) aber\\
        \(n_* = \min A\)\\
        Somit kann kein \(m \in \Z\) existieren, dass \(\frac{m^2}{n^2} = 2\) für beliebiges \(n \in \N\). Also ist \(\sqrt{2}\) per Definition der rationalen Zahlen in \(\R \setminus \Q\).
    \end{bew}
    \begin{satz}
        Sei \(k \in \N\), dann gilt entweder \(\sqrt{k} \in \N\) oder \(\sqrt{k} \in \R \setminus \Q\).
    \end{satz}
    \begin{bew}
        Sei \(k \in \N\) und \(\sqrt{k} \not\in \N \) 
        Angenommen: \(\sqrt{k} \in \Q\), also \(\sqrt{k} = \frac{m}{n} \; n,m \in \Z, 
        n \in \N\)
        \[A := \{n \in \N: \; \exists \, m \in \Z: \frac{m^2}{n^2} = k\} \neq \emptyset\]
        \[\overset{\text{Satz 5.1.2}}{\Rightarrow} n_* = \min A \in \N\]
        sei \(\frac{m}{n_*} = \sqrt{k}\), dann gilt \(m - n_* = \underbrace{(\sqrt{k}-1)}_{<1}n_*\)\\
        Aber wähle \(q \in \N: q \leq \sqrt{k} < q + 1\) existiert nach Lemma 5.2.1\\
        Da \(\sqrt{k} \not \in \N\) gilt \(q < \sqrt{k} < q + 1\)\\
        Also gilt: \(0 < m - qn_* = \underbrace{(\sqrt{k}-q)}_{<1}n_*\)\\
        Somit
        \[\sqrt{k} = \frac{m}{n_*} = \frac{m(m - qn_*)}{n_*(m - qn_*)} = \frac{kn_*^2 - qmn_*}{n_*(m-qn_*)} = \frac{kn_* - mq}{m - qn_*} \text{ \Lightning}\]
        weil \(n_* = \min A\) und \(m - qn_* < n_*\)
    \end{bew}
\end{bsp}

\end{document}