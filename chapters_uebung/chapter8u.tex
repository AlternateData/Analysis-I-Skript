\documentclass[../ana1u.tex]{subfiles}
\begin{document}
\setcounter{section}{7}

\section{Reihenkonvergenz (14.12.18)}
\begin{bsp}
    Grenzwert von:
    \[\sum_{n=1}^{\infty} \frac{1}{4n^2 - 1}\]
    Für \(n \geq 1\) haben wir folgende Zerlegung:
    \[\frac{2}{4n^2 - 1} = \frac{2}{(2n - 1)(2n + 1)} = \frac{1}{2n-1} - \frac{1}{2n+1}\]
    \[s_k = \sum_{n=1}^{k} \frac{1}{4n^2-1} = \frac{1}{2}\left(\sum_{n=1}^{k} \frac{1}{2n-1} 
    - \sum_{n=1}^{k} \frac{1}{2n+1} \right) \]
    \[= \frac{1}{2}\left(\sum_{n=0}^{k-1} \frac{1}{2n+1} 
    - \sum_{n=1}^{k} \frac{1}{2n+1} \right) \]
    \[= \frac{1}{2}\left(1 - \frac{1}{2k+1}\right) \]
    Somit gilt:
    \[\sum_{n=1}^{\infty} \frac{1}{4n^2 - 1} = \limes{k} s_k 
    = \limes{k} \frac{1}{2}\left(1 - \frac{1}{2k+1}\right) = \frac{1}{2} \]
\end{bsp}
\begin{bsp}
    Konvergenz von:
    \[\sum_{n=1}^{\infty} \frac{1}{n^2+1} \left(= \frac{1}{2}(1 + \pi \coth(\pi)) \right) \]
    Wir können den Grenzwert noch nicht berechnen, aber wir können feststellen, ob die 
    Reihe konvergiert oder nicht. Mittel dafür ist das Majorantenkriterium. \\
    Sei \(\sum_{n=0}^{\infty}c_n \) eine konvergente Reihe mit den nicht-negativen 
    Reihengliedern und \((a_n)_n \) eine Folge mit
    \[|a_n| < c_n \; \forall \, n \in \N \Rightarrow \sum_{n=0}^{\infty} a_n 
    \text{ absolut konvergent} \]
    \[\left|\frac{1}{n^2+1}\right| < \frac{1}{n^2} \]
    \(\Rightarrow \sum_{n=1}^{\infty} \frac{1}{n^2} \) konvergiert absolut, da 
    \(\sum_{n=1}^{\infty} \frac{1}{n^2} \) nach Volesung konvergent ist.
\end{bsp}
\begin{bsp}
    Konvergenz von:\[\sum_{n=1}^{\infty} (-1)^n \cdot \binom{2n}{n} \]
    \((a_n)_n\) mit \(a_n = (-1)^n \cdot \binom{2n}{n} \; \forall \, n \). Dann gilt
    \[|a_n| = \frac{(2n)!}{(n!)^2} = \frac{(n+1)(n+2) \cdot \hdots \cdot (n+n)}{1 \cdot 2 
    \cdot \hdots \cdot n} \geq 1 \]
    Somit und da \((a_n)_n\) keine Nullfolge ist, kann die Reihe 
    \(\sum_{n=1}^{\infty} (-1)^n \cdot \binom{2n}{n}\) nicht konvergieren.
\end{bsp}
\begin{bsp}
    Konvergenz von: \\
    Wir definieren \((a_n)_n\) mit \(a_n = \frac{1}{n\sqrt{n}} \; \forall \, n \in \N \)
    \[\sqrt{n} < \sqrt{n+1} \text{ und } n < n+1 
    \Rightarrow \frac{1}{n\sqrt{n}} > \frac{1}{(n+1)\sqrt{n+1}} \]
    somit ist \((a_n)_n\) monoton fallend. Aus dem Cauchy'schen Verdichtungskriterium folgt, 
    dass \(\sum_{n=1}^{\infty} \frac{1}{n\sqrt{n}} \) genau dann konvergiert, wenn die 
    verdichtete Reihe
    \[\sum_{n=1}^{\infty} 2^n a_{2n} = \sum_{n=1}^{\infty} 2^n \frac{1}{2^n\sqrt{2^n}} 
    = \sum_{n=1}^{\infty} \frac{1}{\sqrt{2^n}} \]
    Die Konvergenz von \(\sum_{n=1}^{\infty} \frac{1}{\sqrt{2^n}}\) folgt aus dem Quotientenkriterium
    \[\frac{\left|\frac{1}{\sqrt{2^{n+1}}}\right|}{\left|\frac{1}{\sqrt{2^n}}\right|} 
    = \frac{\sqrt{2^n}}{\sqrt{2^{n+1}}} = \frac{1}{\sqrt{2^n}} < 1 \]
    Da \(a_n > 0\) für alle \(n \in \N\) folgt somit, dass 
    \(\sum_{n=1}^{\infty} \frac{1}{n\sqrt{n}} \) absolut konvergiert.
\end{bsp}
\begin{satz}[Quotientenkriterium]
    Sei \(\sum_{n=0}^{\infty} a_n\) eine Reihe mit \(a_n \neq 0\) für alle 
    \(n \in \N: n \geq n_0 \). \\
    Es gebe eine reelle Zahl \(\varphi\) mit \(0 \leq \varphi < 1\), sodass 
    \(\left|\frac{a_{n+1}}{a_n}\right| \leq \varphi \; \forall \, n \geq n_0 \) \\
    Dann konvergiert die Reihe \(\sum_{n=0}^{\infty} a_n \) absolut. \\
    \textbf{Wichtig:} \\
    Die Bedingung im Quotientenkriterium lautet nicht 
    \[\left|\frac{a_{n+1}}{a_n}\right| \leq 1 \; \forall \, n \geq n_0 \]
    sondern
    \[\left|\frac{a_{n+1}}{a_n}\right| \leq \varphi \; \forall \, n \geq n_0 \]
    für \(\varphi < 1\) unabhängig von \(n \).\\
    Dass die erste Bedingung nicht ausreicht, kann man anhand von 
    \(\sum_{n=0}^{\infty} \frac{1}{n} \) leicht sehen. \\
    Mit \(a_n = \frac{1}{n} \) gilt: 
    \[\left|\frac{a_{n+1}}{a_n}\right| 
    = \frac{n}{\underbrace{n+1}_{\rightarrow \infty \; (n \rightarrow \infty)}} < 1 
    \; \forall \, n \geq 1\]
    Auch bei bekannten konvergenten Reihen ist das Quotientenkriterium unter Umständen 
    nicht anwendbar.
    \[\sum_{n=0}^{\infty} \frac{1}{n^2}; \quad \left|\frac{a_{n+1}}{a_n}\right| 
    = \frac{n^2}{(n+1)^2} < 1 \; \forall \, n \geq 1 \]
    Somit ist das Quotientenkriterium nur eine hinreichende, aber keine notwendige Bedingung 
    für die Konvergenz von Reihen.
\end{satz}
\begin{lem}
    Sei \((a_n)_n \) eine reelle Folge mit \(a_n \neq 0 \) für \(n \in \N \). 
    Dann sind die folgenden Aussagen äquivalent:
    \begin{enumerate}
        \item 
            Die Reihe \(\sum_{n=0}^{\infty} a_n\) konvergiert absolut
        \item
            Es gibt eine Folge \((c_n)_n \) mit \(c_n > 0 \; \forall \, n \in \N \), 
            sodass die Reihe \(\sum_{n=0}^{\infty} c_n \) konvergiert und
            \(\frac{|a_{n+1}|}{|a_n|} \leq \frac{c_{n+1}}{c_n} \) gilt für fast alle 
            \(n \in \N \).
    \end{enumerate}
\end{lem}
\begin{bew}
    (1) \(\Rightarrow \) (2): \\
    Die Reihe \(\sum_{n=1}^{\infty} a_n \) konvergiert absolut. Wähle \(c_n = |a_n| \) 
    für fast alle \(n \in \N \), d.h. es gilt (2). \\
    (2) \(\Rightarrow \) (1): \\
    Es existiert eine Folge \((c_n)_n \) mit \(c_n > 0 \) für fast alle \(n \in \N \). \\
    \(\sum_{n=1}^{\infty} c_n\) konvergiert und \(\left|\frac{a_{n+1}}{a_n}\right| 
    \leq \frac{c_{n+1}}{c_n}\) für alle \(n \geq n_0 \) \\
    Für alle \(m \geq n_0 \) folgt:
    \[\left|\frac{a_{m}}{a_n}\right| = \prod_{n=n_0}^{m-1} \frac{c_{n+1}}{c_n} 
    = \frac{c_{m}}{c_n} \]
    Dann gilt für alle \(m \geq n_0 \)
    \[|a_m| \leq \frac{|a_{n_0}|}{\underbrace{c_{n_0}}_{c \in \R}} \cdot c_m \]
\end{bew}
\begin{bem}
    \(\sum_{n=1}^{\infty} |a_n| \leq c \cdot \sum_{n=1}^{\infty} c_n \leq \tilde{c} \in \R \) \\
    Somit erhalten wir die absolute Konvergenz der Reihe \(\sum_{n=1}^{\infty} a_n\) 
    mit dem Majorantenkriterium.
\end{bem}
\begin{bem}
    Quotientenkriterium \\
    Das Quotientenkriterium ist nicht anwendbar, wenn 
    \(\limes{n} \left|\frac{a_{n+1}}{a_n}\right| = 1 \) \\
    Aber es gibt eine Verschärfung bzw. Abschwächung der Bedingung aus dem 
    Quotientenkriterium.
\end{bem}
\begin{kor}
    Sei \((a_n)_n \) eine reelle Folge mit \(a_n \neq 0 \; \forall \, n \in \N \), dann gilt:
    \[\left|\frac{a_{n+1}}{a_n}\right| \leq 1 - \frac{p}{n} 
    \text{ für } p > 1 \text{ und fast alle } n \in \N \]
    so ist die Reihe \(\sum_{n=1}^{\infty} a_n \) absolut konvergent.
\end{kor}
\begin{bew}
    Wir definieren die Folge \((c_n)_n \) durch \(c_1 := 1 \) und 
    \(c_{n+1} := \frac{1}{n^p} \; \forall \, n \in \N \). \\
    Aus der Vorlesung ist bekannt, dass \(\sum_{n=1}^{\infty} c_n \) genau dann konvergiert, 
    wenn \(p > 1 \) ist. \\
    Weiter gilt für \(n \geq 2 \):
    \[\frac{c_{n+1}}{c_n} = \frac{(n-1)^p}{n^p} 
    = \left(\frac{n-1}{n}\right)^p \overset{\text{Bernoulli}}{\geq} 1 - \frac{p}{n} \]
    Nach Voraussetzung gilt somit:
    \[\left|\frac{a_{n+1}}{a_n}\right| \leq 1 - \frac{p}{n} 
    \leq \frac{c_{n+1}}{c_n} \text{ für fast alle } n \in \N \]
    Nach dem Lemma ist die Reihe \(\sum_{n=1}^{\infty} a_n \) somit absolut konvergent.
\end{bew}

\end{document}