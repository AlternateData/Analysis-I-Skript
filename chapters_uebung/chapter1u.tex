\documentclass[../ana1u.tex]{subfiles}
\begin{document}
\setcounter{section}{0}

\section{Aussagenlogik (19.10.18)}
\begin{defi}
    Eine Aussage ist die gedankliche Widerspiegelung eines Sachverhalts in Form eines Satzes in einer natürlichen oder künstlichen Sprache.
    \vspace{5mm}\\
    Jede Aussage ist entweder wahr oder falsch.
\end{defi}
\begin{bsp}
    Aussagen:
    \begin{enumerate}
        \item Wenn es regnet, dann ist die Straße nass.
        \item Vögel können fliegen
    \end{enumerate}
\end{bsp}
\textbf{Aussagen verknüpfen:}
\begin{align*}
    \neg &A\\
    A &\wedge B\\
    A &\vee B\\
    A &\rightarrow B\\
    A &\Leftrightarrow B
\end{align*}
\textbf{Wahrheitstafel:}
Seien A und B Variablen die entweder wahr oder falsch sind.	
\begin{table}[H]
    \centering
    \begin{tabular}{c | c}			
        A & \(\neg A\) \\ 
        \hline
        \textbf{t} & \textbf{f} \\
        \textbf{f} & \textbf{t}
    \end{tabular}
    \caption{Negation}
\end{table}	
\begin{table}[H]
    \centering
    \begin{tabular}{c | c | c | c | c | c}			
        A & B & \(A \wedge B\) & \(A \vee B\) & \(A \rightarrow B\) & \(A \Leftrightarrow B\) \\
        \hline
        \textbf{w} & \textbf{w} & \textbf{w} & \textbf{w} & \textbf{w} & \textbf{w} \\
        \textbf{w} & \textbf{f} & \textbf{f} & \textbf{w} & \textbf{f} & \textbf{f} \\
        \textbf{f} & \textbf{w} & \textbf{f} & \textbf{w} & \textbf{w} & \textbf{f} \\
        \textbf{f} & \textbf{f} & \textbf{f} & \textbf{f} & \textbf{w} & \textbf{w}
        
    \end{tabular}
    \caption{Konjunktion, Disjunktion, Implikation, Äquivalenz}
\end{table}
\textbf{Ausdrücken der Aussagenlogik:}
\begin{enumerate}
    \item Konstanten und Variablen sind Ausdrücke
    \item Sind A und B Variablen, so sind auch (siehe Wahrheitstabelle) Ausdrücke
\end{enumerate}
\subsection{Grundgesetze der Aussagenlogik}
\begin{enumerate}
    \item Assoziativgesetz:
    \[(A \wedge B) \wedge C \Leftrightarrow A \wedge (B \wedge C)\]
    \[(A \vee B) \vee C \Leftrightarrow A \vee (B \vee C)\]
    \item Kommutativgesetz:
    \[(A \wedge B) \Leftrightarrow (B \wedge A)\]
    \[(A \vee B) \Leftrightarrow (B \vee A)\]
    \item Distributivgesetz:
    \[(A \wedge B) \vee C \Leftrightarrow (A \vee C) \wedge (B \vee C)\]
    \[(A \vee B) \wedge C \Leftrightarrow (A \wedge C) \vee (B \wedge C)\]
    \item Absorptionsgesetz:
    \[A \wedge (A \vee B) \Leftrightarrow A\]
    \[A \vee (A \wedge B) \Leftrightarrow A\]
    \item Idempotenzgesetze:
    \[A \wedge A \Leftrightarrow A\]
    \[A \vee A \Leftrightarrow A\]
    \item ausgeschlossene Dritte:
    \[A \wedge \neg A \Leftrightarrow \textbf{f}\]
    \[A \vee \neg A \Leftrightarrow \textbf{t}\]
    \item de Morgan'sche Gesetze:
    \[\neg (A \wedge B) \Leftrightarrow (\neg A \vee \neg B)\]
    \[\neg (A \vee B) \Leftrightarrow (\neg A \wedge \neg B)\]
    \item Gesetz für true und false:
    \[A \wedge \textbf{t} \Leftrightarrow A\]
    \[A \wedge \textbf{f} \Leftrightarrow \textbf{f}\]
    \[A \vee \textbf{t} \Leftrightarrow \textbf{t}\]
    \[A \vee \textbf{f} \Leftrightarrow A\]
    \[\neg \textbf{f} \Leftrightarrow \textbf{t}\]
    \[\neg \textbf{t} \Leftrightarrow \textbf{f}\]
    \item doppelte Verneinung:
    \[\neg (\neg A) \Leftrightarrow A\]
    \item Gesetz für das Konditional:
    \[A \Rightarrow B \Leftrightarrow (\neg A \vee B)\]
    \[A \Rightarrow B \Leftrightarrow \neg (A \wedge \neg B)\]
    \[A \Rightarrow B \Leftrightarrow \neg B \Rightarrow \neg A\]
    \item Gesetz der Komplementarität:
    \[A \vee \neg A \Leftrightarrow \textbf{t}\]
    \[A \wedge \neg A \Leftrightarrow \textbf{f}\]
\end{enumerate}	
\begin{defi}
    Zwei Aussagenlogische Ausdrücke heißen logisch äquivalent oder wertgleich (A = B), wenn sie die gleichen Wahrheitswerte besitzen.
\end{defi}
\begin{bsp}
    \begin{align*}
        (A \wedge B) \vee C &= (A \vee C) \wedge (B \vee C)\\
        \Leftrightarrow (A \wedge B) \vee C &= (B \wedge (A \vee C)) \vee (C \wedge (A \vee C)))\\
        \Leftrightarrow (A \wedge B) \vee C &=  (B \wedge A) \vee ((B \wedge C) \vee C)\\
        \Leftrightarrow (A \wedge B) \vee C &= (A \wedge B) \vee C
    \end{align*}		
\end{bsp}
\begin{bsp}
    \begin{align*}
        \neg(A \wedge B) &= (\neg A \vee \neg B)\\
        (\neg A \vee \neg B) &= (\neg A \vee \neg B)
    \end{align*}
    Somit sind \(\neg(A \wedge B)\) und \((\neg A \vee \neg B)\) logisch äquivalent und es gilt:
    \begin{align*}
        \neg(A \wedge B) = (\neg A \vee \neg B)
    \end{align*}
\end{bsp}
\subsection{Tautologie}
\begin{defi}
    Ein Aussagenlogischer Ausdruck heißt allgemeingültig oder Tautologie, wenn die Wahrheitsfunktion identisch true ist.
\end{defi}	
\begin{bsp}
    Kontraposition:\\
    \begin{align*}
        A \rightarrow B \Leftrightarrow \neg B \rightarrow \neg A
    \end{align*}
    \begin{align*}
        ((A \rightarrow B) \rightarrow (\neg B \rightarrow \neg A)) &\wedge ((\neg B \rightarrow \neg A) \rightarrow (A \rightarrow B))\\
        ((\neg A \vee B) \rightarrow (B \vee \neg A)) &\wedge ((B \vee \neg A) \rightarrow (\neg A \vee B))\\
        (\neg (\neg A \vee B) \vee (B \vee \neg A)) &\wedge (\neg (B \vee \neg A) \vee (\neg A \vee B))\\
        ((A \wedge \neg B) \vee (\neg A \vee B)) &\wedge ((A \wedge \neg B) \vee (\neg A \vee B))\\
        ((A \vee (\neg A \vee B)) \wedge (\neg B \vee (\neg A \vee B))) &\wedge ((A \vee (\neg A \vee B)) \wedge (\neg B \vee (\neg A \vee B)))\\
        \textbf{t} &\wedge \textbf{t}\\
        &\textbf{t}
    \end{align*}
\end{bsp}

\end{document}