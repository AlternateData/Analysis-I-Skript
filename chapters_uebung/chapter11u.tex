\documentclass[../ana1u.tex]{subfiles}
\begin{document}
\setcounter{section}{10}

\section{Grenzwerte und gleichmäßige Stetigkeit (25.01.19)}
\subsection{Grenzwerte}
\begin{bsp}
    Grenzwerte berechnen:
    \begin{enumerate}
        \item
            \[\limesx{x}{1} \frac{x^n-1}{x-1} = \limesx{x}{1} \sum_{k=0}^{n-1} x^k = n \]
        \item
            \[\limes{x} ((x^{10}+10^{10})^{\minus{1}} \cdot \sum_{k=0}^{1000} (x+k)^{10}) 
            = \limes{x} \frac{\sum_{k=0}^{1000} \left(1+\frac{k}{x}\right)^{10} \cdot x^{10}}
            {\left(1+\left(\frac{10}{x}\right)\right)^{10} \cdot x^{10}} = 1001 \]
        \item 
            Beachte:
            \[2+x\sqrt{1+\frac{4}{x^2}} = 2+x\sqrt{\frac{1}{x^2}(x^2+4)} 
            = 2 + \frac{x}{|x|} \cdot \sqrt{x^2+4}\]
            Dann sind:
            \[\limesx{x}{0+} \left(2+x\sqrt{1+\frac{4}{x^2}}\right)
            =\limesx{x}{0+} 2 + \frac{x}{|x|} \cdot \sqrt{x^2+4} = 4 \]
            und 
            \[\limesx{x}{0-} \left(2+x\sqrt{1+\frac{4}{x^2}}\right)
            =\limesx{x}{0-} 2 + \frac{x}{|x|} \cdot \sqrt{x^2+4} = 0 \]
    \end{enumerate}
\end{bsp}
\subsection{Gleichmäßige Stetigkeit}
\begin{beh}
    Sei \(f:\R \rightarrow \R \) stetig und die Grenzwerte \(\limes{x} f(x)\) und 
    \(\limesx{x}{-\infty} f(x) \) existieren. \\
    Dann ist \(f\) gleichmäßig stetig
\end{beh}
\begin{bew}
    Setze \(\alpha := \limes{x} f(x) \) und \(\beta := \limesx{x}{-\infty} f(x) \) \\
    Sei \(\varepsilon > 0 \). Wähle \(\xi, r > 0 \) mit
    \[|f(x) - \alpha| < \frac{\varepsilon}{4} \; \forall \, x \geq \xi \]
    \[|f(x) - \beta| < \frac{\varepsilon}{4} \; \forall \, x \leq -r \]
    Sind \(x,y \geq \xi \) dann folgt:
    \[|f(x)-f(y)| \leq |f(x) - \alpha| + |\alpha - f(y)| < \frac{\varepsilon}{4} + \frac{\varepsilon}{4} \]
    Genauso folgt:
    \[|f(x)-f(y)| \leq |f(x)-\beta| + |\beta-f(y)| < \frac{\varepsilon}{2} \]
    Insbesondere gilt:
    \[|f(x)-f(y)| < \varepsilon \; \forall \, x,y \in \R \text{ mit } x,y \geq \xi \text{ oder } x,y \leq -r \]
    Da \(f\) stetig ist und \([r, \xi] \) beschränkt und abgeschlossen ist, folgt dass \(f\) gleichmäßig
    stetig auf \([-r,\xi] \) ist. \\
    Es gibt somit ein \(\delta > 0 \), sodass \(\forall \, x,y \in [-r, \xi] \) gilt
    \[|x-y| < \delta \Rightarrow |f(x)-f(y)| < \frac{\varepsilon}{2} \]
    Seien \(x,y \in \R \) mit \(|x-y| < \delta \) \\
    Im Fall \(x,y \in [-r, \xi] \) gilt \(|f(x)-f(y)| < \frac{\varepsilon}{2} < \varepsilon \) \\
    Genauso für \(x,y \in (-\infty, -r] \) und \(x,y \in [\xi, \infty) \) \\
    Sei \(x \in [-r, \xi] \) und \(y > \xi \), dann gilt folgendes 
    \[|x-\xi| = \xi - x \leq \xi - x+y - \xi = |x-y| < \delta \]
    Sei \(y, \xi \geq \xi \) \\
    Damit folgt:
    \[|f(x)-f(y)| \leq |f(x)-f(\delta)| + |f(\delta)-f(y)| < \frac{\varepsilon}{2} + \frac{\varepsilon}{2}
    = \varepsilon \]
    Analog \(x < -r, y \in [-r,\xi] \): \\
    Die Fälle \(y \in [-r, \xi] \) und \(x > \xi \) sowie \(y < -r \) und \(x \in [-r, \xi] \) behandelt
    man durch vertauschen von \(x \) und \(y \).
\end{bew}
\begin{defi*}
    Lipschitz-Stetigkeit \\
    Eine Funktion ist Lipschitz-Stetig (oder dehnungsbeschränkt) auf \(D \subset \R \) (\(f:D \rightarrow \R \)),
    wenn eine Konstante \(L \geq 0 \) existiert, sodass
    \[|f(x)-f(y)| \leq L|x-y| \; \forall \, x,y \in D \]
    \(L \) nennt man Lipschitzkonstante.
\end{defi*}
\begin{bsp}
    \(\sin: \R \rightarrow \R \) ist Lipschitzstetig \\
    Aus den Additionstheoremen folgt:
    \[\sin(2x) = 2 \cos(x)\sin(x) \]
    \[\cos\left(\frac{x+y}{2}\right) = \cos\left(\frac{x}{2}\right) \cos\left(\frac{y}{2}\right)
    - \sin\left(\frac{x}{2}\right) \sin\left(\frac{y}{2}\right)\]
    \[\sin\left(\frac{x+y}{2}\right) = \sin\left(\frac{x}{2}\right) \cos\left(\frac{y}{2}\right)
    - \cos\left(\frac{x}{2}\right) \sin\left(\frac{y}{2}\right)\]
    und wir wissen
    \[\cos^2(x) + \sin^2 (x) = 1 \]
    Wir rechnen:
    \[|\sin(x) - \sin(y)| = \left|2\cos\left(\frac{x}{2}\right)\sin\left(\frac{x}{2}\right)
    - 2\cos\left(\frac{y}{2}\right)\sin\left(\frac{y}{2}\right) \right| \]
    \[= 2 \bigg \vert \cos\left(\frac{x}{2}\right)\sin\left(\frac{x}{2}\right) \cdot
    \left[\cos^2\left(\frac{y}{2}\right) + \sin^2 \left(\frac{y}{2}\right)\right] \]
    \[- \left[\cos^2\left(\frac{x}{2}\right) + \sin^2 \left(\frac{x}{2}\right)\right]  
    \cdot \cos\left(\frac{y}{2}\right)\sin\left(\frac{y}{2}\right) \bigg \vert \]
    \[= 2 \bigg \vert \cos\left(\frac{x}{2}\right)\cos\left(\frac{y}{2}\right) \cdot 
    \left[\sin\left(\frac{x}{2}\right)\cos\left(\frac{y}{2}\right) - \cos \left(\frac{x}{2}\right)
    \sin\left(\frac{y}{2}\right)\right]\]
    \[+ \sin\left(\frac{x}{2}\right)\sin\left(\frac{y}{2}\right) \cdot \left[\cos\left(\frac{x}{2}\right)
    \sin\left(\frac{y}{2}\right) - \sin\left(\frac{x}{2}\right)\cos\left(\frac{y}{2}\right)\right] \bigg \vert \]
    \[= 2\left|\cos\left(\frac{x+y}{2}\right)\right| \cdot \left|\sin \left(\frac{x-y}{2}\right)\right| 
    \leq 2 \cdot \frac{|x-y|}{2} = |x-y| \]
    \[\Rightarrow \sin: \R \rightarrow \R \text{ ist Lipschitzstetig mit } L = 1 \]
\end{bsp}
\begin{beh}
    Lipschitzstetige Funktionen sind gleichmäßig stetig.
\end{beh}
\begin{bew}
    Sei \(D \subset \R, f: D \rightarrow \R \) Lipschitzstetig \\
    Dann gilt:
    \[|f(x)-f(y)| \leq L|x-y| \; \forall \, x,y \in D \]
    Somit existiert zu beliebigen \(\varepsilon > 0 \) ein \(0 < \delta = \frac{\varepsilon}{L} \),
    sodass \(\forall \, x,y \in D \) mit \(|x-y| < \delta \) gilt 
    \[|f(x) - f(y)| \leq L |x-y| < L \delta = \varepsilon \]
\end{bew}
\begin{beh}
    \(f:[0,1] \rightarrow \R; x \mapsto \sqrt{x} \) ist gleichmäßig stetig, aber nicht Lipschitzstetig
\end{beh}
\begin{bew}
    \(f \) ist auf \([0,1] \) gleichmäßig stetig, da jede stetige Funktion auf einen kompakten Intervall
    gleichmäßig stetig ist. \\
    Direkter Beweis: \\
    Sei \(\varepsilon > 0 \) und \(\delta = \varepsilon^2 \). Weiter sei \(x,y \in [0,1] \) beliebig mit
    \(|x-y| < \delta \). \\
    Dann gilt:
    \[|\sqrt{x}-\sqrt{y}| \leq \sqrt{|x-y|} < \sqrt{\delta} = \varepsilon \]
    Somit ist \(f:[0,1] \rightarrow \R; x \mapsto \sqrt{x} \) gleichmäßig stetig. \\
    Nun zur Lipschitzstetigkeit: \\
    Sei \(L \geq 0 \) beliebig. Setze \(x := 0, y := \frac{1}{(L+1)^2} \in [0,1] \) \\
    Dann ist \(x,y \in [0,1] \) und es gilt:
    \[|\sqrt{x}-\sqrt{y}| = \sqrt{\frac{1}{(L+1)^2}} = \frac{1}{L+1} = (L+1) \cdot \frac{1}{(L+1)^2}
    > L \cdot |x-y| \]
    \[\Rightarrow f:[0,1] \rightarrow \R; x \mapsto \sqrt{x} \text{ ist nicht Lipschitzstetig} \]
\end{bew}
\begin{bsp}
    \(f(x) = \frac{1}{x^2} \) mit
    \begin{enumerate}
        \item \(D = (0,1) \) nicht gleichmäßig stetig
        \item \(D = \left[\frac{1}{10}, 1\right] \) gleichmäßig stetig
        \item \(D = (1,\infty) \) Lipschitzstetig \(L=2 \)
        \item \(f(x) = x \sin(x) \) auf \(D(f) = [0, \infty) \) ist nicht gleichmäßig stetig \\
            (\(x_n = 2 \pi n \))
    \end{enumerate}
\end{bsp}
\end{document}