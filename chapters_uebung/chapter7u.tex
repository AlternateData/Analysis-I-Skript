\documentclass[../ana1u.tex]{subfiles}
\begin{document}
\setcounter{section}{6}

\section{Limes Superior und Inferior (07.12.18)}
\begin{defi}
    Sei \(a_n \) eine reelle Folge
    \[\limessup{n} a_n = \limes{n} (\underset{k \geq n}{sup(a_k)}) \]
    \[\limesinf{n} a_n = \limes{n} (\underset{k \geq n}{inf(a_k)}) \]
\end{defi}
\begin{satz}
    Charakterisierung des Limes superior \\
    Sei \(a_n\) eine reelle Folge und \(a \in \R\) \\
    Genau dann gilt:
    \[\limessup{n} a_n = a\]
    Wenn für \(\varepsilon > 0\) die folgenden Bedingungen erfüllt sind.
    \begin{enumerate}
        \item Für fast alle Indizes \(n \in \N\) (d.h für alle bis auf endlich viele) gilt: 
        \(a_n < a+ \varepsilon\)
        \item Es gibt unendlich viele Indizes \(m \in \N\) mit \(a_m > a - \varepsilon\)
    \end{enumerate}
\end{satz}
\begin{bew}
    Wir definieren: \\
    \(s_n := \underset{k \geq n}{sup(a_k)} \) \\
    Für alle \(n \in \N\) gilt: \(\{k \in \N: k \geq n+1\} \subset \{k \in \N: k \geq n\}\) 
    also ist \\
    \(s_{n+1} = sup\{a_k: k \in \N, k \geq n+1\} \subseteq sup\{a_k: k \in \N, k \geq n\} 
    = s_n \) \\
    und damit ist \((s_n)_n\) monoton fallend. \\
    \(\Rightarrow \): Sei \(\limes{n} a_n = a \), d.h. \(\limes{n} s_n = a \) \\
    Weiter \(\varepsilon \geq 0 \) \\
    Da \((s_n)_n \) monoton fallend ist, ist \(s_n \geq a \; \forall \, n \in \N \) \\
    Somit ist Bedingung 2. erfüllt. Weiter muss es ein \(N \in \N\) geben, so dass 
    \(s_n < a + \varepsilon \) für \(n \geq N \) \\
    d.h. \(a_n < a + \varepsilon \) für \(n \geq N \) Also gilt auch 1. \\
    \(\Leftarrow \): seinen Bedingungen 1. und 2. erfüllt. \\
    Aus 2. folgt, \(s_n > a - \varepsilon \) für alle \(n \in \N \) und alle 
    \(\varepsilon > 0 \), also \(\limes{n} s_n \geq a \). \\
    Wegen 1. gibt es zu jedem \(\varepsilon > 0 \) ein \(N \in \N \), sodass 
    \(a_n < a + \varepsilon \) für alle \(n \geq N \). \\
    Woraus folgt:
    \[s_n \leq a + \varepsilon \text{ für alle } n \geq N\]
    Insgesamt gilt also: \(\limes{n} s_n = a\)\\
\end{bew}
\begin{bem}
    Der Limes inferior kann analog definiert werden. \\
    sei \((a_n)_n \) eine reelle Folge und \(a \in \R \) \\
    Dann gilt: \(\limesinf{n} a_n = a \) genau dann,wenn für jedes \(\varepsilon > 0 \) gilt:
    \begin{enumerate}
        \item \(a_n > a - \varepsilon \) für fast alle \(n \in \N \)
        \item \(a_n < a + \varepsilon \) für unendlich viele \(n \in \N \)
    \end{enumerate}
\end{bem}
\begin{bem}
    Da \(\underset{k \geq n}{(sup(a_k))_n}\) monoton fallend und \(\underset{k \geq n}{(inf(a_k))_n}\) monoton wachsend ist, existieren \(\limsup a_n \) und \(\liminf a_n\) immer eigentlich oder uneigentlich, d.h. sie sind entweder reelle Zahlen oder es gibt \(\limessup{n} a_n = \pm \infty \) bzw. \(\limesinf{n} a_n = \pm \infty \) 
\end{bem}
\begin{lem}
    Sei \((a_n)_n \) eine beschränkte Folge reeller Zahlen und \(H((a_n)_n) \) 
    die Menge der Häufungspunkte der Folge \((a_n)_n \). Es gilt:
    \[\limessup{n} a_n = sup(H((a_n)_n))\]
    \[\limesinf{n} a_n = inf(H((a_n)_n))\]
\end{lem}
\begin{bew}
    Wir beweisen \(\limessup{n} a_n = sup(H((a_n)_n)) \), die andere Aussage folgt analog. \\
    Wir definieren: \\
    \(s_n := \underset{k \geq n}{sup(a_k)} \) \\
    \((s_n)_n \) ist monoton fallend. \\
    Da \((a_n)_n \) beschränkt ist, gilt \(\underset{k \in \N}{inf}(a_k) \leq s_n 
    \leq \underset{k \in \N}{sup}(a_k) \; \forall \, \in \N \) \\
    Also ist \((s_n)_n \) beschränkt und \(s_n \in \R \). \\
    Nach dem Monotoniekriterium konvergiert \((s_n)_n \)
    \[s := \limes{n} s_n = \underset{n \rightarrow \infty}{inf}(s_n)_n 
    = \underset{n \in \N}{inf}\underset{k \geq n}{sup}(a_k)_n \in \R\]
    Wir beweisen jetzt:
    \begin{enumerate}
        \item \(s \in H((a_n)_n) \)
        \item \(a \leq s \; \forall \, a \in H((a_n)_n) \)
    \end{enumerate}
    Beweis von 1.: \\
    \(H((a_n)_n) \) ist die Menge aller Grenzwerte von konvergenten Teilfolgen der Folge 
    \((a_n)_n \) somit genügt es zu zeigen, dass zu jedem \(N \in \N \) und 
    \(\varepsilon > 0 \) ein \(n \in \N \) existiert, so dass \((a_n - s) < \varepsilon \) \\
    Da \(s = \limes{n} s_n \), finden wir ein \(m \geq N \), so dass 
    \[|s_m - s| < \frac{\varepsilon}{2} \]
    Nach Definition \(s_m \) gilt es \(n \geq m \), so dass
    \[|a_n - s_m| < \frac{\varepsilon}{2} \]
    Daraus folgt \(n \geq N \) und \(|a_n - s| < \varepsilon \) was 1. zeigt. \\
    Beweis 2.: \\
    Sei \(a \in H((a_n)_n) \), dann ist a Grenzwert einer konvergenten Teilfolge 
    \(({a_n}_k)_{k \in \N}\) von \((a_n)_n \) \\
    Per Definition gilt: \({s_n}_k \geq {a_n}_k \) \\
    Dann folgt:
    \[s = \limes{n} s_n = \limes{k} {s_n}_k \geq \limes{k} {a_n}_k = a \]
    Dies zeigt Bedingung 2. \\
    Aus 1. und 2. folgt direkt
    \[\limessup{n} a_n = sup(H((a_n)_n)) \]
\end{bew}
\begin{lem}
    Sei \((a_n)_n \) eine reelle beschränkte Folge. \\
    Es gilt: 
    \begin{enumerate}
        \item 
            die Folge \(((\underset{k \geq n}{sup}(a_k))\) ist monoton fallend und beschränkt. 
            Weiter gilt:
            \[\limessup{n} a_n = \limes{n} \underset{k \geq n}{sup} (a_k) = \underset{n \in \N}{inf} \; \underset{k \geq n}{sup} (a_k)\]
        \item 
            die Folge \((\underset{k \geq n}{inf} (a_k))\) ist monoton wachsend und beschränkt.
            Weiter gilt:
            \[\limesinf{n} a_n = \limes{n} \underset{k \geq n}{inf} (a_k) 
            = \underset{n \in \N}{sup} \; \underset{k \geq n}{inf} (a_k)\]
    \end{enumerate}		
\end{lem}
\begin{bew}
    Monotonie und Beschränktheit (erledigt)
    \[\limes{n} \underset{k \geq n}{sup} (a_k) = \limessup{n} a_n \]
    \[\limessup{n} a_n \leq \underset{n \in \N}{inf} \; \underset{k \geq n}{sup} (a_k) \]
    (siehe Beweise vorher) \\
    Bleibt zu zeigen:
    \[\limessup{n} a_n \geq \limes{n} \underset{k \geq n}{sup} (a_k) \]
    Wir definieren: \(s_n := \underset{k \geq n}{sup} (a_k) \) und \(s := \limes{n} s_n \) \\
    Wir wollen zeigen: 
    \(s \leq \limessup{n} a_n + \varepsilon \; \forall \, \varepsilon > 0 \) \\
    Sei \(\varepsilon > 0 \) \\
    Nach Satz 1 existiert \(N \in \N \) mit \(a_k \leq \limessup{n} a_n + \varepsilon \) 
    für \(k \geq N \) \\
    Insbesondere gilt: \\
    \(s_n = sup\{a_k: k \geq N\} \leq \limessup{n} a_n + \varepsilon \) \\
    Also auch \(s = \limes{n} s_n \leq s_n \leq \limessup{n} a_n + \varepsilon \) \\
\end{bew}	
\begin{bsp}
    Bestimmung von Limes Superior und Limes Inferior: \\
    \(a_n := (-1)^n \cdot (1 + \frac{1}{n}) \) für alle \(n \in \N \) \\
    Beh.: \(\limessup{n} a_n = 1 \), \(\limesinf{n} a_n = -1 \)
    \begin{bew}
        Wir haben:
        \[\underset{k \geq n}{sup} (a_k) =\begin{cases}
        1 + \frac{1}{n} & \text{n gerade} \\
        1 + \frac{1}{n + 1} & \text{n ungerade}
        \end{cases} \]
        \(\Rightarrow \limes{n} \underset{k \geq n}{sup} (a_k) = 1\) \\
        Weiter gilt:
        \[\underset{k \geq n}{inf} (a_k) =\begin{cases}
        -(1 + \frac{1}{n} )& \text{n ungerade} \\
        -(1 + \frac{1}{n + 1}) & \text{n gerade}
        \end{cases} \]
        \(\Rightarrow \limes{n} \underset{k \geq n}{inf} (a_k) = -1 \) \\
    \end{bew}
\end{bsp}
\begin{bsp}
    \(b_n := (1 + (-1)^n) \cdot (-1)^{\frac{n(n+1)}{2}} \) \\
    Beh.: \(\limessup{n} b_n = 2\), \(\limesinf{n} b_n = -2 \) \\
    Zunächst gilt für \(n \in \N \) \\
    \(n = 4k, k \in \N \Rightarrow \frac{n(n+1)}{2} = 2k(4k+1) \) ist gerade \\
    \(\Rightarrow (-1)^{\frac{n(n+1)}{2}} = 1\)\\
    \(n = 4k - 1, k \in \N \Rightarrow \frac{n(n+1)}{2} = (4k-1)2k \) ist gerade \\
    \(\Rightarrow (-1)^{\frac{n(n+1)}{2}} = 1\)\\
    \(n = 4k - 2, k \in \N \Rightarrow \frac{n(n+1)}{2} = (2k-1)(4k-1) \) ist ungerade \\
    \(\Rightarrow (-1)^{\frac{n(n+1)}{2}} = -1\)\\
    \(n = 4k - 3, k \in \N \Rightarrow \frac{n(n+1)}{2} = (4k-3)(2k-1) \) ist ungerade \\
    \(\Rightarrow (-1)^{\frac{n(n+1)}{2}} = -1 \) \\
    Betrachten wir die Teilfolgen \((b_{4k})_k, (b_{4k-1})_k, (b_{4k-2})_k, (b_{4k-3})_k \) \\
    \(b_{4k} = (1 + (-1)^{4k}) \cdot 1 = 2 \overset{k \rightarrow \infty}{\rightarrow} 2 \) \\
    \(b_{4k-1} = (1 + (-1)^{4k-1}) \cdot 1 
    = 0 \overset{k \rightarrow \infty}{\rightarrow} 0 \) \\
    \(b_{4k-2} = (1 + (-1)^{4k-2}) \cdot (-1) 
    = -2 \overset{k \rightarrow \infty}{\rightarrow} -2 \) \\
    \(b_{4k-3} = (1 + (-1)^{4k-3}) \cdot (-1) = 0 \overset{k \rightarrow \infty}{\rightarrow} 0 \) \\
    Nach Aufgabe \(31 \) von Zettel \(7 \) gilt, dass
    \[H((b_n)_n) = \{-2,0,2\} \]
    Somit gilt nach heutiger Übung
    \[\limessup{n} b_n = 2\]
    \[\limesinf{n} b_n = -2\]
\end{bsp}

\end{document}