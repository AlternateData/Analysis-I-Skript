\documentclass[12pt,a4paper,titlepage]{article} %twopage

%\usepackage{german} %deutsches Format
\usepackage[ngerman]{babel}
\usepackage[utf8]{inputenc} %Umlaute
\usepackage{graphicx} %Grafiken einbinden
\usepackage{amsmath,amssymb,amsthm}
\usepackage{nicefrac}
\usepackage{marvosym}[Lightning]
\usepackage{array}
\usepackage{dsfont} %statt \mathbb{1}
\usepackage{tikz}
\usepackage{float}
\usetikzlibrary{matrix}
\usepackage{hyperref} %Hyperlinks in pdf

\restylefloat{table}

%Metadaten
\hypersetup{
	pdftitle = {Analysis I Übungs Skript (WS 18/19)},
	pdfauthor = {Pavel Zwerschke, Daniel Augustin}}

\theoremstyle{definition}
\newtheorem{satz}{Satz}[subsection]
\newtheorem{kor}[satz]{Korollar}
\newtheorem{lem}[satz]{Lemma}

\newtheorem{defi}[satz]{Definition}
\newtheorem*{beh}{Behauptung}

\theoremstyle{remark}
\newtheorem*{bem}{Bemerkung}
\newtheorem*{bsp}{Beispiel}

\newenvironment{bew}{\begin{proof}[Beweis]}{\end{proof}}
\newcommand{\N}{\mathbb{N}}
\newcommand{\Z}{\mathbb{Z}}
\newcommand{\Q}{\mathbb{Q}}
\newcommand{\R}{\mathbb{R}}
\newcommand{\K}{\mathbb{K}}
\newcommand{\gqq}[1]{\glqq{}#1\grqq{}}
\newcommand{\limes}[1]{\lim\limits_{#1\rightarrow\infty}}
\newcommand{\limessup}[1]{\limsup \limits_{#1\rightarrow\infty}}
\newcommand{\limesinf}[1]{\liminf \limits_{#1\rightarrow\infty}}

\begin{document}
	\title{Analysis I Übung (WS 18/19)}
	\date{\today}
	\author{Pavel Zwerschke, Daniel Augustin}
	\maketitle
	
	%Inhaltsverzeichnis
	\tableofcontents
	\newpage	
	\section{Aussagenlogik (19.10.18)}
	\begin{defi}
		Eine Aussage ist die gedankliche Widerspiegelung eines Sachverhalts in Form eines Satzes in einer natürlichen oder künstlichen Sprache.
		\vspace{5mm}\\
		Jede Aussage ist entweder wahr oder falsch.
	\end{defi}
	\begin{bsp}
		Aussagen:
		\begin{enumerate}
			\item Wenn es regnet, dann ist die Straße nass.
			\item Vögel können fliegen
		\end{enumerate}
	\end{bsp}
	\textbf{Aussagen verknüpfen:}
	\begin{align*}
		\neg &A\\
		A &\wedge B\\
		A &\vee B\\
		A &\rightarrow B\\
		A &\Leftrightarrow B
	\end{align*}
	\textbf{Wahrheitstafel:}
	Seien A und B Variablen die entweder wahr oder falsch sind.	
	\begin{table}[H]
		\centering
		\begin{tabular}{c | c}			
			A & \(\neg A\) \\ 
			\hline
			\textbf{t} & \textbf{f} \\
			\textbf{f} & \textbf{t}
		\end{tabular}
		\caption{Negation}
	\end{table}	
	\begin{table}[H]
		\centering
		\begin{tabular}{c | c | c | c | c | c}			
			A & B & \(A \wedge B\) & \(A \vee B\) & \(A \rightarrow B\) & \(A \Leftrightarrow B\) \\
			\hline
			\textbf{w} & \textbf{w} & \textbf{w} & \textbf{w} & \textbf{w} & \textbf{w} \\
			\textbf{w} & \textbf{f} & \textbf{f} & \textbf{w} & \textbf{f} & \textbf{f} \\
			\textbf{f} & \textbf{w} & \textbf{f} & \textbf{w} & \textbf{w} & \textbf{f} \\
			\textbf{f} & \textbf{f} & \textbf{f} & \textbf{f} & \textbf{w} & \textbf{w}
			
		\end{tabular}
		\caption{Konjunktion, Disjunktion, Implikation, Äquivalenz}
	\end{table}
	\textbf{Ausdrücken der Aussagenlogik:}
	\begin{enumerate}
		\item Konstanten und Variablen sind Ausdrücke
		\item Sind A und B Variablen, so sind auch (siehe Wahrheitstabelle) Ausdrücke
	\end{enumerate}
	\subsection{Grundgesetze der Aussagenlogik}
	\begin{enumerate}
		\item Assoziativgesetz:
		\[(A \wedge B) \wedge C \Leftrightarrow A \wedge (B \wedge C)\]
		\[(A \vee B) \vee C \Leftrightarrow A \vee (B \vee C)\]
		\item Kommutativgesetz:
		\[(A \wedge B) \Leftrightarrow (B \wedge A)\]
		\[(A \vee B) \Leftrightarrow (B \vee A)\]
		\item Distributivgesetz:
		\[(A \wedge B) \vee C \Leftrightarrow (A \vee C) \wedge (B \vee C)\]
		\[(A \vee B) \wedge C \Leftrightarrow (A \wedge C) \vee (B \wedge C)\]
		\item Absorptionsgesetz:
		\[A \wedge (A \vee B) \Leftrightarrow A\]
		\[A \vee (A \wedge B) \Leftrightarrow A\]
		\item Idempotenzgesetze:
		\[A \wedge A \Leftrightarrow A\]
		\[A \vee A \Leftrightarrow A\]
		\item ausgeschlossene Dritte:
		\[A \wedge \neg A \Leftrightarrow \textbf{f}\]
		\[A \vee \neg A \Leftrightarrow \textbf{t}\]
		\item de Morgan'sche Gesetze:
		\[\neg (A \wedge B) \Leftrightarrow (\neg A \vee \neg B)\]
		\[\neg (A \vee B) \Leftrightarrow (\neg A \wedge \neg B)\]
		\item Gesetz für true und false:
		\[A \wedge \textbf{t} \Leftrightarrow A\]
		\[A \wedge \textbf{f} \Leftrightarrow \textbf{f}\]
		\[A \vee \textbf{t} \Leftrightarrow \textbf{t}\]
		\[A \vee \textbf{f} \Leftrightarrow A\]
		\[\neg \textbf{f} \Leftrightarrow \textbf{t}\]
		\[\neg \textbf{t} \Leftrightarrow \textbf{f}\]
		\item doppelte Verneinung:
		\[\neg (\neg A) \Leftrightarrow A\]
		\item Gesetz für das Konditional:
		\[A \Rightarrow B \Leftrightarrow (\neg A \vee B)\]
		\[A \Rightarrow B \Leftrightarrow \neg (A \wedge \neg B)\]
		\[A \Rightarrow B \Leftrightarrow \neg B \Rightarrow \neg A\]
		\item Gesetz der Komplementarität:
		\[A \vee \neg A \Leftrightarrow \textbf{t}\]
		\[A \wedge \neg A \Leftrightarrow \textbf{f}\]
	\end{enumerate}	
	\begin{defi}
		Zwei Aussagenlogische Ausdrücke heißen logisch äquivalent oder wertgleich (A = B), wenn sie die gleichen Wahrheitswerte besitzen.
	\end{defi}
	\begin{bsp}
		\begin{align*}
			(A \wedge B) \vee C &= (A \vee C) \wedge (B \vee C)\\
			\Leftrightarrow (A \wedge B) \vee C &= (B \wedge (A \vee C)) \vee (C \wedge (A \vee C)))\\
			\Leftrightarrow (A \wedge B) \vee C &=  (B \wedge A) \vee ((B \wedge C) \vee C)\\
	 		\Leftrightarrow (A \wedge B) \vee C &= (A \wedge B) \vee C
		\end{align*}		
	\end{bsp}
	\begin{bsp}
		\begin{align*}
			\neg(A \wedge B) &= (\neg A \vee \neg B)\\
			(\neg A \vee \neg B) &= (\neg A \vee \neg B)
		\end{align*}
		Somit sind \(\neg(A \wedge B)\) und \((\neg A \vee \neg B)\) logisch äquivalent und es gilt:
		\begin{align*}
			\neg(A \wedge B) = (\neg A \vee \neg B)
		\end{align*}
	\end{bsp}
	\subsection{Tautologie}
	\begin{defi}
		Ein Aussagenlogischer Ausdruck heißt allgemeingültig oder Tautologie, wenn die Wahrheitsfunktion identisch true ist.
	\end{defi}	
	\begin{bsp}
		Kontraposition:\\
		\begin{align*}
			A \rightarrow B \Leftrightarrow \neg B \rightarrow \neg A
		\end{align*}
		\begin{align*}
			((A \rightarrow B) \rightarrow (\neg B \rightarrow \neg A)) &\wedge ((\neg B \rightarrow \neg A) \rightarrow (A \rightarrow B))\\
			((\neg A \vee B) \rightarrow (B \vee \neg A)) &\wedge ((B \vee \neg A) \rightarrow (\neg A \vee B))\\
			(\neg (\neg A \vee B) \vee (B \vee \neg A)) &\wedge (\neg (B \vee \neg A) \vee (\neg A \vee B))\\
			((A \wedge \neg B) \vee (\neg A \vee B)) &\wedge ((A \wedge \neg B) \vee (\neg A \vee B))\\
			((A \vee (\neg A \vee B)) \wedge (\neg B \vee (\neg A \vee B))) &\wedge ((A \vee (\neg A \vee B)) \wedge (\neg B \vee (\neg A \vee B)))\\
			\textbf{t} &\wedge \textbf{t}\\
			&\textbf{t}
		\end{align*}
	\end{bsp}
	\section{Infimum und Supremum (26.10.18)}
	\begin{satz}
		Seien \(A, B \subseteq \R\) nicht leer und beschränkt.
		\begin{defi}
			\[A + B := \{a + b \; | \; a \in A, b \in B\}\]
			\[A - B := \{a - b \; | \; a \in A, b \in B\}\]
		\end{defi}
		Dann gilt:
		\begin{enumerate}
			\item \(\sup(A + B) = \sup(A) + \sup(B)\)
			\item \(\inf(A + B) = \inf(A) + \sup(B)\)
			\item \(\sup(A - B) = \sup(A) - \inf(B)\)
			\item \(\inf(A - B) = \inf(A) - \sup(B)\)
		\end{enumerate}
		Weil \(A \neq \emptyset\) und \(B \neq \emptyset\) gibt es \(a \in A\) und \(b \in B\).\\
		Dann gilt per Definition \(a+b \in A+B \neq \emptyset\) und \(a-b \in A-B \neq \emptyset\)
	\end{satz}
	\begin{bew}
		Sei \(z \in A + B\) dann existiert ein \(a \in A \wedge b \in B\) , so dass \(a + b = z\) und folgt:
		\[a + b \leq \sup(A) + sup(B)\]	
		also ist \(\sup(A+B) \leq \sup(A) + \sup(B)\) (obere Schranke)\\
		wir nehmen an:
		\[\sup(A+B) < \sup(A) + \sup(B)\]
		dann existiert ein \(\varepsilon > 0\):
		\[\sup(A+B) + \varepsilon = \sup(A) + \sup(B)\]
		nach Lemma 3.3.2 gibt es \(a_0 \in A, b_0 \in B\):
		\[a_0 > \sup(A) - \frac{\varepsilon}{2} \text{ und } b_0 > \sup(B) - \frac{\varepsilon}{2}\]
		per Definition ist \(a_0 + b_0 \in A + B\)\\
		somit gilt:
		\[a_0 + b_0 > \sup(A) - \frac{\varepsilon}{2} + \sup(B) - \frac{\varepsilon}{2} = \sup(A) + \sup(B) - \varepsilon = \sup(A+B) \text{ (Widerspruch)}\]\\
		Sei \(z \in A - B, a \in A, b \in B, z = a - b\)\\
		\(c \in C = \{d \in \R \; | \; d = -b, b \in B\} = -B\)\\
		\(z = a + c = a + (-b) = a - b\) 
		\[\sup(a-B) = \sup(A + (-B)) = \sup(A + C) = \sup(A) + \sup(C)\]
		\[= \sup(A) + \sup(-B) = \sup(A) - \inf(B)\] (siehe Vorlesung)
	\end{bew}
	\section{Körper (26.10.18)}
	\begin{satz}
		Sei \((\K,+,\cdot, >)\) ein angeordneter Körper und seien \(a,b,c,d \in \K\)\\
		Wenn \(b > 0, d > 0\) und \(\frac{a}{b} < \frac{c}{d}\), dann gilt:
		\[\frac{a}{b} < \frac{a+c}{b + d} < \frac{c}{d}\]
	\end{satz}
	\begin{bew}
		\(b > 0 \Rightarrow \frac{1}{b} > 0\)\\
		\(d > 0 \Rightarrow \frac{1}{d} > 0\)\\
		\(b > 0, d > 0: b+d > 0 \Rightarrow \frac{1}{b+d} > 0\)\\
		Per Annahme gilt \(\frac{a}{b} < \frac{c}{d} \Rightarrow ad = \frac{a}{b} \cdot bd < \frac{c}{d} \cdot bd = cb\)
		\begin{align*}
			&\Leftrightarrow ad + ab < cb + ab\\
			&\Leftrightarrow a(d+b) < b(c+a)\\
			&\Leftrightarrow a(d+b) \cdot \frac{1}{b} \cdot \frac{1}{b+c} < b(c+a) \cdot \frac{1}{b} \cdot \frac{1}{b+c}\\
			&\Leftrightarrow \frac{a}{b} < \frac{c+a}{d+b}
		\end{align*}
		Andererseits:
		\begin{align*}
			ad < cb &\Leftrightarrow ad + cd < cb + cd\\
			&\Leftrightarrow d(a+c) < c(b+d)\\
			&\Leftrightarrow \frac{c}{d} > \frac{a+c}{b+d}
		\end{align*}
		Mit Transitivität folgt: \(\frac{a}{b} < \frac{a+c}{b + d} < \frac{c}{d}\)
	\end{bew}
	\begin{satz}
		Seien \(a,b \in \K\)
		\[a^2 < b^2 \Leftrightarrow |a| < |b|\]
	\end{satz}
	\begin{bew}
		"\(\Leftarrow\)":\\
		sei \(0 \leq |a| < |b|\) dann gilt \(|b| > 0 \Leftrightarrow b^2 > 0\)\\
		\[a^2 = a \cdot a \leq |a| \cdot |a| \leq |a| \cdot |b| < |b| \cdot |b| = |b \cdot b| = |b^2| = b^2\]
		somit ist \(a^2 < b^2\)\\
		"\(\Rightarrow\)":\\
		\[a^2 < b^2 \Rightarrow |a| < |b| \Leftrightarrow |b| \leq |a| \Rightarrow b^2 \leq a^2\]
		Sei also \(0 \leq |b| \leq |a|\)\\
		\[b^2 = b \cdot b \leq |b| \cdot |b| \leq |a| \cdot |b| < |a| \cdot |a| = |a \cdot a| = |a^2| = a^2\]
		wenn \(b^2 \leq a^2 \Leftarrow |b| \leq |a|\), dann gilt auch \(|a| < |b| \Leftarrow a^2 < b^2\)
	\end{bew}
	\begin{kor}
		\[ab \leq \left(\frac{a+b}{2}\right)^2\]
		und bei Gleichheit gilt \(a = b\)
	\end{kor}
	\begin{bew}
		(\(\ast\)) \((0 \leq (a-b)^2\) und \(0 = (a-b)^2\) folgt \(a = b\)\\
		Also ist:
		\[0 \leq (a-b)^2 = a^2 - 2ab + b^2 \Leftrightarrow 4ab \leq a^2 +2ab + b^2 = (a+b)^2\]
		\[ab \leq \left(\frac{a+b}{2}\right)^2\]
		Nun sei
		\[ab = \left(\frac{a+b}{2}\right)^2 \Leftrightarrow 0 = (a-b)^2 \overset{(\ast)}{\Rightarrow} a = b\]
	\end{bew}
	\begin{satz}
		\[\forall \varepsilon > 0: 2 \cdot |ab| \leq \varepsilon^2 \cdot a^2 + \frac{1}{\varepsilon^2} \cdot b^2 \quad \forall a,b \in \R\]
	\end{satz}
	\begin{bew}
		mit \(\varepsilon = 1\):\\
		\begin{align*}
			2|ab| &\leq a^2 + b^2\\
			0 &\leq (|a| - |b|)^2 = |a|^2 - 2|a||b| + |b|^2\\
			4|a||b| &\leq a^2 + 2|a||b| + b^2\\
			2|a||b| &\leq a^2 + b^2
		\end{align*}
		\(\varepsilon > 0\) deshalb existiert \(\frac{1}{\varepsilon}\)\\
		somit 
		\[2|a||b| = 2|a \cdot 1 \cdot b| = 2|a \cdot \varepsilon \cdot \frac{1}{\varepsilon} \cdot b| = 2|(a \cdot \varepsilon)(b \cdot \frac{1}{\varepsilon})|\]
		Sei \(\tilde{a} = a\varepsilon\) und \(\tilde{b} = \frac{1}{\varepsilon}b\)
		\[2\left|a\varepsilon \cdot \frac{1}{\varepsilon}b\right| = \tilde{a^2} + \tilde{b^2}\]
		\[2|\tilde{a} \cdot \tilde{b}| \leq a^2\varepsilon^2 + \frac{1}{\varepsilon^2}\] 
	\end{bew}
	\section{Folgenkonvergenz (23.11.18)}
	\begin{bsp}
		Quadratwurzel\\
		\(a > 0, a \in \R\), wir wollen\(\sqrt{a}\) bestimmen.\\
		es gilt \(x^2 = a\), dann \(x = \frac{a}{x}\)\vspace{5mm}\\
		Babylonier verwendeten Iterationsverfahren zur Berechnung der Quadratwurzel, welches in jedem Schritt als Näherungswerte das arithmetische Mittel\vspace{5mm}\\
		\(x := \frac{1}{2}(x+\frac{a}{x})\)
		\begin{beh}
			Seien \(a > 0\) und \(x_0 > 0\) reelle Zahlen\\
			Die Folge \((x_n)_n\) sei gegeben durch \(x_{n+1} := \frac{1}{2}(x_n + \frac{a}{x_n})\) dann konvergiert die Folge gegen \(\sqrt{a}\), d.h. gegen die eindeutig bestimmte positive Lösung der Gleichung \(x^2 = a\)
		\end{beh}		
		Vorgehen im Beweis:
		\begin{enumerate}
			\item \(x_n > 0, \forall n \geq 0\)
			\item \(x_n^2 \geq a, \forall n \in \N\)
			\item \(x_{n+1} \leq x_n, \forall n \in \N\)
			\item Existenz des Grenzwertes
			\item Eindeutigkeit des Grenzwertes
		\end{enumerate}
		\begin{bew}
			Umsetzen des Plans:\\
			\begin{enumerate}				
				\item Schritt:\\
					Beweis durch vollständige Induktion:\\
					(IA):\\
					\(n=0: x_0 > 0\) per Voraussetzung\\
					\(n=1: x_1 = \frac{1}{2}(\underbrace{x_0}_{>0} + \underbrace{\frac{a}{x_0}}_{>0}) > 0\) ist wohldefiniert, da \(x_0 > 0\)\\
					(IS): \(n \rightarrow n+1\): für beliebiges, aber festes n sei gezeigt, dass \(x_n > 0\)\\
					Dann gilt \(x_{n+1} = \frac{1}{2}(\underbrace{x_n}_{>0} + \underbrace{\frac{a}{x_n}}_{>0}) > 0\) wohldefiniert, weil \(x_n > 0\)\\
					Somit gilt nach dem Induktionsprinzip \(x_n > 0\) \(\forall n \in \N\)
				\item Schritt:\\
					\(x_n^2 \geq a \Leftrightarrow x_n^2 - a \geq 0 \)
					\begin{align*}
						x_n^2 - a &= \frac{1}{4}\left(x_{n-1} + \frac{a}{x_{n-1}}\right)^2 - a\\
						&= \frac{1}{4}\left(x_{n-1}^2 + 2a + \frac{a^2}{x_{n-1}^2}\right) - a\\
						&= \frac{1}{4}\left(x_{n-1}^2 + 2a + \frac{a^2}{x_{n-1}^2} - 4a\right)\\
						&= \frac{1}{4}\left(x_{n-1} - \frac{a}{x_{n-1}}\right)^2 \geq 0
					\end{align*}
				\item Schritt:\\
					\begin{align*}
						x_{n+1} \leq x_n &\Leftrightarrow x_n - x_{n+1} \geq 1\\
						x_n - x_{n+1} &= x_n - \frac{1}{2}(x_n + \frac{a}{x_n})\\
						&=\frac{1}{2}x_n - \frac{a}{2x_n}\\
						&= \underbrace{\frac{1}{2x_n}}_{>0} \cdot \underbrace{(x_n^2 - a)}_{>0} \geq 0
					\end{align*}
				\item Schritt:\\
					Wegen Schritt 3 gilt die Folge \((x_n)_n\) ist monoton fallend und durch 0 nach unten beschränkt. Somit gilt nach Satz 8.2, dass \((x_n)_n\) konvergent ist.\\
					Aus Satz 7.1.9:
					\(\limes{n} x_n = x \geq 0\)
					\begin{align*}
						x_{n+1} = \frac{1}{2}(x_n + \frac{a}{x_n}) \overset{\cdot 2x_n} {\Leftrightarrow} 2x_{n+1}x_n = x_n^2 + a \;(\ast)\\						
					\end{align*}
					Betrachte Grenzwert von (\(\ast\))
					\begin{align*}
						\limes{n} (2x_{n+1}x_n) &= \limes{n} (x_n^2 + a)\\
						\overset{\text{7.1.8}}{\Leftrightarrow} 2x^2 &= x^2 + a\\
						\Leftrightarrow x^2 = a						
					\end{align*}
					Somit konvergiert \((x_n)_n\) gegen eine Quadratwurzel von a.
				\item Schritt:\\
					Annahme: \(y\) ist einer weitere Lösung von der Gleichung \(y^2 = a\) mit \(x \neq y\).\\
					Dann gilt:
					\begin{align*}
						0 = x^2 - y^2 = (x-y)\underbrace{(x+y)}_{>0} \text{ \Lightning} x \neq y\\
						\Rightarrow x = y
					\end{align*}
					Also ist x die einzige Lösung der Gleichung \(x^2 = a\).
			\end{enumerate}
		\end{bew}
		\subsection{Geschwindigkeit der Konvergenz}
		Definieren relativen Fehler \(f_n\) im n-ten Schritt\\
		\(x_n = \sqrt{a} \cdot (1+f_a)\)\\
		Da \((x_n)_n\) monoton fallend ist, gilt \(f_n \geq 0 \forall n \geq 1\)\\
		Einsetzen in Rekursionsformel:\\
		\begin{align*}
			x_{x+1} &= \frac{1}{2}\left(x_n + \frac{a}{x_n}\right)\\
			\Leftrightarrow \sqrt{a}(1+f_{n+1}) &= \frac{1}{2}\left(\sqrt{a}(f_n+1) + \frac{a}{\sqrt{a}(1+f_n)} \right)\\
			\Leftrightarrow 1 + f_n &= \frac{1}{2}\left(1+f_n+\frac{1}{f_n + 1}\right)\\
			\Rightarrow f_{n+1} &= \frac{1}{2} \cdot \frac{f_n^2}{1 + f_n}\\
			\text{Nebenüberlegung:}\\
			&f_n < f_n^2: f_n > 1: \frac{f_n^2}{1+f_n} < \frac{f_n^2}{f_n} = f_n\\
			&f_n^2 < f_n: f_n < 1: \frac{f_n^2}{1+f_n} < f_n^2 \\
			\Rightarrow f_{n+1} &= \frac{1}{2} \cdot \frac{f_n^2}{1+f_n} \leq \frac{1}{2} \cdot min(f_n, f_n^2)
		\end{align*}
		Der relative Fehler von \(f_n\) geht sehr schnell gegen Null.
		\begin{bsp}
			\(0\leq f_n < 1 \Rightarrow f_1 < \frac{1}{4} < f_2 < \frac{1}{40} < f_3 < \frac{1}{1600} < \ldots < f_6 < 10^{-30}\)
		\end{bsp}
		Gilt sowas auch für k-te Wurzeln?
		\begin{satz}
			Sei \(k \geq 2 \) eine natürliche Zahl und \(a > 0\) eine positive reelle Zahl, dann konvergiert für jeden Angangswert \(x_0 > 0\) die durch\\
			\(x_{x+1} := \frac{1}{k} \left((k-1)x_n + \frac{a}{x_n^{k-1}}\right)\)\\
			rekursiv definierte Folge \((x_n)_n\) gegen die eindeutig bestimmte positive Lösung der Gleichung \(x^k = a\)
		\end{satz}
		\begin{bew}
			Wie Satz 1 zeigt man, dass es nicht mehr als eine Lösung geben kann.\\Wissen also \(0 \leq z_1 < z_2 \Rightarrow z_1^k < z_2^k\)\\
			Nun zeigen wir, dass der Grenzwert \(x := \limes{n} x_n\) im Fall der Existenz die Gleichung \(x^k = a\) erfüllt.\\
			Wir multiplizieren die Rekursionsgleichung mit \(k \cdot x_n^{k-1}\)\\
			\(\Rightarrow k \cdot x_{n+1} \cdot x_n^{k-1} = (k-1)x_n^k + a\)\\
			Durch Grenzübergang gilt:\\
			\(k \cdot x^k = (k-1)x^k + a \Rightarrow x^k = a \)\\
			Es bleibt zu zeigen, dass die Folge \((x_n)_n\) konvergiert.
			\begin{align*}
				x_{n+1} &= \frac{1}{k}\left((k-1)x_n + \frac{a}{x_n^{k-1}}\right)\\
				&= \frac{1}{k} \cdot x_n \left((k-1) + \frac{a}{x_n^{k}}\right)\\
				&= x_n \left(1 + \frac{1}{k}\left(\frac{a}{x_n^k} - 1\right)\right) \text{ (\(\ast \ast\))}\\
				x_{n+1}^k &= x_n^k\left(1+\frac{1}{k}\left(\frac{a}{x_n^k} - 1 \right)\right)^k\\
				&\overset{\text{Bernoulli}}{\Rightarrow} x_n^k\left(1+\frac{1}{k}\left(\frac{a}{x_n^k} - 1 \right)\right)^k\\
				&\geq x_n^k\left(1+\left(\frac{a}{x_n^k} - 1 \right)\right)^k\\
				&= x_n^k + a - x_n^k = a\\
				\text{Also: } x_{n+1}^k &\geq a \Rightarrow \frac{a}{x_{n+1}^k} \leq 1 \forall n \geq 1\\
			\end{align*}
			Also \((x_n)_n\) monoton fallend und durch 0 nach unten beschänkt. Also existiert \(x = \limes{n} x_n\) nach Satz 8.2 und x erfüllt nach 2) die Gleichung \(x^k = a\)
		\end{bew}
	\end{bsp}
	\begin{bsp}
		Es seien \((f_n)_n\) die Folgen der Fibonacci Zahlen, weiter sei \(a_n := \frac{f_{n+1}}{f_n}\) für \(n \in \N\) und \(a := \frac{1}{2}(1 + \sqrt{5}), b := \frac{1}{2}(1 - \sqrt{5})\)\\
		\begin{enumerate}
			\item \(\frac{1}{\sqrt{5}}\left(a^n-b^n\right)\)
			\item \(\limes{n} a_n = a\)
		\end{enumerate}
		\begin{bew}
			\begin{enumerate}
				\item Aussage:\\
					Vollständige Induktion:\\
					(IA):
					\begin{align*}
						&n = 0: \; f_0 = \frac{1}{\sqrt{5}} \left(a^0-b^0 \right) = 0\\		
						&n = 1: \; f_1 = \frac{1}{\sqrt{5}} \left(a^1-b^1 \right) = \frac{1}{\sqrt{5}} \left(\frac{1}{2}(1 + \sqrt{5})-\frac{1}{2}(1 - \sqrt{5}) \right) = \frac{\sqrt{5}}{\sqrt{5}} = 1\\		
					\end{align*}
					(IS): \(n \rightarrow n+1\)\\
					Angenommen es gilt für ein \(n \in \N: f_k = \frac{1}{\sqrt{5}}\left(a^k-b^k\right) \forall k \geq n\)\\
					Dann folgt:\\
					\(f_{n+1} = f_n + f_n\)\\
					\(= \frac{1}{\sqrt{5}}\left[a^n - b^n +a^{n-1} - b^{n-1}\right]\)\\
					\(= \frac{1}{\sqrt{5}}\left[a^{n-1}(a+1) - b^{n-1}(b+1)\right]\)\\
					\(= \frac{1}{\sqrt{5}}\left[\left(\frac{1}{2}\left(1 + \sqrt{5}\right)\right)^{n-1} \cdot\left(\frac{1}{2}\left(1 + \sqrt{5}\right)+1\right) - \left(\frac{1}{2}\left(1 - \sqrt{5}\right)\right)^{n-1}\cdot\left(\frac{1}{2}\left(1 - \sqrt{5}\right)+1\right)\right]\)\\
					\(= \frac{1}{\sqrt{5}}\left[\left(\frac{1}{2}\left(1 + \sqrt{5}\right)\right)^{n-1}\cdot\left(\frac{1}{4}\left(6 + 2\sqrt{5}\right)\right) - \left(\frac{1}{2}\left(1 - \sqrt{5}\right)\right)^{n-1}\cdot\left(\frac{1}{4}\left(6 - 2\sqrt{5}\right)\right)\right]\)\\
					\(= \frac{1}{\sqrt{5}}\left[\left(\frac{1}{2}\left(1 + \sqrt{5}\right)\right)^{n-1}\cdot\left(\frac{1}{4}\left(1 + 2\sqrt{5} + 5\right)\right) - \left(\frac{1}{2}\left(1 - \sqrt{5}\right)\right)^{n-1}\cdot\left(\frac{1}{4}\left(1 - 2\sqrt{5} + 5\right)\right)\right]\)\\
					\(= \frac{1}{\sqrt{5}}\left[\left(\frac{1}{2}\left(1 + \sqrt{5}\right)\right)^{n-1}\cdot\left(\frac{1}{4}\left(1+\sqrt{5}\right)^2\right) - \left(\frac{1}{2}\left(1 - \sqrt{5}\right)\right)^{n-1}\cdot\left(\frac{1}{4}\left(1 - \sqrt{5}\right)^2\right)\right]\)\\
					\(= \frac{1}{\sqrt{5}}\left[\left(\frac{1}{2}\left(1 + \sqrt{5}\right)\right)^{n+1} - \left(\frac{1}{2}\left(1 - \sqrt{5}\right)\right)^{n+1}\right]\)\\
					\(= \frac{1}{\sqrt{5}}\left[a^{n+1} - b^{n+1}\right]\)\\
					Aus dem Induktionsprinzip folgt somit \(f_n = \frac{1}{\sqrt{5}}\left(a^n-b^n\right) \forall n \in \N \)
				\item Aussage:\\
				\begin{align*}
					a_n = \frac{f_{n+1}}{f_n} \overset{\text{1)}}{=} \frac{a^{n+1} - b^{n+1}}{a^n - b^n} = \frac{a - b\left(\frac{b^n}{a^n}\right)}{1 - \left(\frac{b^n}{a^n}\right)} = \frac{a - b\left(\frac{b}{a}\right)^n}{1 - \left(\frac{b}{a}\right)^n}
				\end{align*}
				Wissen \(a > |b| \Rightarrow \frac{|b|}{a} < 1 \Rightarrow \left(\frac{b}{a}\right)^n \overset{n \rightarrow \infty}{\rightarrow} 0\)\\
				Nach Satz 7.1.8 gilt:
				\begin{align*}
					\limes{n} a_n = \limes{n} \frac{a - b\left(\frac{b}{a}\right)^n}{1 - \left(\frac{b}{a}\right)^n} = \frac{a}{1} = a
				\end{align*}
			\end{enumerate}
		\end{bew}
	\end{bsp}
	\section{k-adische Brüche (30.11.18)}
	\begin{bsp}
		sei \(b \in \N \quad b \geq 2\)\\
		unter einem b-adischen Bruch verstehen wir:\\
		\[\pm \limes{n} \sum_{j=-k}^{n} a_j \cdot b^{-j}\]\\
		Dabei ist \(k \geq 0\) und \(a_j\) sind natürliche Zahlen mit \(0 \leq a_j \leq b\)\\
		Sei die Basis b festgelegt.\\
		\[\pm a_{-k}, \pm a_{-k+1}, \dots ,\pm a_{-1}, \pm a_0, \pm a_1, \pm a_2, \dots\]\\
		Für \(b = 10\) spricht man von Dezimalbrüchen, \(b = 2\) von dyadischen Brüchen, \(b = 3\) von ternären Brüchen.\\
		In der Vorlesung wurde die b-adische Darstellung von \(0 < \alpha < 1\) eingeführt. \\
		Wir können jede positive Zahl \(x \in \R, x > 0\) schreiben als:\\
		\[x = {\lfloor x \rfloor}_{\in \N} + \alpha_{\in [0, 1]} \]
	\end{bsp}
	\begin{defi}
		Abrundungsfunktion:\\
		Für jede reelle Zahl \(x\) ist \(\lfloor x \rfloor\) die größte ganze Zahl, die kleiner oder gleich \(x\) ist.\\
		\[\lfloor x \rfloor = max\{k \in \Z | k \leq x\}\]\\
		\begin{bsp}
			\(x = 25,7365\) dann ist \(\lfloor x \rfloor = 25\) und \(\alpha = 0,7365\)\\
		\end{bsp}
	\end{defi}
	\begin{defi}
		Sei \(b \in \N\) größer als 1\\
		Eine b-adische Darstellung (eine Darstellung zur Basis b) einer natürlichen Zahl \(n\) ist eine Liste \(a_m, \dots, a_0\) von natürlichen Zahlen aus der Menge\\
		\(\{0, \dots, b-1\}\), so dass \(a_m > 0\) und \(n = \sum_{i=0}^{m} a_ib^i\)
	\end{defi}
	\begin{bsp}
		Ternärdarstellung (\(b = 3\)) für die natürlichen Zahlen \(\leq 10\)\\
		\begin{align*}
			&n = 1 = \textbf{0} \cdot 3^2 + \textbf{0} \cdot 3^1 + \textbf{1} \cdot 3^0 &= 1_{(3)}\\
			&n = 2 = \textbf{0} \cdot 3^2 + \textbf{0} \cdot 3^1 + \textbf{2} \cdot 3^0 &= 2_{(3)}\\
			&n = 3 = \textbf{0} \cdot 3^2 + \textbf{1} \cdot 3^1 + \textbf{0} \cdot 3^0 &= 10_{(3)}\\
			&n = 4 = \textbf{0} \cdot 3^2 + \textbf{1} \cdot 3^1 + \textbf{1} \cdot 3^0 &= 11_{(3)}\\
			&n = 5 = \textbf{0} \cdot 3^2 + \textbf{1} \cdot 3^1 + \textbf{2} \cdot 3^0 &= 12_{(3)}\\
			&n = 6 = \textbf{0} \cdot 3^2 + \textbf{2} \cdot 3^1 + \textbf{0} \cdot 3^0 &= 20_{(3)}\\
			&n = 7 = \textbf{0} \cdot 3^2 + \textbf{2} \cdot 3^1 + \textbf{1} \cdot 3^0 &= 21_{(3)}\\
			&n = 8 = \textbf{0} \cdot 3^2 + \textbf{2} \cdot 3^1 + \textbf{2} \cdot 3^0 &= 22_{(3)}\\
			&n = 9 = \textbf{1} \cdot 3^2 + \textbf{0} \cdot 3^1 + \textbf{0} \cdot 3^0 &= 100_{(3)}\\
			&n = 10 = \textbf{1} \cdot 3^2 + \textbf{0} \cdot 3^1 + \textbf{1} \cdot 3^0 &= 101_{(3)}\\
		\end{align*}
	\end{bsp}
	\begin{bsp}
		Basis 10 ist am bekanntesten:\\
		\(354 = \textbf{3} \cdot 10^2 + \textbf{5} \cdot 10^1 + \textbf{4} \cdot 10^0\)\\
		als Darstellung zur Basis \(10\): \(a_2 = 3, a_1 = 5, a_0 = 4\)\\
		\(\Rightarrow 354_{(10)} = 354\)\\
		Wenn \(b = 5\), dann ist\\
		\(354 = \textbf{2} \cdot 5^3 + \textbf{4} \cdot 5^2 + \textbf{0} \cdot 5^1 + \textbf{4} \cdot 5^0 = 2404_{(5)}\)
	\end{bsp}
	\begin{bem} %bessere Struktur?
		Wie findet man eine Darstellung:\\
		Finde größten Index \(m\), so dass \(b^m \leq n\).\\
		Der Koeffizient \(a_m\) ist das größte multiplikative vielfache von \(b^m\), so dass \(n - a_m \cdot b^m \geq 0\).\\
		Dann setzt man \(k=n-a_m \cdot b^m\) und dann setzt man den größten Index \(\tilde{m}\), so dass \(b^{\tilde{m}} \leq k\)
	\end{bem}
	\begin{bsp}
		4-adische Darstellung von 354\\
		\begin{align*}			
			&4^5 = 1024 > 354\\
			&4^4 = 256 < 354\\
			&354 - 256 = 98\\
			&\Rightarrow 354 -1 \cdot 4^4 > 0 \text{ und } 354 - 2 \cdot 4^4 < 0\\
			&\Rightarrow a_4 = 1
		\end{align*}
		Finde Index \(\tilde{m}\), so dass \(4^{\tilde{m}} \leq 98\)			
		\begin{align*}			
			&4^3 = 64 \rightarrow \tilde{m} = 3\\
			&98 - 64 = 34 < 64 \Rightarrow a_3 = 1\\
			&4^2 = 16 < 34, 2 \cdot 16 = 32 < 34 \Rightarrow a_2 = 2\\
			&34 - 2 \cdot 16 = 2 < 4 = 4^1 \Rightarrow a_1 = 0\\
			&\Rightarrow a_0 = 2
		\end{align*}
		Somit ist die 4-adische Darstellung von \(354\) gegeben durch \(11202_{(4)}\).\\
		\(354 = 111010_{(3)} = 101100010_{(2)} = 542_{(8)}\)\\
		Bleibt zu klären, ob das immer geht.
	\end{bsp}
	\begin{satz}
		Sei \(b\) eine natürliche Zahl größer als 1.\\
		Jede Zahl hat eine eindeutige Darstellung zur Basis b.
	\end{satz}
	\begin{bew}
		Konstruiere b-adische Darstellung durch vollständige Induktion nach n.\\
		(IA):\\
		\(n = 0\) hat die Darstellung \(a_0 = 1\)\\
		(IS): \(n - 1 \rightarrow n\): sei \(n > 1\)\\
		Wir nehmen an, dass wir bereits ein b-adische Darstellung für \(n = 1\) konstruiert haben.\\
		\[n-1 = \sum_{j=0}^{m} a_jb^j \quad \text{mit } a_m \neq 0\]\\
		Wenn \(a_m = a_{m-1} = \dots = a_0 = b-1\), dann stellen wir n als\\
		\(a_{m+1} = 1, a_j = 0\) für \(j \leq m\)\\
		Dies funktioniert, das aus der geometrischen Summenformel folgt:\\
		\[n-1 = \sum_{j=0}^{m} (b-1)b^j = (b-1) \cdot \sum_{j=0}^{m}b^j = (b-1) \cdot \frac{1- b^{m+1}}{1-b} = \frac{-(1-b)(1-b^{m+1})}{1-b} = b^{m+1}-1\]\\
		Ist dies nicht der Fall, muss also mindestens ein Koeffizient in der Darstellung von \(n-1\) kleiner als \(b-1\) sein.\\
		Sei \(\varepsilon\) der kleinste Index, so dass \(a_{\varepsilon} < b-1\).\\
		Wir definieren uns \(c_0, \dots, c_m\) durch:\\
		\(c_j = a_j \quad \text{für } j > \varepsilon\)\\
		\(c_t = a_{t+1}\)\\
		\(c_j = 0 \quad \text{für } j < t\)\\
		Da \(a_j = b-1\) für alle \(j<\varepsilon\) erhalten wir\\
		\[\sum_{j=0}^{m} c_jb^j = \underbrace{\sum_{j=0}^{t-1} c_jb^j}_{=0} + \underbrace{c_tb^t}_{=a_{t+1}b^t} + \underbrace{\sum_{j=t+1}^{m} a_jb^j}_{=\sum_{j=t+1}^{m} a_jb^j}\]\\
		Also:
		\[\sum_{j=0}^{t-1} a_jb^j = b^t-1\]\\
		\[= 1+\sum_{j=0}^{t-1} a_jb^j + a_tb^t + \sum_{t+1}^{m}a_jb^j = 1+\sum_{j=0}^{m} a_jb^j = n\]\\
		Nach dem Induktionsprinzip gilt dies somit für alle \(n \in \N\). Wir müssen noch Eindeutigkeit zeigen.\\
		Sei \(a_v,\dots, a_0\) und \(c_s,\dots, c_0\) verschiedene Darstellungen von \(n \in \N\). Wenn \(v \neq s\), können o.B.d.A annehmen, dass \(v > s\). Nun ist die durch \(a_v,\dots, a_0\) dargestellt Zahl mindestens \(b^v\).\\
		Aber die Zahl, die durch \(c_s,\dots, c_0\) dargestellt wird, ist maximal\\
		\[\sum_{j=0}^{v-1}(b-1)b^j = b^v - 1\]\\
		Somit kann der Fall nicht einsetzen. Sei also \(v=s\). Wir wissen, dass \(a_v\) und \(c_s\) ungleich Null sind. Somit gilt: \(a_v - 1 \geq 0, c_s - 1 \geq 0\) und wir erhalten somit verschiedene Darstellungen von Basis \(b\) der kleineren Zahl \(n-b^v\). Dies können wir fortsetzen und sind entweder irgendwann in dem Fall, dass die Indizes nicht übereinstimmen (was nicht sein kann) oder die Darstellungen sind schon gleich.
	\end{bew}
	\begin{bem}
		Sei \(b \in \N, b \geq 2, a \in [0, 1)\)\\
		\[\alpha = \limes{n} \sum_{j = 1}^{n} \frac{b_j}{b_j} \quad \text{mit } b_j \in \{0,1,\dots,b-1\}\]\\
		so dass \(l_{n+1} := \lfloor b^{n+1}(x-\sum_{j=1}^{n} \frac{b_j}{b_j})\rfloor\) (\(*\))\\
		\(b=3\)\\
		\begin{center}
		\begin{tikzpicture}[scale = 4]
			\draw (-1/8,0) -- (9/8,0);
			\draw (0,0) node {\([\)} (1,0) node {\()\)};
			\draw (0,0) node[below=3mm] {\(0\)};
			\draw (1,0) node[below=3mm] {\(1\)};
			\foreach \x in {1,2,4,5,7,8} {
			\draw (\x/9,-1/32) -- (\x/9,1/32);}
			\foreach \x in {3,6} {
			\draw (\x/9,-1/16) -- (\x/9,1/16);}
		\end{tikzpicture}
		\end{center}
		ohne (\(*\)) wäre \(\frac{1}{3} = 0,100000\dots = 0,0222\dots\)
	\end{bem}
	\section{Limes Superior und Limes Inferior (07.12.18)}
	\begin{defi}
		Sei \(a_n\) eine reelle Folge\\
		\[\limessup{n} a_n = \limes{n} (\underset{k \geq n}{sup(a_k)}) \]\\
		\[\limesinf{n} a_n = \limes{n} (\underset{k \geq n}{inf(a_k)}) \]
	\end{defi}
	\begin{satz}
		Charakterisierung des Limes superior\\
		Sei \(a_n\) eine reelle Folge und \(a \in \R\)\\
		Genau dann gilt:\\
		\[\limessup{n} a_n = a\]\\
		Wenn für \(\varepsilon > 0\) die folgenden Bedingungen erfüllt sind.\\
		\begin{enumerate}
			\item Für fast alle Indizes \(n \in \N\) (d.h für alle bis auf endlich viele) gilt: \(a_n < a+ \varepsilon\)
			\item Es gibt unendlich viele Indizes \(m \in \N\) mit \(a_m > a - \varepsilon\)
		\end{enumerate}
	\end{satz}
	\begin{bew}
		Wir definieren:\\
		\(s_n := \underset{k \geq n}{sup(a_k)}\)\\
		Für alle \(n \in \N\) gilt: \(\{k \in \N: k \geq n+1\} \subset \{k \in \N: k \geq n\}\) also ist\\
		\(s_{n+1} = sup\{a_k: k \in \N, k \geq n+1\} \subseteq sup\{a_k: k \in \N, k \geq n\} = s_n\)\\
		und damit ist \((s_n)_n\) monoton fallend.\\
		\(\Rightarrow\): Sei \(\limes{n} a_n = a\), d.h. \(\limes{n} s_n = a\)\\
		Weiter \(\varepsilon \geq 0\)\\
		Da \((s_n)_n\) monoton fallend ist, ist \(s_n \geq a \; \forall n \in \N\)\\
		Somit ist Bedingung 2. erfüllt. Weiter muss es ein \(N \in \N\) geben, so dass \(s_n < a + \varepsilon\) für \(n \geq N\)\\
		d.h. \(a_n < a + \varepsilon\) für \(n \geq N\) Also gilt auch 1.\\
		\(\Leftarrow\): seinen Bedingungen 1. und 2. erfüllt.\\
		Aus 2. folgt, \(s_n > a - \varepsilon\) für alle \(n \in \N\) und alle \(\varepsilon > 0\), also \(\limes{n} s_n \geq a\).\\
		Wegen 1. gibt es zu jedem \(\varepsilon > 0\) ein \(N \in \N\), so dass \(a_n < a + \varepsilon\) für alle \(n \geq N\).\\
		Woraus folgt:\\
		\[s_n \leq a + \varepsilon \text{ für alle } n \geq N\]\\
		Insgesamt gilt also: \(\limes{n} s_n = a\)\\
	\end{bew}
	\begin{bem}
		Der Limes inferior kann analog definiert werden.\\
		sei \((a_n)_n\) eine reelle Folge und \(a \in \R\)\\
		Dann gilt: \(\limesinf{n} a_n = a\) genau dann,wenn für jedes \(\varepsilon > 0\) gilt:
		\begin{enumerate}
			\item \(a_n > a - \varepsilon\) für fast alle \(n \in \N\)
			\item \(a_n < a + \varepsilon\) für unendlich viele \(n \in \N\)
		\end{enumerate}
	\end{bem}
	\begin{bem}
		Da \(\underset{k \geq n}{(sup(a_k))_n}\) monoton fallend und \(\underset{k \geq n}{(inf(a_k))_n}\) monoton wachsend ist, existieren \(\limsup a_n \) und \(\liminf a_n\) immer eigentlich oder uneigentlich, d.h. sie sind entweder reelle Zahlen oder es gibt \(\limessup{n} a_n = \pm \infty \) bzw. \(\limesinf{n} a_n = \pm \infty \) 
	\end{bem}
	\begin{lem}
		Sei \((a_n)_n\) eine beschränkte Folge reeller Zahlen und \(H((a_n)_n)\) die Menge der Häufungspunkte der Folge \((a_n)_n\). Es gilt:\\
		\[\limessup{n} a_n = sup(H((a_n)_n))\]
		\[\limesinf{n} a_n = inf(H((a_n)_n))\]
	\end{lem}
	\begin{bew}
		Wir beweisen \(\limessup{n} a_n = sup(H((a_n)_n))\), die andere Aussage folgt analog.\\
		Wir definieren:\\
		\(s_n := \underset{k \geq n}{sup(a_k)}\)\\
		\((s_n)_n\) ist monoton fallend.\\
		Da \((a_n)_n\) beschränkt ist, gilt \(\underset{k \in \N}{inf}(a_k) \leq s_n \leq \underset{k \in \N}{sup}(a_k) \; \forall \in \N\)\\
		Also ist \((s_n)_n\) beschränkt und \(s_n \in \R\).\\
		Nach dem Monotoniekriterium konvergiert \((s_n)_n\)\\
		\[s := \limes{n} s_n = \underset{n \rightarrow \infty}{inf}(s_n)_n = \underset{n \in \N}{inf}\underset{k \geq n}{sup}(a_k)_n \in \R\]\\
		Wir beweisen jetzt:
		\begin{enumerate}
			\item \(s \in H((a_n)_n)\)
			\item \(a \leq s \; \forall a \in H((a_n)_n)\)
		\end{enumerate}
		Beweis von 1.:\\
		\(H((a_n)_n)\) ist die Menge aller Grenzwerte von konvergenten Teilfolgen der Folge \((a_n)_n\) somit genügt es zu zeigen, dass zu jedem \(N \in \N\) und \(\varepsilon > 0\) ein \(n \in \N\) existiert, so dass \((a_n - s) < \varepsilon\)\\
		Da \(s = \limes{n} s_n\), finden wir ein \(m \geq N\), so dass 
		\[|s_m - s| < \frac{\varepsilon}{2}\]
		Nach Definition \(s_m\) gilt es \(n \geq m\), so dass
		\[|a_n - s_m| < \frac{\varepsilon}{2}\]
		Daraus folgt \(n \geq N\) und \(|a_n - s| < \varepsilon\) was 1. zeigt.\\
		Beweis 2.:\\
		sei \(a \in H((a_n)_n)\), dann ist a Grenzwert einer konvergenten Teilfolge \(({a_n}_k)_{k \in \N}\) von \((a_n)_n\)\\
		Per Definition gilt: \({s_n}_k \geq {a_n}_k\)\\
		Dann folgt:
		\[s = \limes{n} s_n = \limes{k} {s_n}_k \geq \limes{k} {a_n}_k = a\]
		Dies zeigt Bedingung 2.\\
		Aus 1. und 2. folgt direkt
		\[\limessup{n} a_n = sup(H((a_n)_n))\]
	\end{bew}
	\begin{lem}
		Sei \((a_n)_n\) eine reelle beschränkte Folge.\\
		Es gilt: 
		\begin{enumerate}
			\item 
				die Folge \(((\underset{k \geq n}{sup}(a_k))\) ist monoton fallend und beschränkt. Weiter gilt:
				\[\limessup{n} a_n = \limes{n} \underset{k \geq n}{sup} (a_k) = \underset{n \in \N}{inf} \; \underset{k \geq n}{sup} (a_k)\]
			\item 
				die Folge \((\underset{k \geq n}{inf} (a_k))\) ist monoton wachsend und beschränkt. Weiter gilt:
				\[\limesinf{n} a_n = \limes{n} \underset{k \geq n}{inf} (a_k) = \underset{n \in \N}{sup} \; \underset{k \geq n}{inf} (a_k)\]
		\end{enumerate}		
	\end{lem}
	\begin{bew}
		Monotonie und Beschränktheit (erledigt)\\
		\[\limes{n} \underset{k \geq n}{sup} (a_k) = \limessup{n} a_n\]
		\[\limessup{n} a_n \leq \underset{n \in \N}{inf} \; \underset{k \geq n}{sup} (a_k)\]
		(siehe Beweise vorher)\\
		Bleibt zu zeigen:
		\[\limessup{n} a_n \geq \limes{n} \underset{k \geq n}{sup} (a_k)\]
		Wir definieren: \(s_n := \underset{k \geq n}{sup} (a_k)\) und \(s := \limes{n} s_n\)\\
		Wir wollen zeigen: \(s \leq \limessup{n} a_n + \varepsilon \; \forall \varepsilon > 0\)\\
		Sei \(\varepsilon > 0\)\\
		Nach Satz 1 existiert \(N \in \N\) mit \(a_k \leq \limessup{n} a_n + \varepsilon\) für \(k \geq N\)\\
		Insbesondere gilt:\\
		\(s_n = sup\{a_k: k \geq N\} \leq \limessup{n} a_n + \varepsilon\)\\
		Also auch \(s = \limes{n} s_n \leq s_n \leq \limessup{n} a_n + \varepsilon\)\\
	\end{bew}	
	\begin{bsp}
		Bestimmung von Limes superior und Limes inferior:\\
		\(a_n := (-1)^n \cdot (1 + \frac{1}{n})\) für alle \(n \in \N\)\\
		Beh.: \(\limessup{n} a_n = 1\), \(\limesinf{n} a_n = -1\)\\
		\begin{bew}
			Wir haben:
			\[\underset{k \geq n}{sup} (a_k) =\begin{cases}
			1 + \frac{1}{n} & \text{n gerade}\\
			1 + \frac{1}{n + 1} & \text{n ungerade}
			\end{cases} \]
			\(\Rightarrow \limes{n} \underset{k \geq n}{sup} (a_k) = 1\)\\
			Weiter gilt:
			\[\underset{k \geq n}{inf} (a_k) =\begin{cases}
			-(1 + \frac{1}{n} )& \text{n ungerade}\\
			-(1 + \frac{1}{n + 1}) & \text{n gerade}
			\end{cases} \]
			\(\Rightarrow \limes{n} \underset{k \geq n}{inf} (a_k) = -1\)\\
		\end{bew}
	\end{bsp}
	\begin{bsp}
		\(b_n := (1 + (-1)^n) \cdot (-1)^{\frac{n(n+1)}{2}} \)\\
		Beh.: \(\limessup{n} b_n = 2\), \(\limesinf{n} b_n = -2\)\\
		Zunächst gilt für \(n \in \N\)\\
		\(n = 4k, k \in \N \Rightarrow \frac{n(n+1)}{2} = 2k(4k+1)\) ist gerade\\
		\(\Rightarrow (-1)^{\frac{n(n+1)}{2}} = 1\)\\
		\(n = 4k - 1, k \in \N \Rightarrow \frac{n(n+1)}{2} = (4k-1)2k\) ist gerade\\
		\(\Rightarrow (-1)^{\frac{n(n+1)}{2}} = 1\)\\
		\(n = 4k - 2, k \in \N \Rightarrow \frac{n(n+1)}{2} = (2k-1)(4k-1)\) ist ungerade\\
		\(\Rightarrow (-1)^{\frac{n(n+1)}{2}} = -1\)\\
		\(n = 4k - 3, k \in \N \Rightarrow \frac{n(n+1)}{2} = (4k-3)(2k-1)\) ist ungerade\\
		\(\Rightarrow (-1)^{\frac{n(n+1)}{2}} = -1\)\\
		Betrachten wir die Teilfolgen \((b_{4k})_k, (b_{4k-1})_k, (b_{4k-2})_k, (b_{4k-3})_k\)\\
		\(b_{4k} = (1 + (-1)^{4k}) \cdot 1 = 2 \overset{k \rightarrow \infty}{\rightarrow} 2\)\\
		\(b_{4k-1} = (1 + (-1)^{4k-1}) \cdot 1 = 0 \overset{k \rightarrow \infty}{\rightarrow} 0\)\\
		\(b_{4k-2} = (1 + (-1)^{4k-2}) \cdot (-1) = -2 \overset{k \rightarrow \infty}{\rightarrow} -2\)\\
		\(b_{4k-3} = (1 + (-1)^{4k-3}) \cdot (-1) = 0 \overset{k \rightarrow \infty}{\rightarrow} 0\)\\
		Nach Aufgabe \(31\) von Zettel \(7\) gilt, dass
		\[H((b_n)_n) = \{-2,0,2\}\]
		Somit gilt nach heutiger Übung\\
		\[\limessup{n} b_n = 2\]
		\[\limesinf{n} b_n = -2\]
	\end{bsp}
	\section{Reihenkonvergenz (14.12.18)}
	\begin{bsp}
		Grenzwert von:
		\[\sum_{n=1}^{\infty} \frac{1}{4n^2 - 1}\]
		Für \(n \geq 1\) haben wir folgende Zerlegung:
		\[\frac{2}{4n^2 - 1} = \frac{2}{(2n - 1)(2n + 1)} = \frac{1}{2n-1} - \frac{1}{2n+1}\]
		\[s_k = \sum_{n=1}^{k} \frac{1}{4n^2-1} = \frac{1}{2}\left(\sum_{n=1}^{k} \frac{1}{2n-1} - \sum_{n=1}^{k} \frac{1}{2n+1} \right) = \frac{1}{2}\left(\sum_{n=0}^{k-1} \frac{1}{2n+1} - \sum_{n=1}^{k} \frac{1}{2n+1} \right)\]
		\[= \frac{1}{2}\left(1 - \frac{1}{2k+1}\right)\]
		Somit gilt:
		\[\sum_{n=1}^{\infty} \frac{1}{4n^2 - 1} = \limes{k} s_k = \limes{k} \frac{1}{2}\left(1 - \frac{1}{2k+1}\right) = \frac{1}{2} \]
	\end{bsp}
	\begin{bsp}
		Konvergenz von:
		\[\sum_{n=1}^{\infty} \frac{1}{n^2+1} \left(= \frac{1}{2}(1 + \pi \coth(\pi)) \right)\]
		Wir können den Grenzwert noch nicht berechnen, aber wir können feststellen, ob die Reihe konvergiert oder nicht. Mittel dafür ist das Majorantenkriterium.\\
		Sei \(\sum_{n=0}^{\infty}c_n \) eine konvergente Reihe mit den nicht-negativen Reihengliedern und \((a_n)_n\) eine Folge mit
		\[|a_n| < c_n \forall n \in \N \Rightarrow \sum_{n=0}^{\infty} a_n \text{ absolut konvergent}\]
		\[\left|\frac{1}{n^2+1}\right| < \frac{1}{n^2}\]
		\(\Rightarrow \sum_{n=1}^{\infty} \frac{1}{n^2}\) konvergiert absolut, da \(\sum_{n=1}^{\infty} \frac{1}{n^2}\) nach Volesung konvergent ist.
	\end{bsp}
	\begin{bsp}
		Konvergenz von:\[\sum_{n=1}^{\infty} (-1)^n \cdot \binom{2n}{n}\]
		\((a_n)_n\) mit \(a_n = (-1)^n \cdot \binom{2n}{n} \quad \forall n\). Dann gilt
		\[|a_n| = \frac{(2n)!}{(n!)^2} = \frac{(n+1)(n+2) \cdot \hdots \cdot (n+n)}{1 \cdot 2 \cdot \hdots \cdot n} \geq 1\]
		Somit und da \((a_n)_n\) keine Nullfolge ist, kann die Reihe \(\sum_{n=1}^{\infty} (-1)^n \cdot \binom{2n}{n}\) nicht konvergieren.
	\end{bsp}
	\begin{bsp}
		Konvergenz von:\\
		Wir definieren \((a_n)_n\) mit \(a_n = \frac{1}{n\sqrt{n}} \quad \forall n \in \N\)\\
		\[\sqrt{n} < \sqrt{n+1} \text{ und } n < n+1 \Rightarrow \frac{1}{n\sqrt{n}} > \frac{1}{(n+1)\sqrt{n+1}}\]
		somit ist \((a_n)_n\) monoton fallend. Aus dem Cauchy'schen Verdichtungskriterium folgt, dass \(\sum_{n=1}^{\infty} \frac{1}{n\sqrt{n}}\) genau dann konvergiert, wenn die verdichtete Reihe
		\[\sum_{n=1}^{\infty} 2^n a_{2n} = \sum_{n=1}^{\infty} 2^n \frac{1}{2^n\sqrt{2^n}} = \sum_{n=1}^{\infty} \frac{1}{\sqrt{2^n}}\]
		Die Konvergenz von \(\sum_{n=1}^{\infty} \frac{1}{\sqrt{2^n}}\) folgt aus dem Quotientenkriterium
		\[\frac{\left|\frac{1}{\sqrt{2^{n+1}}}\right|}{\left|\frac{1}{\sqrt{2^n}}\right|} = \frac{\sqrt{2^n}}{\sqrt{2^{n+1}}} = \frac{1}{\sqrt{2^n}} < 1\]
		Da \(a_n > 0\) für alle \(n \in \N\) folgt somit, dass \(\sum_{n=1}^{\infty} \frac{1}{n\sqrt{n}}\) absolut konvergiert.
	\end{bsp}
	\begin{satz}[Quotientenkriterium]
		Sei \(\sum_{n=0}^{\infty} a_n\) eine Reihe mit \(a_n \neq 0\) für alle \(n \in \N: n \geq n_0\).\\
		Es gebe eine reelle Zahl \(\varphi\) mit \(0 \leq \varphi < 1\), sodass \(\left|\frac{a_{n+1}}{a_n}\right| \leq \varphi \quad \forall n \geq n_0\)\\
		Dann konvergiert die Reihe \(\sum_{n=0}^{\infty} a_n\) absolut.\\
		\textbf{Wichtig:}\\
		Die Bedingung im Quotientenkriterium lautet nicht 
		\[\left|\frac{a_{n+1}}{a_n}\right| \leq 1 \quad \forall n \geq n_0\]
		sondern
		\[\left|\frac{a_{n+1}}{a_n}\right| \leq \varphi \quad \forall n \geq n_0\]
		für \(\varphi < 1\) unabhängig von \(n\).\\
		Dass die erste Bedingung nicht ausreicht, kann man anhand von \(\sum_{n=0}^{\infty} \frac{1}{n}\) leicht sehen.\\
		Mit \(a_n = \frac{1}{n}\) gilt: 
		\[\left|\frac{a_{n+1}}{a_n}\right| = \frac{n}{\underbrace{n+1}_{\rightarrow \infty \; (n \rightarrow \infty)}} < 1 \quad \forall n \geq 1\]
		Auch bei bekannten konvergenten Reihen ist das Quotientenkriterium unter Umständen nicht anwendbar.
		\[\sum_{n=0}^{\infty} \frac{1}{n^2}; \quad \left|\frac{a_{n+1}}{a_n}\right| = \frac{n^2}{(n+1)^2} < 1 \quad \forall n \geq 1\]
		Somit ist das Quotientenkriterium nur eine hinreichende, aber keine notwendige Bedingung für die Konvergenz von Reihen.
	\end{satz}
	\begin{lem}
		Sei \((a_n)_n\) eine reelle Folge mit \(a_n \neq 0\) für \(n \in \N\). Dann sind die folgenden Aussagen äquivalent:
		\begin{enumerate}
			\item 
				Die Reihe \(\sum_{n=0}^{\infty} a_n\) konvergiert absolut
			\item
				Es gibt eine Folge \((c_n)_n\) mit \(c_n > 0 \quad \forall n \in \N\), so dass die Reihe \(\sum_{n=0}^{\infty} c_n\) konvergiert und	\(\frac{a_{n+1}}{a_n} \leq \frac{c_{n+1}}{c_n}\) gilt für fast alle \(n \in \N\).
		\end{enumerate}
	\end{lem}
	\begin{bew}
		(1) \(\Rightarrow\) (2):\\
		Die Reihe \(\sum_{n=1}^{\infty} a_n\) konvergiert absolut. Wähle \(c_n = |a_n|\) für fast alle \(n \in \N\), d.h. es gilt (2).\\
		(2) \(\Rightarrow\) (1):\\
		Es existiert eine Folge \((c_n)_n\) mit \(c_n > 0\) für fast alle \(n \in \N\).\\
		\(\sum_{n=1}^{\infty} c_n\) konvergiert und \(\left|\frac{a_{n+1}}{a_n}\right| \leq \frac{c_{n+1}}{c_n}\) für alle \(n \geq n_0\)\\
		Für alle \(m \geq n_0\) folgt:
		\[\left|\frac{a_{m}}{a_n}\right| = \prod_{n=n_0}^{m-1} \frac{c_{n+1}}{c_n} = \frac{c_{m}}{c_n}\]
		Dann gilt für alle \(m \geq n_0\)
		\[|a_m| \leq \frac{|a_{n_0}|}{\underbrace{c_{n_0}}_{c \in \R}} \cdot c_m \]
	\end{bew}
	\begin{bem}
		\(\sum_{n=1}^{\infty} |a_n| \leq c \cdot \sum_{n=1}^{\infty} c_n \leq \tilde{c} \in \R\)\\
		Somit erhalten wir die absolute Konvergenz der Reihe \(\sum_{n=1}^{\infty} a_n\) mit dem Majorantenkriterium.
	\end{bem}
	\begin{bem}
		Quotientenkriterium\\
		Das Quotientenkriterium ist nicht anwendbar, wenn \(\limes{n} \left|\frac{a_{n+1}}{a_n}\right| = 1\)\\
		Aber es gibt eine Verschärfung bzw. Abschwächung der Bedingung aus dem Quotientenkriterium.
	\end{bem}
	\begin{kor}
		Sei \((a_n)_n\) eine reelle Folge mit \(a_n \neq 0 \quad \forall n \in \N\), dann gilt:
		\[\left|\frac{a_{n+1}}{a_n}\right| \leq 1 - \frac{p}{n} \text{ für } p > 1 \text{ und fast alle } n \in \N\]
		so ist die Reihe \(\sum_{n=1}^{\infty} a_n\) absolut konvergent.
	\end{kor}
	\begin{bew}
		Wir definieren die Folge \((c_n)_n\) durch \(c_1 := 1\) und \(c_{n+1} := \frac{1}{n^p} \quad \forall n \in \N\).\\
		Aus der Vorlesung ist bekannt, dass \(\sum_{n=1}^{\infty} c_n\) genau dann konvergiert, wenn \(p > 1\) ist.\\
		Weiter gilt für \(n \geq 2\):
		\[\frac{c_{n+1}}{c_n} = \frac{(n-1)^p}{n^p} = \left(\frac{n-1}{n}\right)^p \overset{\text{Bernoulli}}{\geq} 1 - \frac{p}{n}\]
		Nach Voraussetzung gilt somit:
		\[\left|\frac{a_{n+1}}{a_n}\right| \leq 1 - \frac{p}{n} \leq \frac{c_{n+1}}{c_n} \text{ für fast alle } n \in \N\]
		Nach dem Lemma ist die Reihe \(\sum_{n=1}^{\infty} a_n\) somit absolut konvergent.
	\end{bew}
\end{document}