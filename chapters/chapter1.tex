%17.10.2018
\documentclass[../ana1.tex]{subfiles}
\begin{document}

\section{Was ist Analysis?}
\textbf{Mathematik:} Streng logisches Herleiten neuer Aussagen (aus möglichst wenigen Grundannahmen, sog. Axiomen).\\
\textbf{Analysis:} Aus dem altgriech. \glqq Auflösen\grqq. Analysis hat ihre Grundlage in der \glqq Infinitisemalrechnung\grqq  \; von Leibnitz und Newton.\\
\textbf{Zentrale Begriffe:} Grenzwerte von Folgen und Reihen, Funktionen, stetig, differenzierbar, integrieren, Differential- und Integralrechnung, Differentialgleichungen (Newton, Maxwell, Schrödinger), unendlich dimensionale Räume
\begin{bsp}
	\(S = \frac{1}{2} + \frac{1}{4} + \dots + \frac{1}{2^n} + \dots\\
	2S = 1 + \frac{1}{2} + \dots + \nicefrac{1}{2} + \dots\\
	2S = 1 + S\)\\
	\(S\) entspricht der Wahrscheinlichkeit, dass irgendwann mal Kopf in einem Münzwurf kommt.\\
	Vorsicht!\\
	\(S = 1 + 2 + 4 + \dots\\
	2S = 2 + 4 + 8 + \dots = -1 + 1 + 2 + 4 + \dots = -1 + S\\
	S = -1\)\\
	Natürlich Quatsch!\\
	Formales Rechnen kann gefährlich sein!

	\begin{itemize}
		\item Was sind mathematische Aussagen?
		\item Wie macht man Beweise, wie findet man sie? (learning by doing)
		\item logische Zusammenhänge
		\item Was sind Zahlen?
	\end{itemize}
\end{bsp}
\end{document}