\documentclass[../ana1.tex]{subfiles}
\begin{document}
\setcounter{section}{4}
%09.11.2018
\section{Induktion}
\subsection{Starke Induktion und das Wohlordnungsprinzip}
\begin{satz}[starke Induktion]
	Seien $A(n)$ Aussagen für $n\in\mathbb{N}$. Dann gilt
	\begin{enumerate}
		\item $A(1)$ ist wahr
		\item $\forall n\in\mathbb{N}: A(1), \ldots, A(n)$ wahr $\Rightarrow A(n+1)$ ist wahr
	\end{enumerate}
	$\Rightarrow \forall n\in\mathbb{N}$ ist $A(n)$ wahr
\end{satz}
\begin{bew}
	Definiere die Aussage $B(n) := \{$alle $A(k)$ mit $k\leq n$ sind wahr$\}\Rightarrow$
	\begin{enumerate}
		\item $B(1)$ ist wahr
		\item Ist $B(n)$ wahr für ein $n\in\mathbb{N}$, so ist $B(n+1)$ wahr
	\end{enumerate}
	$\Rightarrow B(n)$ ist wahr für alle $n\in\mathbb{N}$.
\end{bew}
\begin{bem}
	$(\forall n\in\mathbb{N}:A(k) \forall k<n \Rightarrow A(n)) \Leftrightarrow \forall n\in\mathbb{N} A(n)$.
\end{bem}
\begin{satz}[Wohlordnungsprinzip für $\mathbb{N}$]
	Jede nichtleere Teilmenge der natürlichen Zahlen $\mathbb{N}$ hat ein kleinstes Element.
\end{satz}
\begin{bew}
	Sei $A(n):= \{$Jede Teilmenge $b\subset\mathbb{N}$ mit $m\in B$ hat ein kleinstes Element$\}$.\\
	Müssen zeigen: $A(n)$ ist wahr für alle $n\in\mathbb{N}$.
	\begin{enumerate}
		\item $A(1)$ ist wahr, denn ist $B\subset \mathbb{N}$ mit $1\in B$, so folgt $\forall k \in B: l\geq 1$. Also ist $1$ kleinstes Element in $B$.
		\item Angenommen für $n\in\mathbb{N}$ sind $A(1),\ldots,A(n)$ wahr. Sei $B\subset \mathbb{N}$ mit $n+1\in B$.\\
		      \underline{1. Fall:} $\{1,\ldots,n\}\cap B = \emptyset \Rightarrow n+1$ ist kleinstes Element in $B$.\\
		      \underline{2. Fall:} $\{1,\ldots,n\} \cap B \neq \emptyset \Rightarrow \exists k\in \{1,\ldots,n\}$ mit $k\in B$.\\
		      Aus der Induktionsannahme folgt also $A(k)$ ist wahr. $\Rightarrow B$ hat ein kleinstes Element.
	\end{enumerate}
	In beiden Fällen hat $B$ ein kleinstes Element, also ist $A(n+1)$ wahr.\\
	$\overset{\text{Satz 1}}{\Rightarrow} \forall n\in \mathbb{N} A(n)$ wahr.
\end{bew}
Notation:\\
Ganze Zahlen $\mathbb{Z} := (-\mathbb{N})\cup \mathbb{N}_0 = \{0, \pm 1, \pm 2, \ldots\} = \{\ldots, -2,-1,0,1,2,\ldots\}$.\\
Rationale Zahlen: $\mathbb{Q} := \{ \frac{m}{n} \vert n\in\mathbb{N}, m\in\mathbb{Z}\}$.
\begin{kor}
	Jede nichtleere, nach unten beschränkte Teilmenge in $\mathbb{Z}$ hat ein kleinstes Element.
\end{kor}
\begin{bew}
	Sei $A\subsetneq \mathbb{Z}, A\neq \emptyset, A\geq \beta$ für $B\in\mathbb{Z}$\\
	Setze $B:= A+\beta +1 =\{\alpha+|\beta|+1\vert \alpha\in A\}\subsetneq \mathbb{N}, B\neq \emptyset\\
		\overset{\text{Satz 2}}{\Rightarrow} \exists n_0:= \min B \Rightarrow z_0 := n_0 - |\beta|-1\in\mathbb{Z}$ ist kleinstes Element von $A$.
\end{bew}
\subsection{Anwendungen}
\begin{lem}
	Sei $a\in\mathbb{R}$ mit $a>0$. Dann existiert $q\in\mathbb{N}_0$ mit $q\leq a<q+1$
\end{lem}
\begin{bew}
	Ist $0<a<1$, so nehme $q = 0$.\\
	Also $a\geq 1$ und setze $B:=\{n\in\mathbb{N}|a<n\}$.\\
	Da $\mathbb{N}$ nicht nach oben beschränkt ist (archim. Prinzip), gilt $B\neq \emptyset$.\\
	$\overset{\text{Satz 2}}{\Rightarrow} m:=\min B$ existiert. Da $m\in B$, ist $m> a \geq 1$.\\
	Somit gilt nach Satz 3.5.8, dass $q:= m-1\in\mathbb{N}$.\\
	Da $m$ die kleinste natürliche Zahl mit $m<a$ ist, folgt $q = m-1 \leq a < m = q+1$.
\end{bew}
\begin{bem}
	Dieselbe Beweisidee zeigt auch $$\forall a\in\mathbb{R}\exists q\in\mathbb{Z} \text{ mit } q\leq a<q+1.$$
\end{bem}
\begin{satz}[$\mathbb{Q}$ ist dicht in $\mathbb{R}$]
	Seien $a,b\in\mathbb{R}, a<b$. Dann existiert $r\in\mathbb{Q}$ mit $a<r<b$.
\end{satz}
\begin{bew}
	O.B.d.A. $b\geq 0$, ansonsten betrachte $a'=-a, b'=-b$.\\
	Weiter können wir $a\geq 0$ annehmen, sonst nehme $r=0$.
	Also sei $0\leq a <b \overset{\text{Archimedes}}{\Rightarrow} \exists n\in\mathbb{N}: n(b-a)>1$.\\
	Setze $B:=\{l\in\mathbb{N}|l>na\} \subset \mathbb{N}$.
	$$\overset{\text{Satz 5.1.2}}{\Rightarrow} m = \min B \text{ existiert}.$$
	Da $m=\min B$ ist, gilt $$m -a\leq na<m,$$ somit gilt auch $$na<m=\underbrace{m-1}_{<na}+\underbrace{1}_{<n(b-a)}=nb$$
	$$\Rightarrow na<m,nb \Leftrightarrow a<\frac{m}{n}<b.$$
\end{bew}
\textbf{Exkurs}\\
Beh.: $\sqrt{2}\in\mathbb{R}\setminus \mathbb{Q}.$
\begin{bew}[Beweis durch Widerspruch]
	Sei $r^2=2$ mit $r = \frac{m}{n}, n\in\mathbb{N}, m\in\mathbb{Z}$.\\
	Wir definieren $$A:= \{n\in\mathbb{N}|\exists m\in\mathbb{Z} \frac{m^2}{n^2}= 2\} \neq \emptyset$$
	$$\overset{\text{Satz 5.1.2}}{\Rightarrow} n_* = \min A \in \mathbb{N}$$
	Also existiert $m\in\mathbb{Z}_+$ mit
	$$m^2 = 2\cdot m_*^2 \Rightarrow m>n_*$$
	Außerdem gilt
	$$m=\sqrt{2}n_* \overset{\sqrt{2}>1}{\Leftrightarrow} 0< \underbrace{m-n_*}_{\in\mathbb{N}} = \underbrace{\overbrace{(\sqrt{2} - 1)}}^{>0}_{<1} n_* < n_*$$
	Nun gilt: $$\sqrt{2} = \frac{m}{n_*} = \frac{m(m-n_*)}{n_*(m-n_*)} \overset{m^2=2n_*^2}{=} \frac{2n_*^2-mn_*}{n_*(m-n_*)} = \frac{2n_*-m}{m-n_*}$$
	\Lightning $2n_* -m \in \mathbb{Z}, m-n_* < n_*$, aber $n_* = \min A$\\
	Somit kann kein $m\in\mathbb{Z}$ existieren, sodass $\frac{m^2}{n^2} = 2$ für beliebiges $n\in\mathbb{N}.$\\
	Also ist $\sqrt{2}$ per Definition der rationalen Zahlen in $\mathbb{R}\setminus\mathbb{Q}$.
\end{bew}
\begin{satz}
	Sei $k\in\mathbb{N}$, dann gilt entweder $\sqrt{k}\in\mathbb{N}$ oder $\sqrt{k}\in\mathbb{R}\setminus\mathbb{Q}$.
\end{satz}
\begin{bew}
	Sei $k\in\mathbb{N}$ und $\sqrt{k} \notin \mathbb{N}$.\\
	Angenommen $\sqrt{k}\in\mathbb{Q}$, also $\sqrt{k} = \frac{m}{n}, m\in\mathbb{Z},n\in\mathbb{N}$\\
	$A:=\{n\in\mathbb{N}|\exists m\in\mathbb{Z} \frac{m^2}{n^2} = k\}$
	$$\overset{\text{Satz 5.1.2}}{\Rightarrow}\exists n_* = \min A \in \N$$
	Sei $\frac{m}{n_*} = \sqrt{k}$, dann gilt
	$$m-n_* = \underbrace{(\sqrt{k}-1)}_{<1} n_*$$ %DURCHGESTRICHEN
	Aber wähle $q\in\N: q\leq \sqrt{k} < q+1$\\
	Existiert nach Lemma 5.2.1. Da $\sqrt{k} \notin \N$ gilt $q<\sqrt{k}<q+1$.\\
	Also gilt:
	$$0\overset{q<\sqrt{k}}{<} \underbrace{m-qn_*}_{\in\mathbb{N}} = (\underbrace{\sqrt{k}-q}_{<1})n_* < n_*$$
	Somit $$\sqrt{k} = \frac{m}{n_*} = \frac{m(m-qn_*)}{n_*(m-qn_*)} = \frac{kn_*^2 - mqn_*}{n_*(m-qn_*)}=\frac{kn_*-mq}{m-qn_*}$$
	\Lightning $n_* = \min A, m-qn_* < n_*$\\
	Somit muss $\sqrt{k}\in\R\setminus\Q$ sein.
\end{bew}

\end{document}