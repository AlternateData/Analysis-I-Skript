\documentclass[../ana1.tex]{subfiles}
\onlyinsubfile{\sectionNumbering} %Use numbering relative to sections and not subsection

\begin{document}
\setcounter{section}{4}

%09.11.2018
\section{Starke Induktion und das Wohlordnungsprinzip}

\begin{satz}[Starke Induktion]\label{satz:starke_ind}
	Seien \(A(n) \) Aussagen für alle \(n \in \N \). Gilt
	\begin{enumerate}[(i)]
		\item \(A(1) \) ist wahr.
		\item \(\forall \, n \in \N \colon \; A(1) \ko \ldots \ko A(n) \) wahr \(\implies A(n + 1) \) wahr.
	\end{enumerate}
	So ist ist \(A(n) \) wahr für alle \(n \in \N \).
\end{satz}
\begin{bew}
	Definiere die Aussage \(B(n) \coloneqq \set{ \text{alle } A(k) \text{ mit } k\leq n \text{ sind wahr} } \). Dann gilt
	\begin{enumerate}[(i)]
		\item \(B(1) \) ist wahr
		\item Ist \(B(n) \) wahr für ein \(n \in\N \), so ist \(B(n+1) \) wahr
	\end{enumerate}
	\(\implies B(n) \) ist wahr für alle \(n \in \N \).
\end{bew}

\begin{bem} \leavevmode \\
	\(\displaystyle\left[\forall \, n \in \N \, \forall \, k < n \colon \; A(k) \implies A(n)\right] \iff \, \forall \, n \in \N \colon \; A(n). \)
\end{bem}

\begin{satz}[Wohlordnungsprinzip für \(\N \)]\label{satz:wohlordprinz} \leavevmode \\
	Jede nichtleere Teilmenge der natürlichen Zahlen \(\N \) besitzt ein kleinstes Element.
\end{satz}
\begin{bew}
	Sei \(A(n)\coloneqq \set{\forall \, \emptyset \neq B \subseteq \N \text{ mit } n \in B \text{ hat ein kleinstes Element}} \).
	Zu zeigen ist also: \(A(n) \) ist wahr für alle \(n \in \N \). Es gilt
	\begin{enumerate}[(i)]
		\item \(A(1) \) ist wahr, denn ist \(B \subseteq \N \) mit \(1 \in B \), so folgt \(\forall \, k \in B \colon \; k \geq 1 \).
			  Also ist \(1 \) kleinstes Element in \(B \).
		\item Angenommen für \(n \in \N \) sind \(A(1) \ko \ldots \ko A(n) \) wahr. Sei \(B \subseteq \N \) mit \(n + 1 \in B \).
			  \begin{faelle}
				\item{Fall 1:} \(\set{1 \ko \ldots \ko n} \cap B = \emptyset \implies n + 1 \) ist kleinstes Element in \(B \).
				\item{Fall 2:} \(\set{1 \ko \ldots \ko n} \cap B \neq \emptyset \implies \, \exists \, k \in \set{1 \ko \ldots \ko n} \) kleinstes Element von \(B \) nach \(\IV \).
			  \end{faelle}
		      \(\implies B \) hat ein kleinstes Element.
	\end{enumerate}
	In beiden Fällen hat \(B \) ein kleinstes Element, also ist \(A(n+1) \) wahr.\\
	\(\overset{\text{\autoref{satz:starke_ind}}}{\implies} \forall \, n \in \N \colon \; A(n) \) wahr.
\end{bew}

\iftoggle{short}{}{\newpage}%Formatierung ausführliches Skript

\begin{notation}\leavevmode
	\begin{enumerate}[(a)]
		\item \(\Z \coloneqq \minus \N \cup \N_0 = \set{0 \ko \pm 1 \ko \pm 2  \ko \ldots} \) ist die Menge der \textit{ganzen Zahlen}.
		\item \(\Q \coloneqq \set{\frac{m}{n} \; \vert \; n \in \N \ko m \in \Z } \) ist die Menge der \textit{rationalen Zahlen}.
	\end{enumerate}
\end{notation}

\begin{kor}
	Jede nichtleere, nach unten beschränkte Teilmenge in \(\Z \) besitzt ein kleinstes Element.
\end{kor}
\begin{bew}
	Sei \(\emptyset \neq A \subsetneq \Z \ko A \geq \beta \) für ein \( \beta \in\Z \) \\
	Setze \(B \coloneqq A + \abs{\beta} + 1 = \set{\alpha + \abs{\beta} + 1 \; \vert \; \alpha \in A} \subseteq \N \ko B \neq \emptyset. \\
	\overset{\text{\autoref{satz:wohlordprinz}}}{\implies} \exists \, n_0 \coloneqq \min B\\
	\overset{\phantom{\text{\autoref{satz:wohlordprinz}}}}{\implies} z_0 \coloneqq n_0 - \abs{\beta} - 1 \in \Z \) ist kleinstes Element von \(A \).
\end{bew}

\begin{lem}\label{satz:gauss_klammer}
	Sei \( a \in\R \) mit \( a > 0 \). Dann existiert \(q \in \N_0 \) mit \(q \leq a < q + 1 \).
\end{lem}
\begin{bew}
	Ist \( 0 < a < 1 \), so nehme \( q = 0 \). Sei also \(a \geq 1 \) und setze \(B \coloneqq \set{ n \in \N \; \vert \; a < n } \). \\
	Da \( \N \) nicht nach oben beschränkt (\autoref{satz:arch_prinz}) ist, gilt \(B \neq \emptyset \). \\
	\(\overset{\text{\autoref{satz:wohlordprinz}}}{\implies} m \coloneqq \min B \) existiert. Da \(m \in B \), ist \(m > a \geq 1 \). \\
	Somit gilt nach \autoref{satz:prop_N}, dass \(q\coloneqq m-1\in\N \).\\
	Da \( m \) die kleinste natürliche Zahl mit \(m < a \) ist, folgt \(m - 1 \leq a < m \).
\end{bew}

\begin{bem}
	Sei \(a \in \R \). Dann existiert \(q \in \Z \) mit \(q \leq a < q + 1 \).
\end{bem}
\begin{bew}
	Analog zu \autoref{satz:gauss_klammer}.
\end{bew}

\begin{satz}[\(\Q \) ist dicht in \(\R \)]
	Seien \(a \ko b \in \R \ko a < b \). Dann existiert \( r \in \Q \) mit \(a < r < b \).
\end{satz}
\begin{bew}
	\Obda{} sei \(b \geq 0 \), ansonsten betrachte \( a^\prime = \minus{b} \ko b^\prime = \minus{a} \). \\
	Weiter können wir \(a \geq 0 \) annehmen, sonst nehme \(r = 0 \). \\
	Sei also \(0 \leq a < b \overset{\text{\autoref{satz:arch_prinz}}}{\implies} \exists \, n \in \N \colon \; n(b-a) > 1 \). \\
	Setze \( B \coloneqq \set{ l \in \N \;\vert \; l > na } \subseteq \N \).
	\( \overset{\text{\autoref{satz:wohlordprinz}}}{\implies} m = \min B \) existiert.\\
	Ferner gilt \(m-a \leq na < m \) und somit
	\[ na < m = \underbrace{m-1}_{< na} + \underbrace{1}_{< n(b-a)} < nb \]
	\(\implies na < m  < nb \iff a < \frac{m}{n} < b\).
\end{bew}

\iftoggle{short}{}{\newpage}%Formatierung ausführliches Skript

\begin{bem}
	\(\sqrt{2} \in \R \setminus \Q \).
\end{bem}
\begin{bew}[Beweis durch Widerspruch]
	Angenommen \( r^2 = 2 \) mit \( r = \frac{p}{q} \ko q \in \N \ko p \in \Z \). \\
	Wir definieren 
	\[A \coloneqq \set{n \in \N \; \vert \; \exists \, m \in \Z \colon \; \frac{m^2}{n^2} = 2} \neq \emptyset \]
	\(\overset{\text{\autoref{satz:wohlordprinz}}}{\implies} n_* = \min A \in \N \) existiert. Also existiert \( m \in \Z_+ \) mit
	\(m^2 = 2n_*^2\). Somit ist \(m > n_* \).
	Ferner gilt
	\[ m= \sqrt{2}n_* \overset{\sqrt{2} > 1}{\iff} 0 < \underbrace{m - n_*}_{\in \N} = \underbrace{(\sqrt{2} - 1)}_{\in (0 \ko 1)} n_* < n_* \]
	Nun gilt: 
	\[ \sqrt{2} = \frac{m}{n_*} = \frac{m(m - n_*)}{n_*(m-n_*)} \overset{m^2 = 2n_*^2}{=} \frac{2n_*^2 - mn_*}{n_*(m - n_*)} = \frac{2n_* - m}{m - n_*} \]
	Somit \(2n_* - m \in \Z \ko m - n_* < n_* \). \Lightning{ zu \(n_* = \min A \)} \\
	Somit kann kein \(m \in \Z \) existieren, sodass \(\frac{m^2}{n^2} = 2 \) für beliebiges \(n \in \N \).
\end{bew}

\begin{satz}\label{satz:sqrt_irr}
	Sei \(k \in \N \), dann ist entweder \(\sqrt{k} \in \N \) oder \(\sqrt{k} \in \R \setminus \Q \).
\end{satz}
\begin{bew}
	Sei \(k \in \N \) und \(\sqrt{k} \notin \N \). \\
	Angenommen \(\sqrt{k} \in \Q \), also \(\sqrt{k} = \frac{m}{n} \ko m \in \Z \ko n \in \N \). \\
	Definiere \(A \coloneqq \set{n \in \N \; \vert \; \exists \, m \in \Z \colon \; \frac{m^2}{n^2} = k} \)
	\(\overset{\text{\autoref{satz:wohlordprinz}}}{\implies}\exists \, n_* = \min A \in \N \). \\
	Sei \(\frac{m}{n_*} = \sqrt{k} \).\\
	Ferner existiert \(q \in \N \) mit \(q \leq \sqrt{k} < q+1 \) nach \autoref{satz:gauss_klammer} \\
	Also gilt
	\[ 0 < \underbrace{m-qn_*}_{\in \N} = (\underbrace{\sqrt{k}-q}_{\in (0 \ko 1)})n_* < n_* \]
	Somit 
	\[ \sqrt{k} = \frac{m}{n_*} = \frac{m(m - qn_*)}{n_*(m - qn_*)} = \frac{kn_*^2 - mqn_*}{n_*(m - qn_*)}=\frac{kn_* - mq}{m - qn_*} \]
	\Lightning{} zu \(n_* = \min A \ko m-qn_* < n_* \)
\end{bew}

\iftoggle{short}{}{\newpage}%Formatierung ausführliches Skript

\begin{bem}(Erweiterung von \autoref{satz:sqrt_irr} auf \(n \)-te Wurzeln) \leavevmode \\
	Seien \(n \in \N \ko k \in \Z \) und \(x \in \R \) mit \(x^n = k \). Dann ist entweder \(x \in \Z \) oder \(x \in \R \setminus \Q \).
\end{bem}
\begin{bew}
	Sei \(\frac{a}{b} \eqqcolon x \) mit \(x^n = k \ko a \in \Z \ko b \in \N \) und \(a \ko b\) teilerfremd, \\
	\dphp{} für \(q \in \N \) mit \(a = qp_a \ko b = qp_b \) für \(p_a \ko p_b \in \Z \) folgt \(q = 1 \).
	Dann gilt
	\[{\left(\frac{a}{b}\right)}^n = \frac{a^n}{b^n} = k \iff \frac{a^n}{b} = \underbrace{kb^{n - 1}}_{\in \Z}. \]
	\(\implies \frac{a^n}{b} \in \Z \). Also muss \(b \) entweder \(1 \) oder \(\minus 1 \) sein.
	\(\implies \frac{a}{b} \in \Z \).
\end{bew}

\end{document}