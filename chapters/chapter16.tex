\documentclass[../ana1.tex]{subfiles}
\begin{document}
\setcounter{section}{15}
\section{Grenzwerte für Funktionen}
\begin{defi}
    Für \( x_0 \in \R^n, \delta > 0 \) ist
    \[ \dot{B}_\delta(x_0) = \set{ x \in \R^n \;\colon \; 
    0 < \abs{x-x_0} < \delta } = B_\delta (x_0) \setminus 
    \set{x_0} \]
    der punktierte \( \delta \)-Ball um \( x_0 \).\\
    Gegeben \( D \subset \R^n, x_0 \in \R^n \) heißt 
    Häufungswert (von \( D \)), falls 
    \[ \forall \, \delta > 0 : \underbrace{ 
    \dot{B}_\delta(x_0) \cap D}_{ = \set{ x\in D \;\colon \;
    0 < \abs{ x - x_0 } < \delta } } \neq \emptyset. \]
    Die Funktion \( f: D\rightarrow \R^m \) konvergiert für 
    \( x\rightarrow x_0 \) gegen \( \alpha \in \R^m \), falls
    \[ \forall \,\varepsilon > 0 \,\exists \, \delta > 0 
    \;\colon \; \abs{ f(x) - \alpha } < \varepsilon \,\forall \, 
    x \in D\text{ mit } 0 < \abs{x-x_0} < \delta. \]
\end{defi}
\begin{notation}
    \[ \limesx{x}{x_0} f(x) = \alpha \text{ oder } f(x) 
    \rightarrow \alpha \text{ für } x\rightarrow x_0 \]
\end{notation}
\begin{bem}\leavevmode
    \begin{enumerate}
        \item \( x_0 \) ist der Häufungswert von \( D \) (Berührpunkt)
        \[ \Leftrightarrow \exists \text{ Folge } {(x_n)}_n \subset D, 
        x_n \neq x_0, x_n \rightarrow x_0. \text{ (HA)} \]
        \item Für die Existenz und den Wert von \( \limesx{x}{x_0} f(x) \)
        ist es egal, ob die Funktion in \( x_0 \) definiert ist oder welchen 
        Wert sie dort hat.

    \end{enumerate}
\end{bem}
\begin{lem}[Stetigkeit und Grenzwert]
    Sei \( D \subset \R^n, f : D \rightarrow \R^m, x_0 \in D \) 
    Häufungspunkt von \(D\). Dann sind äquivalent
    \begin{enumerate}
        \item \( \limesx{x}{x_0} f(x) = f(x_0) \)
        \item \( f \) ist stetig in \( x_0 \).
    \end{enumerate}
\end{lem}
\begin{bew}
    In Formeln lauten die beiden Aussagen 
    \begin{enumerate}
        \item \( \forall \, \varepsilon > 0 \,\exists \, \delta > 0
        \; \colon \; \abs{ f(x) - f(x_0) } < \varepsilon \;\forall \;
        x \in D \) mit \( 0 < \abs{x-x_0} < \delta \).
        \item \( \forall \, \varepsilon > 0 \,\exists \, \delta > 0
        \; \colon \; \abs{ f(x) - f(x_0) } < \varepsilon \;\forall \;
        x \in D \) mit \( \abs{x-x_0} < \delta \).
    \end{enumerate}
    Da sowieso \( \abs{f(x) -f(x_0)} = 0 \) für \( x=x_0 \), sind 1.\ und
    2.\ äquivalent.\\
    Ist \( D\subset \R \) nach oben oder nach unten unbeschränkt, so 
    kann man Grenzwerte \( x \rightarrow \pm \infty \) studieren.
\end{bew}
\begin{defi}[Grenzwert bei \( \pm \infty \)]
    Sei \( D \subset \R \) nach oben unbeschränkt und 
    \( f:D\rightarrow \R^m \). Dann gilt 
    \[ \limes{x} f(x) = \alpha \in \R^m, \]
    falls
    \[ \forall \,\varepsilon > 0 \,\exists \, K\in\R \text{ mit } 
    \abs{f(x) - \alpha} < \varepsilon \,\forall \, x \in D 
    \text{ mit } x > K. \]
    Analog: \( \limesx{x}{-\infty} f(x). \)
\end{defi}
\begin{defi}[Einseitiger Grenzwert]
    Ist \( D \subset \R, f : D\rightarrow \R^m \) und \( x_0 \in D \)
    mit \( (x_0, x_0 + \delta) \cap D \neq \emptyset \) für alle 
    \( \delta > 0 \). Dann ist \( \limesx{x}{x_0+} f(x) = \alpha \) 
    (rechtsseitiger Grenzwert), falls
    \[ \forall \,\varepsilon > 0 \,\exists \, \delta > 0 \;\colon \;
    \abs{ f(x) - \alpha } < \varepsilon \,\forall \, \underbrace{ x\in D \cap 
    (x_0, x_0 + \delta) }_{ \text{oder: } x\in D \text{ mit } 
    x_0 < x < x_0 + \delta }. \]
    Ist \( (x_0 - \delta, x_0) \cap D \neq \emptyset \,\forall \, 
    \delta > 0 \), so ist \( \limesx{x}{x_0-} f(x) = \alpha \), falls
    \[ \forall \, \varepsilon > 0 \,\exists \, \delta > 0 \;\colon \; 
    \abs{f(x) - \alpha} < \varepsilon \,\forall \, x \in D \cap 
    (x_0 - \delta, x_0). \]
\end{defi}
\begin{bsp}
    \[ x\in\R, \mathrm{sgn}(x) := 
    \begin{cases}
        1,  &x>0\\
        0,  &x=0\\
        -1, &x<0
    \end{cases} \]
    \[ \Rightarrow \limesx{x}{0+} \mathrm{sgn}(x) = 1 \quad \limesx{x}{0-} \mathrm{sgn}(x) = -1 \]
    und \( \limesx{x}{0} \mathrm{sgn}(x) \) existiert nicht.
\end{bsp}
\begin{bem}
    Ist \( f: D \rightarrow \R^m, D \leq \R \) und \( x_0 \in D \),\\
    \( D \cap (x_0, x_0 + \delta) \neq \emptyset \,\forall \, 
    \delta > 0 \). Dann gilt: 
    \[ \limesx{x}{x_0} f(x) = a \Leftrightarrow
    \limesx{x}{x_0-} f(x) = a \text{ und } \limesx{x}{x_0+} 
    f(x) = a. \]
\end{bem}
\begin{bew}
    HA
\end{bew}
Um Grenzwerte von Funktionen zu bestimmen, kann man auch 
mit Folgen arbeiten.
\begin{satz}[\( \limesx{x}{x_0} f(x) \) mit Folge]
    Sei \( D \subseteq \R^n \), \(x_0 \) Häufungspunkt von 
    \( D \), \( f: D \rightarrow \R^m \). Für \( a\in \R^m \) 
    sind äquivalent:
    \begin{enumerate}
        \item \( f(x) \rightarrow a \) für \( x\rightarrow x_0 \)
        \item \( \limes{k} f(x_k) = a \,\forall \, {(x_k)}_k 
        \subseteq D \setminus \{x_0\} \) mit \( x_k \rightarrow x_0 \).
    \end{enumerate}
\end{satz}
\begin{bew}
    Analog zu Satz 15.0.4
    \( 1 \Rightarrow 2 \): Sei \( \varepsilon > 0 \). Dann 
    existiert \( \delta > 0 \) mit 
    \[ (*) \abs{f(x) - a} < \varepsilon \,\forall \, x\in D 
    \text{ mit } 0 < \abs{x-x_0} < \delta. \]
    Sei \( {(x_k)}_k \subseteq D \setminus \{x_0\} \) mit 
    \( x_k \rightarrow x_0 \).\\
    Da \( \abs{x-x_0} < \delta \) für fast alle \(k\) gilt, 
    ist wegen \( (*) \)
    \[ \abs{f(x_k) - a} < \varepsilon \text{ für fast alle } k\in\N. \]
    \( \Rightarrow \limes{k} f(x_k) = a \).\\
    \( 2 \Rightarrow 1\): Wir zeigen 
    \( \neg 1 \Rightarrow \neg 2\).
    Ist \( 1 \) falsch, so existiert \( \varepsilon > 0 \), sodass
    gilt: 
    \[ \forall \, \delta > 0 \,\exists \, x \in D \text{ mit } 0 < 
    \abs{f(x_k) - a} \geq \varepsilon. \]
    Wähle \( \delta = \frac{1}{k}, k\in\N \Rightarrow \exists \, x_k 
    \in D\setminus \{ x_0 \} : \abs{x_k - x} < \frac{1}{k} \) und 
    \( \abs{f(x_k) - a} \geq \varepsilon \).\\
    \( \Rightarrow f(x_k) \) konvergiert nicht gegen \(a\). 
    \( \Rightarrow \neg 2 \) ist wahr.
\end{bew}
\begin{satz}[Rechenregel für Grenzwerte]
    Sei \( x_0 \in D \subset \R^n \) Häufungspunkt von \( D \). 
    Dann gilt:
    \begin{enumerate}
        \item Sind \( f, g: D \rightarrow \R^m, f(x) \rightarrow a, 
        g(x) \rightarrow b \) für \( x \rightarrow x_0 \).
        \[ \lambda f(x) + \mu g(x) \rightarrow \lambda a + \mu b
        \text{ für } x \rightarrow x_0. \]
        \item Sind \( f, g: D \rightarrow \C, f(x) \rightarrow a,
        g(x) \rightarrow b \) für \( x\rightarrow x_0 \).\\
        \[ \Rightarrow f(x) g(x) \rightarrow a b \text{ für } 
        x\rightarrow x_0. \]
        Ist \( b\neq 0 \), so ist \( \forall \, x\in 
        D \cap B_\delta(x_0) : g(x) \neq 0 \) und 
        \[ \frac{f(x)}{g(x)} \rightarrow \frac{a}{b} \text{ für } 
        x\rightarrow x_0. \]
        \item Sind \( f: D \rightarrow \R^m, f(D) \subseteq E 
        \subseteq \R^m, g : E \rightarrow \R^l \), gilt 
        \( f(x) \rightarrow y_0 \) für \( x\rightarrow x_0 \) und
        ist \(g\) stetig in \( y_0 \), so folgt
        \[ (g\circ f)(x) = g(f(x)) \rightarrow g(y_0) \text{ für }
        x \rightarrow x_0. \]
        \item Ist \( f: D\rightarrow \R, f(x) \rightarrow a \) für 
        \( x \rightarrow x_0 \) und \( f(x) \geq 0 \,\forall \, 
        x \in D \cap B_\delta(x_0) \\ 
        \Rightarrow a \geq 0 \).
        \item Ist \( f : D \rightarrow (0, \infty) \), so ist 
        \( \limesx{x}{x_0} f(x) = 0 \) äquivalent zu 
        \( \limesx{x}{x_0} \frac{1}{f(x)} = \infty \).
    \end{enumerate}
\end{satz}
\begin{bew}
    Man überlege sich dies in Ruhe selbst.
\end{bew}
\begin{bspe}
    \begin{align*}
        f : \R \rightarrow \R, x \mapsto ax \text{ ist stetig für } a \in \R \\
        \Rightarrow x \mapsto x^2 = x \cdot x \text{ ist stetig.}\\
        \Rightarrow x \mapsto x^3 = x \cdot x^2 \text{ ist stetig.}\\
        \Rightarrow \text{Induktion, } x \mapsto x^n, n\in \N \text{ ist stetig.}\\
    \end{align*}
    \( \Rightarrow \) Summe. Sind \( n\in\N, a_0, \ldots,a_n \in \R \),
    dann ist 
    \[ p : \R \rightarrow \R, x\mapsto p(x) 
    = a_0 + a_1 x + \cdots + a_n x^n 
    \text{ stetig. (reelles Polynom)} \]
    Genauso: \( \R^2 \ni (x,y) \mapsto x + iy \) ist stetig und 
    \( (x,y) \mapsto z^n = {(x + iy)}^n \) ist für alle \( n\in\N \)
    stetig und \( z \mapsto \sum_{l=0}^n a_l z^l = a_0 + a_1 z^1 
    + \cdots + a_n z^n \) ist stetig. (komplexes Polynom)
\end{bspe}
\begin{satz}[Stetigkeit von Potenzreihen]
    Sei \( f(z) := \sum_{n=0}^\infty a_n z^n \) eine 
    Potenzreihe mit Konvergenzradius \( \rho > 0 \). 
    Dann ist \( f: B_\rho(0) \rightarrow \C \) stetig.
\end{satz}
\begin{bew}
    Sei \( z_0 \in B_\rho(0) \) (d.\ h.\  \( \abs{z_0} < \rho \)),
    \( \rho > 0 \), sodass \( \abs{z_0} + \delta < \rho \) und
    \[ {(z_n)}_n \subseteq B_\rho(0) \text{ mit } z_n 
    \rightarrow z_0. \]
    Dann gilt: \( \abs{z_n} \leq \abs{z_0} + \delta := \eta \) 
    für fast alle \(n\in\N \).\\
    Sei nun \( L\in\N \) so groß, dass \( \abs{z_n} \leq \eta 
    \,\forall \, n \geq L + 1 \).\\
    Dann gilt:
    \begin{align*}
        \abs{f(z_n) - f(z_0)} &= \abs{ \sum_{k=0}^\infty a_k z_n^k - \sum_{k=0}^\infty a_k z_0^k }\\
        &= \abs{ \sum_{k=0}^L a_k(z_n^k - z_0^k) + \sum_{k=L+1}^\infty a_k(z_n^k - z_0^k) }\\
        &\leq \sum_{k=0}^L \abs{a_k} \abs{(z_n^k - z_0^k)} + \sum_{k=L+1} \abs{a_k} \underbrace{( \abs{z_n}^k + \abs{z_0}^k )}_{\leq 2 \eta^k}\\
        &\leq \sum_{k=0}^L \abs{a_k} \underbrace{\abs{ z_n^k - z_0^k }}_{\rightarrow 0} + 2 \sum_{k=L+1}^\infty \abs{a_k} \eta^k
    \end{align*}
    Somit ist \( \limessup{n} \abs{f(z_n) 
    - f(z_0)} \leq \sum_{k=0}^L \abs{a_k} 
    \underbrace{\limessup{n} \abs{z_n^k - z_0}}_{=0} 
    + 2 \underbrace{\sum_{k=L+1}^\infty \abs{a_k} \eta^k}_{
        \rightarrow 0 (L\rightarrow \infty)} \) \\
    D.\ h.\  \(f\) ist stetig in \( z_0 \in B_\rho(0) \).\\
    Da \( z_0 \in B_\rho(0) \) beliebig war, 
    folgt die Behauptung.
\end{bew}
\begin{kor}
    Die Funktionen \( \exp, \sin, \cos : 
    \C \rightarrow \C \) sind stetig.
\end{kor}
\end{document}