\documentclass[../ana1.tex]{subfiles}
\onlyinsubfile{\sectionNumbering} %Use numbering relative to sections and not subsection

\begin{document}
\setcounter{section}{18}
\section{Satz von Rolle, Mittelwertsatz, Extrema}
\begin{defi}[Lokale Extrema]
    Die Funktion \( f : [a,b] \rightarrow \R \) (oder \( (a,b) \))
    hat in \( x_0 \in [a,b] \) ein lokales Minimum (oder in 
    \( x_0 \in (a,b) \)), falls \( \exists \, \delta > 0 \), 
    sodass 
    \[ f(x_0) \leq f(x) \, \forall \, x \in U_\delta(x_0)
    \cap [a,b] = (x_0 - \delta, x_0 + \delta) \cap [a,b] \]
    Ist \( x_0 \in (a,b) \), so kann \( \delta > 0 \) so klein 
    gewählt werden, dass \( (x_0 - \delta, x_0 + \delta) 
    \cap [a,b] = (x_0 - \delta, x_0 + \delta) \).
    %BILD
    Ist sogar \( f(x_0) < f(x) \,\forall \, x \in 
    (x_0 - \delta, x_0 + \delta) \cap [a,b] \), so heißt 
    das lokale Minimum isoliert (oder strikt).\\
    Analog: \( f \) hat in \( x_0 \) ein (indirektes) lokales 
    Maximum, falls \(-f\) in \(x_0\) ein (isoliertes) lokales 
    Minimum hat.\\
    Lokale Extrema \(=\) lokale Minima oder Maxima.
\end{defi}
\begin{satz}
    \( f: (a,b) \rightarrow \R \) habe in \(x_0 \in (a,b) \) 
    ein lokales Extremum. Ist \(f\) in \(x_0\) differenzierbar, 
    so gilt \( f'(x_0) = 0 \) (notwendige Bedingung für Extrema).
\end{satz}
\begin{bew}
    O.\ B.\ d.\ A.\ sei \(x_0\) ein lokales Minimum von \(f\) 
    (sonst betrachte \(-f\)).
    \[ \Rightarrow \exists \, \delta > 0 : f(x) \geq f(x_0) 
    \,\forall x \in (x_0 - \delta, x_0 + \delta) \]
    \[ \Rightarrow \frac{f(x) - f(x_0)}{x - x_0} = \begin{cases}
        \geq 0, &x \in (x_0, x_0 + \delta)\\
        \leq 0, &x \in (x_0 - \delta, x_0)
    \end{cases} \]
    \[ \Rightarrow f_+'(x_0) = \limesx{x}{x_0 +} 
    \frac{ f(x) - f(x_0) }{ x - x_0 } \geq 0 \]
    \[ = f_{\minus}'(x_0) = \limesx{x}{x_0 \minus} 
    \frac{ f(x) - f(x_0) }{ x - x_0 } \leq 0 \]
    \[ \Rightarrow f'(x_0) = f_{\pm}'(x_0) = 0. \]
    Rand: \(f: [a,b] \rightarrow \R \) \\
    hat lokales Minimum bei \( x_0 = b \).
    \[ \Rightarrow \frac{ f(x) - f(b) }{ x - b } \leq 0 \]
    \[ \Rightarrow f_{\minus}'(b) \leq 0. \]
\end{bew}
\begin{satz}[Rolle]
    Sei \( f: [a,b] \rightarrow \R \) stetig und differenzierbar 
    in \( (a,b) \) mit \( f(a) = f(b) = 0 \).
    \[ \Rightarrow \exists \, \xi \in (a,b) : f'(\xi) = 0. \]
\end{satz}
\begin{bew}
    \( [a,b] \) ist kompakt und \(f\) ist stetig.
    \[ \Rightarrow \exists \xi_1 \in (a,b), \xi_2 \in [a,b]: \]
    \[ f(\xi_1) = \sup( f(x), x \in [a,b] ), f(\xi_2) 
    = \inf (f(x) : x \in [a,b]). \]
    \( \Rightarrow \) ist \( \xi_1 \in (a,b) \), nehme 
    \( \xi := \xi_1 \overset{\text{Satz 2}}{\Rightarrow}
     f'(\xi) = 0 \).\\
    Ist \( \xi_2 \in (a,b) \), nehme \( \xi = \xi_2 \)
    \( \overset{\text{Satz 2}}{\Rightarrow} f'(\xi) = 0 \).\\
    Somit sind \( \xi_1, \xi_2 \in [a,b] \)
    \[ \sup(f(x) : x\in [a,b]) = \inf(\ldots) = 0. \]
    Also ist \( f(x) = 0 \,\forall \, x\in [a,b] \)
    \[ \Rightarrow f'(x) = 0 \, \forall \, x\in (a,b). \]
\end{bew}
\begin{satz}[Mittelwertsatz]
    
\end{satz}
\end{document}