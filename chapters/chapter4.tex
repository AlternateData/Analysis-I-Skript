%02.11.2018
\documentclass[../ana1.tex]{subfiles}
\begin{document}
\setcounter{section}{3}

\section{Abbildungen und Funktionen}

\subsection{Funktionen als Abbildungen}

\begin{defi}
	Eine Funktion (oder Abbildung) von einer Menge \(A \) in eine Menge \(B \) ordnet jedem Element \(a \in A \)
	ein eindeutiges Element \(b \in B \) zu. Schreibe:
	\[f \colon A \longrightarrow B \ko \quad a \mapsto f(a) \]
	Die Abbildung \(f \colon A \longrightarrow B \) heißt
	\begin{enumerate}[(a)]
		\item \textit{injektiv} \(\longeq \) Aus \(f(a) = f(a^\prime) \) mit \(a \ko a^\prime \in A \) folgt \(a = a^\prime \).
		\item \textit{surjektiv} \(\longeq \forall b \in B \, \exists a \in A \colon \; f(a) = b \).
		\item \textit{bijektiv} \(\longeq f \) ist injektiv und surjektiv. 
	\end{enumerate}
\end{defi}

\begin{bsp}
	\(f \colon \R \longrightarrow \R_{\geq 0} \coloneqq \set{x \in \R \; \vert \; x \geq 0} \ko \; x \mapsto x^{2} \) ist surjektiv und nicht injektiv.
\end{bsp}

\begin{bem}
	Sei \(f \colon A \longrightarrow B \) eine Funktion.
	\begin{enumerate}[(a)]
		\item \(f \injektiv \iff a \ko a^\prime \in A \ko \; a \neq a^\prime \Rightarrow f(a) \neq f(a^\prime) \)
		\item \(f \bijektiv \iff \forall \, b \in B \, \existse \,a \in A \colon \; f(a) = b \)
	\end{enumerate}
\end{bem}

\begin{defi*}
	Es sei \(f \colon A \longrightarrow B \) eine Funktion. \\
	Ist \(f \) bijektiv so definiert
	\[\inverse{f} \colon B \longrightarrow A \ko \quad b \mapsto a \text{ mit } f(a) = b \]
	die \textit{Umkehrfunktion von \(f \)}.
	Ist \(f \) nicht bijektiv, so definiert
	\[ \abb{\inverse{f}}{\PO(B)}{\PO(A)} \ko \quad B \supseteq M \mapsto \set{a \in A \; \vert \; f(a) \in M} \]
	die \textit{Urbildfunktion von \(f \)} eine verallgemeinerte Umkehrfunktion. \\
	Ist nun \(g \colon B \longrightarrow C \) eine weitere Funktion, so heißt
	\[g \circ f \colon A \longrightarrow C \ko \quad (g \circ f)(a) \coloneqq g \circ f(a) \coloneqq g(f(a)) \]
	die \textit{Verkettung von \(f \) und \(g \)}. Diese verkettete Funktion lässt sich durch folgendes Diagramm darstellen:
	\begin{center}
    \begin{tikzcd}
        A \arrow{r}{f \circ g} \arrow{d}{g} & C \\
        B \arrow{ur}{f}
    \end{tikzcd}
	\end{center}
\end{defi*}

\begin{defi*}[Identitätsabbildung]
	Sei \(A \) eine Menge, dann ist die \textit{Identität} \\
	\textit{auf \(A \)} definiert durch
	\[\id_{A} \colon A \longrightarrow A \ko \quad a \mapsto a. \]
\end{defi*}


\begin{bem}
	Sei \(f \colon A \longrightarrow B \) eine Funktion. Dann gilt
	\[f \bijektiv \implies \inverse{f} \circ f = \id_{A} \text{ und } f \circ \inverse{f} = \id_{B}. \]
\end{bem}


\subsection{Abbildungen als Graph}

\begin{defi}
	Seien \(A \ko B \) Mengen und \(a \in A \ko \, b \in B \). \\
	Dann heißt \((a \ko b) \) \textit{Tupel}.
	In der Mengenlehre sind diese definiert durch 
	\[(a \ko b) \coloneqq \set{ \set{\emptyset \ko \set{a}}\ko \set{b} }. \]
	Die Reihenfolge der Elemente spielt also eine wichtige Rolle und im Allgemeinen gilt \((a \ko b) \neq (b \ko a) \).
	Die Menge
	\[A \times B \coloneqq \set{(a \ko b) \; \vert \; a \in A \ko \, b \in B} \]
	heißt \textit{kartesisches Produkt} von \(A \) und \(B \).
\end{defi}

\begin{bsp}\(\R \times \R \) \\
	\begin{figure}[h!]
		\centering
		\subgraphic{0.3}{img03.pdf}
		\caption{Kartesisches Koordinatensystem der Ebene}
	\end{figure}
\end{bsp}

\begin{defi*}
	Seien \(A \ko B \) Mengen. Die Abbildungen
	\[\Pi_1 \coloneqq \Pi_A \colon A \times B \rightarrow A \ko \quad (a \ko b) \mapsto a \]
	\[\Pi_2 \coloneqq \Pi_B \colon A \times B \rightarrow B \ko \quad (a \ko b) \mapsto b \]
	heißen \textit{Projektion auf die erste bzw.\ zweite Koordinate}. \\
	Es gilt also \(\Pi_A(a \ko b) = a \) und \(\Pi_B(a \ko b) = b\).
\end{defi*}

\begin{defi*}
	Es seien \(A_1 \ko \ldots \ko A_n \) Mengen für \(n \in \N \). Dann sei \(A_1 \times A_2 \) definiert wie oben.
	Ferner definiere induktiv
	\[A_1 \times \cdots \times A_j \coloneqq (A_1 \times \cdots \times A_{j-1}) \times A_j \ko \quad j \in \set{2 \ko \ldots \ko n}.\]
\end{defi*}

\begin{bem}
	Seien \(A \ko B \ko C\) Mengen. Dann gilt
	\[(A \times B) \times C = A \times (B \times C) \text{ mit } ((a \ko b) \ko c) = (a \ko (b \ko c)).\]
	Es existiert also eine Bijektion \(\Phi \colon (A \times B) \times C \longrightarrow A \times (B \times C)\).
\end{bem}

\begin{defi}[Graph einer Abbildung]\leavevmode \\
	Sei \(f \colon A \longrightarrow B \) Funktion. Die Menge
	\[\Gamma \coloneqq \Gamma_f \coloneqq \set{(a \ko b) \in A \times B \colon \; b = f(a)} \varsubsetneq A \times B \]
	heißt \textit{Graph von \(f\)}.
	\iftoggle{short}{}{
		\begin{figure}[h!]
			\centering
			\subfloat[Graph]{\subgraphic{0.3}{img05.pdf}}
			\qquad
			\subfloat[Kein Graph]{\subgraphic{0.3}{img06.pdf}}
		\end{figure}
	}
\end{defi}

\begin{notation}[Einschränkung von Funktionen]\leavevmode \\
	Sei \(f \colon A \longrightarrow B \) eine Funktion und \(X \subseteq A\) eine Teilmenge von \(A \).
	\[\restr{f}{X} \colon X \longrightarrow B \ko \quad x \mapsto f(x)\]
	heißt \textit{Einschränkung von \(f \) auf \(X \)}.
\end{notation}

\begin{satz}\label{satz:graph}
	\(\Gamma \subset A \times B \) ist genau dann Graph einer Abbildung \(f \colon A \longrightarrow B \),
	wenn die Projektion \(\restr{\Pi_A}{\Gamma} \colon \Gamma \longrightarrow A \) bijektiv ist.
\end{satz}
\begin{bew}
	\equirl{
		Sei \(\Gamma = \Gamma_f \) wobei \(f \colon A \rightarrow B \)eine Funktion ist.\\
		\(\overunderset{(a \ko b) \in \Gamma_f }{ \Leftrightarrow b = f(a)}{\implies} \forall \, a \in A \)
		existiert genau ein \(b\in B \) mit \(f(a) = b \). \\
		\(\overunderset{\phantom{(a \ko b) \in \Gamma_f}}{\phantom{\Leftrightarrow b = f(a)}}{\implies} \restr{\Pi_A}{\Gamma} \) ist bijektiv.
	}{
		Sei \(\restr{\Pi_A}{\Gamma} \rightarrow A \) bijektiv. Dann gilt für \( (a_j \ko b_j) \in \Gamma \ko j \in \set{1 \ko 2} \) \\
		mit \( \Pi_A(a_1 \ko b_1) = \Pi_A(a_2 \ko b_2) \) auch \((a_1 \ko b_1) = (a_2 \ko b_2) \).\\
		\(\implies a_1 = a_2 \ko \, b_1 = b_2 \implies \forall \, a \in A \, \existse \, b \in B \colon \; (a \ko b) \in \Gamma \),\\
		da \(b = \Pi_B(a \ko b) = \Pi_B(\inverse{\left(\restr{\Pi_A}{\Gamma}\right)}(a)) \).\\
		Definiere \(f \coloneqq \Pi_B \circ \inverse{\left(\restr{\Pi_A}{\Gamma}\right)} \colon A \longrightarrow B\). Nachrechnen zeigt \(\Gamma = \Gamma_f \).\qedhere
		\begin{figure}[H]
			\centering
			\subgraphic{0.25}{img07.pdf}
			\caption{Situation im Beweis von \autoref{satz:graph}}
		\end{figure}
	}
\end{bew}

\begin{bem}
	In \autoref{satz:graph} gilt \(f = \Pi_B \circ \inverse{\left(\restr{\Pi_A}{\Gamma}\right)} \). %ENTSPRICHT 4.2
\end{bem}

\begin{bsp}
	Ist \(f \colon A \longrightarrow B \) bijektiv, also \(f(a) = b \iff \inverse{f}(b) = a\), so gilt \\
	\[\inverse{\Gamma_f} = \set{(b \ko \inverse{f}(b)) \; \vert \; b \in B} = \set{(f(a), a) \; \vert \; a \in A} = S(\Gamma_f) \ko \]
	wobei \(S \colon A \times B \longleftrightarrow B \times A \ko \, (a \ko b) \mapsto (b \ko a)\).
	\begin{figure}[h!]
		\centering
		\subfloat{\subgraphic{0.3}{img08.pdf}}
		\qquad
		\subfloat{\subgraphic{0.3}{img09.pdf}}
		\caption{\(\protect\Gamma_{f^{^{\protect\shortminus 1}}} \) entspricht Spiegeln von \(\protect\Gamma_{f} \) an der Winkelhalbierenden.}
	\end{figure}
\end{bsp}


\subsection{Schubfachprinzip und endliche Mengen}

\begin{notation}
	Sei \(n \in \N \). \([n] \coloneqq \set{1 \ko \ldots \ko n} \) ist induktiv gegeben durch
	\[[1] \coloneqq \set{1} \]
	\[[n + 1] \coloneqq [n] \cup \set{n + 1} \ko \quad n \in \N.\]
	Es bezeichne weiterhin \(\#[n] \coloneqq n \) die \textit{Mächtigkeit} von \([n] \).
\end{notation}

\begin{satz*}[Schubfachprinzip]\label{satz:schubfach}\leavevmode \\
	Ist \(f \colon [m] \longrightarrow [n] \) eine Funktion mit \(m > n \), so existiert ein \(k \in [n]\) mit 
	\(\inverse{f}(\set{k}) \supseteq \set{m_1 \ko m_2} \), wobei \(m_1 \neq m_2 \) ist.
\end{satz*}
\begin{bew}
	Angenommen für alle \(k \in [n] \) gilt \(\inverse{f}(\set{k}) \subseteq \set{j_k} \) für ein \(k_j \in [m] \).
	Ferner gilt \(\#[n] = n \). Da durch \(f \) jedem Element in \([m] \) genau ein Element in \([n] \) zugeordnet wird
	sind die \(j_k \), falls existent, alle unterschiedlich.
	Somit folgt, dass \(\inverse{f}([n]) \subseteq \set{j_1 \ko \ldots \ko j_n} \) für \(j_k \) paarweise verschieden \Lightning{ zu \(m > n\).}
\end{bew}

\begin{kor}\label{satz:schubfach:kor}\leavevmode \\
	Sind \(n \ko m \in\N \) und ist \(f \colon [m] \longrightarrow[n] \) injektiv \(\implies m \leq n \).
\end{kor}
\begin{bew}
	Angenommen \(m > n\), dann folgt aber, dass ein \(k \in [n] \) existiert mit \(\#\inverse{f}(\set{k}) \geq 2 \).
	Es gibt also \(m_1 \ko m_2 \in [m] \ko \; m_1 \neq m_2 \) \\
	mit \(f(m_1) = k = f(m_2) \).\Lightning{ zu \(f \) injektiv.}
\end{bew}
%%%%%%%%%%%%%%%%%%%%%%%%%%%%%%%%%%%%%%%%%%%%%%%%%%%%%%%%%%%%%%%%%%%%%%%%%%%%%%%%%%%%%%%%%%%%%%%%%%%%%%%%%%%%%%%%%%%%%%%%%%%%%%%%%%%%%%%%%
%%%Irgendwie ist dieser Beweis sehr umständlich und Lang. Ich habe mir die Freiheit genommen ihn durch eine kurze Version zu ersetzen.%%%
%%%%%%%%%%%%%%%%%%%%%%%%%%%%%%%%%%%%%%%%%%%%%%%%%%%%%%%%%%%%%%%%%%%%%%%%%%%%%%%%%%%%%%%%%%%%%%%%%%%%%%%%%%%%%%%%%%%%%%%%%%%%%%%%%%%%%%%%%
%%%\begin{bew}
%%%	Fassen obige Aussage als \(A(n) \) auf, die für alle \(m\in\N \) zu zeigen ist. \\
%%%	Induktionsanfang:
%%%	\[ n=1: f:[m] \rightarrow \{1\} \injektiv \Rightarrow m=1, \text{ da sonst } f(1) = 1 = f(2) \text{\Lightning{} zu Injektivität}. \]
%%%	Induktionsschritt: \\
%%%	Induktionsvoraussetzung: \(A(n) \) ist wahr für \(n\in\N \). \\
%%%	Zu zeigen: \(A(n+1) \) ist wahr. \\
%%%	Angenommen, \(f:[m]\rightarrow[n+1] = [n] \cup \{n+1\} \) sei injektiv. \\
%%%	Zu zeigen: \(m\leq n+1 \) \\
%%%	Fallunterscheidung:
%%%	\begin{enumerate}
%%%		\item Ang.\  \(m=1 \Rightarrow m=1\leq n+1\checkmark{} \)
%%%		\item Ang.\  \(m>1, m\in\N \overset{\text{Satz 3.5.8}}{\Rightarrow} m-1\in\N \) \\
%%%		      \((*) \) Beh.: \(\exists \) inj.\  \(\tilde{f}: \{1,\ldots,m-1\}\rightarrow \{1,\ldots,n\} \). \\
%%%		      \[ \overset{(*) + \text{IV}}{\Rightarrow} m-1\leq n, \text{ \dphp } m\leq n+1 \Rightarrow A(n+1) \text{ ist wahr}. \]
%%%	\end{enumerate}
%%%	Beweis von \((*) \): \\
%%%	Angenommen, \(\exists f: [m]\rightarrow[n+1] \) inj. \\
%%%	Dann \(\exists \tilde{f}: [m+1]\rightarrow [m+1]\rightarrow[n] \) inj. \\
%%%	Fallunterscheidung:
%%%	\begin{itemize}
%%%		\item Ang. \(f(k)\in [n] \forall 1\leq k \leq m-1 \). Dann setze \(\tilde{f}(k) \coloneqq f\vert_{[m-1]} \\
%%%			  \tilde{f}(k) \coloneqq f(k), 1\leq k\leq m-1 \) \\
%%%		      (Nachrechnen \(\tilde{f} \) ist injektiv.)
%%%		\item \(\exists j\in\N, 1\leq j\leq m-1 \) mit \(f(j) = n+1 \). \\
%%%		      Dann definiere \(\tilde{f}: [m-1]\rightarrow [n] \)
%%%		      \[\tilde{f}(k) \coloneqq
%%%			      	\begin{cases}
%%%				    	f(k), 1 \leq k \leq m-1, k\neq j \\
%%%				      	f(m), k=j
%%%			      	\end{cases} \]
%%%		      Man prüfe nach \(\tilde{f}: [m-1]\rightarrow [n] \) injektiv!
%%%	\end{itemize}
%%%\end{bew}
%%%%%%%%%%%%%%%%%%%%%%%%%%%%%%%%%%%%%%%%%%%%%%%%%%%%%%%%%%%%%%%%%%%%%%%%%%%%%%%%%%%%%%%%%%%%%%%%%%%%%%%%%%%%%%%%%%%%%%%%%%%%%%%%%%%%%%%%%

\begin{kor}\label{satz:schubfach:kor_2}\leavevmode \\
	Sind \(n, m \in\N \) und ist \(f \colon [m] \longrightarrow[n] \) bijektiv, so gilt \(m = n \).
\end{kor}
\begin{bew}
	Nach Voraussetzung sind sowohl \(f \) als auch \(\inverse{f} \) injektiv. \\
	\(\implies m \leq n \wedge n \leq m \implies m = n \).
\end{bew}

\begin{defi}
	Seien \(M, A, B \) Mengen.
	\begin{enumerate}[(a)]
		\item \(M \) heißt \textit{endlich} 
		\(\longeq M = \emptyset \) oder 
		\(\exists \, n \in \N \, \exists \, 
		f \colon [n] \longrightarrow M \bijektiv \). \\
			  In diesem Fall heißt \( \# M \coloneqq n \) die \textit{Mächtigkeit von M}. \\
			  Für die leere Menge setzen wir \( \# \emptyset \coloneqq 0 \).
		\item \(M \) heißt \textit{unendlich} \(\longeq M \) ist nicht endlich.
		\item \(A, B \) heißen gleichmächtig \(\longeq \exists \, f \colon A \longrightarrow B \, \bijektiv \). \\
			  Schreibe auch \(A \sim B\).
		\item \(M \) heißt \textit{abzählbar} \(\longeq \exists \, f \colon \N \longrightarrow M \, \bijektiv \).
		\item \(M \) heißt \textit{überabzählbar unendlich}\\
		\(\longeq M \) ist unendlich und nicht abzählbar.
	\end{enumerate}
\end{defi}

\begin{bem}
	Ist \(M \) endlich, so ist \( \# M \) wohldefiniert.
\end{bem}
\begin{bew}
	Seien \( \abb{f}{[n]}{[M]}, \abb{g}{[m]}{M} \) bijektiv.
	\[\begin{tikzcd}
		& M &  \\
	   {[n]} \arrow[ru, "f"] &  & {[m]} \arrow[lu, "g"'] \arrow[ll, "h"']
	\end{tikzcd}\]
	\( h \coloneqq \inverse{f} \circ g \colon [m] \longrightarrow [n] \) ist auch bijektiv. \(\oversett{\autoref{satz:schubfach:kor_2}}{\implies} m=n \).
\end{bew}

\iftoggle{short}{}{\newpage}%Formatierung ausführliches Skript

\begin{bem}[Satz von Cantor und Berenstein]\label{satz:cantor_Berenstein}\leavevmode \\
	Seien \(A \ko B \) Mengen und \(f \colon A \longrightarrow B \) sowie \(g \colon B \longrightarrow A \) injektiv.
	Dann existiert eine Bijektion \(h \colon A \longrightarrow B \).
\end{bem}
\begin{bew}
	Siehe \cite[Kolmogorov-Fomin: Introductory Real Analysis]{Kolmogorov}.
\end{bew}

\begin{prosa}
	Die vorangegangene Bemerkung motiviert eine Definition der Form
	\[A \leq B \longeq \exists \, f \colon A \longrightarrow B \injektiv .\]
	Es bleibt dann zu zeigen, dass \(A \leq B \, \wedge \, B \leq A \iff A \sim B \).
\end{prosa}

\begin{defi*}
	Seien \(A \ko B \) Mengen. \\
	Existiert eine injektive Funktion \(f \colon A \longrightarrow B \) so schreibe \(A \leq B \). \\
	In diesem Fall sage auch \gqq{Die Kardinalität von \(A \) ist kleiner gleich der Kardinalität von \(B \)} oder
	\gqq{\(A \) ist weniger mächtig oder gleich mächtig wie \(B\)}.
\end{defi*}

\begin{bem}
	Seien \(A \ko B \) Mengen.
	\begin{enumerate}[(a)]
		\item Ist \(B \subset A \) und \(A \) endlich, so ist \(B \) endlich und \( \#B \leq \# A \).
		\item Sind \(A \ko B \) endlich und disjunkt, also \(A \cap B = \emptyset \) dann gilt \\
			  \(\#(A \cup B) = \#A + \#B \).
	\end{enumerate}
\end{bem}

\begin{satz}\label{satz:mengen_abz}\leavevmode
	\begin{enumerate}[(a)]
		\item Jede Teilmenge einer abzählbaren Menge ist abzählbar.
		\item Für alle \(j \in \N \) sei \(A_j \) eine abzählbare Menge. Dann ist
			  \[\bigcup_{j \in \N}{A_j} \]
			  abzählbar.
	\end{enumerate}
\end{satz}
\begin{bew}\leavevmode
	\begin{enumerate}[(a)]
		\item \begin{faelle}
			  	\item[Fall \(A \) endlich:] Jedes \(B \subseteq A \) ist endlich und somit abzählbar.
			  	\item[Fall \(A \) abzählbar unendlich:] Es existiert eine Bijektion \(f \colon \N \rightarrow A \).\\
				  Wir setzen \(a_n \coloneqq f(n) \, \forall \, n \in \N \). Dann ist \\
				  \[ A = \bigcup_{n \in \N}{\set{a_n}} = \set{a_1 \ko a_2 \ko \ldots} \]
				  Ist \(B \subseteq A \), so existieren \(n_j \in \N \ko 1 \leq n_1 < n_2 < \ldots \) mit \\ 
				  \[ B = \set{a_{n_1} \ko a_{n_2} \ko \ldots}. \]
				  Gibt es nur endlich viele \(n_j \), so ist \(B \) endlich, andernfalls ist \\
				  \(h \colon \N \rightarrow B \ko \, j \mapsto a_{n_j} \) eine Bijektion.
			  \end{faelle}
		\item \obda sind alle \(A_j \) paarweise verschieden, also \(A_l \cap A_m \neq \emptyset \) für \(l \neq m \).
		      Wenn nicht, betrachte
			  \[ B_1 \coloneqq A_1 \ko \quad B_2 \coloneqq A_2 \setminus A_1 \ko \quad B_3 \coloneqq A_3 \setminus \set{A_1 \cup A_2} \]
			  \[ B_{n+1} \coloneqq A_{n+1} \setminus \set{A_1 \cup \ldots \cup A_n} \]
		      Dann sind \(B_n \) paarweise verschieden und \[\bigcup_{j \in \N} B_j = \bigcup_{j \in \N} A_j. \]
			  Schreiben \(A_j \) als Folge \(A_j = \set{a_{1j} \ko a_{2j} \ko \ldots } \) \\
			  \begin{figure}[H]
				\centering
				\begin{tikzpicture}
					\matrix(m)[matrix of math nodes,column sep=1cm,row sep=1cm,nodes={rectangle, minimum height=2em, minimum width=2em,
							   anchor=center, %draw=gray,%align=center,
							   inner sep=0pt, outer sep=0pt}]{
						s_{11}  & s_{12}  & s_{13}  & s_{14}\;\cdots \\
						s_{21}  & s_{22} & s_{23} & s_{24}\;\cdots \\
						s_{31} & s_{32} & s_{33} & s_{34}\;\cdots \\
						\underset{\vdots}{s_{41}} & \underset{\vdots}{s_{42}} & \underset{\vdots}{s_{43}} & \ddots \\
					};
					\draw[->]
					(m-1-1)edge(m-1-2)
					(m-1-2)edge(m-2-1)
					(m-2-1)edge(m-3-1)
					(m-3-1)edge(m-2-2)
					(m-2-2)edge(m-1-3)
					(m-1-3)edge(m-1-4)
					(m-1-4)edge(m-2-3)
					(m-2-3)edge(m-3-2)
					(m-3-2)edge(m-4-1);
				\end{tikzpicture}
			  \end{figure}
			  Jetzt können wir das obige rechteckige Schema diagonal abzählen. \\
			  Dies liefert uns eine Bijektion von \(\N \) nach \(\bigcup_{j \in \N}{A_j} \).\qedhere
	\end{enumerate}
\end{bew}

\begin{bsp}
	\(f \colon \N \longrightarrow \N \times \N \ko \, n \mapsto (u \ko v) \) mit \(n = 2^{u - 1}(2v - 1) \) ist eine Bijektion. 
\end{bsp}

\iftoggle{short}{}{\newpage}%Formatierung ausführliches Skript

\begin{defi}\leavevmode
	\begin{enumerate}[(a)]
		\item Eine bijektive Abbildung
			  \(\sigma \colon \set{1 \ko \ldots \ko n} \rightarrow \set{1 \ko \ldots \ko n} \) heißt \\
			  \textit{Permutation} von \(\set{1 \ko \ldots \ko n} \).
		\item Die Menge
			  \[S_n \coloneqq \set{\sigma \colon \set{1 \ko \ldots n} \rightarrow \set{1 \ko \ldots \ko n \; \vert \; \sigma \bijektiv}} \]
			  bildet mit der Komposition eine Gruppe. \(S_n \) heißt auch \textit{symmetrische} \textit{Gruppe}. 
		\item Für \(n \in \N \) heißt \(n! = \prod_{k=1}^{n}k \) \textit{Fakultät} von \(n \).
	\end{enumerate}
\end{defi}

\begin{satz}
	\( \# S_n = n! \).
\end{satz}
\begin{bew}\leavevmode
	\induktion{
		\(\#S_1 = 1 = 1! \).
	}{
		Für ein \(n \in \N \) gelte \(\#S_n = n! \).
	}{
		Identifiziere \(\sigma \in S_n \) mit \(n \)-Tupeln. Dann lässt sich die Menge \(S_n \) als disjukte Vereinigung von Mengen
		\[S_{n+1 \ko k} \coloneqq \set{\tau \in S_{n+ 1} \; \vert \; \tau_k = n + 1} \ko \quad k = 1 \ko \ldots \ko n + 1 \]
		schreiben. \zB \(S_{3 \ko 2} = \set{(1 \ko 3 \ko 2) \ko (2 \ko 3 \ko 1)} \). \\
		Jedem \(\tau = (\sigma_1 \ko \ldots \ko \sigma_n) \in S_n \) können wir die Permutation
		\[\tau_k \coloneqq (\sigma_1 \ko \ldots \ko \sigma_{k-1} \ko \underbrace{n + 1}_{k \text{-te Stelle}} \ko \sigma_k \ko \ldots \ko \sigma_n) \in S_{n + 1 \ko k} \]
		zuordnen. \(\tau_k \) ist bijektiv, also eine Permutation (Übung). \\
		\(\begin{aligned}[t]
		  	&\implies &\#S_{n + 1 \ko k} &= \#S_n \overset{\IV}{=} n! \\
		  	&\implies &\#S_{n + 1}       &= \#(\bigcup_{k=1}^{n+1} S_{n+1 \ko k} ) = \sum_{k=1}^{n+1}\#S_{n+1 \ko k} = \sum_{k=1}^{n+1} n! \\
		  	&			&		           &= (n+1)n! = (n+1)!
		 \end{aligned}\)
	}
\end{bew}

\begin{defi}[Binomialkoeffizient]
	Seien \(\alpha \in \R \ko k \in \N \).
	\[ \binom{\alpha}{k} \coloneqq \frac{\alpha(\alpha-1)\cdots(\alpha-k+1)}{k!} \ko \quad \binom{\alpha}{0} \coloneqq 1 \]
	heißt Binomialkoeffizient von \(\alpha \) und \(k \).
\end{defi}

\begin{lem}[Rekursionsformel für Binomialkoeffizienten]\label{satz:bin_rek} \leavevmode \\
	Seien \(\alpha \in \R \ko k \in \N \). Dann gilt
	\[ \binom{\alpha + 1}{k} = \binom{\alpha}{k} + \binom{\alpha}{k-1}. \]
\end{lem}
\begin{bew}
	Für \(k=1 \) ist dies einfach zu sehen. Für \(k \geq 2 \) gilt\\
	\[\begin{aligned}[t]
		& \binom{\alpha}{k} + \binom{\alpha}{k-1} &=& \frac{\alpha (\alpha-1)\cdots(\alpha-k+1)}{1\cdot2\cdots k} + \frac{\alpha (\alpha-1)\cdots(\alpha-k+2)}{1\cdot2\cdots (k-1)} \\
		&                                         &=& \frac{\alpha (\alpha-1)\cdots(\alpha-k+2)(\alpha-k+1+k)}{1\cdot2\cdots k} \\
		&                                         &=& \frac{(\alpha + 1)\alpha (\alpha-1)\cdots((\alpha-1)-k+1)}{1\cdot2\cdots k} = \binom{\alpha + 1}{k}
	  \end{aligned}\]
\end{bew}

\begin{bem}[Pascalsches Dreieck] \leavevmode
	\begin{enumerate}[(1)]
		\item Ist \(\alpha = n \in\ N_0 \), so können wir \(\binom{n}{k} \) ausrechnen mit dem Dreiecksschema von Blaise Pascal (1623\textendash1662). \\
		      \begin{tabular}{>{\(n=}l<{ \)\hspace{12pt}}*{13}{c}}
			      0 &   &   &   &   &    			&    		    							 & 1  										  &    			  &    &   &   &   &   \\
			      1 &   &   &   &   &    		    & 1\tikzmark{a} 							 & \tiny+  									  & 1\tikzmark{b} &    &   &   &   &   \\
			      2 &   &   &   &   & 1\tikzmark{c} & \tiny+									 & \tikzmark{d2}\(\overset{\tikzmark{d}}{2}\) &    			  & 1  &   &   &   &   \\
			      3 &   &   &   & 1 &    		    & \(\overset{\tikzmark{e}}{3}\)\tikzmark{e2} & \tiny+   							 	  & \tikzmark{f}3 &    & 1 &   &   &   \\
			      4 &   &   & 1 &   & 4  			&    										 & \(\overset{\tikzmark{g}}{6}\) 			  &   			  & 4  &   & 1 &   &   \\
			      5 &   & 1 &   & 5 &    			& 10 										 &    										  & 10 			  &    & 5 &   & 1 &   \\
			      6 & 1 &   & 6 &   & 15 			&    									     & 20 										  &    			  & 15 &   & 6 &   & 1
			  \end{tabular}
			  \begin{tikzpicture}[overlay, remember picture, yshift=.25\baselineskip, shorten >=.5pt, shorten <=.5pt]
				\draw [->] ([yshift=.75pt]{pic cs:a}) -- ({pic cs:d});
				\draw [->] ([yshift=.75pt]{pic cs:b}) -- ({pic cs:d});
				\draw [->] ([yshift=.75pt]{pic cs:c}) -- ({pic cs:e});
				\draw [->] ([yshift=.75pt]{pic cs:d2}) -- ({pic cs:e});
				\draw [->] ([yshift=.75pt]{pic cs:e2}) -- ({pic cs:g});
				\draw [->] ([yshift=.75pt]{pic cs:f}) -- ({pic cs:g});
			  \end{tikzpicture}
		\item Ist \( \alpha=n \in \N_0 \), so folgt durch Erweitern mit \((n-k)! \)
			  \[\binom{n}{k} = \frac{n!}{k!(n-k)!} = \binom{n}{n-k} \]
			  für \(n \in\N_0, k\in \{0,1,\ldots,n\} \).
	\end{enumerate}
\end{bem}

\begin{satz}[Zahl der Kombinationen]\label{satz:zahl_komb}\leavevmode \\
	Sei \( n \in \N_0 \ko k \in \set{1 \ko \ldots \ko n} \). Dann ist die Anzahl der \(k \)-elementigen Teilmengen von \( \set{1 \ko \ldots \ko n} \) gleich \( \binom{n}{k} \).
\end{satz}
\begin{bew}\leavevmode
	\induktion{
		Die Behauptung gilt für \(k=0 \) und beliebiges \(n \in\N \), da die leere Menge die einzige Teilmenge von \( \set{1 \ko \ldots \ko n} \) mit \(0 \) Elementen ist
		und nach Definition ist \( \binom{n}{0} = 1 \). Insbesondere gilt die Behauptung dann für \(n=0 \). 
	}{
		Für ein \(n \in \N \) gelte für alle \(k \in \set{1 \ko \ldots \ko n} \), dass \\
		\[\#\set{A \subseteq [n] \; \vert \; \#A = k} = \binom{n}{k}. \]
	}{
		Sei \(A \subseteq \set{1 \ko \ldots \ko n + 1} \) mit \(\#A = k \geq 1 \). Dann ist entweder \(A \) eine \(k \)-elementige Teilmenge von \(\set{1 \ko \ldots \ko n} \)
		oder lässt sich als disjunkte Vereinigung \(A = B \cup \set{n + 1} \) schreiben, wobei \(B \) eine \((k-1)\)-elementige Teilmenge von \(\set{1 \ko \ldots \ko n} \) ist.
		Also gilt \\
		\(\begin{aligned}
			\#\set{A \subseteq [n+1] \; \vert \; \#A = k}&=\phantom{+} \#\set{A \subseteq [n] \; \vert \; \#A = k} \\
			                               			     &\phantom{=}+\#\set{B \subseteq [n] \; \vert \; \#B = k - 1} \\
			  		                   	    \overset{\IV}&{=} \binom{n}{k} + \binom{n}{k-1} \\
					   	 \overset{\text{\autoref{satz:bin_rek}}}&{=} \binom{n+1}{k}.
		  \end{aligned}\)
	}
\end{bew}

\begin{satz}[Binomische Formel] Seien \(a \ko b \in \R \ko n \in \N \). Dann gilt
	\[{(a+b)}^n = \sum_{k=0}^{n} \binom{n}{k} a^k b^{n-k}. \]
\end{satz}
\begin{bew}\leavevmode
	\induktion{
		\({(a+b)}^1 = a+b = \sum_{k=0}^{1}\binom{1}{k} a^k b^{1-k} \).
	}{
		Für ein \(n \in \N \) gelte \({(a+b)}^n = \sum_{k=0}^{n} \binom{n}{k} a^k b^{n-k} \), für alle \(a \ko b \in \R \).
	}{
		\(\!\begin{aligned}[t]
			{(a+b)}^{n+1} =& (a+b) {(a+b)}^n = (a+b) \sum_{k=0}^{n} \binom{n}{k} a^k b^{n-k} \\
		   				  =& \sum_{k=1}^{n+1} \binom{n}{k-1} a^k b^{n-(k-1)} + \sum_{k=0}^{n} \binom{n}{k} a^k b^{n+1-k} \\
						  =& \binom{n}{n} a^{n+1} + \sum_{k=1}^{n} \binom{n}{k-1} a^k b^{n+1-k} \\
						   &+ \sum_{k=1}^{n} \binom{n}{k} a^k b^{n+1-k} + \binom{n}{0} b^{n+1} \\
		   				  =& a^{n+1} + \sum_{k=1}^{n} \underbrace{\left(\binom{n}{k-1} + \binom{n}{k}\right)}_{=\binom{n+1}{k}} a^k b^{n+1-k} + b^{n+1} \\
		   				  =& \sum_{k=0}^{n+1} \binom{n+1}{k} a^k b^{n+1-k}. \qquad\qquad\qquad\qquad\qquad\qquad\quad\;\;\, \qedhere
	   \end{aligned}\)%%qed ist sehr schlecht gemacht grade
	}
\end{bew}

\begin{bspe}\leavevmode
	\begin{enumerate}[(1)]
		\item \({(a+b)}^1 = a+b \)
		\item \({(a+b)}^2 = a^2 + 2ab + b^2 \)
		\item \({(a+b)}^3 = a^3 +3a^2b + 3ab^2 +b^3 \)
	\end{enumerate}
\end{bspe}

\begin{defi*}
	Sei \(A \ko B \) eine Menge.
	\[B^A \coloneqq \set{f \; \vert \; f \colon A \longrightarrow B } \]
	ist die Menge aller \(B\)-wertigen Funktionen mit Definitionsbereich \(A \). \\
	Insbesondere betrachten wir Funktionen der Form
	\[\set{0 \ko 1}^A \coloneqq \set{f \; \vert \; f \colon A \longrightarrow \set{0 \ko 1} }. \]
\end{defi*}

\begin{satz}\label{satz:kard_fn}
	Sei \(A \neq\emptyset \) eine endliche Menge. Dann ist \(\#( \set{0,1}^A) = 2^{\#A}\).
\end{satz}
\begin{bew}
	Sei \(\#A \eqqcolon n \in \N \). Also existiert eine Bijektion \(h \colon [n] \longrightarrow A\) und
	wir können \obda{} annehmen, dass \(A = \set{1 \ko 2 \ko \ldots \ko n} \).
	Es bleibt also zu zeigen, dass \(\#\set{0 \ko 1}^{[n]} = 2^n \).
	\induktion{
		Für \(n = 1 \) existieren genau zwei Funktionen \(f_1 \ko f_2 \)von \(A = \set{1} \) nach \(\set{0 \ko 1} \).
		Diese sind gegeben durch \(f_1(1) = 0 \ko f_2(1) = 1 \).
	}{
		Für ein \(n \in \N \) gelte \(\#\set{0 \ko 1}^{[n]} = 2^n \).
	}{
		Sei \(f \colon [n + 1] \longrightarrow \set{0 \ko 1} \) eine Funktion. Dann gehört \(f \) zu genau einer von zwei
		disjunkten Mengen \(S_0 \) und \(S_1 \) mit
		\[S_0 = \set{f \colon \set{1 \ko \ldots \ko n} \rightarrow \set{0 \ko 1} \; \vert \; f(n + 1) = 0 }\ko\]
		\[S_0 = \set{f \colon \set{1 \ko \ldots \ko n} \rightarrow \set{0 \ko 1} \; \vert \; f(n + 1) = 1 }.\]
		\(\begin{aligned}[t]
			\implies& S_0 \cap S_1 = \emptyset \ko \set{0 \ko 1}^{[n]} = S_0 \cup S_1 \text{ und} \\
			        & \#S_0 = \#S_1 = \#(\set{0 \ko 1}^{[n]}) \overset{\IV}{=} 2^n \\
			\implies& \#(\set{0 \ko 1}^{[n + 1]}) = \#S_0 + \#S_1 = 2^n + 2^n = 2^{n + 1}.
		  \end{aligned}\)
	}
\end{bew}

\begin{kor}
	Sei \(A \) eine endliche Menge. Dann gilt \(\#\PO(A) = 2^{\#A} \).
\end{kor}
\begin{bew}
	Sei \(A = \emptyset \implies \PO(\emptyset) = \set{\emptyset} \), also \(\#\PO(A) = 1 = 2^0 \). \\
	Nach \autoref{satz:kard_fn} genügt zu zeigen, dass eine Bijektion \(\varphi \colon \PO(A) \longrightarrow \set{0 \ko 1}^A \) existiert.
\end{bew}

\begin{lem}\label{satz:bij_pow}
	Sei \(A \neq \emptyset \). Dann sind \( \PO(A) \) und \(\set{0 \ko 1}^A \) gleichmächtig.
\end{lem}
\begin{bew}
	Sei \(B \subseteq A \) und definiere die Indikatorfunktion auf \(B \) folgendermaßen:
	\[ \1_B(x) \coloneqq \begin{cases}
			1 \ko & x \in B \\
			0 \ko & x \in A \setminus B
		\end{cases}\]
	Beachte: \(B = \set{x \in A \; | \; \1_B(x) = 1 } = \inverse{\1_B}(\set{1}) \). \\
	Definiere nun \( \varphi \colon \PO(A) \longrightarrow \set{0 \ko 1}^A \ko \, B \mapsto \1_B \). \\
	Sei \(f \in \set{0 \ko 1}^A \). Mit \(B_f \coloneqq \inverse{f}(\set{1}) = \set{a \in A \; \vert \; f(a) = 1} \) \\
	folgt \(\varphi(f) = \1_{B_f} = f \). Also ist \(\varphi \) surjektiv. \\
	Seien nun \(B_1 \ko B_2 \subseteq A \) mit \(B_1 \neq B_2 \). \obda{} gelte \(B_1 \setminus B_2 \neq \emptyset \). \\
	Dann existiert ein \(x \in B_1 \setminus B_2 \subseteq A \) und es gilt \(\1_{B_1}(x) = 0 \neq 1 = \1_{B_2}(x) \). \\
	Es folgt \(\varphi(B_1) = \1_{B_1} \neq \1_{B_2} = \varphi(B_2) \). Also ist \(\varphi \) injektiv und damit bijektiv.
\end{bew}

\begin{lem}
	Es existiert keine surjektive Funktion \(f \colon A \longrightarrow \PO(A) \).
\end{lem}
\begin{bew}
	Sei \(f \colon A \longrightarrow \PO(A) \) eine Funktion, also \(f(a) \subseteq A \) für alle \(a \in A \). \\
	Betrachte \(R \coloneqq \set{a \in A \; \vert \; a \notin f(a)} \subseteq A \) und nehme an \(f \)  ist surjektiv. \\
	Somit existiert ein \(a \in R \) mit \(R = f(a)\).
	\begin{faelle}
		\item[Fall \(a \in R \):] \(a \in f(a) = R =  \set{a \in A \; \vert \; a \notin f(a)} \). \Lightning{}
		\item[Fall \(a \notin R\):] \(a \notin f(a) \implies a \in R \). \Lightning{} 
	\end{faelle}
	Folglich kann \(f \) nicht surjektiv sein.
\end{bew}

\onlyinsubfile{
	\bibliographystyle{unsrt}
	\IfFileExists{ana-skript.bib}{\bibliography{ana-skript}}{\bibliography{../ana-skript}}
}
\end{document}