\documentclass[../ana1.tex]{subfiles}
\onlyinsubfile{\sectionNumbering} %Use numbering relative to sections and not subsection

\begin{document}
\setcounter{section}{19}
\section{Funktionenfolgen}
Im Folgenden ist \( D \subset \R \) (man kann aber auch 
für die Stetigkeit \( D \subset \R^d, d \in \N \) betrachten). \\
Gegeben eine Folge von Funktionen \( {(f_n)}_n \).
\[ f_n: D \rightarrow \R \ (\text{oder } f_n: D \rightarrow \C) \]
Welche Art von Konvergenzbegriff ist sinnvoll?
\begin{defi}[Punktweise Konvergenz]
    \( f_n : D \rightarrow \C \) konvergiert punktweise 
    gegen \( f: D \rightarrow \C \), falls 
    \[ \limes{n} f_n(x) = f(x) \; \forall \, x \in D. \]
\end{defi}
\begin{bsp}
    \[ f_n: [0,1] \rightarrow \R, f_n(x) = \begin{cases}
        1 - nx, &0 \leq x \leq \frac{1}{n} \\
        0, &\frac{1}{n} < x \leq 1
    \end{cases}. \]
    \( f_n \) konvergiert punktweise gegen 
    \[ f: [0,1] \rightarrow \R, f(x) = 
    \begin{cases}
        1, &x = 0 \\
        0, &0 < x \leq 1
    \end{cases}. \]
    Beachte: Die Grenzfunktion \(f\) ist nicht stetig 
    (in \(0\)).
\end{bsp}
\begin{defi}
    \( f: D \rightarrow \C \) heißt beschränkt, falls 
    \[ \underset{x\in D}{\sup} \abs{f(x)} < \infty \]
    \begin{align*}
        B(D, \C) :=\ &\set{ f: D \rightarrow \C, 
        f \text{ ist beschränkt} } \\
        =\ &\text{Vektorraum der beschränkten 
        komplexwertigen Funktionen} \\
        B(D, \R) :=\ &\set{ f: D \rightarrow \R, 
        f \text{ ist beschränkt} } \\
        =\ &\text{Vektorraum der beschränkten reellen Funktionen}
    \end{align*}
    Für \( f\in B(D,\R) \) (oder \( B(D,\C) \)) setzen wir 
    \[ ||f||_\infty := \underset{x\in D}{\sup} \abs{f(x)} 
    \text{ (Supremumsnorm)} \]
    hat alle Eigenschaften einer Norm
    \begin{enumerate}
        \item \( ||f||_\infty \geq 0 \) und 
        \( ||f||_\infty = 0 \Leftrightarrow f = 0 \) 
        (Nullfunktion)
        \item \( ||\lambda f||_\infty = |\lambda| \cdot 
        ||f||_\infty \) (homogen), \( \lambda \in \R \)
        \item \( ||f+g||_\infty \leq ||f||_\infty 
        + ||g||_\infty \) (Dreiecksungleichung) \\
        für \( f, g \in B(D, \R) \) (bzw. \( B(D,\C) \))
    \end{enumerate}
    Dies gilt, da
    \begin{align*}
        \abs{ f(x) + g(x) } &\leq \abs{f(x)} + \abs{g(x)} \\
        &\leq \underset{y\in D}{\sup} \abs{ f(y) } 
        + \underset{y\in D}{\sup} \abs{g(y)}\\
        &= ||f||_\infty + ||g||_\infty \\
        \Rightarrow ||f + g||_\infty &= \underset{x\in D}{\sup}
        \abs{f(x) + g(x)} \leq ||f||_\infty + ||g||_\infty
    \end{align*}
\end{defi}
\begin{defi}[gleichmäßige Konvergenz]
    Eine Folge \( \abb{f_n}{D}{\R} \) (oder \( \C \)) 
    konvergiert gleichmäßig gegen \( f: D \rightarrow \R \), 
    falls 
    \[ \limes{n} ||f_n - f ||_\infty = 0. \]
\end{defi}
\begin{bem} \leavevmode
    \begin{enumerate}
        \item Wir können auch erlauben, dass \(D \subset \C \) 
        ist. Dann ist 
        \[ ||f_n-f||_\infty = \underset{z \in D}{\sup}|f_n(z) 
        - f(z)|. \]
        \item Großer Unterschied zwischen punktweiser und 
        gleichmäßiger Konvergenz. Sei \( f_n : D \rightarrow \C \) 
        \begin{enumerate}[(a)]
            \item \( f_n \rightarrow f \) punktweise: 
            \[ \forall \, \varepsilon > 0 \; \forall \, x \in 
            D \; \exists \, K \in \N: \abs{ f_n(x) - f(x) } 
            < \varepsilon \; \forall \, n > K. \]
            \item \( f_n \rightarrow f \) gleichmäßig
            \[ \forall \, \varepsilon > 0 \; \exists \, K \in \N 
            \;\forall \, x \in D: \abs{ f_n(x) - f(x) } < \varepsilon 
            \;\forall \, n > K. \]
        \end{enumerate}
        Das heißt bei punktweiser Konvergenz darf \( K \) von 
        \( \varepsilon \) und \( x \) abhängen, bei gleichmäßiger 
        Konvergenz nur von \( \varepsilon \).
    \end{enumerate}
\end{bem}
\begin{lem}[lokal gleichmäßige Konvergenz von Potenzreihen]\label{satz:glmKonvPotenz}
    Sei \( P(z) = \sum_{n=0}^\infty a_n z^n \) eine Potenzreihe, 
    die für \( z_0 \in \C \setminus \set{0} \) konvergiert. \\
    Dann gibt es ein \( M \in [0,\infty) \) mit 
    \begin{align*}
        \abs{ P(z) - \sum_{l=0}^k a_l z^l } 
        &= \abs{ \sum_{l=k+1}^\infty a_l z^l }\\
        &\leq \sum_{l=k+1}^\infty \abs{ a_l } \abs{ z }^l \\
        &\leq \frac{ M }{ 1 - \frac{\abs{z}}{\abs{z_0}} } 
        {\left( \frac{\abs{z}}{\abs{z_0}} \right)}^{k+1} 
        \; \forall \, \abs{z} < \abs{z_0}
    \end{align*}
    Insbesondere gilt \( \forall \, 0 < \delta < 1 \)
    \[ \limes{k} \underset{\abs{z} \leq \delta \abs{z_0}}{\sup} 
    \abs{ P(z) - \sum_{l=0}^k a_l z^l } = 0. \]
    \Dphp{} das Polynom 
    \[ P_k(z) := \sum_{l=0}^k a_l z^l 
    = a_0 + a_1 z + \cdots + a_k z^k \]
    konvergiert auf der Kreisscheibe 
    \[ \set{ z \in \C \;\vert \; \abs{z} \leq 
    \delta \abs{z_0} } \]
    gleichmäßig gegen \( P \).
\end{lem}
\begin{bew}
    Da \( \sum_{n=0}^\infty a_n z_0^n \) konvergiert, ist 
    \( a_n z_0^n \) eine Nullfolge.
    \[ \Rightarrow M := \underset{n\in\N_0}{\sup} 
    \abs{a_n z_0^n} \in [0,\infty) \]
    \[ \Rightarrow \abs{a_n z^n} = \abs{a_n} \abs{z_0}^n 
    {\left(\frac{\abs{z}}{\abs{z_n}}\right)}^n 
    \leq M {\left(\frac{\abs{z}}{\abs{z_n}}\right)}^n \]
    Sei \( P_k(z) = \sum_{l=0}^k a_l z^l \)
    \begin{align*}
        \Rightarrow \abs{ P(z) - P_k(z) } 
        &= \abs{ \sum_{l=k+1}^\infty a_l z^l } \\
        &\leq \sum_{l=k+1}^\infty \abs{a_l z^l} \\
        &\leq \sum_{l=k+1}^\infty \abs{ a_l z_0^l } 
        {\left( \frac{\abs{z}}{\abs{z_0}} \right)}^l \\
        &\leq M \sum_{l=k+1}^\infty 
        {\left( \frac{\abs{z}}{\abs{z_0}} \right)}^l \\
        &\leq M {\left( \frac{\abs{z}}{\abs{z_0}} \right)}^{k+1} 
        \cdot \underbrace{\sum_{l=0}^\infty 
        {\left( \frac{\abs{z}}{\abs{z_0}} \right)}^l}_{
            \text{konvergiert, falls } \frac{\abs{z}}{\abs{z_0}} < 1
        }\\
        &= \frac{ M }{ 1-\frac{\abs{z}}{\abs{z_0}} } 
        {\left( \frac{\abs{z}}{\abs{z_0}} \right)}^{k+1}.
    \end{align*}
    \begin{align*} 
        \underset{\abs{z} \leq \delta \abs{z_0}}{\sup} 
        \abs{P(z) - P_k(z)} 
        &\leq \underset{\abs{z} \leq \delta \abs{z_0}}{\sup} 
        \frac{ M }{ 1 - \frac{\abs{z}}{\abs{z_0}} } 
        {\left( \frac{\abs{z}}{\abs{z_0}} \right)}^{k+1}\\
        &\leq \frac{M}{1-\delta} \cdot \delta^{k+1}\\
        &\overset{k\rightarrow\infty}{\longrightarrow} 0 
        \text{ falls } 0 < \delta < 1.
    \end{align*}
\end{bew}
\begin{bem}
    Ist \(0 < R < \infty \) der Konvergenzradius der Potenzreihe 
    \(P(z) = \sum_{n=0}^{\infty} a_n z^n \), so konvergiert
    \(P(k) = \sum_{l=0}^{k} a_l z^l \) gleichmäßig auf
    \([-R + \delta, R - \delta] \) gegen \( P \) für alle 
    \( 0 < \delta < R \)
    \[ \underset{\abs{z} \leq R-\delta}{\sup} \abs{P(z)-P_k(z)}
    \overset{k \rightarrow \infty}{\longrightarrow} 0 \]
    Ist \( R = \infty \), so konvergiert \( P_k \) auf 
    \( [-L,L] \) gleichmäßig gegen \( P \) für jedes feste
    \( L > 0 \) (lokal gleichmäßige Konvergenz)
\end{bem}
\begin{bsp}
    Für alle \( 0 < L < \infty \) konvergiert 
    \[ \sum_{l=0}^k \frac{z^l}{l!} \]
    gleichmäßig gegen
    \[ \sum_{l=0}^\infty \frac{z^l}{l!} = \exp z \]
    auf \( \set{ z \in \C \; \vert \; \abs{z} \leq L } \).
\end{bsp}
\begin{satz}[Gleichmäßige Stetigkeit und Stetigkeit]\label{satz:glmStetigUStetig}
    Seien \( f_n : D \rightarrow \C, n\in\N \) eine 
    stetige Funktionenfolge auf \( D \subset \R \) 
    (oder \( D = \C \)), die gleichmäßig gegen 
    \( f: D \rightarrow \C \) konvergiert, d.\ h.\ 
    \[ ||f_n - f||_\infty \rightarrow 0 
    \text{ für } n\rightarrow\infty. \]
    Dann ist \(f\) auch stetig auf \(D\).
\end{satz}
\begin{bew}
    \( \frac{\varepsilon}{3} \)-Argument. \\
    Sei \( x_0 \in D \) fest, aber beliebig gewählt. \\
    Dann folgt \( \forall \, x \in D, n \in \N \) gilt:
    \begin{align*}
        \abs{ f(x) - f(x_0) } 
        &= \abs{ f(x) - f_n(x) + f_n(x) - f_n(x_0) 
        + f_n(x_0) - f(x_0) } \\
        (*) &\leq \abs{ f(x) - f_n(x) } + \abs{f_n(x) - f_n(x_0)} 
        + \abs{f_n(x_0) - f(x_0)}.
    \end{align*}
    Sei \( \varepsilon > 0 \).
    Da \( ||f-f_n||_\infty \rightarrow 0 \) für 
    \(n \rightarrow \infty \) gibt es ein \( n \in \N \)
    \[ ||f-f_n||_\infty = \underset{x \in D}{\sup} 
    \abs{f(x)-f_n(x)} < \frac{\varepsilon}{3} 
    \; \forall \, n \geq N \]        
    Da \( f_N \) stetig ist, existiert somit \( \delta > 0 \): 
    \[ \overset{(*)}{\Rightarrow} \forall \, x\in D, 
    \abs{x-x_0} < \delta. \]
    \[ \abs{ f(x) - f(x_0) } < \frac{\varepsilon}{3} 
    + \frac{\varepsilon}{3} + \frac{\varepsilon}{3}  = \varepsilon \]
    also ist \( f \) stetig in \( x_0 \). \\            
\end{bew}
\begin{bew} [Alternativer Beweis]
    Aus \( (*) \) folgt 
    \begin{align*}
        \abs{f(x) - f_n(x_0)} &\leq \underset{y \in D}{\sup}
        \abs{f(y)-f_n(y)} + \abs{f_n(x) - f_n(x_0)}
        + \underset{y \in D}{\sup} \abs{f_n(y) - f(y)} \\
        &= 2 ||f-f_n||_\infty + \abs{f_n(x) -f_n(x_0)} 
        \; \forall \, n \in \N 
    \end{align*}
    Da für festes \(n \in \N \) die Funktion \( f_n \) stetig
    in \( x_0 \) ist
    \[ \limsup\limits_{x\rightarrow x_0} \abs{f(x)-f(x_0)}
    \leq 2 ||f-f_n||_\infty \rightarrow 0 \; (n \rightarrow \infty) \]    
    \[ \Rightarrow \limsup\limits_{x\rightarrow x_0} 
    \abs{ f(x) - f(x_0)} = 0 \Leftrightarrow 
    f \text{ stetig in } x_0. \]
\end{bew}
\begin{satz}[Stetigkeit von Potenzreihen]
    Sei \( P(z) = \sum_{n=0}^\infty a_n z^n \) eine komplexe 
    Potenzreihe mit \( R > 0 \). Dann ist \( P \) stetig auf 
    der Kreisscheibe 
    \( B_R(0) = \set{ z\in\C \; \vert \; \abs{z} < R } \).
\end{satz}
\begin{bew}
    Die Polynome \( P_k(z) = \sum_{l=0}^k a_l z^l = a_0 + a_1 z 
    + \cdots + a_k z^k \) sind stetig auf \( \C \). \\
    Nach \autoref{satz:glmKonvPotenz} konvergiert \( P_k \) 
    gegen \(P\) gleichmäßig auf jeder Kreisscheibe 
    \( \set{ z \in \C \; \vert \; \abs{z} \leq \rho } \) mit 
    \( 0 < \rho < R \).
    \[ \oversett{\autoref{satz:glmStetigUStetig}}{\Rightarrow} 
    \set{z \in \C \; \vert \; \abs{z \leq \rho}}
    \ni z \mapsto P(z) \text{ ist stetig.} \]
    Da für \( z_0 \in \C, \abs{z_0} < R \) ein 
    \( 0 < \rho < R \) existiert, sodass aus 
    \( z_0 \leq \rho \) folgt, dass \( P \) stetig 
    in jedem \( z_0 \in \C \) mit \( \abs{z_0} < R \) ist.
\end{bew}
\begin{bem}
    Beim Vertauschen von Ableitungen und Grenzwerten von 
    Funktionenfolgen muss man aufpassen! \\
    Auch wenn die Funktionenfolge gleichmäßig konvergiert.
\end{bem}
\begin{bsp}
    \( f_n(x) = \frac{1}{n} \sin (n^2 x) 
    \oversett{glm.}{\rightarrow} 0 (n\rightarrow\infty) \).
    Aber \( f_n'(x) = n^2\cos (n^2 x) \) konvergiert nicht.
\end{bsp}
\begin{satz}[Vertauschen von Konvergenz und Ableitung]
    Seien \( f_n : I \rightarrow \R \) (oder \( \C \)) stetig 
    und differenzierbar auf \( I = (a,b) \subset \R \).
    Es gelte 
    \begin{enumerate}[(a)]
        \item \( \set{f_n(x_0)}_{n\in\N} \) konvergiert für 
        \( x_0 \in I \).
        \item Die Folge der Ableitungen 
        \( \set{ f_n' }_{n \in \N} \) 
        konvergiert gleichmäßig gegen eine Funktion 
        \( g: I \rightarrow \R \) (oder \( \C \)), also 
        \( ||f_n' - g||_\infty 
        \overset{n\rightarrow\infty}{\longrightarrow} 0 \).
        Dann konvergiert \( \set{f_n}_n \) gleichmäßig gegen 
        \( f: I \rightarrow \R \) (oder \( \C \)). \\
        \( f \) ist differenzierbar und \( f' = g \).
    \end{enumerate}
\end{satz}
\begin{bem}
    Konvergenz ist lokal gleichmäßig falls 
    \[ a = -\infty \text{ oder } b = \infty. \]
\end{bem}
\begin{bew}\hfill \\
    1. Schritt: Beh.:\ Sei \( f_n: I \rightarrow \R, x\in I \) 
    beliebig \( \rightarrow f(x) = \limes{n} f_n(x) \) existiert. \\
    Bew.:\ Seien \( n,m \in \N, h(x) := f_n(x) - f_m(x) \).
    \( \overref{satz:mittelwert}{MWS}{\Rightarrow} 
    \exists \, \rho \) zwischen \(x\) und \(x_0\):
    \begin{align*}
        \abs{ h(x) - h(x_0) } 
        &= \abs{ f_n(x) - f_m(x) - f_n(x_0) + f_m(x_0) } \\
        &\leq \abs{ f_n'(\rho) - f_m'(\rho) }\abs{x-x_0}
    \end{align*}
    \begin{align*}
        \abs{ f_n(x_0) - f_m(x_0) }
        &\leq 
        \underbrace{\abs{ f_n(x_0) - f_m(x_0) }}_{\rightarrow 0 \text{ siehe (a)}} 
        + \abs{ f_n'(\rho) 
        - f_m'(\rho) }\abs{x-x_0} \\
        &\rightarrow \abs{ f_n'(\rho) - f_m'(\rho) } \abs{x-x_0} \\
        &\leq \underbrace{||f_n' - f'||_\infty}_{\rightarrow 0 
        \text{ für } n,m \rightarrow \infty \text{ siehe (b)}} 
        \abs{x-x_0}        
    \end{align*}
    \( \Rightarrow \set{f_n(x)}_n \) ist Cauchy für jedes feste 
    \( x\in I \). \\
    \( \Rightarrow \limes{n} f_n(x) = f(x) \) existiert. \\
    2.\ Schritt: Beh.: \( \forall x_0 \in I \) ist \( f \) 
    differenzierbar in \(x_0\) und \( f'(x_0) = g(x_0) \). \\
    Bew.: Sei \(x \neq x_0 \in I\).
    \begin{align*}
        \frac{f(x) - f(x_0)}{x-x_0} - g(x_0)
        &= \frac{ f(x) - f(x_0) }{ x-x_0 } \\
        &- \frac{ f_n(x) - f_n(x_0) }{ x-x_0 } 
        + \frac{ f_n(x) - f_n(x_0) }{ x-x_0 } \\
        &- f_n'(x_0) + f_n'(x_0) - g(x_0)
    \end{align*}
    \begin{align*}
        \Rightarrow &\abs{ \frac{f(x)-f(x_0)}{x-x_0} 
        - g(x_0) } \\
        \leq &\abs{ \frac{f(x)-f(x_0)}{x-x_0} 
        - \frac{f_n(x)-f_n(x_0)}{x-x_0} } \\
        + &\abs{ \frac{f_n(x)-f_n(x_0)}{x-x_0} 
        - f_n'(x_0) } \\
        + &\abs{ f_n'(x_0)
        - g(x_0) } \; (**)
    \end{align*}
    \begin{align*}
        &\overunderref{satz:schranken}{Schranken-}{satz}{\Rightarrow}
        \abs{\frac{f_l(x)-f_l(x_0)}{x-x_0}-\frac{f_n(x)-f_n(x_0)}{x-x_0}} \\
        &= \abs{\frac{(f_l-f_n)(x)-(f_l - f_n)(x_0)}{x-x_0}} \\
        &\leq \underbrace{||(f_l-f_n)'||_\infty}_
        {=\underset{x \in (a,b)}{\sup} \abs{f_l'(x)-f_n'(x)}}.
    \end{align*}
    Da \( ||f_l' - f_n'||_\infty \leq ||f_l' - g||_\infty 
    + ||g - f_n||_\infty \) folgt 
    \[ \abs{ \frac{ f(x) - f(x_0) }{ x-x_0 } 
    - \frac{ f_n(x) - f_n(x_0) }{ x-x_0 } } 
    = \limes{l} \abs{ \frac{ f_l(x) - f_l(x_0) }{ x-x_0 } 
    - \frac{ f_n(x) - f_n(x_0) }{ x-x_0 } } 
    \leq \limes{l} ||f_l' - f_n'||_\infty \leq ||g-f_n'||\infty \]
    \[ \overset{(**)}{\Rightarrow} \abs{ \frac{ f(x) - f(x_0) }{ x-x_0 } 
    - g(x_0) } 
    \leq ||g - f_n' ||_\infty 
    + \abs{ \frac{ f_n(x) - f_n(x_0) }{ x-x_0 } - f_n'(x_0) } 
    + \abs{ f_n'(x_0) - g(x_0) } (***) \]
    \[ \leq 2\cdot ||g - f_n' ||_\infty 
    + \abs{ \frac{ f_n(x) - f_n(x_0) }{ x-x_0 } - f_n'(x_0) }  \]
    Nun bilde Grenzwert \( x \rightarrow x_0 \)
    \[ \overset{(***)}{\Rightarrow} 
    \limsup\limits_{x\rightarrow x_0} 
    \abs{ \frac{ f_n(x) - f_n(x_0) }{ x-x_0 } - g(x_0) } 
    \leq 2 ||g - f_n'||_\infty 
    \overset{n\rightarrow \infty}{\longrightarrow} 0. \]
    \( \Rightarrow f'(x_0) \) existiert und es gilt 
    \( f'(x_0) = g(x_0) \) für alle \( x_0 \in I \).
\end{bew}
\begin{prosa}
    Anwendung auf Potenzreihe:\\
    \( P(x) = \sum_{n=0}^\infty a_n x^n \) mit 
    Konvergenzradius \( R > 0 \).\\
    Informell ableiten ergibt 
    \[ g(x) = \sum_{n=0}^\infty n a_n x^{n-1} 
    = \sum_{n=1}^\infty n a_n x^{n-1} \]
    \[ = \sum_{n=0}^\infty (n+1) a_{n+1} x^n 
    = a_1 + 2 a_2 x + 3a_3 x^2 + \cdots + (n+1) a_{n+1} x^n + \cdots \]
    hat auch Konvergenzradius \(R\).\\
    \( \forall \, 0 < \tilde{R} < R \) konvergiert 
    \[ \sum_{l=0}^k l a_l x^{l-1} 
    = \left( \sum_{l=0}^k a_l x^l \right)' = P_k'(x) \]
    gleichmäßig auf \( (-\tilde{R}, \tilde{R}) \) gegen \(g\) (Lemma 4)\\ %href
    \[ \oversett{Satz 7}{\Rightarrow} P(x) 
    = \limes{k} P_k(x) = \limes{k} \left( 
    \sum_{l=0}^k a_l x^l \right) \]%href 
    ist differenzierbar auf \( (-\tilde{R},\tilde{R}) \) und 
    \[ P'(x) = \limes{k} P_k'(x) 
    = \sum_{l=1}^\infty l a_l x^{l-1} (*) \]
    Da \( \tilde{R} \) beliebig nahe bei \(R\) sein kann, gilt:\\
    Jede Potenzreihe \( P(x) = \sum_{n=0}^\infty a_n x^n \) mit 
    Konvergenzradius \( R > 0 \) ist auf \( (-R, R) \) differenzierbar 
    und ihre Ableitung ist gegeben durch 
    \[ P'(x) = \sum_{n=0}^\infty n a_n x^{n-1} \] 
    kann man beliebig oft wiederholen.\\
    Das heißt \(P\) ist beliebig oft differenzierbar auf 
    \( (-R,R) \) und die \(n\)-te Ableitung ist gegeben durch 
    \[ P^{(m)}(x) 
    = \sum_{l=m}^\infty l (l-1)\cdots(l-m) a_l x^{l-m}. \]
\end{prosa}
\end{document}