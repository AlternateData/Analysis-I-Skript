\documentclass[../ana1.tex]{subfiles}
\onlyinsubfile{\sectionNumbering} %Use numbering relative to sections and not subsection

\begin{document}
\setcounter{section}{19}
\section{Funktionenfolgen}

Im Folgenden ist \( D \subset \R \) (man kann aber auch 
für die Stetigkeit \( D \subset R^d, d \in \N \) betrachten). \\
Gegeben eine Folge von Funktionen \( {(f_n)}_n \).
\[ f_n: D \rightarrow \R (\text{oder } f_n: D \rightarrow \C) \]
Welche Art von Konvergenzbegriff ist sinnvoll?
\begin{defi}[Punktweise Konvergenz]
    \( f_n : D \rightarrow \C \) konvergiert punktweise 
    gegen \( f: D \rightarrow \C \), falls 
    \[ \limes{n} f_n(x) = f(x) \; \forall \, x \in D. \]
\end{defi}
\begin{bsp}
    \[ f_n: [0,1] \rightarrow \R f_n(x) = \begin{cases}
        1 - nx, &0 \leq x \leq \frac{1}{n} \\
        0, &\frac{1}{n} < x \leq 1
    \end{cases}. \]
    \( f_n \) konvergiert punktweise gegen 
    \[ f: [0,1] \rightarrow \R, f(x) = 
    \begin{cases}
        1, &x = 0 \\
        0, &0 < x \leq 1
    \end{cases}. \]
    Beachte: Die Grenzfunktion \(f\) ist nicht stetig 
    (in \(0\)).
\end{bsp}
\begin{defi}
    \( f: D \rightarrow \C \) heißt beschränkt, falls 
    \[ \underset{x\in D}{\sup} \abs{f(x)} < \infty \]
    \begin{align*}
        B(D, \C) &:= \set{ f: D \rightarrow \C, 
        f \text{ ist beschränkt} } \\
        &= \text{Vektorraum der beschränkten 
        komplexwertigen Funktionen} \\
        B(D, \R) &:= \set{ f: D \rightarrow \R, 
        f \text{ ist beschränkt} } \\
        &= \text{VR der beschränkten reellen Funktionen}
    \end{align*}
    Für \( f\in B(D,\R) \) (oder \( B(D,\C) \)) setzen wir 
    \[ ||f||_\infty := \underset{x\in D}{\sup} \abs{f(x)} 
    \text{ (Supremumsnorm)} \]
    hat alle Eigenschaften einer Norm
    \begin{enumerate}
        \item \( ||f||_\infty \geq 0 \) und 
        \( ||f||_\infty = 0 \Leftrightarrow f = 0 \) 
        (Nullfunktion)
        \item \( ||\lambda f||_\infty = |\lambda| \cdot 
        ||f||_\infty \) (homogen), \( \lambda \in \R \)
        \item \( ||f+g||_\infty \leq ||f||_\infty 
        + ||g||_\infty \) (Dreiecksungleichung) \\
        für \( f, g \in B(D, \R) \) (bzw. \( B(D,\C) \))
    \end{enumerate}
    Dies gilt, da
    \begin{align*}
        \abs{ f(x) + g(x) } &\leq \abs{f(x)} + \abs{g(x)} \\
        &\leq \underset{y\in D}{\sup} \abs{ f(y) } 
        + \underset{y\in D}{\sup} \abs{g(y)}\\
        &= ||f||_\infty + ||g||_\infty \\
        \Rightarrow ||f + g||_\infty &= \underset{x\in D}{\sup}
        \abs{f(x) + g(x)} \leq ||f||_\infty + ||g||_\infty
    \end{align*}
\end{defi}
\end{document}