\documentclass[../ana1.tex]{subfiles}
\begin{document}
\setcounter{section}{1}

\section{Etwas Logik}
\begin{defi*}
	Eine (mathematische) Aussage ist ein Ausdruck, der entweder wahr oder falsch ist.
\end{defi*}

\begin{bspe}\leavevmode
	\begin{enumerate}[(1)]
		\item \(A\): \gqq{\(1+1=2\).}  (auch \gqq{\(1+1=3\)}, \gqq{\(1+1=0\)})
		\item \(B\): \gqq{Es gibt unendlich viele Primzahlen.}
		\item \(C\): \gqq{Es gibt unendlich viele Primzahlen \(p\), für die \(p + 2\) auch eine Primzahl ist.} (Primzahlzwillingsvermutung)
		\item \(D\): \gqq{Die Gleichung \(m \ddot{x} = F\) hat, gegeben \(\dot{x}(0) = v_0, x(0) = x_0\), immer genau eine Lösung.} (Lösung der Newtonschen Gleichung)
		\item \(E\): \gqq{Jede gerade natürliche Zahl größer als 2 ist die Summe zweier Primzahlen.} (Goldbach Vermutung)
		\item \(F\): \gqq{Morgen ist das Wetter schön.}
		\item \(G\): \gqq{Ein einzelnes Atom im Vakuum mit der Kernladungszahl \(Z\) kann höchstens \(Z+1\) Elektronen binden.} (Ionisierungsvermutung, es ist noch nicht einmal bekannt, ob es eine Zahl \(Z\) gibt, sodass höchstens \(Z+1\) Elektronen gebunden werden.)
		\item \(H(k,m,n)\): \gqq{Es gilt: \(k^2 + m^2 = n^2\).} (z.B. \(H(3,4,5)\) ist wahr.)
	\end{enumerate}
\end{bspe}

\begin{bem}
	Sei  \(A(n)\) eine Aussage für jede natürliche Zahl \(n\in \N \). Dann gilt, dass
	\(A(n)\) für jedes \(n\) wahr ist, genau dann, wenn gilt
	\begin{enumerate}[(i)]
		\item \(A(1)\) ist wahr.
		\item \(A(n)\) wahr \(\implies A(n+1)\) ist wahr.
	\end{enumerate}
\end{bem}

\begin{bsp}
	\(A(n): 1+2+3+\cdots+n = \frac{n(n+1)}{2}\).
	\begin{bew} (Durch vollständige Induktion)
		\induktion{
			\(1 = \frac{1(1+1)}{2}\).
		}{
			\(A(n)\) gelte für ein \(n\in\mathbb{N}\).
		}{
			\(\begin{aligned}[t]
				\underbrace{1+2+\cdots+n}_{=\frac{n(n+1)}{2}\IV} + (n+1) &= \frac{n(n+1)}{2}+(n+1)
				=\frac{n(n+1) + 2(n+1)}{2}\\
				&= \frac{(n+1)(n+2)}{2} = \frac{(n+1)((n+1)+1)}{2}.
			\end{aligned}\)
			\newline\qedhere
		}
	\end{bew}
\end{bsp}
%18.10.2018
\begin{bem}[Gaußsche Summenformel]\leavevmode
	\begin{enumerate}[(a)]
		\item \(\begin{aligned}[t]
				&	         		 & S &= 1 + 2 + 3 + \cdots + n = n + (n-1)+ (n-2) + \cdots + 2 + 1\\
				&\implies 		     &2S &= \underbrace{(n+1)+(n+1)+\cdots+(n+1)}_{n\text{-mal}}\\
				&\Longleftrightarrow & S &= \frac{n(n+1)}{2}.
			  \end{aligned}\)
		\item \( S_{n} \coloneqq 0 + 1 + 2 + \cdots + n = \) entspricht der Anzahl der Punkte in einem 
			  rechtwinkligen Dreieck mit dem Flächeninhalt \(\frac{1}{2}\cdot n \cdot n\).\newline
			  Ansatz (\glqq{}geschicktes Raten\grqq, \glqq{}scientific guess\grqq, englisch: ansatz):
			  \[S_{n} = \underbrace{a_{2}n^{2} + a_{1}n + a_{0}\text{,}}_{\text{Polynom }2\text{. Grades in }n}\quad a_{2} = \frac{1}{2}\]
			  Wie bekommt man die Werte von \(a_0, a_1,\) und \(a_2\)?
			  \[\begin{aligned}[t]
				n=0\colon& S_{0} = 0 = a_{2}0^{2} + a_{1}0 + a_{0}               &\implies& a_{0} = 0.\\ 
				n=1\colon& S_{1} = 1 = a_{2}1^{2} + a_{1}1 = \frac{1}{2} + a_{1} &\implies& a_{1} = \frac{1}{2}.
			  \end{aligned}\]
			  \(\implies S_{2} = \frac{1}{2}n^{2} + \frac{1}{2}n = \frac{n(n+1)}{2}\).
	\end{enumerate}
\end{bem}

\subsection{Grundbegriffe}
\begin{notation}\leavevmode
	\begin{center}
		\begin{tabular}{r|l}
			\(\colon \) & \glqq{}so, dass gilt\grqq{}\\
			\(\exists \) & \glqq{}es gibt mindestens ein\grqq{}, \glqq{}es existiert\grqq{}\\
			\(\forall \) & \glqq{}für alle\grqq{}\\
			\(\Rightarrow \) & \glqq{}impliziert\grqq(\(A \Rightarrow B\) \glqq{}aus \(A\) folgt \(B\)\grqq)\\
			\(\Leftrightarrow \) & \glqq{}genau dann, wenn\grqq{}\\
			\(\neg A\) & nicht \(A\) \\
			\(A \wedge B\) & \(A\) und \(B\) \\
			\(A \vee B\) & \(A\) oder \(B\) \\
			\(A \coloneqq B\) & \(A\) ist per Definition gleich \(B\)
		\end{tabular}
	\end{center}
\end{notation}

\begin{satz}
	Folgende Aussagen sind allein aus logischen Gründen immer wahr.
	\begin{center}
		\begin{tabular}{rl}
			\(\neg(\neg A) \Leftrightarrow A\)                            & Gesetz der doppelten Verneinung \\
			\(A \Rightarrow B \Leftrightarrow \neg B \Rightarrow \neg A\) & Kontraposition                  \\
			\(A \Rightarrow B \Leftrightarrow (\neg (A \wedge \neg B))\)  & beim Widerspruchsbeweis         \\
			\(\neg(A \wedge B) \Leftrightarrow (\neg A \vee \neg B)\)     & de Morgan                       \\
			\(\neg(A \vee B) \Leftrightarrow (\neg A \wedge \neg B)\)     & de Morgan                       \\
		\end{tabular}
	\end{center}
\end{satz}

\begin{bem}\leavevmode
	\begin{enumerate}[(a)]
		\item \(\begin{aligned}[t]
					(A \Rightarrow B) &\Leftrightarrow B \text{ ist mindestens so wahr wie } A\\
									&\Leftrightarrow A \text{ ist mindestens so falsch wie } B\\
									&\Leftrightarrow \neg B \Rightarrow \neg A.
			    \end{aligned}\)
		\item \(\begin{aligned}[t]
			    	(A \Leftrightarrow B) &\Leftrightarrow ((A \Rightarrow B) \wedge (B \Rightarrow A)).
			    \end{aligned}\)
	\end{enumerate}
\end{bem}
\begin{defi*}
	Sein \(n \in \N \), dann definiere:
	\begin{enumerate}[(a)]
		\item \(n\) ist gerade \(\longeq \exists k \in \N \colon n = 2k\).
		\item \(n\) ist ungerade \(\longeq \exists k \in \N_{0} \colon n = 2k + 1\).
	\end{enumerate}
\end{defi*}
\begin{bsp}
	Es gilt \(n\) ist gerade \(\iff n^2\) ist gerade.
	\begin{bew}
		\equirl{
			\(n\) gerade \(\implies n=2k\) für ein \(k \in \N \).\\
			\(\implies n^{2} = {(2k)}^{2} = 4k^{2} = 2(2k^{2})\) ist gerade.
		}{
			(Durch Kontraposition) \(n\) ungerade \(\implies n=2k + 1\) für ein \(k \in \N_{0}\).\\
			\(\implies n^{2} = {(2k+1)}^{2} = 4k^{2} + 4k + 1 = \underbrace{2(2k^{2} + 2k)}_{\text{gerade}} + 1\) ist ungerade.
		}
	\end{bew}
\end{bsp}

\begin{defi*}[Informelle Mengendefinition nach Cantor]\leavevmode \\
	Eine Menge ist eine Sammlung von Objekten (Elemente) zu einem neuen Objekt.
\end{defi*}

\begin{notation}\leavevmode
	\begin{enumerate}[(a)]
		\item \(a\) ist ein Element von \(M \longeq a \in M\).\\
			  \(a\) ist kein Element von \(M \longeq a \notin M\).
		\item Beschreibung durch Auflisten: \(M = \set{x_{1} \ko x_{2} \ko x_{3} \ko \ldots \ko x_{17}}\).\\
			  Beschreibung durch Eigenschaften: \(M = \set{a \; \vert \; a \text{ hat Eigenschaft } E}\).
	\end{enumerate}
\end{notation}

\begin{bspe}\leavevmode
	\begin{enumerate}[(1)]
		\item \(\N \coloneqq \set{1 \ko 2 \ko 3 \ko \ldots}\)
		\item \(-\N \coloneqq \set{-n \; \vert \; n \in \N}\)
		\item \(\Z \coloneqq \set{x \; \vert \; x \in \N \; \vee \; x \in -\N \; \vee \; x = 0 }\)
	\end{enumerate}
\end{bspe}

\begin{defi}
	Sei \(M\) eine Menge und \(A(x)\) Aussagen mit \(x\in M\).
	\begin{enumerate}[(a)]
		\item \(\forall x \in M \colon A(x)\) ist wahr, falls alle \(A(x)\) wahr sind.
		\item \(\exists x \in M \colon A(x)\) ist wahr, falls mindestens eine Aussage \(A(x)\) wahr ist.
	\end{enumerate}
	Achtung: Reihenfolge der Quantoren ist wichtig!
\end{defi}

\begin{bsp}
	Töpfe \(\coloneqq \) Menge der Töpfe, Deckel \(\coloneqq \) Menge der Deckel.\\
	\(\forall T \in \text{Töpfe} \quad \exists D \in \text{Deckel} \colon D \text{ passt auf } T\) \\
	\(\iff \) Für jeden Topf gibt es mindestens einen Deckel, der passt.
	\(\exists D \in \text{Deckel} \quad \forall T \in \text{Töpfe} \colon  D \text{ passt auf } T\) \\
	\(\iff \) Es existiert mindestens ein Deckel, der auf alle Töpfe passt.
\end{bsp}

\begin{bem}[Negation von quantifizierten Aussagen]
	\begin{flalign*}
		\neg (\forall x \in M \colon A(x)) &\Longleftrightarrow \exists x \in M \colon \neg A(x).&&\\
		\neg (\exists x \in M \colon A(x)) &\Longleftrightarrow \forall x \in M \colon \neg A(X).&&
	\end{flalign*}
\end{bem}

\begin{defi}[Mengenoperationen]
	Seien \(M,N,I\) Mengen und für \(i \in I\) sei \(M_{i}\) eine Menge.
	\begin{enumerate}[(a)]
		\item \(\emptyset \) ist die Menge ohne Elemente (leere Menge).
		\item \(M \subseteq N \longeq \forall x \in M \colon x \in N\). M ist Teilmenge von N.
		\item \(M \subsetneq N \longeq M \subseteq N \vee M \neq N\). M ist echte Teilmenge von N. 
		\item \(M \cap N \coloneqq \set{x \; \vert \; x \in M \wedge x \in N}\) heißt Schnitt von \(M\) und \(N\).
		\item \(M \cup N \coloneqq \set{x \; \vert \; x \in M \vee x \in N}\) heißt Vereinigung von \(M\) und \(N\).
		\item \(M \setminus N \coloneqq \set{x \; \vert \; x \in M \wedge x \notin N}\) heißt Differenz von \(M\) und \(N\).
		\item \(\PO(M) \coloneqq \set{A \; \vert \; A \subset M}\) heißt Potenzmenge von \(M\).
		\item \(\bigcap\limits_{i \in I} M_{i} \coloneqq \set{x \; \vert \; \forall i\in I \colon x \in M_{i}}\).
		\item \(\bigcup\limits_{i \in I} M_{i} \coloneqq \set{x \; \vert \; \exists i\in I \colon x \in M_{i}}\).
	\end{enumerate}
	Ist \(M \cap N = \emptyset \) so heißen \(M\) und \(N\) disjunkt.
\end{defi}

\begin{bsp}\leavevmode
	\begin{enumerate}[(1)]
		\item Es gilt immer \(\emptyset \subseteq M\), für jede Menge \(M\).
		\item \(M = \set{1\ko2} \implies \PO(M) = \set{\emptyset\ko\set{1}\ko\set{2}\ko\set{1\ko2}}\)
	\end{enumerate}
\end{bsp}

\begin{bem}[Eigenschaften von \(\subseteq \)] Seien \(M\ko N\ko A\ko B\ko C\) Mengen. Dann gilt:
	\begin{enumerate}[(a)]
		\item \(\emptyset \subseteq M\).
		\item \(M \subseteq M\).
		\item \(M = N \iff M \subseteq N \vee N \subseteq M\).
		\item \(A\subset B \wedge B \subset C \iff A \subset C\).
		\item \(\begin{rcases*}
				(A\cup B) \cup C = A \cup (B \cup C)\; \\
				(A\cap B) \cap C = A \cap (B \cap C)\;
			  \end{rcases*}\) Assoziativität.
		\item \(\begin{rcases*}
			    A\cup B = B \cup A\; \\
		        A\cap B = B \cap A\;
			  \end{rcases*}\) Kommutativität.
		\item \(\begin{rcases*}
			    A \cap (B\cup C) = (A\cap B) \cup (A\cap C)\; \\
			    A \cup (B\cap C) = (A\cup B) \cap (A\cup C)\;
			  \end{rcases*}\) Distributivgesetze.
	\end{enumerate}
\end{bem}

\end{document}