\documentclass{article}

%\usepackage{german} %deutsches Format
\usepackage[ngerman]{babel}
\usepackage[utf8]{inputenc} %Umlaute
\usepackage{graphicx} %Grafiken einbinden
\usepackage{amsmath,amssymb,amsthm}

\newenvironment{mathalign}{\end{aligned}\end{math}}{\begin{math}\begin{aligned}}
\newenvironment{alignenum}[2]{%
    \providecommand{\FormatLHS}{}%
    \renewcommand*{\FormatLHS}[1]{\makebox[\widthof{\ensuremath{#1}}][r]{\ensuremath{##1}}}%
    %\let\olditem\item%
    %\renewcommand{\item}{o}{%
    %    \end{math} \olditem \begin{math}
    %}
    \begin{enumerate}[#2]\end{mathalign}
}{ 
    \begin{mathalign}
    \end{enumerate}%
    %\renewcommand{\item}{o}{\olditem}%
}
\newcommand{\aitem}{
    \begin{mathalign} \item \end{mathalign}
}

\theoremstyle{definition}
\newtheorem{satz}{Satz}[subsection]

\begin{document}

\begin{satz}\leavevmode
	\begin{alignenum}{(A \Leftrightarrow B)}{(a)}
		\aitem \FormatLHS{(A \Rightarrow B)} &\Leftrightarrow B \text{ ist mindestens so wahr wie } A\\
										  &\Leftrightarrow A \text{ ist mindestens so falsch wie } B\\
										  &\Leftrightarrow \neg B \Rightarrow \neg A.
		\aitem \FormatLHS{(A \Leftrightarrow B)} &\Leftrightarrow ((A \Rightarrow B) \wedge (B \Rightarrow A)).
	\end{alignenum}
\end{satz}

\end{document}
