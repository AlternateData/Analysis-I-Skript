\documentclass[../ana1.tex]{subfiles}
\onlyinsubfile{\sectionNumbering} %Use numbering relative to sections and not subsection

\begin{document}
\setcounter{section}{13}
\section{Potenzreihen}

\begin{prosa}
    Eine Potenzreihe ist eine Reihe der Form
    \[ \sum_{n=0}^\infty a_n z^n, a_n \in\C, z \in\C \] 
    %unter a_n pfeil mit Koeffizienten und unter z Pfeil mit Variable
    Partialsummen \( S_n(z) = \sum_{j=0}^n a_j z^j \) (Polynome).\\
    Frage nach der Konvergenz?\\
    Definiere \( R:= \sup \{ \abs{z} \;\vert \; z\in\C, \sum_{n=0}^\infty a_n z^n \text{ konvergent} \} \)
    (Konvergenzradius). (\( R = 0, \infty \) sind erlaubt.)
\end{prosa}
\begin{lem}
    Die Potenzreihe \( \sum_{n=0}^\infty a_n z^n \) konvergiert absolut für jedes \( z\in\C \) mit \( \abs{z} < R \) (d.\ h.\  \( z\in B_R(0) \)) und divergiert für jedes \( \abs{z} > R \).
\end{lem}
\begin{bew}
    Sei \( z_1 \neq 0, \sum_{n=0}^\infty a_n z_1^n \) konvergent.
    \( \Rightarrow a_n z_1^n \) ist eine Nullfolge.
    \( \Rightarrow \, \exists \, K \geq 0 \; \colon \; \abs{a_n z_1^n} \leq K \quad \forall \, n \) \\
    Nehme \( 0<r < \abs{z_1}, \abs{z} \leq r \)
    \[ \Rightarrow \abs{a_n z^n} = \abs{a_n} \abs{z}^n = \abs{a_n} { \left( \frac{\abs{z}}{\abs{z_1}} \right) }^n \abs{z_1}^n \]
    \[ \leq \abs{a_n}\abs{z_1}^n {\left( \underbrace{\frac{r}{\abs{z_1}}}_{=: \theta} \right)}^n = \abs{a_n z_1^n} \cdot \theta^n \leq K \theta^n, \theta = \frac{r}{\abs{z_1}} < 1 \]
    \( \Rightarrow \) konvergente Majorante \( K \theta^n \) \\
    \[ \abs{ \sum_{n=0}^\infty a_n z^n } \leq \sum_{n=0}^\infty \abs{ a_n z^n } \leq K \sum_{n=0}^\infty \theta^n = \frac{K}{1 - \theta} \leq \infty \]
    \[ \Rightarrow \sum_{n=0}^\infty a_n z^n \text{ konvergiert absolut für alle } \abs{z} < \abs{z_1}. \]
    Angenommen \( \exists z\in\C \) mit \( \abs{z} > R \) und \( \sum_{n=0}^\infty a_n z^n \) konvergiert.
    \( \Rightarrow \) Konvergenzradius \(\geq \abs{z} > R \) \Lightning{} Def.\ von \(R\).
\end{bew}
\begin{bsp}
    Definiere
    \[ \underbrace{\exp z}_{=e^z} := \sum_{n=0}^\infty \frac{z^n}{n!} =: \sum_{n=0}^\infty b_n \]
    Quotientenkriterium:
    \[ \frac{\abs{b_{n+1}}}{\abs{b_n}} = \frac{ \abs{ \frac{ z^{n+1} }{ (n+1)! } } }{ \abs{ \frac{ z^n }{ n! } } } = \abs{ \frac{ z^{n+1} }{ z^n } } \cdot \frac{ n! }{ (n+1)! } = \frac{\abs{z}}{n+1} \rightarrow 0, n\rightarrow\infty. \]
    \[ \Rightarrow \sum_{n=0}^\infty \frac{z^n}{n!} \text{ konvergiert absolut } \forall z\in\C. \]
\end{bsp}
\begin{satz}[Additionstheorem]
    \[ \forall \, z,w\in\C : \exp(z) \exp(w) = \exp(z+w) \]
    Insbesondere ist \( \exp(z) \neq 0 \;\forall \, z\in\C \) und \( \exp(x)>0 \,\forall \, x\in\R \).
\end{satz}
\begin{bew}
    \[ z\in\C, a_n = \frac{ z^n }{ n! }, b_n = \frac{ w^n }{ n! } \]
    \[ \Rightarrow \exp(z)\exp(w) = \left( \sum_{k=0}^\infty a_k \right) \left( \sum_{l=0}^\infty b_l \right) \overset{\text{Cauchyprod.}}{=} \sum_{n=0}^\infty c_n \]
    \[ c_n = \sum_{k+l=n} a_k b_l = \sum_{k+l=n} \frac{z^k}{k!} \frac{w^l}{l!} = \sum_{k=0}^n \frac{ z^k w^{n-k} }{ k!(n-k)! } \]
    \[ = \frac{1}{n!} \sum_{k=0}^n \frac{ n! }{ k!(n-k)! } z^k w^{n-k} = \frac{1}{n!} \underbrace{ \sum_{k=0}^n \binom{n}{k} z^k w^{n-k} }_{\text{binom. Formel } {(z+w)}^n} = \sum_{n=0}^\infty \frac{1}{n!} {(z+w)}^n \]
    \[ = \exp(z+w) \]
    Insbesondere ist \( \exp(z) \exp(-z) = \exp(z-z) = \exp(0) = 1 \) \\
    \( \Rightarrow \exp(z) \neq 0 \;\forall \, z\in\C \) und \( \inverse{(\exp(z))} = \exp(-z) \).
\end{bew}
\subsection*{Sinus, Cosinus und Eulersche Formel}
Man ersetzt \( z \) durch \( iz \) in \( e^z \).
\[ e^{iz} = \sum_{n=0}^{\infty} \frac{{(iz)}^n}{n!} = \sum_{n=0}^{\infty} i^n \frac{z^n}{n!} \]
Beachte \( \N_0 \ni n \mapsto i^n \) hat Periode \( 4 \).
\[ n = 2k, k\in\N_0 \Rightarrow i^n = i^{2k} = {(i^2)}^k = {(-1)}^k \]
\[ n = 2k+1, k\in\N_0 \Rightarrow i^n = i^{2k+1} = i^{(2k)} \cdot i = {(-1)}^k \cdot i \]
\[ \Rightarrow e^{iz} = \limes{L} \sum_{n=0}^L i^n \frac{z^n}{n!} = \limes{L} \left( \sum_{\substack{0\leq n \leq L\\ n\text{ gerade}}} i^n \frac{z^n}{n!} + \sum_{\substack{0 \leq n \leq L\\ n\text{ ungerade}}} i^n \frac{z^n}{n!} \right) \]
\[ = \limes{L} \left( \sum_{\substack{0\leq 2k \leq L\\ k\in\N_0}} {(-1)}^k \frac{z^{2k}}{(2k)!} + i \cdot \sum_{\substack{0\leq 2k+1 \leq L\\ k\in\N_0 }} {(-1)}^k \frac{z^{2k+1}}{(2k+1)!} \right) (*) \]
und: Umordnung ist okay wegen absoluter Konvergenz und Umordnungssatz.\\
(Nachrechnen mit Quotientenkriterium, dass beide Reihen in \( (*) \) für alle \( z\in\C \) konvergieren. (HA))
\begin{defi}
    Wir setzen für \( z\in\C \)
    \[ \cos z := \sum_{n=0}^\infty {(-1)}^n \frac{z^{2n}}{2n!} \qquad \sin z  := \sum_{n=0}^\infty {(-1)}^n \frac{z^{2n+1}}{(2n+1)!}. \]
    Dann gilt
    \[ e^{iz} = \cos z + i \sin z \quad \forall z\in\C. \]
\end{defi}
\begin{prosa}
    Beachte: wenn \( z=t\in\R \), so sind \( \cos t,\, \sin t \in\R \) und es gilt
    \[ e^{it} = \cos t + i \sin t \quad \forall t\in\R \text{ (Eulersche Formel)} \]
    Die Eulersche Formel gibt eine Darstellung der komplexen Zahl
    \[ e^{it} = \exp(it) \]
    in Real- und Imaginärteil
    \begin{align*}
        \Re (e^{it}) = \cos t = \frac{1}{2} (e^{it} + \overline{e^{it}})\\
        \Im (e^{it}) = \sin t = \frac{1}{2i} (e^{it} - \overline{e^{it}})
    \end{align*}
    \[ \text{Sei } \exp_n (z) = \sum_{n=0}^n \frac{z^k}{k!} = 1 + z + \frac{z^2}{2} + \cdots + \frac{z^n}{n!} \]
    \[ \Rightarrow \overline{\exp_n(z)} = 1 + \overline{z} + \frac{ \overline{z}^2 }{2} + \cdots + \frac{ \overline{z}^n }{n!} = \exp_n (\overline{z}) \rightarrow \exp(\overline{z}) \]
    \[ \Rightarrow \overline{\exp(z)} = \exp(\overline{z}) \quad \forall \, z\in\C. \]
    Somit folgt: \( \overline{e^{it}} = \overline{\exp(it)} = \exp(\overline{it}) = \exp(\minus it) = e^{\minus i t} \quad \forall \, t\in\R \).\\
    Daher 
    \begin{align*}
        \cos t = \Re(e^{it}) = \frac{1}{2} \left( e^{it} + e^{\minus it} \right) \\
        \sin t = \Im(e^{it}) = \frac{1}{2i} \left( e^{it} - e^{\minus it} \right)
    \end{align*}
\end{prosa}
\begin{satz}
    \[ \forall \, t\in\R \text{ ist } \abs{\exp(it)} = 1. \]
\end{satz}
\begin{bew}
    \begin{align*}
        \abs{\exp(it)}^2 = \overline{\exp(it)} \cdot \exp(it)\\
        = e^{\minus i t} \cdot e^{it}\\
        = e^{\minus it + it} = e^0 = 1.
    \end{align*}
\end{bew}
\begin{kor}
    \( \forall \, t\in\R \) ist 
    \[ \cos^2 (t) + \sin^2 (t) = 1. \]
\end{kor}
\begin{bew}
    HA
\end{bew}
\begin{lem}
    Für den Konvergenzradius \( R \) von \( \sum_{n=0}^\infty a_n z^n \) gillt
    \[ R = \frac{1}{\limessup{n} \sqrt[n]{\abs{a_n}}} \]
    wobei \(\frac{1}{0} := \infty \) und \( \frac{1}{\infty} := 0 \) gesetzt wird.
\end{lem}
\begin{bew}
    Aus dem Wurzelkriterium folgt, dass \( \sum_{n=0}^\infty a_n z^n \) konvergiert, falls
    \[ \underbrace{ \limessup{n} \abs{a_n z^n}^{1/n} }_{ =\limessup{n} \abs{a_n}^{1/n} \abs{z} \ (*) } < 1 \]
    \[ r_0 := \frac{1}{ \limessup{n} \sqrt[n]{\abs{a_n}} } \]
    Schritt 1: 
    \begin{enumerate}[(a)]
        \item \( r_0 = \infty (\Leftrightarrow \limessup{n} \sqrt[n]{ \abs{a_n} } = 0) \) \\
        Dann konvergiert (wegen \( (*) \)) \( \sum_{n=0}^\infty a_n z^n \quad \forall \, z\in\C \)
        \[ \Rightarrow R = \infty. \]
        \item Ist \( 0 < r_0 < \infty \), so konvergiert \( \sum_{n=0}^\infty a_n z^n \) für \( \abs{z} \leq r_0 \ (\leadsto (*) ) \) \\
        Also ist \( R \geq r_0 \) falls \( 0 < r_0 < \infty \).
    \end{enumerate}
    Schritt 2: \( 0 \leq r_0 < \infty \) und \( \abs{z} > r_0 \)
    \[ q := \limessup{n} \sqrt[n]{\abs{a_n}} \abs{z} > \limessup{n} \sqrt[n]{\abs{a_n}} r_0 = 1. \]
    Also gibt es für jede \( \varepsilon > 0 \) \( \infty \)-viele \( n\in\N \,\colon \, \sqrt[n]{\abs{a_n}} \abs{z} \geq q - \varepsilon \).\\
    Wähle \(\varepsilon \) so klein, dass \( q\geq 1 \):\\
    \[ \Rightarrow \, \exists \, \infty \text{-viele } n\in\N_0 \,\colon \, \sqrt[n]{\abs{a_n}} \abs{z} \geq 1 \Leftrightarrow \abs{a_n} \abs{z}^n \geq 1. \]
    \[ \Rightarrow \text{für jede } \abs{z} > r_0 \text{ ist } a_n z^n \text{ keine Nullfolge} \Rightarrow \sum_{n=0}^\infty a_n z^n \text{ konvergiert nicht.} \]
    \[ \Rightarrow \text{nach Def.\ von } R \text{ ist } R \leq r_0 \text{, falls } 0\leq r_0 < \infty. \]
    Schritt 3: Aus Schritt 1 und Schritt 2 folgt \( R = r_0 \), falls \( 0 < r_0 < \infty \).\\
    Ist \( r_0 = 0 \), so folgt \(0 \leq R \leq r_0 = 0 \Rightarrow R = 0 \) \\
    Ist \( r_0 = \infty \), so folgt \(0 \leq R \geq r_0 = \infty \Rightarrow R = \infty \) \\
\end{bew}
\begin{kor}
    Die Potenzreihen \( \sum_{n=0}^\infty a_n z^n \) und \( \sum_{n=1}^\infty n \cdot a_n z^{n-1} \) haben denselben Konvergenzradius.
\end{kor}
\begin{bew}
    Da \( n^{\nicefrac{1}{n}} \rightarrow 1 \) für \( n\rightarrow \infty \)
    \[ \limessup{n} \sqrt[n]{\abs{a_n}} = \limessup{n} \sqrt[n]{n \cdot \abs{a_{n-1}}}. \]
\end{bew}
\begin{lem}
    Konvergiert \( \sum_{n=0}^\infty a_n z^n \) für ein \( z = z_1 \neq 0 \), 
    so ist sie auf der Kreisscheibe \( \overline{B_r (0)}  = \{ z\in\C \;\vert \; \abs{z}\leq r \} \)
    beschränkt für jedes \( 0 < r < \abs{z_1} \).\\
    D.\ h.\  \( \exists \, M(r) \geq 0 \), sodass
    \[ \abs{ \sum_{n=0}^\infty a_n z^n } \leq M(r) \quad \forall \, \abs{z} < r. \]
\end{lem}
\begin{bew}
    Ist \( \abs{z} \leq r < \abs{z_1} \), so ist
    \[ \abs{a_n z^n} = \abs{a_n} \abs{z^n} = \abs{a_n} \underbrace{ {\left( \frac{\abs{z}}{\abs{z_1}} \right)}^n }_{\leq \frac{r}{\abs{z_1}} = \theta < 1} \abs{z_1}^n \]
    Da \( \sum_{n=0}^\infty a_n z^n \) konvergent ist, ist \( a_n z^n \) eine Nullfolge und somit insbesondere beschränkt.
    \[\text{D.\ h.\ } \exists \, 0 \leq K < \infty \,\colon \, \abs{a_n z^n} \leq K \]
    \[ \Rightarrow \abs{a_n z^n} \leq K \theta^n \quad \forall \, \abs{z}\leq r, \quad \theta := \frac{r}{\abs{z_1}} < 1 \]
    \[ \Rightarrow \abs{ \sum_{n=0}^\infty a_n z^n } \leq \sum_{n=0}^\infty \abs{a_n z^n} \leq K \cdot \sum_{n=0}^\infty \theta^n = K \frac{1}{1 - \theta} \]
    \[ = \frac{K}{1 - \frac{r}{\abs{z_1}}} = \frac{ K \abs{z_1} }{\abs{z_1} - r} =: M(r). \]
    Dann gilt also
    \[ \abs{ \sum_{n=0}^\infty a_n z^n } \leq M(r) \quad \forall \, \abs{z} \leq r. \]
\end{bew}
\begin{kor}
    Ang.\  \( \sum_{n=0}^\infty a_n z^n \) konvergiert für ein \( z = z_1 \neq 0 \). Dann gibt es zu jedem
    \[ 0 < r < \abs{z_1} \text{ und } k\in\N_0\]
    ein \( M(r,k) > 0 \) mit
    \[ \abs{ \sum_{n=1}^\infty a_n z^n } \leq M(r,k) \abs{z}^{k+1} \quad \forall \, \abs{z}\leq r \]
\end{kor}
\begin{bew}
    Da \( \sum_{n=0}^\infty a_n z_1^n \) konvergiert, konvergiert somit
    \[ z_1^{\minus(k+1)} \sum_{n=0}^\infty a_n z_1^n = \sum_{n=k+1}^\infty a_n z_1^{n-(k+1)} = \sum_{n=0}^\infty a_{n+k+1} z_1^n \]
    \[ \overset{\text{Lem.\ 1}}{\Rightarrow} \text{der Konvergenzradius } R \text{ von } \sum_{n=0}^\infty a_{n+k+1} z^n \geq \abs{z_1} \]
    \[ \overset{\text{Lem.\ 8}}{\Rightarrow} \text{für jede } 0 < r < \abs{z_1} \text{ gibt es ein } M = M(k,r) \text{, sodass} \]
    \[ \abs{ \sum_{n=0}^\infty a_n z^{n-(k+1)} } = \abs{ \sum_{n=0}^\infty a_{n+k+1} \abs{z}^n } \leq M(k,r) \quad \forall \, \abs{z} \leq r \]
    \[ \Rightarrow \abs{ \sum_{n=k+1}^\infty a_n z^n } = \abs{z}^{k+1} \abs{ \sum_{n=k+1}^\infty a_n z^{n-(k+1)} } \leq \abs{z}^{k+1} M(k,r) \quad \forall \, \abs{z} < r. \]
\end{bew}
\begin{satz}
    Es gebe eine Folge \( {(z_k)}_j, z_j \in\C, z_j \neq 0 \) mit \( z_j \rightarrow 0 \), sodass \( \sum_{n=0}^\infty a_n z_j^n = 0 \quad \,\forall \, j\in\N \).
    \[ \Rightarrow a_n = 0 \quad \forall n\in\N_0. \]
\end{satz}
\begin{bew}
    Angenommen, die Behauptung ist falsch. \\
    \( \Rightarrow \exists \) kleinstes \( k\in\N_0 \) mit \(a_k \neq 0 \) und \( a_0 = a_1 = \cdots = a_{k-1} = 0 \)
    \[ 0 = \sum_{n=0}^\infty a_n z_j^n = a_k z_j^k + \sum_{n=k+1}^\infty a_n z_j^n. \]
    Sei \( 0 < r < \abs{z_1} \). Dann folgt aus Korollar 9 für \( j\in\N \) mit \( \abs{z_j} \leq r \)
    \[ \abs{a_k} \abs{z_j}^k = \abs{ \sum_{n=k+1}^\infty a_n z_j^n } \overset{\text{Kor.\ 9}}{\leq} M(k,r) \abs{z_j}^{k+1}. \]
    Da \( z_j\rightarrow 0 \) folgt für fast alle \( j \)
    \( \abs{ a_k } \leq M(k,r) \abs{z_j} \rightarrow 0 \) \Lightning{} \(a_k \neq 0\)
    \[ \Rightarrow \text{alle } a_n = 0. \]
\end{bew}
\end{document}