\documentclass[../ana1.tex]{subfiles}
\onlyinsubfile{\sectionNumbering} %Use numbering relative to sections and not subsection

\begin{document}
\setcounter{section}{6}

\section{Folgen und Konvergenz}

\begin{defi}
	Sei \( X \neq \emptyset \) eine Menge. Eine Folge in \(X \) oder auch \(X \)-wertige Folge ist eine Funktion 
	\[ \abb{f}{\N}{X} \ko \quad n \mapsto f(n) \in X. \]
	Wir setzen \(a_n \coloneqq f(n) \ko \, n \in \N \).\\
	\(a_n \) heißt \(n \)-tes Folgenglied. Wir schreiben auch \({(a_n)}_{n \in \N}\) oder kurz \({(a_n)}_n \).\\
	Ist \(X = \R \), so heißt die Folge reellwertig oder reelle Folge. \\
	Dabei gilt \({(a_n)}_{n\in\N} \subset \R \).
\end{defi}

\begin{defi}[Konvergenz reeller Folgen] \leavevmode \\
	Eine reelle Folge \({(a_n)}_{n \in \N} \) konvergiert gegen ein \(a \in \R \), falls 
	\[\forall \, \varepsilon > 0 \, \exists \, k \in \N \colon \; \forall \, n \geq k \text{ folgt } \abs{a_n - a} < \varepsilon. \]
	Die Zahl \(a\) heißt Grenzwert der Folge, wir schreiben \(\limes{n}a_n = a\) oder \(a_n\rightarrow a\) (für \(n\rightarrow\infty \)).\\
	Eine (reelle) Folge heißt konvergent, falls ein \(a\in\R \) der Grenzwert der Folge ist, andernfalls heißt die Folge divergent.
\end{defi}

\begin{bem}
	\[\limes{n}a_n = a \Leftrightarrow \forall \, \varepsilon > 0 \exists \, k\in\N: \forall \, n\geq k \text{ folgt } |a_n -a|< \varepsilon.\]
\end{bem}

\begin{bsp}
	Beweis mit \(\varepsilon \)-Methode:
	\begin{enumerate}
		\item \(a_n \coloneqq \frac{1}{n}\) konvergiert gegen \(0\). Denn zu geg.\  \(\varepsilon > 0\) wähle \(k\in\N \) mit \(k>\frac{1}{\varepsilon}\). Dann gilt für \(n\geq k\)
		\[ \abs{a_n-a} = \abs{\frac{1}{n} - 0} = \frac{1}{n} \leq \frac{1}{k} < \varepsilon.\]
		\item Konstante Folge. Sei \(a\in\R \) und sei \(a_n = a\) für \(n\in\N \).\\
		Dann folgt \(\limes{n} a_n = a\), denn für \(\varepsilon > 0\)
		\[ \abs{a_n - a} = |a-a|=0<\varepsilon \text{, wähle } k=1\]
		\item Sei \(a_n \coloneqq {(-1)}^n\), also \(a_1 = -1, a_2 = 1, a_3 = -1, \ldots \) \\
		Dann ist \({(a_n)}_{n\in\N}\) nicht konvergent.
		\begin{bew}
			Angenommen: \({(a_n)}_n\) konvergiert und \(a\in\R \) ist Grenzwert. Zu \(\varepsilon = 1\) existiert dann \(k\in\N \) so, dass \(|a_n - a| < \varepsilon = 1 \quad \forall n\geq k\) \\
			Also gilt für \(n\geq k\):
			\[2 = |a_n - a_{n+1}| = |a_n - a + a - a_{n+1}| \leq |a_n - a| + |a-a_{n+1}| < 1 + 1 = 2 \text{ \Lightning}\]
		\end{bew}
		\item Die Folge \((a_n)\) konvergiert gegen \(a\). Dann konvergiert auch \({(|a_n|)}_n\) gegeen \(|a|\). (Hinweis: Umgekehrte Dreiecksungleichung)
		\item Geometrische Folge:\\
		Sei \(q\in\R, |q|<1\). Dann gilt
		\[\limes{n} q^n = 0.\]
		\begin{bew}
			Annahme: \(q\neq 0\), dann gilt \(\frac{1}{|q|}>1\) und es existiert \(x>0\), sodass \(\frac{1}{|q|} = 1 + x\).\\
			Aus Bernoullischer Ungleichung folgt 
			\[ {(1+x)}^n \geq 1+nx \]
			und somit 
			\[ |q^n-0| = |q^n| = |q|^n = \frac{1}{{(1+x)}^n} \leq \frac{1}{1+nx}. \]
			Also zu \(\varepsilon > 0\) wähle \(k\in\N \; \forall \, n\geq k\) gilt \(nx > \frac{1}{\varepsilon}\).
			\[|q^n-0|\leq \frac{1}{1+nx} \leq \frac{1}{nx} < \varepsilon \text{ für } n\geq k.\]
		\end{bew}
		\item Sei \(a\in\R \) mit \(a>0\). Dann konvergiert die \(a_n = a^{\nicefrac{1}{n}}\) gegen \(1\).
		\begin{bew}\hfill \\
			Fall 1: Die Beh.\ stimmt für \( a=1 \).\\
			Fall 2: \(a>1\). Dann ist 
			\(a_n = a^{\nicefrac{1}{n}}>1\) 
			und somit 
			\[ q_n \coloneqq a_n-1 
			= a^{\nicefrac{1}{n}} -1 >0. \]
			\[a_n = a^{\frac{1}{n}} = 1+q_n \Rightarrow a = {(1+q_n)}^n \overundersett{Bern.}{Ungl.}{\geq} 1+ nq_n\]
			\[\Rightarrow 0 \leq q_n \leq \frac{a-1}{n} \; \forall \, n\in\N \]
			Zu \(\varepsilon > 0\) wähle \(K\in\N \) mit \(K>\frac{a-1}{\varepsilon}\).\\
			Dann \(n\geq K\)
			\[|a_n-1| = |a^{\nicefrac{1}{n}}-1|= a^{\nicefrac{1}{n}} -1 = q_n \leq \frac{a-1}{n} < \varepsilon.\]
			Fall 3: \(0<a<1\). Dann ist \(b \coloneqq \frac{1}{a}>1\).
			\[\overset{\text{Fall 2}}{\Rightarrow} \limes{n} b^{\frac{1}{n}} = 1\]
			\begin{align*}
				|a^{\nicefrac{1}{n}}-1|&=a^{\nicefrac{1}{n}}\left|1-\frac{1}{a^{\nicefrac{1}{n}}}\right|\\
				&= a^{\nicefrac{1}{n}}\left| 1 - {\left(\frac{1}{a}\right)}^{\nicefrac{1}{n}} \right|\\
				&= a^{\nicefrac{1}{n}} \left|1-b^{\nicefrac{1}{n}}\right|\\
				&\leq \left|1-b^{\nicefrac{1}{n}} \right|\underset{n\rightarrow\infty}{\longrightarrow} 0
			\end{align*}
			Somit gilt
			\[\limes{n} a^{\nicefrac{1}{n}} = 1\]
		\end{bew}
	\item Es gilt \(\limes{n}n^{\nicefrac{1}{n}}=1\).
	\begin{bew}
		1. Versuch:\\
		Setze \(q_n \coloneqq n^{\nicefrac{1}{n}} -1>0\) für \(n>1\) \\
		\[\Rightarrow n={(1+q_n)}^n \geq 1+nq_n\]
		\[\Rightarrow |n^{\nicefrac{1}{n}}-1| = q_n \leq \frac{n-1}{n} = 1-\frac{1}{n}\]
		funktioniert nicht\dots \\
		Frage: Kann Bernoullische Ungleichung verbessert werden?\\
		\begin{align*}
			{(1+q)}^n &= \sum_{k=0}^{n} \binom{n}{k} q^k1^{n-k}\\
			&=1+\binom{n}{1}q + \binom{n}{2}q^2 + \sum_{k=3}^{n}\binom{n}{k} q^k1^{n-k}\\
			&\geq 1+nq + \frac{n(n-1)}{2} q^2\\
			&\geq 1+\frac{n(n-1)}{2} q^2 \quad\text{falls }	q\geq 0. \tag{\(*\)}
		\end{align*}
		Setzen \(q_n \coloneqq n^{\nicefrac{1}{n}} -1 >0\) für \(n\geq 2\).
		\[\Rightarrow n = {(1+q_n)}^n \overset{(*)}{\geq}1+\frac{n(n-1)}{2} q_n^2\]
		\[\Rightarrow q_n^2 \leq \frac{2(n-1)}{n(n-1)} = \frac{2}{n}\]
		\[\Rightarrow q_n \leq \sqrt{\frac{2}{n}} \forall n\geq 2\]
		Zu \(\varepsilon >0\) wähle \(K\in\N \) mit \(\sqrt{\frac{2}{K}}<\varepsilon \).
		\[\Rightarrow |n^{\nicefrac{1}{n}} -1| = q_n \leq \sqrt{\frac{2}{n}} \overset{n\geq K}{<} \varepsilon.\]
		Somit gilt \(\forall\varepsilon>0\exists k\in\N \), sodass für \(n\geq K\) gilt
		\[|n^{\nicefrac{1}{n}} - 1| <\varepsilon.\]
		Also per Definition
		\[\limes{n} n^{\nicefrac{1}{n}} = 1.\]
	\end{bew}
	\end{enumerate}
\end{bsp}

\begin{satz}
	Falls die reelle Folge \({(a_n)}_{n\in\N}\) konvergiert, so ist ihr Grenzwert eindeutig bestimmt.
\end{satz}
\begin{bew}
	Annahme: \({(a_n)}_{n\in\N}\) konvergiert gegen \(a\) und \(b\in\R \). Und \(a\neq b\) o.\ B.\ d.\ A.\ gilt \(a<b\).
	Wissen: 
	\begin{align*}
		\forall \varepsilon>0\exists K_1\in\N: \forall n\geq K_1 \quad |a_n -a| <\varepsilon \\
		\forall \varepsilon>0\exists K_2\in\N: \forall n\geq K_2 \quad |a_n -b| <\varepsilon
	\end{align*}
	Setze \(\varepsilon\coloneqq\frac{b-a}{2} >0\).\\
	Dann folgt für \(n\geq \max \{K_1,K_2\} \)
	\[b-a = b-a_n + a_n -a \leq \underbrace{|b-a_n|}_{<\varepsilon} + \underbrace{|a_n-a|}_{<\varepsilon} < 2\varepsilon = b-a \text{\Lightning}\]
	Somit muss \(a=b\) gelten!
\end{bew}

Bild:\\
\begin{center}
\begin{tikzpicture}[scale=2]
	\draw (0,0) -- (3,0);
	\draw (1,0) node[below=2mm] {\(a\)} (2,0) node[below=2mm] {\(b\)};
	\draw (1,-1/32) -- (1,1/32);
	\draw (2,-1/32) -- (2,1/32);
	\draw[<->] (1.5,1/8) -- (2,1/8);
	\draw (0.5,0) node {\((\)} (1.5,0) node {\()(\)} (2.5,0) node {\()\)};
	\draw (1.75, 1/8) node[above] {\(\varepsilon= \frac{b+a}{2} \)};
	\draw (1.5,-0.5) node {\(a_n\)};
	\draw[->] (1.5,-0.4) -- (1.75,-0.1);
	\draw[->] (1.5,-0.4) -- (1.25,-0.1);
\end{tikzpicture}
\end{center}

\begin{defi}[\(\varepsilon \)-Umgebung]
	Die \(\varepsilon \)-Umgebung um \(a\in\R \) ist die Menge
	\[U_\varepsilon(a)\coloneqq \{x\in\R: |x-a| <\varepsilon \} = (a-\varepsilon, a+\varepsilon).\]
\end{defi}

Beobachtung: Sei \({(a_n)}_{n\in\N}\) konvergent gegen \(a\in\R \).
\[\Leftrightarrow \forall \varepsilon >0 \,\exists K\in\N: a_n \in U_\varepsilon(a) \;\forall n\geq K.\]

\begin{defi}[Beschränktheit von Folgen]
	Eine Folge \({(a_n)}_{n\in\N} \subset \R \) heißt beschränkt, wenn für \(C\geq 0\) gilt \(|a_n|\leq C \quad \forall n\in\N \) \\
	nach oben beschränkt, wenn es ein \(C\in\R \) gibt mit \(a_n\leq C \quad \forall n\in\N \) \\
	nach unten beschränkt, wenn es ein \(C\in\R \) gibt mit \(a_n\geq C \quad \forall n\in\N \).
\end{defi}

\begin{bem}
	beschränkt \(\Leftrightarrow \) nach oben und nach unten beschränkt
\end{bem}

\begin{satz}
	Jede konvergente Folge ist beschränkt.
\end{satz}
\begin{bew}
	Sei \(\limes{n} a_n = a\). Zu \(\varepsilon = 1\) wähle \(K\in\N \).
	\(|a_n-a|<1 \quad \forall n\geq K\).
	\begin{align*}
		n\geq K &\Rightarrow |a_n| = |a_n -a+a| \leq |a_n -a| + |a| < 1 + |a|\\
		n\leq K-1 &\Rightarrow |a_n| \leq \max \{|a_1|,\ldots,|a_{K-1}|\}.
	\end{align*}
	Setze \(C\coloneqq \max \{|a_1|,\ldots,|a_{K-1}|, 1 + |a|\} \), so folgt
	\[|a_n| \leq C \quad\forall n\in\N.\]
\end{bew}

\begin{lem}
	Die Folge \({(b_n)}_n \subset \R \) konvergiert gegen \(b\neq 0\). Dann existiert \(K\in\N \), sodass 
	\[|b_n| \geq \frac{|b|}{2} \,\forall n > K .\]
\end{lem}
\begin{bew}
	Bild:\\
	\begin{center}
	\begin{tikzpicture}
		\draw (-1,0) -- (5,0);
		\foreach \x in {0, 2, 4} 
		{
			\draw (\x,-1/8) -- (\x,1/8);
		}
		\draw (0,0) node[below=2mm] {\(0\)};
		\draw (2,0) node[below=2mm] {\(\frac{b}{2}\)}; 
		\draw (2,-1.5) node {\(b_n\)};
		\draw[->] (2,-1.2) -- (3,-0.1);
		\draw (4,0) node[below=2mm] {\(b\)};
	\end{tikzpicture}
	\end{center}
	Setze \(\varepsilon \coloneqq \frac{|b|}{2}>0\). Dann existiert \(K\in\N \) mit 
	\[|b_n-b| < \varepsilon = \frac{|b|}{2} \quad\forall n\geq K, n\geq K\]
	\begin{align*}
		&\Rightarrow |b| = |b-b_n + b_n| \leq |b-b_n| + |b_n| \overset{n\geq K}{<} \frac{|b|}{2} + |b_n|\\
		&\Rightarrow |b_n| > |b| - \frac{|b|}{2} = \frac{|b|}{2} \quad \forall n\geq K.
	\end{align*}
\end{bew}

\begin{satz}[Rechenregel für Grenzwerte]
	Es gelte \(a_n\rightarrow a, b_n \rightarrow b\) für \(n\rightarrow\infty \).
	\begin{enumerate}
		\item \(\forall \lambda, \mu \in\R \) ist \( {(\lambda a_n + \mu b_n)}_{n\in\N} \) konvergent mit Grenzwert 
		\[\limes{n} (\lambda a_n + \mu b_n) = \lambda a + \mu b.\]
		\item Die Folge \( {(a_n b_n)}_{n\in\N} \) konvergiert mit Grenzwert 
		\[\limes{n} {(a_n b_n)} = ab.\]
		\item Falls \(b\neq 0\), so gibt es ein \(K_0 \in\N \) mit \(b_n \neq 0 \, \forall \, n\geq K\) und die Folge \( {\left(\frac{a_n}{b_n}\right)}_{n\geq K_0}\) ist konvergent mit Grenzwert 
		\[\limes{n} \frac{a_n}{b_n} = \frac{a}{b}.\]
	\end{enumerate}
\end{satz}
\begin{bew}\leavevmode
	\begin{enumerate}
		\item 1. Fall \(\lambda = \mu = 1\).\\
		Zu \(\varepsilon>0 \,\exists K_1, K_2 \in\N \), sodass
		\begin{align*}
			|a_n-a|<\frac{\varepsilon}{2} \quad \forall n\geq K_1\\
			|b_n-b|<\frac{\varepsilon}{2} \quad \forall n\geq K_2
		\end{align*}
		Setze \(K \coloneqq \max \{K_1, K_2\} \). Dann folgt 
		\begin{align*}
			\hphantom{=} |a_n + b_n - (a+b)| 
			&= |(a_n-a)+(b_n-b)| \\
			&\leq |a_n-a|+|b_n-b| \\
			&< \frac{\varepsilon}{2} + \frac{\varepsilon}{2} \\
			&= \varepsilon \quad \forall n\geq K.
		\end{align*}
		Also ist \(\limes{n}a_n+b_n=a+b\).\\
		Fall 2: allg.\  \(\lambda,\mu\in\R \) \\
		Aus 2.\ folgt 
		\[\limes{n} \lambda a_n = \lambda \limes{n} a_n\]
		\[\limes{n} \mu b_n = \mu \limes{n} b_n \tag{\(*\)} \]
		\(\oversett{Fall 1}{\Rightarrow} \lambda a_n + \mu b_n\) ist konvergent und 
		\begin{align*}
			&\limes{n} (\lambda a_n + \mu b_n) \\
			= &\limes{n} (\lambda a_n) + \limes{n} (\mu b_n) \\
			\overset{(*)}{=} &\lambda \limes{n} a_n + \mu \limes{n} b_n \\
			= &\lambda a + \mu b.
		\end{align*}
		\[\]
		\item Sei \(n\in\N \). Dann folgt \(a_n b_n-ab = a_n b_n -a_n b + a_n b - ab=a_n(b_n-b)+(a_n-a)b\)
		\[\Rightarrow |a_n b_n -ab|\leq |a_n| |b_n-b| + |a_n-a||b|.\]
		Nach Satz 6 existiert \(C\geq 0\) mit \(|a_n|\leq C \forall n\in\N \). Setze \(D\coloneqq \max \{C, |b|\} \).
		\[|a_n b_n - ab| \leq D(|a_n-a|+|b_n-b|) \,\forall \, n\in\N.\]
		Zu \(\varepsilon > 0\) wähle \(K_1,K_2\in\N \) mit 
		\[|a_n-a|<\frac{\varepsilon}{2(0+1)} \quad \forall n\in K_1\]
		\[|b_n-b|<\frac{\varepsilon}{2(0+1)} \quad \forall n\in K_2\]
		Dann folgt \(\forall n\geq K \coloneqq \max \{K_1, K_2\} \)
		\[|a_n b_n - ab|<\frac{\varepsilon}{2} + \frac{\varepsilon}{2} = \varepsilon.\]
		Also \(\limes{n} a_n b_n = ab\).
		\item \Obda{} \(a_n = 1\). (aus 2.\ folgt dann der allgemeine Fall mit \( \frac{a_n}{b_n} = a_n \cdot\frac{1}{b_n} \))\\
		Da \(b_n \rightarrow b \neq 0\), folgt mit Lemma 7, dass ein \(K_0\in\N \) existiert mit \(|b_n| >\frac{|b|}{2} \) für \(n\geq K_0\) 
		\[\frac{1}{b_n} \text{ ist wohldefiniert } \forall n\geq K_0.\]
		Es gilt: \( \frac{1}{b} - \frac{1}{b_n} = \frac{b_n - b}{b b_n} \) und somit 
		\[ \left| \frac{1}{b} - \frac{1}{b_n} \right| = \frac{|b_n-b|}{|b| \cdot |b_n|} \leq \frac{2 |b_n - b|}{|b|^2}. \]
		Zu \(\varepsilon >0 \) wähle \(K_1 \in \N \) mit \(|b_n - b| < \frac{|b|^2\varepsilon}{2} \quad \forall n\geq K_1 \).\\
		Dann folgt 
		\[ \left| \frac{1}{b} - \frac{1}{b_n} \right| \leq \frac{2 \cdot |b_n - b|}{|b|^2} < \varepsilon \quad \forall n\geq\max \{ K_0,K_1\}. \]
		Somit folgt \( \frac{1}{b_n} \rightarrow \frac{1}{b} \) (für \(n\rightarrow \infty \)). Somit folgt die allg. Aussage aus Teil 2 von Satz 7.1.8.
	\end{enumerate}
\end{bew}

%20.11.2018
reelle Folgen \(f = {(f_n)}_n, g = {(g_n)}_n  \) \\
\[ {(f+g)}_n \coloneqq f_n + g_n, \quad n\in\N \]
\(|{(\lambda f)}_n \coloneqq \lambda f_n \Rightarrow {(\lambda f + \mu g)}_n = \lambda f_n + \mu g_n \) ist eine Linearkombination.\\
\(\Rightarrow{}\) Raum der reellen Folgen ist ein reeller Vektorraum.
\begin{align*}
	&\{ \text{Raum der (reellen) Folgen} \} \\
	&\supsetneq \{ \text{Raum der beschränkten (reellen) Folgen} \} \\
	&\supsetneq \{ \text{Raum der (reellen) konvergenten Folgen} \} \\
	&\supsetneq \{ \text{Raum der (reellen) Nullfolgen} \}.
\end{align*}
\({(f_n)}_{n\in\N}\) ist eine Nullfolge, falls \( \limes{n} f_n = 0 \).

\begin{bsp}[1]
	\( p,q \) Polynome vom Grad \(m,n\in\N \).\\
	D. h. \[ p(x) = a_m x^m + a_{m-1} x^{m-1} + \cdots + a_1 x + a_0 \quad \forall x\in\R \]
	\[q(x) = b_n x^n + b_{n-1} x^{n-1} + \cdots + b_1 x + b_0 \quad b_n \neq 0 \neq a_m \]
	\begin{align*}
		k\in\N. \frac{p(k)}{q(k)} = \frac{a_m k^m + a_{m-1} k^{m-1} + \cdots + a_1 k + a_0}{b_n k^n + b_{n-1} k^{n-1} + \cdots + b_1 k + b_0}\\
		= k^{m-n} \frac{a_m + a_{m-1} k^{-1} + \cdots + a_1 k^{1-m} + a_0 k^{-m} }{b_n + b_{n-1} k^{-1} + \cdots + b_1 k^{1-n} + b_0 k^{-n}}\\
		\overset{\text{Satz 8}}{\longrightarrow}
		\begin{cases}
			0, \text{ falls } n>m.\\
			\frac{a_n}{b_n}, \text{ falls } n = m.
		\end{cases}
	\end{align*}
\end{bsp}

\begin{bsp}[Geometrische Reihe]
	\begin{align*}
		-1<q<1.\\
		a_n &\coloneqq 1 + q + q^2 + \cdots + q^n\\
		&= \sum_{l=0}^{n} q^l \overset{\text{Satz 3.5.7}}{=} \frac{1-q^{n+1}}{1-q}\\
		\Rightarrow \limes{n} a_n &= \frac{1- \limes{n} q^{n+1}}{1-q} = \frac{1}{1-q}.
	\end{align*}
	Da \(q^n \rightarrow 0, n\rightarrow\infty \), Bsp. 6 oben.
	Schreiben hierfür \[ \sum_{l=0}^{\infty} q^l = \frac{1}{1-q}, \quad -1<q<1 \]
\end{bsp}

\begin{bsp}
	Ist \( {(b_n)}_n \) beschränkt, \( {(a_n)}_n \) Nullfolge. \( \Rightarrow {(b_n a_n)}_n \) Nullfolge. (Hausaufgabe)
\end{bsp}

\textbf{Notation:} Wir sagen die Aussagen \( A(n), n\in\N \) gelten für fast alle \( n\in\N \), falls \( K_0\in\N \) existiert, sodass \(A(n)\) wahr ist für alle \( n\geq K_0 \) (d.\ h.\ für alle genügend großen \( n \), d.\ h.\  \( A(n) \) wahr für alle bis auf endlich viele \(n\in\N \)).

\begin{bsp}
	\[ a_n\rightarrow a,n\rightarrow\infty \Leftrightarrow \forall \varepsilon > 0 \text{ ist } a_n\in U_\varepsilon(a) (= a-\varepsilon, a+\varepsilon) \text{ für fast alle } n.  \]
\end{bsp}

\begin{satz}
	Seien \( {(a_n)}_n, {(b_n)}_n\) konvergente reelle Folgen, \( a_n\rightarrow a, b_n \rightarrow b, n\rightarrow\infty \). Dann gilt
	\begin{enumerate}
			\item Aus \(a_n\leq b_n\) für fast alle \(n\) folgt \(a\leq b\).
			\item Sind \(c,d\in\R, c\leq a_n\leq d \) für fast alle \(n \Rightarrow c\leq a\leq d\)
			\item (Sandwichlemma) Ist \(a_n \leq c_n \leq b_n \) für fast alle \(n\) (\( {(c_n)}_n \) weitere reelle Folge) und \( a=b \Rightarrow {(c_n)}_n \) konvergiert und \( \lim\limits_{b\rightarrow\infty}c_n = a \) (\( =b \)).
	\end{enumerate}
\end{satz}
\begin{bew}\leavevmode
	\begin{enumerate}
		\item Bild: Ang.\  \( a>b \) \\
		\( \varepsilon = \frac{b-a}{2} >0 \)
		\begin{center}
		\begin{tikzpicture}[scale=2]
			\draw (0,0) -- (4,0);
			\draw (1,-1/8) -- (1,1/8);
			\draw (1,0) node[below=2mm] {\(b\)};
			\draw (3,-1/8) -- (3,1/8);
			\draw (3,0) node[below=2mm] {\(a\)};
			\draw (2,0) node {\()(\)} (1.5,0) node[below] {\(b_n\)} (2.5,0) node[below] {\(a_n\)};
			\draw[<->] (2,1/4) -- (3,1/4);
			\draw (2.5,1/4) node[above] {\(\varepsilon\)};
		\end{tikzpicture}
		\( \Rightarrow b_n > a_n \) \Lightning{}
		\end{center}
		\textit{Formal}: \(\exists K_0 \in\N:a_n\leq b_n \quad \forall n\geq K_0. \)
		\begin{align*}
		&\forall \varepsilon > 0 \exists K_1 \in\N,K_2\in\N: &a_n \in U_\varepsilon(a) &\forall n\geq K_1, &a - \varepsilon < a_n < a+\varepsilon \\
		&\Rightarrow K \coloneqq \max(K_0, K_1, K_2) &b_n \in U_\varepsilon (b) &\forall n\geq K_2,  &b-\varepsilon < b_n < b+\varepsilon.
		\end{align*}
		Ang. \(a>b: \varepsilon \coloneqq \frac{a-b}{2} > 0 \Rightarrow \) \\
		\(K \) wie oben \(:\Rightarrow a< a_n + \varepsilon \leq b_n + \varepsilon < b + 2\varepsilon = b + 2 \frac{a-b}{2} = a \)
		\( \Rightarrow a<a\) \Lightning{} \( \Rightarrow a\leq b \checkmark{}\).
		Andere Möglichkeit:\\
		\[ a_n \leq b_n, \forall \varepsilon > 0: a-\varepsilon < a_n<a + \varepsilon, b-\varepsilon < b_n < b + \varepsilon \quad \forall n\geq K. \]
		\[ a < a_n + \varepsilon \leq b_n + \varepsilon < b + 2\varepsilon \Rightarrow \underbrace{a-b < 2\varepsilon \quad \forall \varepsilon > 0}_{\Rightarrow a-b\leq 0 \Leftrightarrow a\leq b.}. \]
		\item Nehme \(b_n = c, b_n \rightarrow c\).\\
		Da \(b_n = c \leq a_n \overset{1.}{\Rightarrow} c= \lim b_n \leq \lim a_n = a \).\\
		Nehme auch \(b_n = d, a_n \leq d = b_n \overset{1.}{\Rightarrow} a \leq \lim b_n = d. \checkmark \)
		\item Haben \(\forall \varepsilon > 0\).
		\begin{align*}
			&\exists K_0 \in\N: &a_n \leq c_n \leq b_n \quad \forall n\geq K_0\\
			&\exists K_1, K_2 \in\N: &a-\varepsilon < a_n < a + \varepsilon \quad \forall n\geq K_1\\
			&&\underbrace{b-\varepsilon}_{=a-\varepsilon} < b_n < \underbrace{b + \varepsilon}_{=a+\varepsilon} \, \forall \, n \geq K_2. \text{ (da }b=a)
		\end{align*}
		\[\forall  n\geq K: a-\varepsilon < a_n \leq c_n \leq b_n < a+\varepsilon \]
		\[ \Rightarrow a-\varepsilon < c_n < a_n + \varepsilon \Leftrightarrow c_n\in U_\varepsilon(a) \quad \forall n\geq K \] 
		\( \Leftrightarrow \) konvergiert \( {(c_n)}_n \) gegen \( a \)!
	\end{enumerate}
\end{bew}

Achtung! \( a_n < b_n \forall n, a_n \rightarrow a, b_n \rightarrow b \nRightarrow  a<b\).\\
Bsp. \(a_n = 0, b_n = \frac{1}{n}\).

\begin{defi}[Uneigentliche Konvergenz]
	Die Folge \( {(a_n)}_n \) konvergiert uneigentlich (divergiert bestimmt) gegen \(+\infty \), falls 
	\[ \forall R>0 \exists K\in\N \text{ mit } a_n > R \quad \forall n\geq K. \]
	Schreiben \( \limes{n} a_n = \infty \) oder \( a_n \rightarrow +\infty, n\rightarrow \infty \)
	Analog für \( \limes{n} a_n = -\infty \), falls 
	\[ \forall R<0 \exists K\in\N: a_n < R \forall n\geq K. \]
\end{defi}

\begin{bsp}
	Ist \(a>1 \Rightarrow \limes{n}q^n = +\infty, 0< \frac{1}{q} < 1. \)
\end{bsp}
\begin{bew}
	\( \frac{1}{q^n} = {\left( \frac{1}{q} \right)}^n \rightarrow 0, n\rightarrow \infty \) \\
	d.\ h.\ zu \(R>0 \exists K\in\N: \frac{1}{q^n} < \frac{1}{R} \quad \forall n\geq K. \) \\
	\( \Leftrightarrow q^n > R \quad \forall n\geq K. \) Also \(\lim q^n = +\infty \) nach Def.\\
	Insgesamt: 
	\begin{align*}
		&q>1 &\Rightarrow \limes{n}q^n = +\infty.\\
		&q=1 &\Rightarrow \limes{n}q^n = 1.\\
		&-1<q<1 & \Rightarrow \limes{n} q^n = 0.\\
		&q\leq -1 & \Rightarrow {(q^n)}_n \text{ ist nicht konvergent.}
	\end{align*}
	Ist \( q<-1 \Rightarrow {(q_n)}_n \) nicht beschränkt ist.
\end{bew}

\begin{satz}[Kehrwerte]\leavevmode
	\begin{enumerate}
		\item Aus \( |a_n| \rightarrow \infty, n\rightarrow\infty \) folgt \(\frac{1}{a_n} \rightarrow0,n\rightarrow\infty \).
		\item Aus \( a_n\rightarrow 0, a_n > 0 \) (bzw. \(a_n<0\)) \( \forall n \) folgt \( \frac{1}{a_n} \rightarrow\infty, n\rightarrow\infty \) (\( \frac{1}{a_n} \rightarrow -\infty, n\rightarrow\infty \)).
	\end{enumerate}
\end{satz}
\begin{bew}
	Übung.\phantom{\qedhere}
\end{bew}

\end{document}