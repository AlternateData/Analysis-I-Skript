\documentclass[../ana1.tex]{subfiles}
\onlyinsubfile{\sectionNumbering} %Use numbering relative to sections and not subsection

\begin{document}
\setcounter{section}{10}

\section{Konvergente Reihen, Teil 1}
hatten: endl. Summen: \[ a_1+a_2+\cdots+a_n = \sum_{k=1}^{n}a_k \]
Ziel: unendl. Summen: \[ a_1+a_2+a_3+\cdots = \sum_{k=1}^{\infty}a_k = \sum_{n=1}^{\infty}a_n \]
Was soll das sein?
\begin{defi}
	Dieses Symbol \[ \sum_{n=1}^{\infty}a_n \] steht für die Folge \({(S_n)}_n\) der Partialsummen: \[S_n := \sum_{k=1}^{n}a_k .\]
	Die Reihe \( \sum_{k=1}^{\infty}a_k \) konvergiert, falls die Folge ihrer Partialsummen konvergiert. In diesem Fall setzen wir 
	\[ \sum_{k=1}^{\infty}a_k := \limes{n} \sum_{k=1}^{\infty}a_k = \limes{n} S_n \]
	Sagen auch die Reihe \( \sum_{n=1}^{\infty}a_n \) konvergiert. Die Reihe \( \sum_{n=1}^{\infty}a_n \) divergiert, falls \( {(S_n)}_n \) divergiert. Reihe \( \sum_{n=1}^{\infty}a_n \) heißt bestimmt divergent, falls \({(S_n)}_n\) bestimmt gegen \(+ \infty \) oder \(-\infty \) divergiert. Setzen dann \( \sum_{n=1}^{\infty}a_n =-\infty \), falls \(\limes{n} \sum_{k=1}^{n}a_k = -\infty \). \( \sum_{n=1}^{\infty}a_n = +\infty \), falls \( \limes{n}\sum_{k=1}^{n}a_k=+\infty \)
\end{defi}
\begin{bem}
	Sei \({(b_n)}_n\) eine Folge. \(a_1 = b_1, a_n = b_n-b_{n-1}, n\geq 2 \Rightarrow \sum_{k=1}^{n}a_k = b_n \).\\
	Auch Reihen der Form \( \sum_{n=0}^{\infty}a_n \) oder \( \sum_{n=42}^{\infty}a_n\), (\(v\in\Z \)) \(\sum_{n=v}^{\infty}a_n\).
\end{bem}
Beobachtung: Sind \(a_n \geq 0 \Rightarrow S_n = \sum_{k=1}^{\infty}a_k \) monoton wachsend in \(n\).\\
\( \overset{\text{Mon. Konv.}}{\Rightarrow}\) \\
entweder \({(S_n)}_n\) ist nach oben beschränkt \( \Rightarrow \limes{n} S_n \in [0,\infty) \) \\
oder \({(S_n)}_n\) ist nach oben beschränkt \( \Rightarrow \limes{n} S_n  = +\infty \) \\
\( \Rightarrow \)
\begin{satz}
	Sind \(a_n \geq 0, n\in\N \), dann gilt\\
	entweder ist \(S_n = \sum_{k=1}^{n}a_n\) nach oben beschränkt und dann ist \[ \sum_{k=1}^{\infty}a_k = \limes{n}S_n \in [0,\infty) \]
	oder \(S_n \rightarrow \infty, n\rightarrow\infty \) und dann ist \[ \sum_{k=1}^{\infty}a_k = \limes{n}S_n = +\infty. \]
\end{satz}
\begin{bew}
	Oben.
\end{bew}
\begin{kor}
	Sind \(a_n \geq 0, n\in \N \), so ist
	\begin{align*}
		\text{entweder } &\sum_{n=1}^{\infty}a_n < \infty &\text{in diesem Fall ist die Reihe nach oben beschränkt.}\\
		\text{oder } &\sum_{n=1}^{\infty}a_n = +\infty &\text{nach oben unbeschränkt.}
	\end{align*}
\end{kor}
\begin{satz}[Cauchy-Kriterium für Reihen]
	Seien \(a_n \in\R, n\in\N \). Dann gilt: 
	\[ \sum_{n=1}^{\infty} a_n \text{ konvergiert } \Leftrightarrow \forall \varepsilon >0 \exists K \in \N : | \sum_{j=n+1}^{m}a_j|<\varepsilon \quad \forall m>n\geq K.\]
\end{satz}
\begin{bew}
	Reihe konvergiert per Def.\ genau dann, wenn Folge der Partialsummen \( {(S_n)}_n, S_n = \sum_{k=1}^{n}a_k \) konvergiert.
	\[ \overset{\text{Cauchy-Krit.\ Folgen}}{\Longleftrightarrow} \forall \varepsilon > 0 \exists K\in\N: |\underbrace{S_m-S_n}_{=\sum_{k=1}^{m}a_k-\sum_{k=1}^{n}a_k=\sum_{k=n+1}^{m}a_k}|<\varepsilon \quad \forall m>n\geq K. \]
\end{bew}
%06.12.2018
\begin{bew}
	Partialsummen \({(S_n)}_n\)
	\[ S_n := \sum_{j=1}^{n} a_j \]
	Cauchy-Kriterium  für Folgen:
	\[ \Rightarrow {(S_n)}_n \text{ konvergiert } \Leftrightarrow \forall \varepsilon > 0 \exists K\in\N:|S_m-S_n|<\varepsilon \quad \forall m>n\geq K. \]
	\[ S_m - S_n = \sum_{j=1}^{m}a_j - \sum_{j=1}^{n} a_j = \sum_{j=n+1}^{m} a_j. \]
	D. h. \({(S_n)}_n\) konvergiert \[ \Leftrightarrow \forall \varepsilon > 0 \exists K\in \: \left|\sum_{j=n+1}^{m}a_j\right|<\varepsilon \quad \forall m>n\geq K. \]
\end{bew}
\begin{bsp}[Geometrische Reihe]
	Sei \(|q|<1\).
	\[ \Rightarrow \sum_{n=0}^{\infty} q^n = \frac{1}{1-q} \]
\end{bsp}
\begin{bsp}
	\[S_n = \sum_{j=0}^{n} q^n = \frac{1-q^{n+1}}{1-q} \rightarrow \frac{1}{1-q} \text{ (da } \limes{n}q^{n+1}=0 \text{)} \]
\end{bsp}
\begin{bsp}
	\[\sum_{n=2}^{\infty}\frac{1}{n(n-1)}=1 \text{ (teleskopierende Summe)} \]
\end{bsp}
\begin{bew}
	\begin{align*}
		\frac{1}{n(n-1)} =\frac{1}{n-1} - \frac{1}{n} \quad\forall n\geq 2.\\
		\Rightarrow S_n &= \sum_{j=2}^{n} \frac{1}{j(j-1)} = \sum_{j=2}^{n} \left( \frac{1}{j-1} - \frac{1}{j} \right)\\
		&= \sum_{j=2}^{n} \frac{1}{j-1} - \sum_{j=2}^{n} \frac{1}{j} = \sum_{j=1}^{n-1}\frac{1}{j} - \sum_{j=2}^{n}\frac{1}{j}\\
		&= 1 - \frac{1}{n} \rightarrow 1, n\rightarrow \infty
	\end{align*}
\end{bew}
\begin{bsp}
	\[ \sum_{n=1}^{\infty} \frac{1}{n^2} \]
\end{bsp}
\begin{bew}
	\begin{align*}
		S_n = \sum_{j=1}^{n}\frac{1}{j^2} = 1 + \sum_{j=2}^{n} \frac{1}{j^2}\\
		\frac{1}{j^2} \leq \frac{1}{j(j-1)} \quad \forall j\geq 2.\\
		\Rightarrow S_n = 1 + \sum_{j=2}^{n} \frac{1}{j^2}.
	\end{align*}
	Also folgt aus \( \leq 1 + \sum_{j=2}^{n} \frac{1}{j(j-1)} = 1 + 1 - \frac{1}{n} \).\\
	Monotone Konvergenz, da \({(S_n)}_n\) konvergiert, also konvergiert \(\sum_{j=1}^{\infty} \frac{1}{j^2} = 2 - \frac{1}{n} \leq 2 \quad\forall n \).
\end{bew}
\begin{kor}
	Wenn die Reihe \(\sum_{n=1}^{\infty} a_n\) konvergent, so ist \({(a_n)}_n\) eine Nullfolge.
\end{kor}
\begin{bew}
	Satz 4 \( \Rightarrow \forall \varepsilon >0 \exists K\in\N:|\sum_{j=n+1}^{m}a_j| < \varepsilon \quad \forall m>n\geq K \).\\
	Setze \[ m=n+1 : \sum_{j=n+1}^{n+1} a_j = a_{n+1} \Rightarrow |a_{n+1}|<\varepsilon \quad \forall n\geq K. \]
	\[ \Rightarrow \limes{n}a_{n+1}=0=\limes{n}a_n. \]
\end{bew}
\begin{bem}
	Warnung: die Umkehrung gilt nicht!
\end{bem}
\begin{bsp}[harmonische Reihe]
	%BEGIN ALIGN MACHEN TODO
	\[ \sum_{n=1}^{\infty} \frac{1}{n} \text{ divergiert, obwohl } \frac{1}{n} \rightarrow 0. \]
	\[ S_n = \sum_{j=1}^{n}\frac{1}{j} \]
	\[ S_m -S_n \text{ wähle }m=2n \]
	\[ S_{2n} - S_n = \sum_{j=1}^{2n}\frac{1}{j} - \sum_{j=1}^{n}\frac{1}{j} = \sum_{j=n+1}^{2n}\frac{1}{j} \]
	\[ = \frac{1}{n+1} + \frac{1}{n+2} + \cdots + \frac{1}{2n} \]
	\[ \leq \frac{1}{2n} + \frac{1}{2n} + \cdots + \frac{1}{2n} \]
	\[ = \frac{1}{2} \]
	\[ \Rightarrow |S_{2n}-S_n| = S_{2n}-S_n \rightarrow \frac{1}{2} \quad\forall n>1 \]
	also kann \({(S_n)}_n\) nicht konvergieren!
	\[ \Rightarrow \sum_{n=1}^{\infty} \frac{1}{n} = +\infty. \]
\end{bsp}
\begin{satz}
	Gilt \(0\leq a_n \leq b_n \quad \forall n\in\N \), so folgt:
	\begin{enumerate}
		\item Ist \( \sum_{n=1}^{\infty}b_n \) konvergent, so konvergiert auch \( \sum_{n=1}^{\infty}a_n \) und \( \sum_{n=1}^{\infty} a_n \leq \sum_{n=1}^{\infty}b_n \).
		\item Divergiert \( \sum_{n=1}^{\infty}a_n \), so divergiert auch \( \sum_{n=1}^{\infty}b_n \) (gegen \(+\infty \)).
	\end{enumerate}
\end{satz}
\begin{bew}
	\begin{align*}
		S_n = \sum_{j=1}^{n}a_j\\
		t_n = \sum_{j=1}^{n}b_j\\
		\Rightarrow S_n \leq t_n \quad \forall n\in\N \\
		t_{n+1}\geq t_n, S_{n+1}\geq S_n
	\end{align*}
	\begin{enumerate}
		\item Ist \( \sum_{n=1}^{\infty}b_n \) konvergent, so konvergiert \({(t_n)}_n\).\\
		\[ t:= \limes{n}t_n\in [0,\infty) \Rightarrow t_n \leq t. \]
		\[ \Rightarrow S_n \leq t_n \leq t \Rightarrow {(S_n)}_n \text{ nach oben beschränkt.} \]
		\[ \overset{\text{Mon. Konv.}}{\Rightarrow} s = \limes{n} S_n \text{ existiert und } \underbrace{S}_{= \sum_{n=1}^{\infty}a_n} \leq t = \sum_{n=1}^{\infty} b_n. \]
		\item Aus 1.\ folgt ist \( \sum_{n=1}^{\infty}a_n =\infty \) divergent, so muss auch \( \sum_{n=1}^{\infty}b_n = \infty \).
	\end{enumerate}
\end{bew}
\begin{satz}
	Sind \( \sum_{n=1}^{\infty}a_n, \sum_{n=1}^{\infty}b_n \) konvergente (reelle Reihen), so konvergiert auch \( \sum_{n=1}^{\infty}(\lambda a_n + \mu b_n) \quad \forall \lambda, \mu \in\R \) und 
	\[ \sum_{n=1}^{\infty} (\lambda a_n + \mu b_n) = \lambda \sum_{n=1}^{\infty}a_n + \mu \sum_{n=1}^{\infty}b_n. \]
\end{satz}
\begin{bew}
	\[ S_n = \sum_{j=1}^{n}a_j, t_n = \sum_{j=1}^{n}b_j, S = \limes{n}S_n, t = \limes{n}t_n \]
	\[ \sum_{j=1}^{n}(\lambda a_n + \mu b_n) = \lambda\sum_{j=1}^{n}a_j + \mu \sum_{j=1}^{n}b_j = \lambda S_n + \mu t_n \rightarrow \lambda S + \mu t, n\rightarrow\infty. \]
\end{bew}
%11.12.2018
\begin{satz}[Cauchyscher Verdichtungssatz]
	Sei \({(a_n)}_n\) monoton fallende Nullfolge. Dann gilt
	\[\sum_{n=1}^{\infty} a_n \text{ konvergiert }\]
	\[\Leftrightarrow \text{ die \glqq{}verdichtete\grqq{} Reihe } \sum_{n=0}^{\infty}2^n a_{2^n} = a_1 + 2a_2 + 4a_4 + 8 a_8 + \dots \text{ konvergiert}.\]
\end{satz}
\begin{bew}
	\( a_{n+1} \leq a_n \quad\forall n, a_n\rightarrow0 \quad n \rightarrow\infty \Rightarrow a_n \geq 0 \quad\forall n\in\N \).
	\[ S_n := \sum_{j=1}^{n}a_j, t_n := \sum_{j=0}^{K}2^j a_{2^j} \text{ sind mon.\ wachsende Folgen.} \]
	\glqq{}\(\Leftarrow{}\)\grqq{}: Beobachtung: Jedes \(n\in\N \) ist in genau einem \glqq{}dyadischen\grqq{} Intervall. \( I_l := \{ 2^l, 2^l + 1, \dots, 2^{l+1} - 1 \} \)
	\[ I_0 = \{1\}, I_1 = \{2,3\}, I_3 = \{4,5,6,7\}, \#I_l = 2^{l+1} - 2^l = 2^l \]
	Ang., \(n<2^k\). 
	\[ S_n = \sum_{j=1}^{n}a_j \leq \sum_{j=1}^{2^k - 1}a_j = \sum_{l=0}^{k}\sum_{j\in I_l} \underbrace{a_j}_{\leq a_{2^l}} \]
	Bemerkung: \(I_l \cap I_m = \emptyset \; l \neq m, \quad \bigcup_{l=0}^{k} I_l = \{n \in \N | n \leq 2^k -1\} \)
	\[\leq \sum_{l=0}^{k} \#I_l \cdot a_{2^l} = \sum_{l=0}^{k} 2^l \cdot a_{2^l} = t_k\]
	\[ \Rightarrow S_n \leq t_k, \text{ falls } n\leq 2^k - 1 \]
	Annahme: \( t = \lim t_k \) existiert. \( \Rightarrow t_k \leq t \quad\forall k \).
	\[ \Rightarrow S_n \leq \limes{k}t_k = t. \]
	\[S_n \leq t \quad\forall n\in\N.\]
	\( \overset{\text{Mon. Konv.}}{\Rightarrow} \limes{n} S_n \) existiert. \checkmark{}\\
	\glqq{}\(\Rightarrow \)\grqq{}: Beachte: Jedes \(n \in\N, n\geq 2 \) ist in genau einem Block.
	\[ \tilde{I}_l = \{ 2^{l-1} + 1, 2^{l-1} + 2, \dots, 2^l \}, l\in\N. \]
	Sei \(n \geq 2^k \Rightarrow \)
	\begin{align*}
		S_n = \sum_{j=1}^{n}a_j \geq \sum_{j=1}^{2^k}a_j = a_1 + \sum_{j=2}^{2^k} a_j\\
		= a_1 + \sum_{l=1}^{k}\sum_{j\in\tilde{I}_l} a_j \geq a_1 + \sum_{l=1}^{k} \underbrace{\# \tilde{I}_l}_{=2^{l-1}} a_{2^l}
	\end{align*}
	Für \(n \geq 2^k \) ist
	\begin{align*}
		S_n &\geq a_1 + \sum_{l=1}^{k} 2^{l-1}a_{2^l}\\
		&=a_1 + \frac{1}{2} \underbrace{ \sum_{l=1}^{k}2^l a_{2^l} }_{=t_k - a_1}\\
		&= \frac{1}{2}a_1 + \frac{1}{2}t_k
	\end{align*}
	\[ \Rightarrow t_k \leq 2S_n \quad\forall n\geq 2^k. \]
	Wir nehmen an, dass \( \limes{n}S_n = S \) existiert.
	\[ S_n \leq S_{n+1} \leq \cdots \leq S \]
	Halte \( k\in\N \) fest.
	\begin{align*}
		&\Rightarrow t_k \leq 2S_n \text{ für fast alle } n.\\
		&\Rightarrow t_k \leq \limes{n} 2S_n = 2S\\
		&\Rightarrow t_k \leq 2S \quad\forall k\in\N.
	\end{align*}
	\[ \overset{\text{Mon. Konv.}}{\Rightarrow}\limes{k}t_k \text{ existiert.} \]
\end{bew}
\begin{bsp}
	\[ \sum_{n=1}^{\infty} \frac{1}{n^p} \text{ konv. } \Leftrightarrow p>1. \]
\end{bsp}
\begin{bew}
	Verdichtete Reihe ist 
	\[ \sum_{l=0}^{\infty}2^l \frac{1}{{(2^l)}^p} = \sum_{l=0}^{\infty} {(2^{1-p})}^l. \]
	geometrische Reihe, sie konvergiert genau dann, wenn \( 2^{1-p} < 1 \Leftrightarrow p>1. \)
\end{bew}
\begin{bsp}
	\[ \sum_{n=1}^{\infty} \frac{1}{n{(1+\ln_2 n )}^p}.\]
\end{bsp}
\begin{satz}[Leibniz]
	Ist \( {(a_n)}_n \) eine monoton fallende Nullfolge, so konvergiert die alternierende Reihe
	\[ a_1 - a_2 + a_3 - a_4 + \cdots = \sum_{n=1}^{\infty} {(-1)}^{n+1} a_n \]
\end{satz}
\begin{bsp}
	\[ 1 - 1/2 + 1/3 - 1/4 + \cdots = \sum_{n=1}^{\infty}{(-1)}^{n+1}\frac{1}{n} \text{ konv. } (=\log 2). \]
	\[ 1 - 1/3 + 1/5 - 1/7 + \cdots = \sum_{n=0}^{\infty}{(-1)}^n \frac{1}{2n+1} \text{ konv. } (=\frac{\pi}{4}) \]
	Beachte: \( \sum_{n=1}^{\infty}\frac{1}{n} = +\infty \).
\end{bsp}
\begin{bew}
	Aus \( a_{n+1} \leq a_n, a_n \rightarrow 0 \Rightarrow a_n \geq 0 \quad\forall n\in\N. \)
	\[ S_k := \sum_{j=1}^{k} {(-1)}^{j+1} a_j \quad k\in\N. \]
	\[S_{2n} = a_1 - a_2 + a_3 - a_4 + \cdots + a_{2n-1} - a_{2n} = \underbrace{(a_1 - a_2)}_{\geq 0} + \underbrace{(a_3 - a_4)}_{\geq 0} + \cdots + \underbrace{(a_{2n-1} - a_{2n})}_{\geq 0} \]
	\begin{align*}
		S_{2n+1} = \sum_{j=1}^{2n+1}{(-1)}^{j+1}a_j &= a_1 - a_2 + a_3 - \cdots - a_{2n} + a_{2n+1}\\
		&= a_1 - \underbrace{a_2 - a_3}_{\geq 0} - \cdots - \underbrace{(a_{2n} - a_{2n+1})}_{\geq 0}\\
		\Rightarrow S_{2(n+1)} = S_{2n+2} = S_{2n} + (a_{2n+1} - a_{2n+2}) \geq S_{2n}.\\
		S_{2(n+1)+1} = s_{2n+3} = S_{2n+1} - (a_{2n+2} - a_{2n+3}) \leq S_{2n+1}
	\end{align*}
	\[ \text{und } 0 \leq S_{2n} = \sum_{j=1}^{2n} {(-1)}^{j+1} a_j = S_{2n+1} - {(-1)}^{2n+2} \cdot a_{2n+1} = S_{2n+1} - a_{2n+1} \leq S_{2n+1}\]
	\[\Rightarrow S_{2n} \leq S_{2n+2}, \quad S_{2n+1} \geq S_{2n+3} \text{ und } 0 \leq S_{2n} \leq S_{2n+1} - a_{2n+1} \leq S_{2n+1} \leq a_n\]
	\( \overset{\text{Mon. Konv.}}{\Rightarrow} S_1 = \limes{n}S_{2n} \) und \( S_2 = \limes{n}S_{2n+1} \) existieren.\\	
	und: 
	\[ \underbrace{S_{2n} - S_{2n+1}}_{S_1 - S_2} = a_{2n+1} \rightarrow \quad n\rightarrow \infty \]
	\[ \Rightarrow S_1 - S_2 = \limes{n}(S_{2n} - S_{2n+1}) = \limes{n}a_{2n+1} = 0 \]
	\[ \Rightarrow S_1 = S_2 \Rightarrow {(S_n)}_n \text{ konvergiert auch gegen } S_1 (=S_2). \]
\end{bew}
\begin{defi}
	Eine Reihe \( \sum_{n=1}^{\infty} a_n \) heißt absolut konvergent, falls \( \sum_{n=1}^{\infty}|a_n| \) konvergiert, d.\ h.\  \(\sum_{n=1}^{\infty}|a_n|<\infty \).
\end{defi}
\begin{satz}
	Eine absolut konvergente Reihe \( \sum_{n=1}^{\infty}a_n, a_n \in\R \) ist konvergent, und
	\[ \left| \sum_{n=1}^{\infty}a_n \right| \leq \sum_{n=1}^{\infty}|a_n|.\] (Dreiecksungleichung für Reihen)
\end{satz}
\begin{bew}
	Annahme: \( \sum_{n=1}^{\infty} |a_n| \) konvergiert.
	\[ \overset{\text{Cauchy Krit.}}{\Leftrightarrow} \forall \varepsilon>0 \exists K\in\N: \sum_{j=n+1}^{m} |a_j| < \varepsilon\quad\forall m>n\geq K. \]
	Beachte: 
	\[ \left| \sum_{j=n+1}^{m}a_j \right| \leq \sum_{j=n+1}^{m}|a_j| < \varepsilon \quad \forall m>n\geq K \]
	\[\Rightarrow \forall \varepsilon > 0 \exists k \in \N: \left| \sum_{j = n+1}^{m} a_j \right| < \varepsilon \quad \forall m>n\geq K\]
	\[\overset{\text{Cauchy}}{\underset{\text{Kriterium}}{\Rightarrow}} \sum_{n=1}^{\infty} a_n \text{ konvergiert}\]
	Auch: \(m\) fest.
	\[ \underbrace{\left| \sum_{n=1}^{m}a_n \right|}_{=S_m}  = |S_m| \leq \sum_{n=1}^{m}|a_n| \leq \sum_{n=1}^{\infty}|a_n| \]
	\[ \Rightarrow |S_m| \leq \sum_{n=1}^{\infty}|a_n| \]
	Wissen \( S_m\rightarrow \sum_{n=1}^{\infty}a_n, m\rightarrow\infty \)
	\[ \Rightarrow \left|\sum_{n=1}^{\infty}a_n \right| = |S| = |\limes{m}S_m| = \limes{m}|S_m| \leq \sum_{n=1}^{\infty}|a_n|. \]
\end{bew}
\begin{bem}
	Warnung: Umkehrung von Satz 11 ist falsch! (Bsp.\ alternierende harmonische Reihe)
\end{bem}
\begin{defi}
	Wir nennen eine Reihe \( \sum_{n=0}^{\infty}c_n \) eine Majorante, von \( \sum_{n=0}^{\infty}a_n, a_n\in\R \), falls \( |a_n| \leq c_n \) für fast alle \(n\).
\end{defi}
\begin{kor}
	Hat die Reihe \( \sum_{n=0}^{\infty}a_n \) eine konvergente Majorante, so konvergiert \( \sum_{n=0}^{\infty}a_n \) absolut und ist somit auch konvergent.
\end{kor}
\begin{bew}
	Folgt direkt aus Satz 6, Def. 12 und Satz 11.
\end{bew}
\begin{satz}[Quotientenkriterium]
	Sei \( \sum_{n=0}^{\infty}a_n \) Reihe, \(a_n\neq0\), und es gebe ein \(q\) mit \(0<q<1\), sodass
	\[(*) \frac{|a_{n+1}|}{|a_n|} \leq q \text{ für fast alle }n. \]
	Dann ist \( \sum_{n=0}^{\infty}a_n \) absolut konvergent.
\end{satz}
\begin{bew}
	\begin{enumerate}
		\item \( (*) \Rightarrow \limsup\limits_{n\rightarrow\infty} \frac{|a_{n+1}|}{|a_n|} \leq q. \)
		\item \( \tilde{q} = \limsup\limits_{n\rightarrow\infty} \frac{|a_{n+1}|}{|a_n|} \Rightarrow \forall\varepsilon>0: \frac{|a_{n+1}|}{|a_n|} \leq \tilde{q} + \varepsilon \) für fast alle \(n\).
	\end{enumerate}
	\[ (*)\Rightarrow\exists n_0 \in\N: \frac{|a_{n+1}|}{|a_n|} \leq q \quad\forall n\geq n_0. \]
	\[ p\in\N_0, n=n_0 + p \]
	\begin{align*}
		\Rightarrow |a_{n+1}| \leq q |a_n| \leq q^2|a_{n-1}| \leq \cdots\leq q^{p+1}|a_{n_0}|\\
		\Rightarrow |a_n| \leq q^p |a_{n_0}| = q^n \underbrace{q^{-n_0}|a_{n_0}|}_{=M} \leq M q^n =: c_n
	\end{align*}
	d.\ h.\  \( \sum_{n=0}^{\infty}a_n \) hat \( \sum_{n=0}^{\infty}M q^n \) (geom. Reihe, sie konvergiert, da \( 0<q<1 \)) als Majorante.
\end{bew}
\begin{bem}
	\( a_n = M q^n \)
	\[ \Rightarrow \left| \frac{a_{n+1}}{a_n} \right| = \frac{M q^{n+1}}{M q^n} = q \]
	\[ a_n = \frac{1}{n^p} \quad \frac{a_{n+1}}{a_n} = \frac{1}{{(n+1)}^p} \frac{n^p}{1} = {\left(\frac{n}{n+1}\right)}^p = {\left(1 - \frac{1}{n+1}\right)}^p \rightarrow 1, n\rightarrow\infty \]
	\[ \Rightarrow \limsup\limits{n\rightarrow\infty} \frac{a_{n+1}}{a_n} = \limes{n} \frac{a_{n+1}}{a_n} = 1. \]
\end{bem}
\begin{satz}[Wurzelkriterium]
	\( \sum_{n}a_n \) Reihe mit \[ (**) \limsup\limits_{n\rightarrow\infty} |a_n|^{\frac{1}{n}} = \limsup\limits_{n\rightarrow\infty} \sqrt[n]{|a_n|} < 1 \Rightarrow \sum_n a_n \text{ konv.\ abs.}.\]
	Ist \( \limsup\limits_{n\rightarrow\infty} |a_n|^{\frac{1}{n}} > 1 \), so ist die Reihe divergent. (ohne Bew.) 
\end{satz}
\begin{bem}
	Bei \( \limsup\limits_{n\rightarrow\infty} |a_n|^{\frac{1}{n}} = 1 \) ist keine Aussage möglich.
\end{bem}
\begin{bem}
	\( \limsup\limits_{n\rightarrow\infty} |a_n|^{\frac{1}{n}} \leq \limsup\limits_{n\rightarrow\infty} \frac{|a_{n+1}|}{|a_n|}, a_n \neq 0, \text{ für fast alle } n \). (H. A.)
\end{bem}
\begin{bew}
	\( (**) \Rightarrow \exists 0<q<1. K\in\N \) mit \[ |a_n|^{\frac{1}{n}} \leq q \quad\forall n\geq K. \]
	\[\left(\tilde{q} = \limessup{n} |a_n|^{\frac{1}{n}} = \limes{n} \underbrace{\sup |a_k|^{\frac{1}{n}}}_{\rightarrow \tilde{q} < 1 (n \rightarrow \infty)}< 1\right)\] \\
	\(\forall \varepsilon > 0 \exists k \in \N: \underset{k \geq n}{\sup} |a_k|^{\frac{1}{k}} < \tilde{q} + \varepsilon \quad \forall n \geq K \) \\
	d.h.\\
	\[\underset{k \geq n}{\sup} |a_k|^{\frac{1}{k}} < \tilde{q} + \varepsilon = q \; (< 1)\]
	\(\tilde{q} < 1\) wähle \(s > 0: q = \tilde{q} + s < 1\) \\
	\[\Rightarrow |a_n| \leq q^n \quad \forall n \geq K \]
	Damit haben wir konv. Majorante \( \sum_n q^n \).
\end{bew}
\begin{satz}[\glqq{}Mutter aller Konvergenzkriterien\grqq{}]
	Sei \( \sum_n a_n \) Reihe mit \(a_n \neq 0 \) für fast alle \(n\). Dann gilt:
	\[ \sum_n a_n \text{ konv.\ abs.\ } \Leftrightarrow \exists c_n > 0, \sum_n c_n < \infty \text{ und } \frac{|a_{n+1}|}{|a_n|} \leq \frac{c_{n+1}}{c_n} \text{ für fast alle }n. \]
\end{satz}
\begin{bew}Übung.
	Scharfes Hinschauen auf Beweis des Quotientenkriteriums.
\end{bew}
\begin{kor}
	\( \sum_n a_n \) und \( p>1: \frac{|a_{n+1}|}{|a_n|} \leq 1 - \frac{p}{n+1} \) für fast alle \(n \Rightarrow \sum_n a_n \) konv.\ abs.\ (Übung).
\end{kor}
\begin{bem}
	Ist \( \liminf\limits_{n\rightarrow\infty} |a_n|^{\frac{1}{n}} > 1 \Rightarrow a_n \) divergent.\\
	Ist \( \liminf\limits_{n\rightarrow\infty} \frac{|a_{n+1}|}{|a_n|} > 1 \Rightarrow \sum_n a_n \) divergent.\\
	Ist \( \liminf\limits_{n\rightarrow\infty} |a_n|^{\frac{1}{n}} = \limes{n}(\underset{k\geq n}{\inf} |a_n|^{\frac{1}{k}}) > 1 \Rightarrow \exists q>1: k \in\N \).
	\[ |a_n|^{1/n} \geq \underset{k\geq n}{\inf} |a_k|^{1/k} \geq q > 1 \quad\forall n\geq K. \]
	\[ \Rightarrow |a_n| > q^n \rightarrow \infty, n\rightarrow\infty \Rightarrow a_n \text{ keine Nullfolge } \Rightarrow \sum_n a_n \text{ divergent}. \]
\end{bem}
\begin{bsp}[Exponentialreihe]
	\[ x\in\R,\qquad \sum_{n=0}^{\infty} \underbrace{\frac{x^n}{n!}}_{=a_n} =: \exp x. \]
	\begin{align*}
		|\frac{a_{n+1}}{a_n}| = |\frac{x^{n+1}}{(n+1)!} \frac{n!}{x^n}| = \frac{|x|}{n+1} \rightarrow 0, n\rightarrow\infty \\
		\Rightarrow \frac{|a_{n+1}}{a_n} \leq 1/2 \text{ für fast alle } n.\\
		\Rightarrow \sum_{n=0}^{\infty} \frac{x^n}{n!} = 1 + x + 1/2 x^2 + 1/6 x^3 + \dots \text{ konv.\ absolut.}
	\end{align*}
\end{bsp}
\begin{bem}
	\begin{align*}
		S_n(x):= \sum_{j=1}^{n} \frac{x^j}{j!} = 1 + x + 1/2x^2 + \cdots + \frac{x^n}{n!}\\
		S'_n(x) = 1 + x + 1/2x^2 + \cdots + \frac{x^{n-1}}{(n-1)!} = S_{n-1}(x)\\
		\Rightarrow S'_n(x) = S_{n-1}(x) = S_n(x) - \frac{x^n}{n!}
	\end{align*}
	Falls gilt: \( S'(x) = ( \limes{n}S_n(x) )' = \limes{n}(S'_n(x)) = \limes{n}S_{n-1}(x) = S(x) \)
	\( \Rightarrow \exp'x = \exp x? \) (im Allg.\ falsch, aber für Potenzreihen wahr)
\end{bem}
\end{document}