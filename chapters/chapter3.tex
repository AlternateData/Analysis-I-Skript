\documentclass[../ana1.tex]{subfiles}
\begin{document}
\setcounter{section}{2}

\section{Die reellen Zahlen}

\subsection{Körperaxiome}
\begin{prosa}
	In diesem Kapitel sei immer \(\K  \) eine Menge mit zwei Operationen \gqq{\(+  \)} \, und 
	\gqq{\(\cdot \)}, \, für die gilt:
	\[a + b \in \K \ko  \; a \cdot b \in \K \quad \forall \, a \ko b \in \K  \]
\end{prosa}

\begin{defi}[Körperaxiome]
	\((\K \ko + \ko \cdot) \) heißt \underline{Körper}, wenn folgende Axiome erfüllt sind:
	\begin{enumerate}[label= (K\arabic*)]
		\item\label{ax:K1}Kommutativgesetze: \\
			  \(\forall \, a \ko b \in \K \colon \; a+b=b+a \ko  \; a \cdot b = b \cdot a \).
		\item\label{ax:K2}Assoziativgesetze: \\
			  \(\forall \, a \ko b \ko c \in \K \colon \; a+(b+c) = (a+b)+c \ko  \; a \cdot (b \cdot c) = (a \cdot b)\cdot c \).
		\item\label{ax:K3}Distributivgesetz: \\
			  \(\forall \, a \ko b \ko c \in \K \colon \; a \cdot (b+c) = a \cdot b + a \cdot c \)
		\item\label{ax:K4}Existenz von neutralen Elementen: \\
			  \(\exists \, 0 \in \K \colon \; a + 0 = 0 + a = a \quad \forall \, a \in \K. \\
				\exists \, 1 \in \K \colon \; a \cdot 1 = 1 \cdot a = a \quad \forall \, a \in \K \). \\
			  Wobei \(0 \neq 1 \) gilt. 
		\item\label{ax:K5}Existenz der inversen Elemente: \\
		      \(\forall \, a \in \K \, \exists  \, \minus a \in \K \colon \; a + (\minus a) = 0. \\
			    \forall \, a \in \K \setminus \set{0} \, \exists \, a^{\minus 1} \in \K \colon \; a \cdot a^{\minus 1} = 1 \). \\
	\end{enumerate}
\end{defi}

%23.10.2018
\begin{bspe}\leavevmode
	\begin{enumerate}[(1)]
		\item \(\Q \coloneqq \set{\frac{m}{n}  \; \vert  \; m \in \Z \ko  \; n \in \N} \) ist ein Körper.
		\item \(\F_{2} \coloneqq \set{0 \ko 1} \) wird mit folgenden Rechenregeln zu einem Körper:
			  \begin{center}
				\begin{tabular}{c|cc}
					\(+ \) & \(0 \) & \(1 \) \\
					\midrule
					\(0 \) & \(0 \) & \(1 \) \\
					\(1 \) & \(1 \) & \(0 \)
				\end{tabular}
				\qquad
				\begin{tabular}{c|cc}
					\(\cdot \) & \(0 \) & \(1 \) \\
					\midrule
					\(0 \)     & \(0 \) & \(0 \) \\
					\(1 \)     & \(0 \) & \(1 \)
				\end{tabular}
			   \end{center}
	\end{enumerate}
\end{bspe}

\iftoggle{short}{}{\newpage}%Formatierung ausführliches Skript

\begin{bem}\leavevmode
	\begin{enumerate}[(a)]
		\item \((\K \ko +) \) und \((\K \ko \cdot) \) sind kommutative Gruppen.
		\item Die neutralen Elemente \(0 \) und \(1 \) sind eindeutig bestimmt.
	\end{enumerate}
\end{bem}
\begin{bew}
	\begin{enumerate}[(b)]
		\item Seien \(0_{1} \) und \(0_{2} \) neutrale Elemente von \(\K \). \\
			  \(\implies 0_{1} \overset{ \hyperref[ax:K4]{(K4)} }{=} 0_{1} + 0_{2} \overset{ \hyperref[ax:K1]{(K1)} }{=} 0_{2} + 0_{1} \overset{ \hyperref[ax:K4]{(K4)} }{=} 0_{2} \). \\
			  (Der Beweis geht analog für \(1_{1} \ko 1_{2} \) und \gqq{\(\cdot \)})\qedhere
	\end{enumerate}
\end{bew}

\begin{defi}
	Seien \(a \ko b \in \K \).
	\begin{enumerate}[(a)]
		\item \(a - b \coloneqq a + (\minus b) \).
		\item \(\frac{a}{b} \coloneqq a \cdot b^{\minus 1} \) für \(b \neq 0 \).
		\item \(ab \coloneqq a \cdot b \).
	\end{enumerate}
\end{defi}

\begin{lem}[Rechnen in einem Körper]\label{satz:krechenregeln} Seien \(a \ko b \ko c \in \K \).
	\begin{enumerate}[(a)]
		\item Umformen von Gleichungen:
			\begin{enumerate}[(i)]
				\item \(a + b = c \implies a = b - c \).
				\item \(a \cdot b = c \) und \(b \neq 0 \implies a = \frac{c}{b} \).
			\end{enumerate}
		\item Allgemeine Rechenregeln:
			\begin{alignenum}{a (b-c)}{(i)}
				\begin{aitem}
					\Format{\minus (\minus a)} & = a \\
				\end{aitem}\begin{aitem}
					\Format{(a^{\minus 1})^{\minus 1}} & = a \text{, falls } a \neq 0 \\
				\end{aitem}\begin{aitem}
					\Format{\minus (a+b)} & = (\minus a) + (\minus b) \\
				\end{aitem}\begin{aitem}
					\Format{(a\cdot b)^{\minus 1}} & = b^{\minus 1} \cdot a^{\minus 1}=a^{\minus 1} \cdot b^{\minus 1} \\
				\end{aitem}\begin{aitem}
					\Format{a \cdot 0} & = 0 \\
				\end{aitem}\begin{aitem}
					\Format{a (\minus b)} & = \minus (ab) \ko (\minus a) (\minus b) = ab \\
				\end{aitem}\begin{aitem}
					\Format{a (b - c)} & = ab - ac \\
				\end{aitem}\begin{aitem}
					\Format{ab} & = 0 \iff a = 0 \vee b = 0 \text{ (Nullteilerfreiheit)} \\
				\end{aitem}
			\end{alignenum}
	\end{enumerate}
\end{lem}
\begin{bew}\leavevmode
	\begin{enumerate}[(a)]
		\item \begin{enumerate}
				\item[(i)] \(0 = a + (\minus a) = (\minus a) + a \implies \minus(\minus a) = a \).
				\item[(iii)] \((a+b) + ((\minus a)+(\minus b)) = (a + (\minus a)) + (b + (\minus b)) = 0 + 0 = 0 \\
							 \implies \minus(a+b) = (\minus a) + (\minus b) \).
				\item[(ii, iv)] Analog zu (i) und (iii).
				\item[(v)] \(a\cdot 0 = a \cdot (0+0) =a \cdot 0 + a \cdot 0 = a \cdot 0 + 0 \\
						   \implies a \cdot 0 = a \cdot 0 - a \cdot 0 = 0 \).
				\item[(vi)] \(a \cdot b + a \cdot (\minus b) = a \cdot (b + (\minus b)) = a \cdot 0= 0 \\
							\overunderset{\text{Eind.}}{\text{d.\ Inv.}}{\implies} \minus ab = a(\minus b) \). \\
							Somit auch \((\minus a)(\minus b) = \minus((\minus a)b) = \minus(b(\minus a)) = (\minus ba) = \minus(\minus ab) = ab \).
				\item[(vii)] \(a(b - c) = a(b + (\minus c)) = ab + a(\minus c)= ab + (\minus ac)= ab - ac \).
				\item[(viii)] Ist \(ab = 0 \) und \(a \neq 0 \implies 0 = (ab)\frac{1}{a} = \frac{1}{a} \cdot (ab) = (\frac{1}{a} \cdot a)b = 1b = b \).
							  Also ist \(b = 0 \).\qedhere
			  \end{enumerate}
	\end{enumerate}
\end{bew}

\begin{satz}[Bruchrechnen]\label{satz:bruchrechnen}
	Seien \(a \ko b \ko c \ko d \in \K \) mit \(c\neq 0 \ko d\neq 0 \). Dann gilt
	\begin{enumerate}[(a)]
		\item \(\frac{a}{c} + \frac{b}{d} = \frac{ad + bc}{cd} \).
		\item \(\frac{a}{c} \cdot \frac{b}{d} = \frac{ab}{cd} \).
		\item \(\frac{\nicefrac{a}{c}}{\nicefrac{b}{d}} = \frac{ad}{bc} \), falls auch \(b\neq 0 \) ist.
	\end{enumerate}
\end{satz}
\begin{bew}
	Übung.\phantom{\qedhere}
\end{bew}


\subsection{Die Anordnungsaxiome}
\setcounter{satz}{-1} %%Bedingt durch Nummerierung in der Vorlesung

\begin{defi}
	Sei \((\K \ko + \ko \cdot) \) ein Körper.
	Dann heißt \(> \) eine \underline{Anordnung} falls folgende Axiome erfüllt sind:
	\begin{enumerate}[label= (A\arabic*)]
		\item\label{ax:A1}Für jedes \(a \in \mathbb{K} \) gilt genau eine der Aussagen \(a > 0 \ko a = 0 \ko \minus a > 0 \). \\
		      Ist \(a \in \K \) mit \(a > 0 \), so heißt a positiv.
		\item\label{ax:A2}Aus \(a>0 \) und \(b>0 \) folgt \(a + b > 0 \) und \(a \cdot b > 0 \).
	\end{enumerate}
	Wir nennen \((\K \ko + \ko \cdot \ko >) \) einen \underline{angeordneten Körper}.
\end{defi}

\begin{notation}\leavevmode
	\begin{enumerate}[(a)]
		\item \(a < 0 \longeq \minus a > 0 \).
		\item \(a > b \longeq a - b > 0 \). \qquad \qquad \quad \raisebox{-0.8em}{\subgraphic{0.4}{img02.pdf}}
		\item \(a < b \longeq a - b < 0 \).
		\item \(a \geq b \longeq a > b  \; \vee  \;  a = b \).
		\item \(a \leq b \longeq a < b  \; \vee  \;  a = b \).
	\end{enumerate}
\end{notation}

\begin{satz}\label{satz:arechenregeln}
	Sei \((\K \ko + \ko \cdot \ko >) \) ein angeordneter Körper. Dann gilt
	\begin{enumerate}[(a)]
		\item \(a \ko b \in \K \) erfüllen genau eine der Relationen \(a > b \ko  \; a = b \ko  \; a < b \). \\
			  (Trichotomie)
		\item aus \(a > b \ko  \; b > c \) folgt \(a > c \). (Transitivität)
		\item aus \(a > b \) folgt
		      \(\begin{lrcases}
					&a + c > b + c \ko & \forall \, c \in \K \\
				    &ac > bc \ko 	   & c > 0 \\
				    &ac < bc \ko 	   & c < 0
			    \end{lrcases} \).
		\item aus \(a > b \) und \(c>d \) folgt
		      \(\begin{lrcases}
				      &a + c > b + d  \\
					  &ac > bd \ko \text{ falls } c \ko d > 0
			      \end{lrcases} \).
		\item für \(a\neq 0 \) ist \(a^{2} > 0 \).
		\item aus \(a > 0 \) folgt \(\frac{1}{a} > 0 \).
		\item aus \(a > b > 0 \) folgt \(0 < \frac{1}{a} < \frac{1}{b} \).
		\item aus \(a > b \ko  \; 0 < \lambda < 1 \) folgt \(b < \lambda b + (1 - \lambda)a < a \).
		\item \(a \geq b \) und \(a \leq b \iff a = b \).
	\end{enumerate}
\end{satz}
\begin{bew}\leavevmode
	\begin{enumerate}[(a)]
		\item Folgt direkt aus (A1) und der Definition von \(a > b \).
		\item \(a - c = \underbrace{(a - b)}_{> 0} + \underbrace{(b - c)}_{> 0} \overset{\text{(A2)}}{>} 0 \).
		\item Im allgemeinen gilt \((a + c) - (b + c) = a - b > 0 \). \\
			  Fall \gqq{\(c > 0 \)}: \(ac - bc = \underbrace{(a - b)}_{> 0} \cdot c \overset{\text{(A2)}}{>}0 \). \\
			  Fall \gqq{\(c < 0 \)}: Es gilt \(\minus c > 0 \implies bc - ac = \underbrace{(a - b)}_{> 0} \cdot \underbrace{(\minus c)}_{> 0} \overset{\text{(A2)}}{>} 0 \).
		\item \((a+c)-(b+d) = (a-b)+(c-d)>0 \) nach (A2). \\
		      \(ac-bd=ac-bc+bc-bd=\underbrace{(a-b)}_{>0} \cdot \underbrace{\phantom{(}c\phantom{)}}_{>0} + \underbrace{\phantom{(}b\phantom{)}}_{>0} \cdot \underbrace{(c-d)}_{>0} \overset{\text{(A.2)}}{>}0 \).
		\item Fall \gqq{\(a > 0 \)}: \(a^2 = a\cdot a > 0 \) (A2). \\
			  Fall \gqq{\(a < 0 \)}: \(a^2 = (-a)\cdot(-a) > 0 \) (A2).
		\item \(\frac{1}{a} > 0 \iff \frac{1}{a} = \underbrace{{\left(\frac{1}{a}\right)}^2}_{>0} \cdot \underbrace{a}_{>0} > 0 \).
		\item \(\frac{1}{b} - \frac{1}{a} = \frac{1}{b}(a - b)\frac{1}{a} > 0 \).
		\item \(a > b\ko  \; 0 > \lambda > 1 \implies \lambda > 0 \wedge 1 - \lambda > 0 \\
			  \implies b = \lambda b + \underbrace{(1 - \lambda)b}_{< (1 - \lambda)a} < \lambda b + (1 - \lambda) a < \lambda a + (1 - \lambda)a = a \\
			  \implies b < \lambda b + (1 - \lambda)a = a \). \\
			  Insbesondere gilt mit \(\lambda = \frac{1}{2} \Rightarrow b < \frac{1}{2} b + \frac{1}{2}a = \frac{a + b}{2} < a \).
		\item Leichte Übung.\qedhere
	\end{enumerate}
\end{bew}

\begin{bem}
	Auf \(\F_2 \) existiert keine Anordnung.
\end{bem}

\begin{defi}[Betrag]
	Sei \((\K \ko + \ko \cdot \ko >) \) ein angeordneter Körper. Der \underline{Betrag} von 
	\(a \in \K \) ist gegeben durch
	\[\abs{a} \coloneqq
		\begin{lrcases}
			&a \ko  &\text{ falls } a \geq 0  \\
			&\minus a \ko &\text{ falls } a < 0
		\end{lrcases}. \]
	Für \(a \ko b \in \K \) definieren
	\[\max(a \ko b) \coloneqq
		\begin{cases}
			a \ko &\text{ falls } a \geq b  \\
			b \ko &\text{ falls } a < b
		\end{cases}
		\qquad
	  \min(a \ko b) \coloneqq
		\begin{cases}
			a \ko &\text{ falls } a \leq b  \\
			b \ko &\text{ falls } a > b
		\end{cases} \]
	das \underline{Maximum} und das \underline{Minimum} von \(a \) und \(b \).
\end{defi}

\begin{bem}\label{bem:abstand}
	Seien \(a \ko b \in \K \)
	\begin{enumerate}[(a)]
		\item \(\abs{a - b} \) ist der Abstand von \(a \) zu \(b \). \\
		      \(\abs{a} = \abs{a - 0} \) ist der Abstand von \(a \) zu \(0 \).
		\item \(\abs{a} = \max(a \ko \minus a) \).
	\end{enumerate}
\end{bem}

\iftoggle{short}{}{\newpage}%Formatierung ausführliches Skript

\begin{satz}\label{satz:bgleichungen}
	Sei \((\K \ko + \ko \cdot \ko >) \) ein angeordneter Körper und seien \(a \ko b \in \K \). Dann gilt
	\begin{enumerate}[(a)]
		\item \(\abs{\minus a} = \abs{a} \) und \(a \leq \abs{a} \).
		\item \(\abs{a} \geq 0 \) und \(\abs{a} = 0 \iff a = 0 \).
		\item \(\abs{ab} = \abs{a}\abs{b} \).
		\item \(\abs{a + b} \leq \abs{a} + \abs{b} \). (Dreiecksungleichung)
		\item \(\abs{\abs{a} - \abs{b}} \leq \abs{a - b} \). (umgekehrte Dreiecksungleichung)
	\end{enumerate}
\end{satz}
\begin{bew}\leavevmode
	\begin{enumerate}
		\item[(a)] \(\abs{\minus a} =
			       \begin{lrcases}
				      &\minus a \ko & \minus a \geq 0  \\
				      &\minus(\minus a) \ko & \minus a \leq 0
			       \end{lrcases}
			       = \begin{lrcases}
				      &\minus a & \ko a \leq 0  \\
				      &a \ko  & a \geq 0
			       \end{lrcases}
			       = \abs{a} \\
				   \abs{a} - a = 
				   \begin{lrcases}
				      &a - a \ko & a \geq 0  \\
				      &\minus a - a \ko & a < 0
			       \end{lrcases} =
			       \begin{lrcases}
				      &0 \ko & a \geq 0  \\
				      &\minus(a + a) \ko & a < 0
			       \end{lrcases} \geq 0 \). \\
		      	   Alternativ: \(a \leq \max(a \ko \minus a) = \abs{a} \). %Bis hierher formatiert
		\item[(c)] o.\ B.\ d.\ A.\  \(a \ko b \geq 0 \), da \(\minus a \cdot \minus b = a \cdot b \).
		      	   \(\implies \abs{ab} = ab = \abs{a}\abs{b} \).
		\item[(d)] Nach \autoref{satz:arechenregeln} gilt \( \abs{a + b}^{2} = {(a+b)}^{2} = a^{2} + 2ab + b^{2} \\
				   = \abs{a}^{2} + 2ab + \abs{b}^{2} \leq \abs{a}^{2} 2\abs{ab} + \abs{b}^{2} \). \\
				   \(\implies {\abs{a + b}}^{2} \leq {(\abs{a} + \abs{b})}^{2} \)
			       \(\overset{\text{Übung}}{\Rightarrow} \abs{a + b} \leq \abs{a} + \abs{b} \). \\
		     	   Übung: Aus \(\abs{c}^{2} \leq \abs{d}^{2} \) folgt \(\abs{c} \leq \abs{d} \) (Kontraposition).
		\item[(e)] \(\abs{a} = \abs{a - b + b} = \abs{(a - b) + b} \overset{\text{(d)}}{\leq} \abs{a - b} + \abs{b} \\
			       \implies \abs{a} - \abs{b} \leq \abs{a - b} \). \\
				   Durch vertauschen von \(a \) und \(b \) folgt \\
				   \(\abs{b} - \abs{a} \leq \abs{b - a} = \abs{\minus b - a} = \abs{a - b} \). \\
				   Also \(\abs{b} - \abs{a} \leq \abs{a - b} \) und \(\abs{a} - \abs{b} \leq \abs{a - b}. \\
				   \begin{aligned}
					\implies &\abs{\abs{a} - \abs{b}} = \max(\abs{a} - \abs{b} \ko  \minus(\abs{a} - \abs{b})) \\
					         &= \max(\abs{a} - \abs{b} \ko \abs{b} - \abs{a}) \leq \abs{a - b}.
				   \end{aligned} \)
	\end{enumerate}
\end{bew}

\begin{bsp}
	Seien \(a \ko b \in \K \) und \(\K \) ein angeordneter Körper. \\
	Aus \(\abs{b - a} \leq \frac{b}{2} \) folgt \(a \geq \frac{b}{2} \).
	%%Bild: %BILD EINFUEGEN \\
\end{bsp}
\begin{bew}
	\(b-a \leq \abs{b-a} \leq \frac{b}{2} \implies a \geq b - \frac{b}{2} = \frac{b}{2} \).
\end{bew}

\begin{kor}[Geometrisch-Arithmetische Ungleichung]\label{kor:gaungl}\leavevmode \\
	Seien \((\K \ko + \ko \cdot \ko >) \) ein angeordneter Körper und \(a \ko b \in \K \). Dann gilt \\
		\[ab \leq {\left( \frac{a + b}{2}\right)}^{2}. \] \\
	Wenn Gleichheit gilt, so folgt \(a=b \).
\end{kor}
\begin{bew}
	Übung.\phantom{\qedhere}
\end{bew}

\begin{bem}
	In jedem angeordneten Körper gilt \(0 < 1 \).
\end{bem}
\begin{bew}
	Übung.\phantom{\qedhere}
\end{bew}


\subsection{Obere/Untere Schranken, Supremum/Infimum}
\begin{prosa}
	Im folgenden sei \(\K \) immer ein angeordneter Körper. Weiter seien \(A \ko B \subset \K \) Teilmengen mit \(A \ko B \neq \emptyset \) und
	\(\gamma \in \K \).
\end{prosa}

\begin{notation}
	Sei \(a \in \K \). Dann heißt \(a \) \underline{nicht negativ}, falls \(a \geq 0 \) ist.
\end{notation}

\begin{defi*}\leavevmode
	\begin{enumerate}[(a)]
		\item \(\gamma \) heißt \(\begin{lrcases}
									\text{obere} \\
									\text{untere}
								 \end{lrcases} \) \underline{Schranke} von \(\begin{lrcases}A \\B\end{lrcases} \)
			  \(\longeq \begin{lrcases}
							\forall \, a \in A \colon \; a \leq \gamma \\
							\forall \, b \in B \colon \; b \geq \gamma  
						\end{lrcases} \). \\
			  Schreibe auch \(\begin{lrcases}
								 A \leq \gamma \\ 
								 B \geq \gamma 
							  \end{lrcases} \)
			  und definiere \(A < \gamma \ko  \; B > \gamma \) analog. \\
			  Besitzt \(A \) eine \(\begin{lrcases}
									  \text{obere} \\
									  \text{untere}
								   \end{lrcases} \) Schranke,
			  so heißt \(A \) nach \(\begin{lrcases}
									  \text{oben} \\
									  \text{unten}
								   \end{lrcases} \) beschränkt. \\
			  A heißt \underline{beschränkt} \(\longeq \) A ist nach oben und unten beschränkt.
		\item Ist \(\begin{lrcases}
						A \leq \alpha \text{ mit } \alpha \in A \\
						B \geq \beta \text{ mit } \beta \in B
					\end{lrcases} \) so definiere
			  \(\begin{lrcases}
					\alpha \coloneqq \max(A) \\
					\beta \coloneqq \min(B)
				\end{lrcases} \).
	\end{enumerate}
\end{defi*}

\begin{bem}
	\(\max(A) \) und \(\min(A) \) sind eindeutig, falls sie existieren.
\end{bem}
\begin{bew}
	Übung.\phantom{\qedhere}
\end{bew}

\begin{bsp}
	\([0\ko1) \coloneqq \set{x  \; \vert  \; 0 \leq x < 1} \) besitzt kein Maximum, aber ein Minimum, nämlich \(\min[0\ko1) = 0 \).
\end{bsp}

\begin{defi}\leavevmode \\
	\(\gamma \) ist die \(\begin{lrcases}
							\text{kleinste obere Schranke oder \underline{Supremum} von } A \\
							\text{größte untere Schranke oder \underline{Infimum} von } B
						\end{lrcases} \) \\
	\(\longeq 
	\begin{lrcases}
		A \leq \gamma \text{ und aus } A \leq \tilde{\gamma} \text{ folgt } \gamma \leq \tilde{\gamma} \\
		B \geq \gamma \text{ und asu } B \geq \tilde{\gamma} \text{ folgt } \tilde{\gamma} \leq \gamma	 
	\end{lrcases} \). \\
	Schreibe \(\begin{lrcases}
				   \gamma &\eqqcolon \sup A &\eqqcolon \sup(A) \\
				   \gamma &\eqqcolon \inf B &\eqqcolon \inf(B)
			   \end{lrcases} \).
\end{defi}

\begin{bsp}
	Für \(P := \set{x \in \K  \; \vert  \; x > 0 } \) gilt
	\begin{enumerate}[(i)]
		\item \(P \) ist nicht nach oben beschränkt.
		\item \(P \) hat kein Minimum, aber \(\inf P = 0 \).
	\end{enumerate}
\end{bsp}

\begin{bew}\leavevmode
	\begin{enumerate}[(i)]
		\item Angenommen \(\gamma \) ist obere Schranke für \(P \). Also \(\forall \, x \in P \) folgt \( 0 < x \leq \gamma \) \\
			  \(\implies \gamma > 0 \implies \gamma \in P \implies 0 < \gamma = \gamma + 0 < \gamma + 1 \in P \) \\
			  \(\implies \gamma + 1 \in P \) und \(\gamma + 1 > \gamma \). \\
			  Also ist \(\gamma \) keine obere Schranke für \(P \) \Lightning.
		\item Angenommen \(\min P \coloneqq \eta \) existiert. Dann folgt \( \eta \in P \ko  \; \eta > 0 \) und
		      \(0 < \tilde{x} \coloneqq \frac{\eta}{2} < \eta \) ist ein weiteres Minimum \Lightning. Also kann kein Minimum existieren. \\
			  Offensichtlich gilt \(0 < P \), also ist \(0 \) eine untere Schranke für \(P \). \\
			  \(0 \) ist außerdem die größte untere Schranke, denn nach obigem Argument ist jede Zahl \(\eta > 0 \) keine untere Schranke für \(P \). \\
			  Also ist \(\inf P = 0 \). \qedhere
	\end{enumerate}
\end{bew}

\begin{lem}\label{satz:epskrit_inf_sup}\leavevmode
		\begin{alignenum}{\alpha = \sup A}{(a)}
			\begin{aitem}
				\Format{\alpha = \sup A} &\iff \alpha \geq A  \, \wedge  \, \forall \, \varepsilon > 0  \, \exists a \in A \colon \; \alpha - \varepsilon < a.
			\end{aitem}\begin{aitem}
				\Format{\beta  = \inf B} &\iff \beta  \leq B  \, \wedge  \, \forall \, \varepsilon > 0  \, \exists b \in B \colon \; b < \beta + \varepsilon.
			\end{aitem}
		\end{alignenum}
\end{lem}
\begin{bew}\leavevmode
	\begin{enumerate}[(a)]
		\item \equirl{
				Sei \(\alpha = \sup A \) und \(\varepsilon > 0 \). \(\implies \alpha - \varepsilon \) ist keine obere Schranke für 
				\(A \implies \exists a \in A \colon \; x > \alpha - \epsilon \).
			  }{
				Es gelte \(\alpha \geq A  \, \wedge  \, \forall \, \varepsilon > 0  \, \exists a \in A \colon \; \alpha - \varepsilon < a \).
				Also ist \(\alpha \) eine obere Schranke für \(A \). Sei nun \(\tilde{\alpha} < \alpha \). \\
				Setze \(\varepsilon \coloneqq \alpha - \tilde{\alpha} > 0 \). \(\implies \exists a \in A \colon \; \tilde{\alpha} = \alpha - \varepsilon < a \).
				\(\implies \tilde{\alpha} \) ist keine obere Schranke für \(A \). Also ist \(\alpha \) die kleinste obere Schranke.
			  }
		\item \(A \coloneqq \minus B = \set{\minus b  \;\vert  \;b \in B} \). Mit der Identität
			  \(\sup A = \sup(\minus B) = \minus \inf B \) folgt die Behauptung.\qedhere
	\end{enumerate}
\end{bew}


%30.10.2018
\subsection{Das Vollständigkeitsaxiom}

\begin{defi}
	Ein angeordneter Körper \((\K \ko + \ko \cdot \ko >) \) erfüllt das Voll\-ständig\-keits\-axiom, falls gilt:
	\begin{enumerate}
		\item[(V)]\label{ax:V}Jede nichtleere, nach oben beschränkte Teilmenge von \(\K \) hat ein Supremum.
	\end{enumerate}
	Solch einen Körper nennt man \underline{ordnungsvollständig}. \(\R \), der Körper der reellen Zahlen, ist \textbf{der} ordnungsvollständige Körper. (Im Wesentlichen gibt es nur einen!)
\end{defi}

%%$\mathbb{Q}; A:= \{r\in\mathbb{Q}|r^2 < 2\}$ Kontext??

\begin{notation}
	Seien \(a \ko b \in \R \) mit \(a \leq b \). Dann heißt
	\providecommand{\Format}{}%
	\renewcommand*{\Format}[1]{\makebox[\widthof{\((\minus \infty \ko a] \)}][r]{\(#1 \)}}
	\begin{enumerate}[ ]
		\item \(\Format{[a \ko b]} \coloneqq \set{x \in \R  \; \vert  \; a \leq x \leq b} \) abgeschlossenes Intervall.
		\item \(\Format{(a \ko b)} \coloneqq \set{x \in \R  \; \vert  \; a < x < b} \) offenes Intervall.
		\item \(\Format{[a \ko b)} \coloneqq \set{x \in \R  \; \vert  \; a < x < b} \) nach rechts halboffenes Intervall.
		\item \(\Format{(a \ko b]} \coloneqq \set{x \in \R  \; \vert  \; a < x < b} \) nach links halboffenes Intervall.
		\item \(\begin{rcases}
					\Format{(\minus \infty \ko a]} \coloneqq \set{x \in \R  \; \vert  \; x \leq a} \\
					\Format{(\minus \infty \ko a)} \coloneqq \set{x \in \R  \; \vert  \; x < a} \\
					\Format{[a \ko \infty)} \coloneqq \set{x \in \R  \; \vert  \; a \leq x} \\
					\Format{(a \ko \infty)} \coloneqq \set{x \in \R  \; \vert  \; a < x}
			    \end{rcases} \) unbeschränktes Intervall.
	\end{enumerate}
	Dabei ist die Intervalllänge in den ersten vier Fällen gegeben durch \(b - a \).
\end{notation}


\subsection{Die natürlichen Zahlen \(\N \)}

\begin{prosa}
	Wir wollen eine definition der natürlichen Zahlen geben. Definitionen wie \(\N \coloneqq \set{1 \ko 2 \ko 3 \ko \ldots} \) sind jedoch
	sehr vage und informell. Ferner wäre eine Definition der folgenden Art ein zirkulärer Schluss:
	\[ n = \underbrace{1 + 1 + \cdots + 1}_{n \text{-mal}} \]
	Stattdessen wollen wir eine Definition geben. welche die Induktiven Eigenschaften der natürlichen Zahlen ausnutzt, die bereits in den vorangegangen
	Kapitel behandelt wurde.
\end{prosa}

\iftoggle{short}{}{\newpage}%Formatierung ausführliches Skript

\begin{defi}
	Eine Teilmenge \(M \subseteq \R \) heißt \underline{induktiv}, falls gilt:
	\begin{enumerate}[label= (I\arabic*)]
		\item\label{ax:I1} \(1 \in M \).
		\item\label{ax:I2} Aus \(n \in M \) folgt \(n + 1 \in M \).
	\end{enumerate}
\end{defi}

\begin{bspe}\leavevmode
	\begin{enumerate}[(1)]
		\item \([1 \ko \infty) \) ist induktiv.
		\item \(\R \) ist induktiv.
		\item \((1 \ko \infty) \) ist nicht induktiv.
		\item \(\set{1} \cup [1+1 \ko \infty) \) ist induktiv.
	\end{enumerate}
\end{bspe}

\begin{bem}
	Ein beliebiger Schnitt induktiver Mengen ist wieder induktiv.
\end{bem}
\begin{bew}
	Sei \(J \) eine beliebige Indexmenge und für alle \(j \in J \) sei \(A_{j} \subseteq \R \) induktiv. \\
	\(\implies \forall \, j \in J \colon \; 1 \in A_{j} \Rightarrow 1 \in \underset{j \in J}{\bigcap}A_{j} \). \\
	\(\implies n \in \underset{j \in J}{\bigcap}A_{j} \Rightarrow \forall \, j \in J \colon \; x + 1 \in A_{j} \Rightarrow x + 1 \in \underset{j \in J}{\bigcap}A_{j} \).
\end{bew}

\begin{defi}[Natürliche Zahlen]
	\[\N \coloneqq \set{x \in \R  \; \vert  \; \forall \, M \subseteq \R \text{ induktiv } \colon \; x \in M} = \smashoperator{\bigcap_{M \subseteq \R \text{ induktiv}}} M \]
	definiert die \underline{natürlichen Zahlen}.
\end{defi}

\begin{bem}
	\(\N \) ist die kleinste induktive Teilmenge von \(\R \).
\end{bem}

\begin{satz}[Archimedisches Prinzip für \(\R \)]\label{satz:arch_prinz}\leavevmode
	\begin{enumerate}[(a)]
		\item \(\N \) ist (in \(\R \)) \textbf{nicht} nach oben beschränkt.
		\item \(\forall \, x \in \R \) mit \(x > 0 \colon \; \exists \, n \in \N \colon \; \frac{1}{n} < x \).
	\end{enumerate}
\end{satz}
\begin{bew}\leavevmode
	\begin{enumerate}[(a)]
		\item Angenommen, \(\N \subseteq \R \) ist nach oben beschränkt durch \(\alpha \). \\
		      Es gilt \(\N \neq \emptyset \) da \(1 \in \N \) nach \autoref{ax:I1}. \\
		      Nach \hyperref[ax:V]{Vollständigkeitsaxiom} gilt \(\alpha = \sup \N \in \R \). \\
		      Setze \(\varepsilon = 1 \) in \autoref{satz:epskrit_inf_sup}, dann gilt \(\alpha - 1 \) ist keine obere Schranke für \(\N \).
		      und \(\exists n \in \N \colon \; n > \alpha - 1 \).
			  \(\implies n+1 > \alpha \in \N \) \Lightning.
		\item Sei \(x > 0 \overset{\autoref{satz:arechenregeln}{(f)}}{\implies} \frac{1}{x} > 0 \implies \, \exists n \in \N \colon \; n > \frac{1}{x} \\
		      \underset{\autoref{satz:arechenregeln}{(g)}}{\implies} x = \frac{1}{\frac{1}{x}} > \frac{1}{n} \).\qedhere
	\end{enumerate}
\end{bew}

\begin{satz}[Induktionsprinzip]\label{satz:ind_prinz}
	Sei \(M \subseteq \N \) mit
	\begin{enumerate}[(i)]
		\item \(1\in M \).
		\item \(x \in M \implies x + 1 \in M \).
	\end{enumerate}
	Dann ist \(M = \N \).
\end{satz}
\begin{bew}
	\(M \) ist induktiv. Da \(\N \) die kleinste induktive Teilmenge von \(\R \) ist gilt \(\N \subseteq M \).
	Da \(M \subseteq \N \) folgt \(M = \N \).
\end{bew}

\begin{kor}[Vollständige Induktion]\label{satz:vollst_ind}\leavevmode \\
	Für \(n \in \N \) seien \(A(n) \) Aussagen. Es gelte:
	\begin{enumerate}[(i)]
		\item \(A(1) \) ist wahr.
		\item Aus \(A(n) \) ist wahr folgt \(A(n+1) \) ist wahr.
	\end{enumerate}
	Dann sind alle \(A(n) \) wahr.
\end{kor}
\begin{bew}
	Definiere \(M \coloneqq \set{n \in \N  \; \vert  \; A(n) \text{ ist wahr}} \subset \N \). Dann gilt:
	\begin{enumerate}[(i)]
		\item \(1\in M \), da \(A(1) \) wahr ist.
		\item Für \(n\in M \) gilt \(n + 1 \in M \), da wenn \(A(n) \) wahr ist auch \(A(n + 1) \) wahr ist.
	\end{enumerate}
	\(\overunderset{\hyperref[satz:ind_prinz]{\text{Induktions-}}}{\hyperref[satz:ind_prinz]{\text{prinzip}}}{\implies} M = \N \), also sind alle \(A(n) \) wahr.
\end{bew}

\begin{defi*}[Summen]Seien \(n \ko m \in \N \) und für alle \(k \in \N \) sein \(a_{k} \in \R \).
	\[\sum_{k = 1}^{1} a_{k} \coloneqq a_{1} \ko \quad \sum_{k = 1}^{n + 1} a_{k} \coloneqq \left(\sum_{k = 1}^{n} a_{k} \right) + a_{n + 1} \]
	Der Laufindex kann auch verschoben werden:
	\[\sum_{k = m}^{n} a_{k} = \sum_{j = m - 1}^{n - 1} a_{j + 1} = \ldots = \sum_{l = 0}^{n - m} a_{l + m} \]
	Falls \(m > n \) so definiere \(\sum_{k = m}^{n} a_{k} \coloneqq 0 \) als die leere Summe.
\end{defi*}

\begin{defi*}[Produkte]Seien \(n \ko m \in \N \) und für alle \(k \in \N \) sein \(a_{k} \in \R \), sowie \(a \in \R \).
	\[\prod_{k = 1}^{1} a_{k} \coloneqq a_{1} \ko \quad \prod_{k = 1}^{n + 1} \coloneqq \left(\prod_{k = 1}^{n} a_{k} \right) \cdot a_{n + 1} \]
	Falls \(m > n \) so definiere \(\prod_{k = m}^{n} a_{k} \coloneqq 1 \) als das leere Produkt. \\
	Ferner sei \(a^{n} \coloneqq \prod_{k = 1}^{n} a \), also \(a^{0} = 0 \ko  \; a^{1} = 1 \ko  \; a^{n + 1} = a^{n} \cdot a \).
\end{defi*}

\begin{bem}[Rechenregeln für Summen und Produkte]\leavevmode \\
	Seien \(a_{j} \ko b_{j} \ko c \in \R \) für \(j = 1 \ko \ldots \ko n \ko  \; n \in \N \).
	\begin{enumerate}[(a)]
		\item \(\sum_{k = 1}^{n} (a_{k} + b_{k}) = \sum_{k = 1}^{n} a_{k} + \sum_{k = 1}^{n} b_{k}. \)
		\item \(\prod_{k = 1}^{n} (a_{k} \cdot b_{k}) = \prod_{k = 1}^{n} a_{k} \cdot \prod_{k = 1}^{n} b_{k}. \)
		\item \(\sum_{k = 1}^{n} c \cdot a_{k} = c \cdot \sum_{k = 1}^{n} a_{k}. \)
		\item \(\prod_{k = 1}^{n} c \cdot a_{k} = c \cdot \prod_{k = 1}^{n} a_{k}. \)
	\end{enumerate}
\end{bem}

\begin{satz}[Bernoullische Ungleichung]\label{satz:bern_ungl}
	Sei \(x \in \R \ko  \; x \geq \minus 1 \ko  \; n \in \N_{0} \coloneqq \N \cup \set{0} \). Dann gilt
	\[{(1 + x)}^{n} \geq 1 + nx. \]
	Es gilt \gqq{\(> \)}, falls \(n>1 \ko  \; x \neq 0 \).
\end{satz}
\begin{bew} (Durch vollständige Induktion)
	\induktion{
		\((1 + x)^{0} = 1 = 1 + 0x \).
	}{
		Es gelte \({(1 + x)}^{n} \geq 1 + nx \) für ein \(n \in \N_{0} \).
	}{  
		\(\!\begin{aligned}[t]
			 &(1 + x)^{n + 1} = \underbrace{(1 + x)^{n}}_{\geq 1 + nx} \cdot \underbrace{(1 + x)}_{\geq 0} \geq (1 + nx)(1 + x) \\
			 &= 1 + (n + 1)x + {nx}^{2}
			 \begin{cases}
				\geq 1 + (n+1)x \ko & x > \minus 1  \\
				> 1+(n+1)x\ko       & x > \minus 1 \ko  \; x \neq 0
			 \end{cases}.
		  \end{aligned} \)
	}
\end{bew}

\begin{satz}[Geometrische Summe]\label{satz:geom_sum}
	Sei \(x\neq 1 \), dann ist 
	\[\sum_{k = 0}^{n}{x}^{k} = \frac{1 - x^{n + 1}}{1 - x.} \]
\end{satz}

\iftoggle{short}{}{\newpage}%Formatierung ausführliches Skript

\begin{bew} (Durch vollständige Induktion)
	\induktion{
		\(\sum_{k = 0}^{0} x^{k} = x^{0} = 1 = \frac{1 - 0}{1 + 0} \).
	}{
		Für ein \(n \in \N \) gelte \(\sum_{k = 0}^{n} x^k = \frac{1 - x^{n + 1}}{1 - x} \).
	}{
		\(\!\begin{aligned}[t]
			\sum_{k = 0}^{n} x^k + x^{n + 1} \overset{\text{IV}}&{=} \frac{1 - x^{n + 1}}{1 - x} + x^{n + 1}  \\
			&= \frac{1 - x^{n + 1} + (1 - x)x^{n + 1}}{1 - x} = \frac{1 - x^{n + 2}}{1 - x}.
		  \end{aligned} \)
	}
\end{bew}
\begin{bew} (Ohne vollständige Induktion)\leavevmode \\
	\[\begin{aligned}[t]
			&        & S_n         & \coloneqq \sum_{k = 0}^{n} x^k \\
			&\implies& x \cdot S_n & = \sum_{k = 0}^{n} x^k = \sum_{k = 0}^{n} x^{k + 1} = \sum_{j = 1}^{n + 1} x^j \\
			&\implies& (1 - x)S_n  & = S_n - x S_n = \sum_{k = 0}^{n} x^k - \sum_{k = 1}^{n + 1} x^k = x^0 - x^{n + 1} = 1 - x^{n + 1} \\
			&\implies& S_n         & = \frac{1 - x^{n + 1}}{1 - x}
	  \end{aligned} \]
\end{bew}

\begin{satz}[Eigenschaften von \(\N \)]\label{satz:prop_N}
	Es seien \(m \ko n \in \N \). Dann gilt
	\begin{enumerate}[(a)]
		\item \(n + m \in \N \) und \(n \cdot m \in \N \).
		\item \(n = 1 \) oder \(n - 1 \in \N \).
		\item \(m \leq n \implies n - m \in \N_{0} \).
		\item es gibt kein \(k \in \N \) mit \(n < k < n + 1 \).
	\end{enumerate}
\end{satz}

\iftoggle{short}{}{\newpage}%Formatierung ausführliches Skript

\begin{bew}\leavevmode
	\begin{enumerate}[(a)]
		\item Sei \(m \in \N \) und definiere \(A \coloneqq \set{n \in \N  \; \vert  \; n + m \in \N} \subseteq \N \). Es gilt:
			  \begin{enumerate}[(i)]
				\item \(1 \in A \), denn aus \(m \in \N \) folgt \(m + 1 \in \N \).
				\item Ist \(n \in A \), so ist \((n + 1) + m = \underbrace{\in \N}{n + m} + 1 \in \N \).
					  Also auch \(n + 1 \in A \).
			  \end{enumerate}
			  \(\overset{\autoref{satz:ind_prinz}}{\implies} A = \N \).
		\item Definiere \(B \coloneqq \set{n \in \N  \; \vert  \; n = 1  \; \vee  \; n - 1 \in \N} \). Es gilt:
			  \begin{enumerate}[(i)]
				\item \(1 \in B \).
				\item Ist \(n \in B \), so ist \(n = (n + 1) - 1 \in \N \), also \(n + 1 \in B \).
			  \end{enumerate}
			  \(\overset{\autoref{satz:ind_prinz}}{\implies} B = \N \).
		\item Definiere \(C \coloneqq \set{n \in \N  \; \vert  \; \forall \, m \in \N \text{ mit } m \leq n \text{ ist } n - m \in \N_{0}} \). Es gilt:
			  \begin{enumerate}[(i)]
				\item \(1 \in C \), denn ist \(n = 1 \in \N \), so folgt mit (b), dass auch \(m = 1 \) ist und somit
					  \(n - m = 1 - 1 = 0 \in \N_{0} \).
				\item Ist \(n \in C \) und \(m \in \N \) mit \(m \leq n + 1 \) so gilt: \\
					  \case{Fall \(n = 1 \)}{&\(n + 1 - m = (n + 1) - 1 = n \in \N \).} \\
					  \case{Fall \(n > 1 \)}{& \(m - 1 \in \N \) und \(m - 1 \leq (n + 1) - 1 = n \). \\
					  						& Da \(n \in C \) folgt \(n - (m - 1) \in \N_{0} \) \\
					  						& \(\implies n + 1 \in C \).}
			  \end{enumerate}
		\item Übung.\qedhere
	\end{enumerate}
\end{bew}

\end{document}
