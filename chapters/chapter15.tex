\documentclass[../ana1.tex]{subfiles}
\begin{document}
\setcounter{section}{14}
\section{Stetigkeit}
Sei \( D \subset \R^n, f \colon D \rightarrow \R^m, 
x \mapsto f(x) \) eine Funktion.\\
Im Fall \( m=1 \) heißt die Funktion reellwertig. 
Ist \( D = I \subset \R \) ein Intervall, so bezeichnet 
man \( C: I\rightarrow R^m \) als einen Weg oder auch Kurve.\\
z.\ B.\ 
\[ c(t) = (x(t), y(t), z(t)), t=\text{ Zeit} \]

\begin{bspe}\leavevmode
    \begin{enumerate}
        \item horizontaler Wurf aus der Höhe \( h \) mit 
        Geschwindigkeit \(v\) \\
        Bild:\\
        %BILD
        \( c: \left[0, \sqrt{\frac{2 h}{g}} \right] \Rightarrow 
        \R^3, c(t) = (vt, 0, h - \frac{1}{2} g t^2) \)
        \item Abstand zu einer Menge \( E \subset \R^n \):\\
        \( dist_E : \R^n \rightarrow \R, dist_E(x) 
        := \inf \{ \abs{x-y} \;\vert \; y\in E \} \)
        %dist beide Male als Math Operator
        %BILD
    \end{enumerate}
\end{bspe}
\begin{defi}[Stetigkeit]
    Die Funktion \( f: D\rightarrow \R^m \) ist stetig 
    in \( x_0 \in D \), falls
    \[ \forall \, \varepsilon > 0 \,\exists \, \delta > 0 
    \, \colon \, \underbrace{ \abs{ f(x) - f(x_0) 
    }}_{\text{Abstand in }R^m} < \varepsilon \quad 
    \forall x\in D \text{ mit } \underbrace{\abs{x - x_0} 
    }_{\text{Abstand in } R^n} < \delta. \]
    \( f: D\rightarrow \R^m \) heißt stetig, falls \( f \) 
    stetig in allen \(x_0 \in D \) ist.
\end{defi}
\begin{bem}
    Mittels der Bälle 
    \[ B_r^d(y_0) := \{ y\in\R^d \;\colon \; 
    \abs{y-y_0}<r \} \]
    kann man Stetigkeit von \( f \) in \( x_0 \) auch wie folgt
    auffassen.
    \[ \forall \, \varepsilon > 0 \,\exists \, \delta > 0 
    \;\colon \; f(D\cap B_\delta^n(x_0)) \subset 
    B_\varepsilon^m (f(x_0)) \]
\end{bem}
\begin{bspe}\leavevmode
    \begin{enumerate}
        \item Die konstante Funktion \( f: D\rightarrow \R^m, 
        x\mapsto c\in\R^m \) fest.
        Sei \( x_1\in D \)
        \[ \Rightarrow f(x) - f(x_2) = c - c = 0 \]
        \[ \Rightarrow \abs{f(x_1) - f(x_2)} = 0 \quad \forall \, 
        x_2\in D \]
        \(\Rightarrow f\) ist stetig.
        \item \( a\in\R, f:\R \rightarrow \R, x\mapsto ax \)
        \[ \Rightarrow \abs{ f(x) - f(x_0) } = 
        \abs{ ax - ax_0 } = \abs{a}\abs{x-x_0} \]
        \[ \Rightarrow \text{Wähle }\delta = 
        \frac{\varepsilon}{1+\abs{a}} 
        (a=0 \text{ ist erlaubt}) \]
        \[ \Rightarrow \abs{f(x) - f(x_0)} = \abs{a}
        \abs{x-x_0} < \abs{a} \frac{\varepsilon}{1+\abs{a}}
        < \varepsilon \quad \forall \abs{x-x_0} \leq \delta. \]
        Somit ist \(f\) stetig in \(x_0\).
        \item \( f:\R\rightarrow\R, x\mapsto x^2 \)
        ist stetig, denn
        \begin{align*}
            \abs{ f(x) - f(x_0) } &= \abs{x^2 - x_0^2} 
        = \abs{ (x+x_0)(x-x_0) }\\
            &\leq \underbrace{( \abs{x} + \abs{x_0} )}_{
                \leq 2\abs{x_0}+1, \text{ falls } \abs{x-x_0} < 1
            } \abs{x-x_0}
        \end{align*}
        Zu \( \varepsilon > 0 \) und \( x_0 \in\R \) wähle
        \[ \delta := \min(1, \frac{\varepsilon}{1 + 2\abs{x_0}})
        = \delta(\varepsilon, x_0) > 0 \]
        \begin{align*}
            \Rightarrow &\abs{ f(x) - f(x_0) }\\
            = &\abs{x^2 - x_0^2}\\
            \leq &(2\abs{x_0} + 1) \abs{x-x_0}\\
            < &\varepsilon \quad \forall \, \abs{x-x_0} < \delta.
        \end{align*}
        Also ist \(f\) stetig in jedem \( x_0\in\R \).
        \item Indikatorfunktion 
        \[ E\subset \R, 
        \mathds{1}_E : \R\rightarrow\R, x\mapsto 
        \begin{cases}
            1, x\in E\\
            0, x\in\R \setminus E
        \end{cases} \]
        Sei \( \mathds{1} \) stetig in \( x_0\in E \). Dann existiert
        \( \varepsilon = \frac{1}{2} \) ein \(\delta > 0\) so, dass
        \[ \abs{ \mathds{1}_E(x) - \mathds{1}_E(x_0) < \frac{1}{2} 
        \quad \forall x\in (x_0 - \delta, x_0 + \delta) }. \]
        Ist \( \mathds{1}_E(x_0) = 1 \Rightarrow \mathds{1}_E(x) = 1 \)
        somit \( (x_0 - \delta, x_0 + \delta) \subset E \).\\
        Ist \( \mathds{1}_E(x_0) = 0 \Rightarrow \mathds{1}_E(x) = 0 \)
        somit \( (x_0 - \delta, x_0 + \delta) \subset E \).\\
        Spezialfall \( E = \Q, \mathds{1}_\Q \) heißt Dirichletfunktion.\\
        Fakt: \( \Q \) ist dicht in \( \R \), d.\ h.\ in jedem Intervall
        \( (x_0 - \delta, x_0 + \delta) \) für \( x_0 \in\Q \) gibt es
        reelle Zahlen.\\
        Andererseits ist die Menge \( \R\setminus\Q \) auch dicht in \(\R \), 
        denn \( \sqrt{2} \notin \Q \) und \( \sqrt{2}+\Q \subset \R\setminus\Q \) \\
        \( \Rightarrow \) im Intervall \( (x_0 - \delta, x_0 + \delta) \)
        gibt es Punkte aus \( \R\setminus\Q \) \\
        \( \Rightarrow \) die Dirichletfunktion \( \mathds{1}_\Q \) ist
        in jedem Punkt \( x_0 \in\R \) unstetig.
        \item \[ x\in \R^d, \abs{x} = 
        {\left( \sum_{j=1}^d \abs{x_j}^2 \right)}^{\nicefrac{1}{2}} \]
        \( f:\R^d \rightarrow \R, x\mapsto \abs{x} \) ist stetig, denn
        \[ \abs{ f(x) - f(x_0) } = \abs{ \abs{x} - \abs{x_0} } 
        \overset{\text{Umgekehrte Dreiecksungl.}}{\leq} \abs{x - x_0}. \]
    \end{enumerate}
\end{bspe}
\begin{satz}[Folgenkriterium für Stetigkeit]
    Sei \( D\subset \R^n, f:D\rightarrow \R^m, x_0 \in D \). Dann sind
    äquivalent
    \begin{enumerate}
        \item \( f \) ist stetig in \( x_0 \).
        \item Für jede Folge \( {(x_k)}_{k\in\N} \) mit \( x_k \in D \)
        und \( \limes{k} x_k = x_0 \) folgt 
        \[ \limes{k} f(x_k) = f(x_0) \]
        (d.\ h.\  \( f(\limes{k} x_k) = \limes{k} f(x_k) \))
    \end{enumerate}
\end{satz}
\begin{bew}\leavevmode \\
    1. \(\Rightarrow \) 2. Sei \(x_k \in D \) und \( x_k 
    \rightarrow x_0 \) für \( k\rightarrow \infty \).
    Zu \( \varepsilon > 0 \) wähle man \( \delta > 0 \) mit
    \[ \abs{ f(x) - f(x_0) } < \varepsilon \quad \forall \, x\in D 
    \text{ mit } \abs{x-x_0}<\delta. \]
    \[ \Rightarrow \abs{ f(x_k) - f(x_0) } < \varepsilon
    \text{ für fast alle } k\in\N \]
    \[ \Rightarrow \limes{k} \abs{ f(x_k) - f(x_0) } = 0. \]
    2. \(\Rightarrow \) 1. Kontraposition. Sei \( f \) nichtstetig
    in \( x_0 \in D \).\\
    Dann gibt es ein \( \varepsilon > 0 \)
    \[ \forall \, \delta > 0 \,\exists \, x\in D : \abs{x-x_0} 
    < \delta : \abs{ f(x) - f(x_0) } \geq \varepsilon.\ (*) \]
    Wähle \( \delta = \frac{1}{n} \overset{(*)}{\Rightarrow}
    \exists \) Folge \( {(x_n)}_n, x_n \in D, x_n \rightarrow x_0 \)
    mit \( \abs{ f(x_n) - f(x_0) } \geq \varepsilon. \)
    \[ \Rightarrow f(x_n) \text{ konvergiert nicht gegen } f(x_0). \]
    \[ \neg 1. \Rightarrow \neg 2.\text{, d.\ h.\ }2.\Rightarrow 1. \]
\end{bew}
\begin{satz}[Verkettung]
    Seien \( f:D\rightarrow \R^m, f(D) \subset E \in\R^m, 
    g: E\rightarrow\R^k \). Ist \( f \) stetig in \( x_0 \)
    und \( g \) stetig in \( y_0 := f(x_0) \), so ist 
    \( g \circ f : D\rightarrow \R^k, x \mapsto (g\circ f)(x) 
    = g(f(x)) \) stetig in \( x_0 \).
\end{satz}
\begin{bew}
    Sei \( {(x_n)}_n \) eine Folge in \( D \) mit 
    \( x_n \rightarrow x_0 \).
    \[ \overset{\text{Satz 2}}{\Rightarrow} f(x_n) \rightarrow f(x_0)
    = y_0 \text{, da } f \text{ stetig ist.} \]
    \[ \overset{\text{Satz 2}}{\Rightarrow} g(f(x_n))
    \rightarrow g(y_0) = g(f(x_0)) \text{, da } g \text{ stetig ist.} \]
    \[ \overset{\text{Satz 2}}{\Rightarrow} g\circ f 
    \text{ ist stetig in } x_0. \]
\end{bew}
\end{document}