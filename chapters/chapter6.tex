\documentclass[../ana1.tex]{subfiles}
\begin{document}
\setcounter{section}{6}

\section{Existenz von Wurzeln (in $\R$)}
Sei $n\in\N$ und $a>0$. Dann heißt eine Zahl $\alpha$ $n$-te Wurzel von $a$, schreiben $\alpha = a^{\frac{1}{n}}$ oder $\sqrt[n]{a}$, falls $a^n = a$.
\begin{satz}
	Sei $\alpha \in\R, a>0$ und $n\in\N$. Dann existiert die $n$-te Wurzel von $a$ als reelle Zahl. D.h. $\exists ! \alpha\in\R$ mit $\alpha>0$ und $\alpha^n = a$.
\end{satz}
\begin{bem}
	Für die rationalen Zahlen ist dies falsch!
\end{bem}
\begin{bew}
	Angenommen, die Beh. gilt für $a\geq 1$. Sei $0<b<1$. Setze $a :=\frac{1}{b}>1\Rightarrow \exists !\alpha>0:\alpha^n = a = \frac{1}{b}$. Setze $\beta := \frac{1}{\alpha}$.\\
	Dann gilt also $$\beta^n = \left(\frac{1}{\alpha}\right)^n = \frac{1}{\alpha^n} = \frac{1}{a} = b.$$
	Also nehme an $a\geq 1$. Ist $a=1$, so ist $\alpha = 1$ die einzige Lösung von $\alpha^n = 1$. Außerdem können wir $n\in\N$ mit $n>1$ wählen. Also sei $a>1,n\in\N,n>1$. Setze
	$$A := \{x\in\R|0<x, x^n<a\}$$
	Dann ist $1\in A$ und somit $A\neq\emptyset$. Außerdem ist $A$ nach oben beschränkt, denn ist $y \geq a$, so folgt
	$$y^n \geq a^n = \underbrace{a \cdot a \ldots \cdot a}_{n\text{-mal}} > \underbrace{1 \cdot 1 \ldots \cdot 1}_{n\text{-mal}} \cdot a = a$$
	Also ist $A \leq a$.
	$$\overset{\text{Vollst.axiom}}{\Rightarrow} \alpha := \sup A \in\R \text{ existiert}.$$
	Da $1\in A \Rightarrow \alpha \geq 1 > 0$.\\
	$\alpha^n$ ist eine reelle Zahl für die gilt nach Anordnungsaxiom entweder $\alpha^n <a, \alpha^n >a$ oder $\alpha^n = a$.\\
	1. Fall: Annahme: $\alpha^n <a$.\\
	Sei $0<\delta\leq 1$. Dann gilt (binom. Formel)
	\begin{align*}
		(\alpha + \delta)^n & = \sum_{k=0}^{n} \binom{n}{k} \alpha^k\delta^{n-k}                                                                                                            \\
		                    & = \alpha^n + \sum_{k=0}^{n-1} \binom{n}{k} \alpha^k\delta^{n-k}                                                                                               \\
		                    & = \alpha^n + \sum_{k=1}^{n-1} \binom{n}{k-1} \underbrace{\alpha^{k-1}}_{\leq a^{k-1}}\underbrace{\delta^{n+1-k}}_{\leq \delta \cdot \delta^{n-k} \leq \delta} \\
		                    & \leq \alpha^n \delta \sum_{k=1}^{n-1} \binom{n}{k-1} \alpha^{k-1}                                                                                             \\
		                    & \leq \alpha^n \delta \sum_{k=0}^{n} \binom{n}{k} \alpha^{k}                                                                                                   \\
		                    & = \alpha^n + \delta(a+1)^n (*)
	\end{align*}
	$\alpha^n <a$ nach Annahme und $\delta := \frac{1}{2} \min\left(1, \frac{a-\alpha^n}{(a+1)^n}\right)$\\
	Dann gilt $0<\delta\leq 1$ und $(*)$
	$$(\alpha + \delta)^n \leq \alpha^n + \delta(a+1)^n \leq \alpha^n + \frac{1}{2} (a-\alpha^n) = \frac{1}{2} (\alpha^n+a) < \frac{1}{2} (a+a) =a$$
	Somit ist $\alpha < \alpha + \delta$ \Lightning $\alpha$ ist $\sup A$.\\
	2. Fall: Annahme: $\alpha^n > a, 0<\delta \leq 1$
	\begin{align*}
		\Rightarrow (\alpha - \delta)^n & = \sum_{k=0}^{n} \binom{n}{k} \alpha^{n-k} (-\delta)^k = \sum_{k=0}^{n} \binom{n}{k} \alpha^{n-k} (-1)^k\delta^k \\
		                                & = \alpha^n + \sum_{k=0}^{n-1} \binom{n}{k+1} \alpha^{n-1-k} (-1)^{k+1}\delta^{k+1}                               \\
		                                & = \alpha^n - \sum_{k=0}^{n-1} \binom{n}{k+1} \alpha^{n-1-k} (-1)^{k} \delta^k                                    \\
		                                & \geq \alpha^n - \delta \sum_{k=0}^{n-1} \binom{n}{k+1} a^{n-k+1}                                                 \\
		                                & = \alpha^n - \delta \sum_{k=1}^{n} \binom{n}{k} a^k                                                              \\
		                                & \geq \alpha^n - \delta (a+1)^n (**)
	\end{align*}
	Setze $\delta := \frac{1}{2} \min \left(1, \frac{\alpha^n-a}{(a+1)^n} \right)$. Dann gilt $0\leq\delta\leq \frac{1}{2}<1$.
	$$(\alpha -\delta)^n \geq \alpha^n - \frac{1}{2} (\alpha^n + a) = \frac{1}{2} (\alpha^n + a) > \frac{1}{2} (a + a) \geq a.$$
	Somit $\alpha - \delta$ eine obere Schranke für $A$. Da $\alpha - \delta<\alpha$, ist das ein Widerspruch zu $\alpha = \sup A$. Somit bleibt nur $\alpha^n = a$.
\end{bew}

\end{document}