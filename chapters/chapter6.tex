\documentclass[../ana1.tex]{subfiles}
\onlyinsubfile{\sectionNumbering} %Use numbering relative to sections and not subsection

\begin{document}
\setcounter{section}{5}

\section{Existenz von Wurzeln in \(\R \)}

\begin{notation}
	Sei \( n \in\N \) und \(a > 0 \). Eine Zahl \(\alpha \in \R \) 
	heißt dann \textit{\(n\)-te Wurzel von \(a \)}, 
	falls \(\alpha^n = a \).\\
	In diesem Fall schreibe auch 
	\( \alpha \eqqcolon a^{\frac{1}{n}}\) 
	oder \(\alpha \eqqcolon \sqrt[n]{a} \).
\end{notation}

\begin{satz}\label{satz:ex_wurzel}
	Sei \(\alpha \in\R, a>0\) und \(n \in \N \). Dann existiert die \(n\)-te Wurzel von \(a\) als eindeutige reelle Zahl,
	\dphp{} \(\existse \, \alpha \in \R \) mit \(\alpha > 0 \) und \(\alpha^n = a \).
\end{satz}
\begin{bew}
	Angenommen, die Behauptung gilt für \(a \geq 1 \). \\
	Sei \(0 < b < 1\) und setze \(a \coloneqq \frac{1}{b} > 1 \implies \existse \, \alpha > 0 \colon \alpha^n = a = \frac{1}{b} \). \\
	Setze ferner \(\beta \coloneqq \frac{1}{\alpha}\). \\
	Dann gilt also
	\[ \beta^n = {\left(\frac{1}{\alpha}\right)}^n = \frac{1}{\alpha^n} = \frac{1}{a} = b.\]
	Sei also \(a \geq 1 \). Ist \(a = 1 \), so ist \(\alpha = 1 \) die einzige Lösung von \(\alpha^n = 1 \). Außerdem können wir annehmen, dass \(n > 1 \).
	Es seien nun also \(a > 1 \ko n > 1 \). Setze
	\[ A \coloneqq \set{ x \in \R \; \vert \; 0 < x \ko x^n< a }. \]
	Dann ist \(1 \in A \) und somit \(A \neq \emptyset \). Außerdem ist \(A \) nach oben beschränkt, denn ist \(y \geq a\), so folgt
	\[ y^n \geq a^n = \underbrace{a \cdot a \cdots a}_{n\text{-mal}} > \underbrace{1 \cdot 1 \cdots \cdot 1}_{n-1\text{-mal}} \cdot a = a \]
	Somit ist \(A \leq a \). \(\overset{\text{\autoref{ax:V}}}{\implies} \alpha \coloneqq \sup A \in \R \) existiert. Da \(1 \in A \) folgt \(\alpha \geq 1 > 0 \). \\
	Für \(\alpha \) gilt entweder \(\alpha^n <a \ko \alpha^n > a \) oder \(\alpha^n = a \). Ist \(\alpha^n = a \), so sind wir fertig. Bleibt also noch zu zeigen,
	dass die anderen beiden Fälle nicht auftreten können.
	\begin{faelle}
		\item[Fall \(\alpha^n < a\):] Sei \(0 < \delta \leq 1 \). Dann gilt
			\!\begin{align*}
				{(\alpha + \delta)}^n & = \sum_{k=0}^{n} \binom{n}{k} \alpha^k\delta^{n-k} \\
								  	  & = \alpha^n + \sum_{k=0}^{n-1} \binom{n}{k} \alpha^k\delta^{n-k} \\
									  & = \alpha^n + \sum_{k=1}^{n-1} \binom{n}{k-1} \underbrace{\alpha^{k-1}}_{\leq a^{k-1}}\underbrace{\delta^{n+1-k}}_{\leq \delta \cdot \delta^{n-k} \leq \delta} \\
									  & \leq \alpha^n \delta \sum_{k=1}^{n-1} \binom{n}{k-1} \alpha^{k-1} \\
									  & \leq \alpha^n \delta \sum_{k=0}^{n} \binom{n}{k} \alpha^{k} \\
									  & = \alpha^n + \delta {(a+1)}^n. \tag{\(*\)}
			\end{align*}
			Es gilt \(\alpha^n < a\) nach Annahme 
			und \(\delta \coloneqq \frac{1}{2} 
			\min{\left( 1 \ko \frac{a-\alpha^n}{{(a+1)}^n} \right)} \) \\
			Dann gilt \(0<\delta\leq 1\) und \((*) \). Also
			\begin{align*}
				&{(\alpha + \delta)}^n \\
				\leq \; &\alpha^n + \delta {(a+1)}^n \\
				\leq \; &\alpha^n + \frac{1}{2} (a-\alpha^n) \\
				= \; &\frac{1}{2} (\alpha^n+a) \\
				< \; &\frac{1}{2} (a+a) \\
				= \; &a
			\end{align*}
			Somit ist \(\alpha < \alpha + \delta \). \Lightning{} zu \(\alpha \) ist \(\sup A\). \\
		\item[Fall \(\alpha^n > a\):] Sei \(0 < \delta \leq 1 \). Dann gilt
			\!\begin{align*}
				{(\alpha - \delta)}^n & = \sum_{k=0}^{n} \binom{n}{k} \alpha^{n-k} {(-\delta)}^k = \sum_{k=0}^{n} \binom{n}{k} \alpha^{n-k} {(-1)}^k\delta^k \\
								      & = \alpha^n + \sum_{k=0}^{n-1} \binom{n}{k+1} \alpha^{n-1-k} {(-1)}^{k+1}\delta^{k+1} \\
								   	  & = \alpha^n - \sum_{k=0}^{n-1} \binom{n}{k+1} \alpha^{n-1-k} {(-1)}^{k} \delta^k \\
									  & \geq \alpha^n - \delta \sum_{k=0}^{n-1} \binom{n}{k+1} a^{n-k+1} \\
									  & = \alpha^n - \delta \sum_{k=1}^{n} \binom{n}{k} a^k \\
					 				  & \geq \alpha^n - \delta {(a+1)}^n \tag{\(**\)}
			\end{align*}
			Setze nun \( \delta \coloneqq \frac{1}{2} \min \left(1 \ko \frac{\alpha^n-a}{{(a+1)}^n} \right) \). Dann gilt \(0 \leq \delta \leq \frac{1}{2} < 1 \). Somit gilt
			\[{(\alpha -\delta)}^n \geq \alpha^n - \frac{1}{2} (\alpha^n + a) = \frac{1}{2} (\alpha^n + a) > \frac{1}{2} (a + a) \geq a. \]
			Also ist \( \alpha - \delta \) eine obere Schranke für \(A \). Da \(\alpha - \delta < \alpha \), ist das ein Widerspruch zu \(\alpha = \sup A\). Somit bleibt nur \(\alpha^n = a\).
	\end{faelle}
\end{bew}

\begin{bem}
	Für die rationalen Zahlen ist die Aussage aus \autoref{satz:ex_wurzel} falsch. \zB{} \(\sqrt{2} \notin \Q \).
\end{bem}

\end{document}